% $Id$
% Copyright 2009-2013 CV Radhakrishnan.
% Released under LPPLv1.3+.
% 
% TeX4ht and \Configure, part 2.

\ifx\texhtstandalonedoc1
\documentclass[a4paper]{article}
\usepackage{xspace,graphicx,blog}
\begin{document}
\section{Cross-linking}
\else
\section{\textbackslash{}Configure Cross-linking}
\fi

\begin{verbatim}
  \Link[@1 @2]{@3}{@4}...\EndLink
\end{verbatim}

\noindent  Creates

\begin{verbatim}
  <a href="@1#@3" name="@4" @2>...</a>
\end{verbatim}

\begin{itemize}
\item   When \Verb=@1= is empty, \texht will derive its value automatically.
    The derived value will be the file name containing the target \Verb=@3=.

  \item  \Verb=\Link= may be followed by \Verb=-=, if \texht needs not automatically
    determine (for other \Verb=\link= commands) the file containing \Verb=@4=.
    In the present of such a flag, \texht can spare a definition of
    one macro.

  \item The component \Verb=[@1 @2]= is optional. If omitted,
    \Verb=@1= and \Verb=@2= are  assumed to be empty.

  \item  The \Verb=href= attribute is omitted when \Verb=@1= and
    \Verb=@3= are empty. 

  \item  The name attribute is omitted when \Verb=@4= is empty.
  
\end{itemize}
\ifx\cvrtexht\undefined\else\Tg</ul>\fi

  \Example

\begin{verbatim}
  \Link{a}{}...\Endlink .....  \Link{}{b}...\EndLink
  \Link[http://foo  id="fooo"]{a}{b}...\EndLink

  \Configure{Link}..............4
\end{verbatim}

\noindent   Configures \Verb=\Link...\EndLink= so that

\begin{tabular}{ll}
\fline     \#1 replaces \Verb=a=

\fline     \#2 replaces \Verb+href=+

\fline     \#3 replaces \Verb+name=+

\fline     \#4 replaces `\#'.  If empty, the older value remains in effect.

\end{tabular}

\Example

\begin{verbatim}
  \Configure{Link}{a}{href=}{name=}{}
  \Configure{Link}{ref}{target=}{id=}{\empty}

  \Configure{?Link}..............1
\end{verbatim}

\begin{tabular}{ll}
\fline   \#1 insertion before broken links

\end{tabular}
\par\medskip

\noindent   To help with debugging

\begin{verbatim}
\LinkCommand...................1 <= i <= 6
\end{verbatim}
\noindent
   Creates a \Verb=\Link=-like command

\begin{tabular}{ll}

\fline   \#1   tag name

\fline   \#2   href-like attribute

\fline   \#3   name-like attribute

\fline   \#4   insertion

\fline   \#5   \Verb=/=, if empty element

\fline   \#6  replacement for \#  (ignored if absent)

\end{tabular}

\Example

\begin{verbatim}
  \LinkCommand\JSLink{a,\noexpand\jsref,name}
  \def\jsref="#1"{href="javascript:window.open('#1')"}

  \JSLink{a}{}xx\EndJSLink
  \Link{}{a}\EndLink       % or \JSLink{}{a}\EndJSLink

  \Configure{XrefFile}.....................1
\end{verbatim}

\begin{tabular}{ll}

\fline \#1 names cross-references of files (appends \#1 to \Verb=)F= and
      \Verb=)Q= entries of the \Verb=.xref= files). Applicable mainly
      implicitly within \Verb=\Link= commands.

\end{tabular}

\begin{verbatim}
  \Tag.....................................2
\end{verbatim}

\begin{tabular}{ll}
\fline   \#1  label

\fline   \#2  content

\end{tabular}

\begin{verbatim}
  \Ref.....................................1
  \LikeRef.................................1
\end{verbatim}

\begin{tabular}{ll}
 \fline  \#1  label

\end{tabular}

\par\medskip

   \Verb=\Tag= and \Verb=\Ref= are \Verb=tex4ht.sty= commands
   introduced cross-referencing 
   content through \Verb=.xref= auxiliary files.

   \Verb=\LikeRef= is a variant of \Verb=\Ref= which doesn't verify whether the
   labels exit.  It is mainly used in \Verb=\Link= and \Verb=\edef= environments.

\begin{verbatim}
  \ifTag ..................................3
\end{verbatim}

\begin{tabular}{ll}
\fline   \#1  questioned tag

\fline   \#2  true part

\fline   \#3  false part

\end{tabular}

\begin{verbatim}
  \LoadRef-[prefix]+{filename.ext}{pattern}
\end{verbatim}

\noindent   Load the named xref-type file
\fspace=20mm
\begin{tabular}{ll}
\fline .xref  optional --- \Verb=.xref= is assume for a default\par

\fline   +    optional --- asks \Verb=\Ref= and \Verb=\LikeRef= commands
              to use expanded tags \Verb=filename::tag=, instead of
              just \Verb=tag=

\fline[prefix]   optional --- asks just for tags starting with the
              specified prefix.

\fline   -    optional --- deletes the prefixes from the loaded tags

\fline   \{pattern\}  to be included only when \Verb=[prefix]= or
\Verb=+= are included. 
              States how tags are to be addressed, with the parameter
              symbol `\#1'  specifying the loaded part.

\end{tabular}

\fspace=6mm

   \Example

\begin{verbatim}
  % a.tex
  \LoadRef-[to:]{b}{from:#1}      \Ref{from:filename}
                                  \LikeRef{from:filename}

  % b.tex
  \Tag{to:filename}{\FileName}
\end{verbatim}

   \Example

\begin{verbatim}
  \LoadRef-[)F]{file}{)Ffoo##1}
  \LoadRef-[)Q]{file}{)Qfoo##1}
  \Configure{XrefFile}{foo}   \Link...\EndLink
  \LoadRef{another-file}
\end{verbatim}

\ifx\texhtstandalonedoc1 \end{document}\fi
