%$Id: tex4ht-process.tex,v 1.1 2010/09/05 09:02:33 cvr Exp cvr $

\documentclass[a4paper]{article}

 \usepackage{xspace,graphicx,blog}

\begin{document}


\section{Cross-linking}

\begin{verbatim}
  \Link[@1 @2]{@3}{@4}...\EndLink
\end{verbatim}

\noindent  Creates

\begin{verbatim}
  <a href="@1#@3" name="@4" @2>...</a>
\end{verbatim}

\begin{itemize}
\item   When \Verb=@1= is empty, \texht will derive its value automatically.
    The derived value will be the file name containing the target \Verb=@3=.

  \item  \Verb=\Link= may be followed by \Verb=-=, if \texht needs not automatically
    determine (for other \Verb=\link= commands) the file containing \Verb=@4=.
    In the present of such a flag, \texht can spare a definition of
    one macro.

  \item The component \Verb=[@1 @2]= is optional. If omitted,
    \Verb=@1= and \Verb=@2= are  assumed to be empty.

  \item  The \Verb=href= attribute is omitted when \Verb=@1= and
    \Verb=@3= are empty. 

  \item  The name attribute is omitted when \Verb=@4= is empty.
  
\end{itemize}
\ifx\cvrtexht\undefined\else\Tg</ul>\fi

  \Example

\begin{verbatim}
  \Link{a}{}...\Endlink .....  \Link{}{b}...\EndLink
  \Link[http://foo  id="fooo"]{a}{b}...\EndLink

  \Configure{Link}..............4
\end{verbatim}

\noindent   Configures \Verb=\Link...\EndLink= so that

\begin{tabular}{ll}
\fline     \#1 replaces \Verb=a=

\fline     \#2 replaces \Verb+href=+

\fline     \#3 replaces \Verb+name=+

\fline     \#4 replaces `\#'.  If empty, the older value remains in effect.

\end{tabular}

\Example

\begin{verbatim}
  \Configure{Link}{a}{href=}{name=}{}
  \Configure{Link}{ref}{target=}{id=}{\empty}

  \Configure{?Link}..............1
\end{verbatim}

\begin{tabular}{ll}
\fline   \#1 insertion before broken links

\end{tabular}
\par\medskip

\noindent   To help with debugging

\begin{verbatim}
\LinkCommand...................1 <= i <= 6
\end{verbatim}
\noindent
   Creates a \Verb=\Link=-like command

\begin{tabular}{ll}

\fline   \#1   tag name

\fline   \#2   href-like attribute

\fline   \#3   name-like attribute

\fline   \#4   insertion

\fline   \#5   \Verb=/=, if empty element

\fline   \#6  replacement for \#  (ignored if absent)

\end{tabular}

\Example

\begin{verbatim}
  \LinkCommand\JSLink{a,\noexpand\jsref,name}
  \def\jsref="#1"{href="javascript:window.open('#1')"}

  \JSLink{a}{}xx\EndJSLink
  \Link{}{a}\EndLink       % or \JSLink{}{a}\EndJSLink

  \Configure{XrefFile}.....................1
\end{verbatim}

\begin{tabular}{ll}

\fline   \#1 names cross-references of files (appends \#1 to \Verb=)F= and \Verb=)Q=
      entries of the \Verb=.xref= files). Applicable mainly implicitly
      within \Verb=\Link= commands.

\end{tabular}

\begin{verbatim}
  \Tag.....................................2
\end{verbatim}

\begin{tabular}{ll}
\fline   \#1  label

\fline   \#2  content

\end{tabular}

\begin{verbatim}
  \Ref.....................................1
  \LikeRef.................................1
\end{verbatim}

\begin{tabular}{ll}
 \fline  \#1  label

\end{tabular}

\par\medskip

   \Verb=\Tag= and \Verb=\Ref= are \Verb=tex4ht.sty= commands
   introduced cross-referencing 
   content through \Verb=.xref= auxiliary files.

   \Verb=\LikeRef= is a variant of \Verb=\Ref= which doesn't verify whether the
   labels exit.  It is mainly used in \Verb=\Link= and \Verb=\edef= environments.

\begin{verbatim}
  \ifTag ..................................3
\end{verbatim}

\begin{tabular}{ll}
\fline   \#1  quetioned tag

\fline   \#2  true part

\fline   \#3  false part

\end{tabular}

\begin{verbatim}
  \LoadRef-[prefix]+{filename.ext}{pattern}
\end{verbatim}

\noindent   Load the named xref-type file
\fspace=20mm
\begin{tabular}{ll}
\fline .xref  optional --- \Verb=.xref= is assume for a default\par

\fline   +    optional --- asks \Verb=\Ref= and \Verb=\LikeRef= commands
              to use expanded tags \Verb=filename::tag=, instead of
              just \Verb=tag=

\fline[prefix]   optional --- asks just for tags starting with the
              specified prefix.

\fline   -    optional --- deletes the prefixes from the loaded tags

\fline   \{pattern\}  to be included only when \Verb=[prefix]= or
\Verb=+= are included. 
              States how tags are to be addressed, with the parameter
              symbol `\#1'  specifying the loaded part.

\end{tabular}

\fspace=6mm

   \Example

\begin{verbatim}
  % a.tex
  \LoadRef-[to:]{b}{from:#1}      \Ref{from:filename}
                                  \LikeRef{from:filename}

  % b.tex
  \Tag{to:filename}{\FileName}
\end{verbatim}

   \Example

\begin{verbatim}
  \LoadRef-[)F]{file}{)Ffoo##1}
  \LoadRef-[)Q]{file}{)Qfoo##1}
  \Configure{XrefFile}{foo}   \Link...\EndLink
  \LoadRef{another-file}
\end{verbatim}

\end{document}

\section{Files}

\begin{itemize}

\item \Verb=\FileName=:    Holds the name of the current hypertext file.
\item \Verb=\FileNumber=:  Holds the internal number of the current hypertext file.

\end{itemize}

\begin{verbatim}
  \RefFileNumber...........................1
   #1  File number
\end{verbatim}

\noindent    Provides the file name.

\begin{verbatim}
  \NextFile.................................1
   #1 Requested name for the next file
  \Hinput{#1}
\end{verbatim}
\noindent
    The command asks to load the configuration files associated
    with mark \Verb=#1=.

\begin{verbatim}
  \Hinclude[#1]{#2}
\end{verbatim}
\noindent
    The command associates configuration file \Verb=#2= with mark \Verb=#1=.  If
    the mark is the star character \Verb=*=, the configuration files is
    associated to all marks.  The command is applicable until the
    \Verb=\Preamble= command is processed.
   For instance,  \Verb=\Hinclude[*]{html4.4ht}....\Hinput{latex}=.

\begin{verbatim}
  \Hinclude{#1}{#2}
\end{verbatim}
\noindent
    The command is applicable while the \Verb=\Preamble= command is
    processed. Its purpose is to load \Verb=*4ht= hook files within
    the fragments of code specified in \Verb=#1=.
   For instance, \Verb=\Hinclude{\input plain.4ht}{plain}=.


\section{Fonts}

\begin{verbatim}
  \Configure{htf}...............................9

    #1       label (integer 0--255)
    #2       delimiter (a character not appearing in #3,...,#9)
             even label             odd label
    #3       start opening tag      start empty tag
    #4       name                   alt
    #5       size                   name
    #6       mag                    size
    #7                              mag
    #8       end the tag            ord
    #9       closing tag            end the tag
\end{verbatim}

    The \Verb=htf= fonts assign a content and a label to each symbol (possibly
    followed by a comment).  For instance,

\begin{verbatim}
        'e'    '1'    epsilon
        'z'    '3'    zeta
\end{verbatim}

    An even label asks that the content itself will be used for the
    symbol, and an odd label asks that the symbol will be represented by a
    bitmap.  In the later case,  the content serves as a substitution for
    browsers which don't exhibit bitmaps.

    The \Verb=\Configure{htf}...= command provides label-dependent wrappers to
    chosen representations.

    If they are not empty, \Verb=mag= and \Verb=ord= must be c-type
    patterns for integer arguments, and \Verb=name= and \Verb=size=
    should be a patterns for strings.  The \Verb=mag= entry is
    ignored for fonts of the default dimension. Together
    they specify a attribute-value format, mainly for references
    in the \Verb=css= code.

  \Example

\begin{verbatim}
  \Configure{htf}{0}{+}{<span\Hnewline
     class="}{\%s}{-\%s}{x-x-\%d}{}{">}{</span>}
  \Configure{htf}{1}{+}{<img\Hnewline
     src="}{" alt="}{" class="}{\%s}{-\%d}{x-x-\%x}{" />}
\end{verbatim}

\begin{verbatim}
  \Configure{htf-attr}....................... 2
\end{verbatim}
  
\fline \#1  c-pattern for the font name and size

\fline     \#2  c-pattern for font magnification
\par\medskip

     Specify the format of the selectors within the \textsc{CSS} files.

   \Example
\begin{verbatim}
  \Configure{htf-attr}{.\%s-%s}{--\%s}
  \Configure{htf-css}....................... 2
\end{verbatim}

\fline    \#1  font name or label

\fline    \#2  css entry
\par\medskip

    A variant of the \Verb=\Css= command. If \Verb=#1= is a font name,
    the contribution replaces the one given within the
    \Verb=htf= font definition. If \Verb=#1= is a label for an entry
    of a \Verb=htf= font, the contribution is added to the \textsm{CSS}
    file.  The contribution is offered, only when the
    font is in use.

  \Example

\begin{verbatim}
   \Configure{htf-css}{4}{.small-caps{font-variant: small-caps;}}
\end{verbatim}

\section{Bitmaps}

\begin{verbatim}
  \Configure{Picture}....................... \#1
\end{verbatim}

\fline  \#1  Extension name for bitmap files of \Verb=dvi= pictures,
      stored in \Verb=\PictExt=.
\par\medskip

  \textbf{Default:} \Verb=\Configure{Picture}{.png}=

  The extension names of bitmap files of glyphs of \Verb=htf= fonts may be
  determined within a g-entry in the environment file \Verb=tex4ht.env=, or a
  g-flag of the \Verb=tex4ht.c= utility.

\end{document}

\begin{verbatim}
  \Configure{Picture-alt}......................1

  \#1  alt value for \Picture+{...}  and \Picture*{...}

\begin{verbatim}
  \Configure{Picture+}.........................2
  \Configure{Picture*}.........................2

  \#1  before the dvi picture code
  \#2  after the dvi picture code

  Typically, the plus `+' variant is introduced as an inline
  contribution into paragraphs, and the star `*' variant as an
  independent block between paragraphs.

\begin{verbatim}
  \Configure{PictureAlt}........................2
  \Configure{PictureAlt*+}......................2
  \Configure{PictureAlt*+[]}....................2

  \#1 definitions before alt
  \#2 definitions after alt

 Apply to \Picture{...}, \Picture*+{...}, and \Picture*+[...]{...}

\begin{verbatim}
  \Configure{PictureAlt}........................1
  \Configure{PictureAlt*+}......................1
  \Configure{PictureAlt*+[]}....................1

  \#1 definition for attributes (introduced through
     a parameter named `\#1')

   Apply to \Picture{...}, \Picture*+{...}, and \Picture*+[...]{...}

\begin{verbatim}
  \Configure{IMG}...............................5

  \#1 before file name
  \#2 between file name and alt
  \#3 close alt for  \Picture without * or +
  \#4 close alt for  \Picture with * and +
  \#5 right delimiter

  \Example

\begin{verbatim}
     \Configure{IMG}
        {\ht:special{t4ht=<img src="}}
        {\ht:special{t4ht=" alt="}}
        {" }
        {\ht:special{t4ht=" }}
        {\ht:special{t4ht=/>}}

\NextPictureFile.............................1

   Requests a file name for the next created picture.

\begin{verbatim}
  \PictureFile.............................0

   Records the filename of the most recent created picture.

Math
----

\begin{verbatim}
\Configure{$}................................2
\Configure{$$}...............................2
\Configure{DviMath}..........................2

\DviMath ... \EndDviMath
\MathClass ... \EndMathClass
\PicMath ... \EndPicMath
\DisplayMath ... \EndDispalyMath

   \Example

\begin{verbatim}
     \Configure{$} {\Tg<math>\DviMath} {\EndDviMath\Tg</math>} {}



\begin{verbatim}
\Configure{PicMath}..........................4

   \Example

\begin{verbatim}
       \Configure{PicMath}{}{}{}{ class="math" }

     \Configure{()}{\protect\PicMath$}{$\protect\EndPicMath}

\Configure{SUB}..............................2
\Configure{SUP}..............................2
\Configure{SUBSUP}...........................3
\Configure{SUPSUB}...........................3
\Configure{SUB/SUP}..........................6

\Configure{putSUB}...........................1
\Configure{putSUP}...........................1

     \#1 the code to be used for realizing subscripts and postcripts

\begin{verbatim}
\Configure{afterSUB}.........................2

     \#1 look ahead token after subscript
     \#2 the code to be used for realizing subscripts having \#1 for
        lookahead token

\begin{verbatim}
\Configure{over}.............................2
\Configure{atop}.............................2
\Configure{above}............................2
\Configure{overwithdelims}...................2
\Configure{atopwithdelims}...................2
\Configure{abovewithdelims}..................2

   \#1 before \over, \atop, \above
             \overwithdelims, \atopwithdelims, \abovewithdelims
   \#2 after  \over, \atop, \above <dimension>
             \overwithdelims <del1> <del2>
             \atopwithdelims <del1> <del2>
             \abovewithdelims <del1> <del2> <dimension>

   \Example

\begin{verbatim}
     \Configure{over}
         {\Send{GROUP}{0}{[before]}[before-rule]}
         {[before-argument]\Send{EndGROUP}{0}{[after]}}


\begin{verbatim}
\Configure{MathClass}........................5

   \#1  class number
          0: mathord, 1: mathop, 2: mathbin, 3: mathrel,
          4: mathopen, 5: mathclose, 6: mathpunc
   \#2  delimiter
   \#3  before
   \#4  after
   \#5  characters

   Extra support:

\begin{verbatim}
      \PauseMathClass
      \EndPauseMathClass
      \NewMathClass<new control sequence>  (7, 8, ...)

\begin{verbatim}
\Configure{FormulaClass}.....................4

   \#1  class number
          0: mathord, 1: mathop, 2: mathbin, 3: mathrel,
          4: mathopen, 5: mathclose, 6: mathpunc
   \#2  a character not in \#3 and \#4
   \#3  before
   \#4  after

   If \#2 is empty, the formula gets the same marking as a
   single character of the specified type

\begin{verbatim}
\Configure{FormulaClass*}....................4

   Like the previous case, but allow marking in the
   nested content.

\begin{verbatim}
\Configure{MathDelimiters}...................2

   \#1  left
   \#2  right

\begin{verbatim}
\Configure{mathbin*}.........................4
\Configure{mathclose*}.......................4
\Configure{mathop*}..........................4
\Configure{mathopen*}........................4
\Configure{mathord*}.........................4
\Configure{mathpunct*}.......................4
\Configure{mathrel*}.........................4

   \#1  a character not presented in \#2\#3\#4
   \#2  code before
   \#3  code after
   \#4  possible definitions for successive cases

  \Example

\begin{verbatim}
     \Configure{mathop*}{*}{}{}
        {\Configure{mathop}{*}{<mo>}{</mo>}{}}
     \mathop{\overline{x \mathop{op} y}} \limits^{a=3}

\Configure{mathbin}..........................4
\Configure{mathclose}........................4
\Configure{mathopen}.........................4
\Configure{mathop}...........................4
\Configure{mathord}..........................4
\Configure{mathpunct}........................4
\Configure{mathrel}..........................4

Variants of the above group, requesting to supress nested marks.

\begin{verbatim}
\Configure{nolimits}.........................1

\MathSymbol

AtBeginDocument
---------------

\begin{verbatim}
\Configure{AtBeginDocument}..................2

    \#1  before the corresponding hook of latex
    \#2  after

   Insertions are accumulative, and can be erased by providing
   two empty arguments

Other Hooks
-----------

\begin{verbatim}
  \Configure{HChar}...................1

    \#1  a character

    The \HChar{i} instruction inserts the character code i with the
    font information of character \#1, when i is positive. If i is
    negative, the font info is not included.


\begin{verbatim}
\Configure{Canvas}
\Configure{ExitHPage}
\Configure{LinkHPage}......................1
\Configure{FontCss}
\Configure{HVerbatim+}
\Configure{MiniHalign}
\Configure{Needs-}
\Configure{Needs}


\Configure{TraceTables}
\Configure{edit}
\Configure{halignTB}
\Configure{halignTD}
\Configure{halign}
\Configure{hooks}
\Configure{moveright}

\Configure{noalign-}
\Configure{pic-halign}

\Configure{accent}
\Configure{mathaccent}
\Configure{accented}
\Configure{accenting}


\section{Back-end Specials}

\begin{verbatim}

                                                    insertions
                                                    ----------
  =    \special{t4ht=...content...}
           Insert the specified content to the html output, under
           edef mode of processing, and without using the mapping
           of the htf fonts.  Used in \HCode{...}.
  @    \special{t4ht@...integer...}
           Insert the absolute value as character code to the output.
           Positve values ask the insertion to be included in place
           of the next chracter, together with the font information
           of that character.
                                                    files
                                                    -----
  >    \special{t4ht>...file-name...}
           Open a new file, if needed, and direct future output
           to the specified file.  Used in \File{...}.
  <    \special{t4ht<...file-name...}
           Close the specified file.  If it is the current file,
           activate the youngest file. Used in \EndFile{...}.
  >*   \special{t4ht*>...file-name...}
           Declare the file to be the oldest.
       \special{t4ht*>}
           Reactivate the file that activated the current file.
  *<   \special{t4ht*<file}
           Input file (with no processing)
  +    \special{t4ht++file-name}...dvi...\special{t4ht+}
           Pipe the dvi code into a dvi page in the secondary dvi file
           `jobname.idv'.  Used by \Picture{...}, e.g., for requesting
           gif's.
  +    \special{t4ht+embeded-specials within idv}
  .    \special{t4ht.ext}
           Change default ext of root file
  @D   \special{t4ht@D....} Writes the content, augmented with a
           loc stamp, to the .lg file.  The locations stamp consists
           a byte-address in a named output file.

                                                    character maps
                                                    --------------
  !    \special{t4ht!...optional-parameters....}...dvi...\special{t4ht!}
           Create an approximated character map for the dvi code.
           Used in \Picture{...}, e.g., for ALT of IMG
  |    \special{t4ht|}...\special{t4ht|}
           Use the non-pictorial characters of the htf fonts.
           Used for character maps of \Picture{....}
  @    \special{t4ht@-}....\special{t4ht@-}
           Remove left margin from character map.  Used in \Picture{...}.

                                                    character settings
                                                    ------------------
  @    \special{t4ht@@}....\special{t4ht@@}
           Insert the character codes, instead of their mappings through
           the htf fonts.  Used in \JavaScript...
  @    \special{t4ht@...integer...}, \special{t4ht@-...integer...}
           Introduce the character code into the output.
           Used by \HChar{...} and \HChar{-...}. The earlier one
           also inherites the current font info.
  @    \special{t4ht@+...string...}
           Replace the character code introduced by the next character
           with the specified string.  The decoration of the character
           code is inherited, when the string is not empty. The string
           might include character codes by enclosing them between braces.
  @    \special{t4ht@*...string...}
           A variant \special{t4ht@+...string...} that inserts the content
           after the character instead of replacing it.
  @    \special{t4ht@(}
           Ignore spaces
  @    \special{t4ht@)}
           End ignore spaces
  @    \special{t4ht@[}
           Ignore chs and spaces
  @    \special{t4ht@]}
           End ignore chs and spaces
  @    \special{t4ht@[...}...\special{t4ht@]...}\special{t4ht@?...}
           Ignore chs and spaces, if the specials  have the above
           syntax on identical strings.
  @    \special{t4ht@!}
           Get the last ignored spaces (none, if from previous lines).
  @    \special{t4ht@_....}
           Output character for rulers. Empty string is also allowed.
  @    \special{t4ht@.''''}
           Output for line break characters (empty
                                         content resets the default).
  @    \special{t4ht@,''''}
           Output for space characters (empty content resets the default).

                                                    dvi tracing
                                                    -----------
  @    \special{t4ht@%X}...\special{t4ht@%x}
           Request dvi tracing.

               X  x
               P  p      groups
               C  c      characters
               H  h      horizontal spaces
               V  v      vertical spaces
               R  r      rulers

       \special{t4ht@%%X*...open-del....*...close-del....}
       \special{t4ht@%%x*...open-del....*...close-del....}
           Tailor dvi tracing

  @    \special{t4ht@/}
           On/off tracing of specials.
  @    \special{t4ht@e...}
           String for tracing errors into the output.
  ;    \special{t4ht;....}
           Decorations for htf characters (e.g., css)
              8    pause
              9    end pause
              \let \prOteCt \relax \Protect \csname acp:c\endcsname {14}.. patt
ern
              =... show font name of char
              %... show font size of char
              ,    don't report next htf class to lg
              -    set default font info
              +    unset default font info
  ^    \special{t4ht^i}$symbols$\special{t4ht^}}'
           Requests math class i for the listed math symbols.

           \tex assignes class numbers 0--7 to the atoms of math
           formulas: 0--ordinary symbol, 1--large operator, 2--binary
           operation, 3--relational operation, 4--math delimiter,
           5--right delimiter, 6--punctuation mark, and 7--adjustable.
           TeX4ht adds classes  8 and 9, while using
           class 7 independently
       \special{t4ht^}
           on/off processing delimiters
       \special{t4ht^-}
           pause processing delimeters
       \special{t4ht^+}
           continue processing delimeters
       \special{t4ht^i}
           on/off processing delimiters of class i
       \special{t4ht^i*...*...}
           configure delimiters for class i. * can be any
           character distinguished for the group.
       \special{t4ht^i(}
           put delimiters of class i on next group
       \special{t4ht^i)}
           As before, but ignore the delimeters within the sub-group.
       \special{t4ht^)*...*...}
           put the specified delimiters on next group.
           Ignore delimeters within the group.
       \special{t4ht^<*...*...}
           put the specified delimiters on the next group.
           Don't ignore delimeters within the group.
                                                    dvi arithmetic
                                                    --------------
  :    \special{t4ht:....}
           Dvi-mode arithmetics.
             :+...  increment by 1( define, if not defined)
             :-...  decrement by 1
             :>...  push current value
             :<...  pop current value
             :!...  display current value
             :\let \prOteCt \relax \Protect \csname acp:c\endcsname {14}..  dis
play top value

                                                    messages to lg file
                                                    -------------------
  +    \special{t4ht+@...message...}
           Send message to the lg file.  Used in the \Needs{...} command.
  @    \special{t4ht@D....}
           Send message to the lg file, together with location and file
           stamp.

                                                    positional code
                                                    -------------------
  "    \special{t4ht"}
           Start/end positional env
       \special{t4ht"* before-all * after-all ...** before-char
                     * after-char * rect
                     *%A*%B*%C*%D*%E
                     * optional
           Configure positioned code
             * before-all
             * after-all      %...
             ** before-char  %x %y
             * after-char
             * rect          %x1 %y1 ...
             * x,x1-coefficients %A(x) + %B
             * y1-coefficients %C(y1 - %E(height)) + %D
             * y-coefficients  %C(y) + %D
             * optional: 1, 2

               %x1 %y1 %length %thickness          default
               %x1 %y1 %x1+length %y1+thickness    1
               %x1 %y1 %x1+length %y1 %thickness   2

           A-magnification, B-displacement
           C-magnification, D-displacement,
           E- origin (0: top, 0.5: mid, 1: bot)

           The %...'s should be c-type templates (e.g., "%.2f"; "%.0f"
              gives an integer)

           Multiple after-all templates are allowed. The leading
             character is a code specifying the dimension type.
                x           min x
                X           max x
                y           min y
                Y           max y
                d           dx
                D           dy
                otherwise   a string with no values
           The delimiter `*' can be substitued by another character.


  ~    \special{t4ht~...}
           Grouped-base two-way delivery for content created by
           inline commands like \over.

       \special{t4ht~}...\special{t4ht~}   on/off

       ~<i...  send forward to the start of the group nested
                                                at relative level i.
       ~>i...  send forward to the end of the group nested
                                                at relative level i.
               i=0, current group


       ~<*...  send back to start of previous token / group.
       \special{t4ht~<)}...\special{t4ht~<(}
           activate / deactivate  back token / group submissions
       \special{~<[}...\special{t4ht~<]}'
           hide region from back submissions over token / group
       \special{t4ht~<-} ... \special{t4ht~<+}
           latex back token / group

       \special{t4ht\string~!...path...<...content}
           insertion at the start of the group reached by the path
       \special{t4ht\string~!...path...>...content}
           insertion at the end of the group reached by the path
       \special{t4ht\string~!...path.../}
           ignore content within the group reached by the path
       \special{t4ht\string~!...path...-}
           ignore rulers from the group reached by the path
           until the start of the next group
        A path may consist only of `e' and `s' characters for,
        respectively, entering and skipping groups

  *!   \special{t4ht*! system command}
           System call
  *^   accent specials
        t text accent \#1\#2\#1\#3\#1\#4\#1\#5\#1\#6 pattern
                      empty                insertion point
        m math accent \#1\#2\#1\#3\#1\#4\#1\#5\#1   pattern
                      empty                insertion point
        a accented    \#1\#2\#1\#3\#1\#4\#1\#5\#1\#6
        i             \#1\#2\#1\#3\#1\#1
  *@   halign specials
\end{verbatim}

\begin{verbatim}
\HCode...............................1
\end{verbatim}

   A wrapper for \special{t4ht=...}.

   The sharp symbol \# may be accessed indirectly through the command.

\begin{verbatim}
\Hnewline............................0
\end{verbatim}

   Requests new lines within specials

Sectioning
----------

\begin{verbatim}
\Configure{@sec @ssect}
\end{verbatim}


Tables of Contents
------------------


\begin{verbatim}
\Configure{tableofcontents}........................5
\end{verbatim}

   \#1 before
   \#2 at end
   \#3 after
   \#4 at indented paragraph break
   \#5 at non-indented paragraph break

   The \tableofcontents command may be followed by a comma separated
   list of sectioning unit names to be included in the table of
   contents.  The list should be enclosed within square brackets.
   Alternatively, a command of the form \TableOfContents[...] might
   be used.

Lists
-----

\begin{verbatim}
\ConfigureList.....................5
\end{verbatim}

   \#1   type of list (e.g., itemize, description, enumerate,
                            list, trivlist)
   \#2   before list
   \#3   after  list
   \#4   before label
   \#5   after label

   \DeleteMark   removes latex's label; to be placed at the end of \#4
   \AnchorLabel  defines an anchor for \label in current item; to
                 be placed in \#5


Tables
------

\begin{verbatim}
\Configure{tabular}...................6
\Configure{array}.....................6
\end{verbatim}

    \#1   before table         \#2   after table
    \#3   before row           \#4   after row
    \#5   before cell          \#6   after cell

    \HRow         current row number
    \HCol         current column number
    \HMultispan   number of cells covered by the current cell
    \ar:cnt       number of columns in the table

  NOTE: Table require a number of compilations that depends
        on the number of columns.

  \Example

\begin{verbatim}
   \Configure{tabular}
       {\HCode{<table>}}
       {\HCode{</table>}}
       {\HCode{<tr class="row-\HRow">}}
       {\HCode{</tr>}}
       {\HCode{<td
               \ifnum \HMultispan>1 colspan="\HMultispan"\fi >}}
       {\HCode{</td>}}


\Configure{VBorder}...................4
\end{verbatim}

    Break points, when scanning the pattern of column desriptions, at

    \#1  at start of pattern
    \#2  at |
    \#3  at a non-@ entry
    \#4  at a @ entry

  Applies to \begin{tabular / array}{...pattern...}

    \ar:cnt    index of entry in pattern
    \ch:class  records the current alignment type: -,<,>,p,...
    \HColAlign produces the \Configure{halignTD} contribution
               for the current alignment type
    \HColWidth holds the width of the current p column

\begin{verbatim}
\Configure{HBorder}..................10
\end{verbatim}

   hline:
    \#1  insert at start of row (e.g., <tr>)
    \#2  insert at each cell    (e.g., <td><hr/></td>)
    \#3  insert at end of row   (e.g., </tr>)

   cline:
    \#4  insert at start of row         (e.g., <tr>)
    \#5  insert at each `extra' cell    (e.g., <td></td>)
    \#6  insert at each cell            (e.g., <td><hr/></td>)
    \#7  insert at end of row           (e.g., </tr>)

   vspace:
    \#8  insert at start of row (e.g., <tr>)
    \#9  insert at each cell    (e.g., <td>&nbsp;</td>)
    \#10 insert at end of row   (e.g., </tr>)

   The contributions are collected into \HBorder.  (The \InitHBorder
   clears \HBorder.)

\begin{verbatim}
\Configure{putHBorder}...............1
\end{verbatim}

    \#1 Specifies how \HBorder is to be used.

   \Example 
\begin{verbatim}
\Configure{putHBorder}{\HCode{\HBorder}}
\end{verbatim}



\begin{verbatim}
\Configure{halignTD}..................2 + 2i + {}
\end{verbatim}

  interpretation for character codes referenced in \HAlign

  e.g.,

\begin{verbatim}
   \Configure{halignTD}
   {}{}
   {<}{\HCode{style="text-align:left"}}
   {-}{\HCode{style="text-align:center"}}
   {>}{\HCode{style="text-align:right"}}
   {^}{\HCode{style="vertical-align:top"}}
   {=}{\HCode{style="vertical-align:baseline"}}
   {|}{\HCode{style="vertical-align:middle"}}
   {_}{\HCode{style="vertical-align:bottom"}}
   {p}{\HCode{style="text-align:left"}}
   {}
\end{verbatim}

   \halignTD can be used in td elements to extract the alignment.
   It recieves information from \halignTB.

\begin{verbatim}
\Configure{halignTB}..................2
\end{verbatim}

   delimiters for \halignTB{tabular}

   \Example
\begin{verbatim}
       \Configure{halignTB}{\HCode{<table }}{\HCode{>}}
\end{verbatim}

\begin{verbatim}
\Configure{tabbing}[mag]..................4
\end{verbatim}

   \#1 before each line
   \#2 after each line
   \#3 before each entry
   \#4 after each entry

   [mag] optional parameter specifying the magnification desired
         for the dimensions.  When offered, the other parameters
         have no effect if all of them are assigned empty arguments

   \TabType   \` or \relax
   \TabWidth  Provides the entry width; 0 at trailing entry that is
              not flushed rightward


Cross References
----------------

\begin{verbatim}
\Configure{ref}.......................3
\end{verbatim}

  \#1   \Link-type command
  \#2   \EndLink-type command
  \#3   anchor (the system anchor is
               reachable through the parameter name \#1)
  \RefArg   Holds the argument of \ref

  If \#1 is empty, the hyper links are ignored
  If \#3 is empty, the anchor is the one provided by the system

  \Example

\begin{verbatim}
    \Configure{ref}{\Link}{\EndLink}{{\bf \#1}}

\Configure{pageref}...................3
\end{verbatim}

  \#1   before
  \#2   after
  \#3   anchor (system anchor, if parameter is empty)

\begin{verbatim}
\Configure{newlabel}..................2
\end{verbatim}

  \#1   address for hyperlink (\cur:th \:currentlabel, if empty)
  \#2   anchor (the system anchor is
               reachable through the parameter name \#1)

\begin{verbatim}
\Configure{@newlabel}.................1
\end{verbatim}

   \#1  modifications to the newlabel environment

\begin{verbatim}
\Configure{newlabel-ref}..............1
\end{verbatim}

   \#1  an intermediate link command for the aux file
       (Configured by \Configure{ref}...)

   The default for \#1 is \rEfLiNK

\begin{verbatim}
\Configure{cite}......................4
\end{verbatim}

  \#1   before
  \#2   after
  \#3   \Link-type command
  \#4   \EndLink-type command

  If \#3 is empty, the hyper links are ignored.

\begin{verbatim}
\Configure{bibitem}...................2
\end{verbatim}

  \#1   \Link-type command
  \#2   \EndLink-type command

\begin{verbatim}
\Configure{bibcite}...................1
\end{verbatim}

  \#1  configurations for content transfered by bibitem to the aux file

  \Example
       \Configure{bibcite}
                 {\def\hookrightarrow{\string\hookrightarrow}}
       \bibitem[$\hookrightarrow$...]{...}


\begin{verbatim}
\LoadLabels[\#1]{\#2}....................
\end{verbatim}

  [\#1]   optional group name
  \#2     aux file name, without the extension

  Loads labels of another file, under the specified group name

\begin{verbatim}
\RefLabel.............................2
\end{verbatim}

  \#1     group name (for separating files and labels from
                     different sources)
  \#2     label

  A variant of \ref for loading labels produced for other files

  \Example
    file1.tex:  \label{foo}

    file2.tex:  \LoadLabels[x]{file1}
                \RefLabel{x}{foo}

\begin{verbatim}
\SkipRefstepAnchor.....................0
\end{verbatim}

    No \Link anchor for next \refstepcounter

\begin{verbatim}
\ShowRefstepAnchor.....................0
\AutoRefstepAnchor.....................0
\end{verbatim}


Bibliography on bibtex2 option:

\begin{verbatim}
  \Configure{bibliography2}........................ 4
\end{verbatim}

     \#1 before anchor
     \#2 anchor
     \#3 after anchor
     \#4 link attributes

    \Example

\begin{verbatim}
       \Configure{bibliography2}
          {\bgroup ~~[\Configure{Link}{a}{target="x"  href=}{ name=}{}}
          {more} {]\egroup}

  \Configure{bibitem2}..............................3
\end{verbatim}

     \#1 at start of bibitem
     \#2 at end of bibitem
     \#3 separator after label

\begin{verbatim}
  \Configure{bibliographystyle2}....................1
\end{verbatim}

     \#1 an empty argument asks for the same style as the
        normal aux file (still bibtex may produce different
        output).

Note: Option `bibtex2' requires compilation
      of `\jobname j.aux' with bibtex.

Captions
--------

refcaption

  An option for \Preamble, requesting anchors at \caption. The default
  setting sends them back to the start of the floating environment.


Theorems
--------

\begin{verbatim}
\Configure{newtheorem} ......................3
\end{verbatim}

  \#1 before theorem
  \#2 between title and body
  \#3 after theorem

Math
----
\begin{verbatim}
\Configure{()}...............................2
\Configure{[]}...............................2
\end{verbatim}

   \Example

\begin{verbatim}
     \Configure{()}{\protect\PicMath$}{$\protect\EndPicMath}
     \Configure{[]} {\Tg<display>\DviMath$$} {$$\EndDviMath\Tg</display>}

\Configure{equation}.........................3
\end{verbatim}

   \#1    at start
   \#2    between the equation and its numbering
   \#3    at end

  \Examples:

\begin{verbatim}
      \Configure{equation}
           {\IgnorePar\EndP\bgroup \Configure{HtmlPar}{}{}{}{}%
                    \HCode{<table class="equation"><tr><td>}\IgnorePar
           }
           {\HCode{</td><td class="eq-number">}}
           {\HCode{</td></tr></table>}\egroup}

      %%%%%%%%%%%%%%%%%%%%%%%%%%%%%%%%%%%%%%%%%%%%%

      \Configure{equation}
         {\IgnorePar\EndP \bgroup \Configure{$$}{}{}{}%
          \Configure{@math}{display="inline"}\DviMath
                  \HCode{<mtable class="equation"><mtr><mtd>}\IgnorePar
         }
         {\IgnorePar\HCode{</mtd><mtd class="eq-number">}}
         {\HCode{</mtd></mtr></mtable>}\EndDviMath\egroup}


\Configure{frac}.............................4
\Configure{sqrtsign}.........................2

\Configure{mbox}.............................2
\end{verbatim}

Environments of latex
---------------------

\begin{verbatim}
  \ConfigureEnv{...}.........................4
\end{verbatim}

     \#1 environment name
     \#2 before env
     \#3 after env
     \#4 before underlying list
     \#5 after underlying list

    \#2 and \#3 are ignore when they are both empty as well as
    when there is no underlying list


   array
   center
   flushleft
   flushright
   minipage
   tabbing
   tabular
   verbatim*
   verbatim


\begin{verbatim}
   \Configure{@begin}........................2
\end{verbatim}

      \#1 environment name
      \#2 insertion before the environment

   \Example

\begin{verbatim}
      \Configure{@begin}{theindex}{\section*{\indexname}}
\end{verbatim}


Verbatim
--------

\begin{verbatim}
\Configure{verbatim}......................2
\end{verbatim}

   \#1 at start of line
   \#2 space character

\begin{verbatim}
\Configure{verb}..........................2
\end{verbatim}

   \#1 before
   \#2 after

\begin{verbatim}
\Configure{obeylines}.....................3
\end{verbatim}

   \#1 before
   \#2 at start of line
   \#3 after

\begin{verbatim}
\ScriptEnv................................3
\end{verbatim}

  Introduces a verbatim environent

   \#1 name
   \#2 before
   \#3 after

  A `-' immediately after \begin{...} designate as an escape symbol
  the character following the dash

  \Example

\begin{verbatim}
     \ScriptEnv{foo}
       {\HCode{<myscript>}\NoFonts\hfill\break }
       {\EndNoFonts \HCode{</myscript>}}

     \begin{foo}
     ....
     ....
     \end{foo}

     \begin{foo}-\relax \unhbox \voidb@x \special {t4ht@+&{35}x00A0{59}}x     .
...
     ....
     \end{foo}
\end{verbatim}


Fonts
-----

\begin{verbatim}
\Configure{texttt}........................2
\Configure{textit}........................2
\Configure{textrm}........................2
\Configure{textup}........................2
\Configure{textsl}........................2
\Configure{textsf}........................2
\Configure{textbf}........................2
\Configure{textsc}........................2
\Configure{emph}..........................2
\end{verbatim}

  \#1  before content
  \#2  after content


Footnotes
---------

\begin{verbatim}
\Configure{footnotetext}..................3
\end{verbatim}

    \#1 before footnote
    \#2 between mark and content
    \#3 after footnote

   \FNnum     footnote number

\begin{verbatim}
\Configure{footnotemark}..................2
\end{verbatim}

    \#1 before
    \#2 after

Pictures
--------

\begin{verbatim}
\Configure{picture}.......................2
\end{verbatim}

    \#1  before
    \#2  after

Other Hooks
-----------

\begin{verbatim}
\Configure{ }.........................1
\end{verbatim}

   \#1 representation for non-breaking space ch

\begin{verbatim}
\Configure{hline}.....................1

\Configure{hspace} ...................3
\end{verbatim}

   \tmp:dim      register holding the size
   \#1            before the space
   \#2            after the space
   \#3            after \#1 (\tmp:dim mod 6em  copies)

\begin{verbatim}
\Configure{vspace} ...................1
\end{verbatim}

   \#1 the size of space is prvided in a parameter nmaed `\#1'

   \Example

\begin{verbatim}
      \Configure{vspace}
      {\ifhmode
         \HCode{<br />}%
         \ifdim \#1>1ex \HCode{<br />}\fi
       \fi
      }

\Configure{fbox} .................................. 2
\end{verbatim}

   \Examples:

\begin{verbatim}
        \Configure{fbox}
          {\HCode{<div class="fbox">}\bgroup \fboxrule=0pt}
          {\egroup\HCode{</div>}}
        \Css{div.fbox {border: 1pt solid black;}}

        \Configure{fbox}
          {\HCode{<table cellspacing="0pt"
              border="1"><tr><td>}\bgroup \fboxrule=0pt}
          {\egroup\HCode{</td></tr></table>}}

\Configure{'} ..................................... 3
\end{verbatim}

   \#1 at entry to math prime environment
   \#2 at exit
   \#3 content of \prime

\begin{verbatim}
\Configure{float}....................................4
\end{verbatim}

   \#1 optional, to appear within brakects [ and ].  An anchor for
      the links preceeding the float, when option refcaption is
      not active
   \#2 Insertion before the links
   \#3 at start
   \#4 at end

\begin{verbatim}
\Configure{textcircled}.............................2n+1
   2i'th     replaced       i=1,...,n
   2i+1'st   replacement
   2n+'nd    empty (terminator)
\end{verbatim}


\begin{verbatim}
\Configure{add accent}{\#1:\#2}{\#3}{\#4}...{}{}
\end{verbatim}

   \#1  encoding
   \#2  font number
   \#3  character under font
   \#4  replacement

   Applies to accents that reach \add@accent

  \Example

\begin{verbatim}
     \Configure{add accent}{OT4:18}
       {E}{\add:acc{00C8}}
       {e}{\add:acc{00E8}}
       {}{}
\end{verbatim}




\begin{verbatim}
\Configure{//[]}
\Configure{AfterTitle}
\Configure{HAccent}
\Configure{InsertTitle}
\Configure{accents}
\Configure{accent}
\Configure{centercr}
\Configure{centerline}

\Configure{displaylines}
\Configure{framebox}

\Configure{leftline}
\Configure{marginpar}
\Configure{mathaccent}
\Configure{newline}
\Configure{oalign}

\Configure{overline}

\Configure{rightline}
\Configure{stackrel}
\Configure{tt}
\Configure{underline}
\Configure{thanks}....................2
\end{verbatim}



\section{Fontmath}
                                         *

\begin{verbatim}
\Configure{mathbf}........................2
\Configure{mathit}........................2
\Configure{mathrm}........................2
\Configure{mathsf}........................2
\Configure{mathtt}........................2
\end{verbatim}

  \#1  before content
  \#2  after content

\begin{verbatim}
\Configure{overbrace}.................3
\Configure{underbrace}................3
\end{verbatim}

--- Note ---  A script of the form 
tex '\def \filename{{%%1}{idx}{4dx}{ind}} \input idxmake.4ht' 
makeindex -o %%1.ind %%1.4dx 
in the env file, automatically calls to the revised makeindex 
command. An extra compilation of the source LaTeX file is required, 
to get the index correctly into the output.


\section{Article}

Title Page
----------

\begin{verbatim}
\Configure{maketitle}.....................4
\end{verbatim}

   \#1 start of maketitle
   \#2 end of maketitle
   \#3 before title
   \#4 after title

\begin{verbatim}
\Configure{thanks author date and}........8
\end{verbatim}

   \#1  before thanks
   \#2  after thanks
   \#3  before author
   \#4  after author
   \#5  before date
   \#6  after date
   \#7  representation of `and'
   \#8  line breaks (= end of rows, for an embedded tabular environment)


Sectioning Commands
-------------------

\begin{verbatim}
\Configure{part}...................4
\Configure{section}................4
\Configure{subsection}.............4
\Configure{subsubsection}..........4
\Configure{paragraph}..............4
\Configure{subparagraph}...........4
\end{verbatim}

   \#1 before division
   \#2 after division
   \#3 before title
   \#4 after title

\begin{verbatim}
\Configure{likepart}...............4
\Configure{likesection}............4
\Configure{likesubsection}.........4
\Configure{likesubsubsection}......4
\Configure{likeparagraph}..........4
\Configure{likesubparagraph}.......4
\end{verbatim}

   starred versions of the sectioning commands

\begin{verbatim}
\Configure{endpart}................1
\Configure{endsection}.............1
\Configure{endsubsection}..........1
\Configure{endsubsubsection}.......1
\Configure{endparagraph}...........1
\Configure{endsubparagraph}........1
\Configure{endlikepart}............1
\Configure{endlikesection}.........1
\Configure{endlikesubsection}......1
\Configure{endlikesubsubsection}...1
\Configure{endlikeparagraph}.......1
\Configure{endlikesubparagraph}....1
\end{verbatim}

   \#1  a comma separated list specifying the end
       points for the configured logical unit

   \Example 

\begin{verbatim}
\Configure{endsection}
              {likesection,chapter,likechapter,appendix,part,likepart}
\end{verbatim}

\begin{verbatim}
\Configure{partTITLE+}
\Configure{sectionTITLE+}
\Configure{subsectionTITLE+}
\Configure{subsubsectionTITLE+}
\end{verbatim}


   \#1 an insertion just before the content of <TITLE>;

   The insertion overrides the one offered by \Configure{CutAtTITLE+}
   for the given section type (the `like' counterparts acn also be
   configured).

   If \#1 is a one parametric macro, it gets the title content for
   an argument.



Bibliography
------------

\begin{verbatim}
\ConfigureList{thebibliography}......4
\end{verbatim}

   \#1   before list
   \#2   after  list
   \#3   before label
   \#4   after label

   \DeleteMark   removes latex's label; to be placed at the end of \#3
   \AnchorLabel  defines an anchor for \label in current item; to
                 be placed in \#4

\begin{verbatim}
\Configure{cite}        see the
\Configure{bibitem}     latex section
\end{verbatim}


Tables of Content
-----------------

\begin{verbatim}
\ConfigureToc
\end{verbatim}

  lof, lot, appendix, chapter, likechapter, likeparagraph, likepart,
  likesection, likesubparagraph, likesubsection, likesubsubsection,
  paragraph, part, section, subparagraph, subsection, subsubsection,


\begin{verbatim}
\Configure{tableofcontents*}.......................1
\end{verbatim}

    \#1  A non-empty  parameter asks to implicitly introduce
        a \tableofcontents upon reaching the first sectioning
        command, if no \tableofcontents command is encountered
        earlier. The parameter assumes a colon-separated list
        of sectioning types to be included in the output
        of \tableofcontents.  Starred variants of sectioning
        types should be referenced with the `like' prefix.

        An empty parameter cancels earlier requests for implicit
        calls to \tableofcontents (e.g., embedded within requests
        to package options 1, 2, 3, 4)

   \Example

\begin{verbatim}
      \Configure{tableofcontents*}{part,likepart,%
           section,likesection,subsection,likesubsection}
\end{verbatim}

\contentsname

   A LaTeX macro stating the title for a table of contents division.



Captions
--------

\begin{verbatim}
\Configure{caption}...............4
\end{verbatim}

   \#1   before number          \#2    after number
   \#3   before title           \#4    after title

Indexes
-------


\begin{verbatim}
\Configure{theindex} ..........................9
\end{verbatim}

     \#1    before-env
     \#2    after-env
     \#3    before-item
     \#4    after-item
     \#5    before-subitem
     \#6    after-subitem
     \#7    before-subsubitem
     \#8    after-subsubitem
     \#9    at-indexspace

    \Example

\begin{verbatim}
       \Configure{theindex}
          {\HCode{<ul class="theindex">}\global\let\IndexSpace=\empty}
          {\HCode{</ul>}}
          {\HCode{<li \IndexSpace>}\global\let\IndexSpace=\empty}
                                         {\HCode{</li>\Hnewline}}
          {\HCode{<li>}\ \ \ \ }         {\HCode{</li>\Hnewline}}
          {\HCode{<li>}\ \ \ \ \ \ \ \ } {\HCode{</li>\Hnewline}}
          {\global\def\IndexSpace{class="indexspace"}}

       \Css{.indexspace{margin-top:1em;}}
\end{verbatim}

    The links are indirectly requested in the idx files within
    \beforeentry macros. For instance, a file try.tex

\begin{verbatim}
       \documentclass{article}
          \makeindex
       \begin{document}

       \section{xx}

       \index{a1}  x
       \index{a2}  x
       \index{a2}  x
       \index{b1}  x
       \index{b2}  x
       \index{b3}  x

       \input \jobname.ind

       \end{document}
\end{verbatim}

    produces a file try.idx of the form

\begin{verbatim}
       \beforeentry{try.html}{dx1-1001}{}
       \indexentry{a1}{1}
       \beforeentry{try.html}{dx1-1002}{}
       \indexentry{a2}{1}
       \beforeentry{try.html}{dx1-1003}{}
       \indexentry{a2}{1}
       \beforeentry{try.html}{dx1-1004}{}
       \indexentry{b1}{1}
       \beforeentry{try.html}{dx1-1005}{}
       \indexentry{b2}{1}
       \beforeentry{try.html}{dx1-1006}{}
       \indexentry{b3}{1}
\end{verbatim}

    where each pair

\begin{verbatim}
        \beforeentry{A}{B}{}\indexetry{C}{D}

    represents an entry of the form

\begin{verbatim}
        \indexentry{\Link[A]{B}{}C\EndLink}{D}
\end{verbatim}

    The makeindex utility ignores the \beforeentry records.  To compensate
    for that, one needs to pre-process the idx file which is introduced to
    the makeindex utility and/or post-process the output of the utility.

    A script consisting of two instructions similar to

\begin{verbatim}
      tex '\def\filename{{try}{idx}{4dx}{ind}} \input idxmake.4ht'
      makeindex -o try.ind try.4dx
\end{verbatim}

    instead of

\begin{verbatim}
      makeindex -o try.ind try.idx
\end{verbatim}

    should do the job.

    On some platforms, the quotation marks ' should be
    replaced by double quotation marks " or eliminated.

\begin{verbatim}
\Configure{makeindex} ..........................1
\end{verbatim}

    The default setting, requests consecutive numbers for the
    pointers in the indexes.  The current command provides the
    means to configure the pointers to other values.

    \Example 

\begin{verbatim}
\Configure{makeindex}{Sec \arabic{section}}
\end{verbatim}


Environments of article:
------------------------

\begin{verbatim}
  \ConfigureEnv{...}.........................4
\end{verbatim}

   abstract
   description
   figure
   figure*
   quotation
   quote
   table
   table*
   thebibliography
   titlepage
   verse

Other Hooks
-----------

\begin{verbatim}
\Configure{listof}
\end{verbatim}

\section{Graphics}

\begin{verbatim}
\Configure{graphics}...............2
\end{verbatim}

    \#1  before \includegraphics
    \#2  after \includegraphics


    \Examples:

\begin{verbatim}
       \Configure{graphics}
          {\Picture+[PIC]{ class="graphics"}}
          {\EndPicture }

       \Configure{graphics}
         {\bgroup
             \Configure{IMG}
                {\ht:special{t4ht=<img src="}}
                {\ht:special{t4ht=" alt="}}
                {" }
                {\ht:special{t4ht=" }}
                {}%
          \Picture+[PIC]{}}
         {\EndPicture
             \def\temp{.pstex}\expandafter\ifx
                              \csname Gin@ext\endcsname\temp
                                         \HCode{ width="75\%" }\fi
             \HCode{ />}%
          \egroup}


\Configure{graphics*}..............2
\end{verbatim}

    \#1  extension name
    \#2  insertion

    \Gin@base (file name), \Gin@ext, \Gin@req@width, \Gin@req@height,
    \noBoundingBox (defined iff bounding box is unknown)

    Allows to configure tex4ht for graphics files named in
    the \includegraphics macro, based on the type of the files.

    An empty insertion \#2 cancels previous requests for the
    specified extension.

    \Example

\begin{verbatim}
       \Configure{graphics*}
         {jpg}
         {\Picture[pict]{\csname Gin@base\endcsname.jpg}}

       \Configure{graphics*}
         {wmf}
         {\Needs{"convert \csname Gin@base\endcsname.wmf
                          \csname Gin@base\endcsname.gif"}%
          \Picture[pict]{\csname Gin@base\endcsname.gif
                      width="\expandafter\the\csname
                                Gin@req@width\endcsname"
                     height="\expandafter\the\csname
                                Gin@req@height\endcsname"}%
         }

       \Configure{graphics*}
         {eps}
         {\openin15=\csname Gin@base\endcsname\PictExt\relax
          \ifeof15
             \Needs{"convert \csname Gin@base\endcsname.eps
                             \csname Gin@base\endcsname\PictExt"}%
          \fi
          \closein15
          \Picture[pict]{\csname Gin@base\endcsname\PictExt}%
         }
\end{verbatim}

  Note: Arguments of the \includegraphics command such as angle and
    scale in

\begin{verbatim}
       \includegraphics[angle=-90,scale=0.5]{fig.eps}
\end{verbatim}

    are not known to the given figure (e.g., to fig.eps). To be
    taken into account, the scripts should handle the transformations
    they request (e.g., in \csname Grot@angle\endcsname,
    \csname Gscale@x\endcsname, \csname Gscale@y\endcsname)





