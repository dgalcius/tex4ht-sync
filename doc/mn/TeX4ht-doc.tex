% save TeX4ht ; save bugfixes (in particular executables)

% dotemp  twice
% or 
% remove TeX4ht/* ;  dofinal twice ; compile bugfixes





%%%%%%%%%%%%%%% comment out for final version %%%%%%%%%%%%%
% \def\UnderRevision{{\bf
%    This site is under revision 
%    until the end of February
% \IgnorePar\EndP \HCode{<hr />}\par
% }}
%%%%%%%%%%%%%%%%%%%%%%%%%%%%%%%%%%%%%%%%%%%%%%%%%%%%%%%%%%%%


\immediate\write16{!!!!!!!!!!!!!!!!!!!!!!!!!!!!!!!!!!!!!!!!!!!!!!!!!!!!!!!}
\immediate\write16{!!!!!!Compile: dotemp or dofinal               !!!!!!!!}
\immediate\write16{!!!!!!NEEDS: htcmd.exe, tex4ht.exe, t4ht.exe   !!!!!!!!}
\immediate\write16{!!!!!!needs: TWO COMPILATIONS to get correct index!!!!!!!!}
\immediate\write16{!!!!!!checkmn in-messages out-clean-messages!!!!!!!!}
\immediate\write16{!!!!!!!!!!!!!!!!!!!!!!!!!!!!!!!!!!!!!!!!!!!!!!!!!!!!!!!}
gowww\immediate\write16{!!!!!!cd /home/httpd/html/applications/tex4ht!!!!!!!!}
\immediate\write16{!!!!!!lynx http://www.cse.ohio-state.edu/\string ~gurari/tex4ht/tex4ht-all.zip}


%+% ftp cam.ctan.org  
%+% anonymous
%+% gurari@cse.ohio-state.edu
%+% ftp> cd incoming
%+% ftp> send tex4ht-all.zip
%+% ftp> send README.tex4ht
%+% ftp> quit
%+% 
%+% 
%+% -- mail the administrators (ctan@dante.de) to report:
%+% 
%+%  Subject: Uploaded current version of tex4ht
%+%
%+%    + what you've uploaded
%+% 
%+% tex4ht-all.zip
%+% README
%+% 
%+%    + which CTAN node you've uploaded to
%+% 
%+% cam.ctan.org  
%+% 
%+%    + where you want the files to go
%+% 
%+% Replace
%+% 
%+%   /tex-archive/support/TeX4ht/tex4ht-all.zip
%+% 
%+%    + what licensing conditions you apply to your software 
%+%
%+% LPPL
%+% 
%+%    + a brief summary of what your upload is intended to do.
%+% 
%+% Update of the TeX4ht system for translating TeX-based
%+% files into hypertext.
%+%
%+% Thanks, -eitan


%+% ftp cam.ctan.org  
%+% anonymous
%+% gurari@cse.ohio-state.edu
%+% ftp> cd incoming
%+% ftp> send AlProTex.sty
%+% ftp> send README.protex
%+% ftp> quit
%+% 
%+% 
%+% -- mail the administrators (ctan@dante.de) to report:
%+% 
%+%  Subject: Uploaded current version of AlProTex.sty
%+%
%+%    + what you've uploaded
%+% 
%+% AlProTex.sty
%+% README
%+% 
%+%    + which CTAN node you've uploaded to
%+% 
%+% cam.ctan.org  
%+% 
%+%    + where you want the files to go
%+% 
%+% Replace
%+% 
%+%   /tex-archive/web/protex/AlProTex.sty
%+%   /tex-archive/web/protex/README
%+% 
%+%    + what licensing conditions you apply to your software 
%+%
%+% LPPL
%+% 
%+%    + a brief summary of what your upload is intended to do.
%+% 
%+% Update the ProTex literate programming system.
%+%
%+% Thanks, -eitan


\long\def\WAIT#1\ENDWAIT{}
%%%%%%%%%%%%%%%%%%% Compiling this program %%%%%%%%%%%%
% For temp directory: dotemp
% For TeX4ht dir    : difinal

%  \def\SysNeeds#1{}


\def\SKIP{\bgroup\catcode`\\=9 \catcode`\{=9 \catcode`\}=9 \foo}
\long\def\foo#1/////////////+/{\egroup}\SKIP




%%%%%%%%%%%%%%%%%%%%%%%%%%%%%%%%%%%%%%
MORE AT THE END
%%%%%%%%%%%%%%%%%%%%%%%%%%%%%%%%%%%%%%

refresh AlProTex.sty in ctan
-------------------------------------------
--------------------------------------------------------------------------------------
-------------------------------------------

--------------------------------------------------------------------------------------
--------------------------------------------------------------------------------------
--------------------------------------------------------------------------------------
--------------------------------------------------------------------------------------
--------------------------------------------------------------------------------------
--------------------------------------------------------------------------------------
http://biostat.mc.vanderbilt.edu/wiki/Main/SweaveConvert
--------------------------------------------------------------------------------------
From: Peter FELECAN <pfelecan@acm.org> 
To: gurari@cse.ohio-state.edu 
Subject: [SPAM]  tex4ht packages for SUN Solaris 
Date: Mon, 18 May 2009 09:50:36 +0200 
 
Eitan, 
 
Just to let you know that I packaged tex4ht --- the latest version --- 
for SUN Solaris and the packages are available in the distribution 
OpenCSW: http://www.opencsw.org/ If you think that it's of interest for 
the tex4ht users community you can mention it on the project page. 
 
Sincerely yours 
--  
Peter FELECAN 

--------------------------------------------------------------------------------------

Bill, 
 
The problem is due to tex4ht assuming that \a maintains its original 
definition as provided in latex.ltx.  The same kind of problem will 
arise with many other macros that are configured by tex4ht.   
 
It is possible to modify tex4ht to provide configurations only for 
control words that maintain their original definitions, but that will 
require a large code overhead.  On the other hand, it can be argued 
that it is the responsibility of authors that change the meaning of 
control words to provide also new tex4ht-based configurations. 
 
I modified the bugfixes distribution to cancel through an instruction 
of the form 
 
   \HRestore\a  
 
the configuration by tex4ht. 
 
-eitan 
 
 > I suppose this could be fixed by now, though the same behavior 
 > is obtained with the texlive 2007 version (version 2006-10-28-15:32). 
 >  
 > (/usr/share/texmf/tex/generic/tex4ht/tex4ht.sty 
 > version 2008-02-25-14:04 
 >  
 > The only notice is: 
 >  
 >    ! Missing $ inserted. 
 >    <inserted text>  
 >                    $ 
 >  
 > Example and xhtml+mathml output appended.  The example runs correctly 
 > through normal latex. 
 
 > ------------------------- 
 > \documentclass{article} 
 > \renewcommand{\a}{\alpha} 
 > \begin{document} 
 >  
 > $\a$ 
 > \end{document} 
 > -----------------------
--------------------------------------------------------------------------------------


\documentclass[english]{article}  
   \listfiles  
   \usepackage{filecontents}  
   \begin{filecontents}{test.bib}  
   @BOOK{book,  
     author = {Surname, Shaun},  
     title = {A book with a long title},  
     shorttitle = {A book},  
     address = {London},  
     year = {2000}  
   }  
   \end{filecontents}  
   \usepackage[style=verbose-note]{biblatex}  
   \bibliography{test}  
  
  \begin{document}  
   \footnote{\cite{book}}  
\footnote{\cite{book}}  
   \end{document}  

--------------------------------------------------------------------------------------
 
here it is a patch to let oolatex test which java implementation is run 
by the command ``java'', and to exit with a meaningful error message (that can b\
e  
improved, of course) 
 
the patch is relevant in the debian context, and is not intended for propagati\
on  
upstream 
 
best regards 
 
                                                                gb 
 
--- /usr/share/tex4ht/oolatex   2008-09-14 05:53:15.000000000 +0200 
+++ oolatex     2008-12-11 15:11:38.000000000 +0100 
@@ -1,4 +1,7 @@ 
 #!/bin/sh 
+ 
+java -version 2>&1 | grep -q 'GNU libgcj' && printf ``oolatex NEEDS Sun's Java\
\n'' && exit 
+ 
 if command -v xhtex > /dev/null 2>&1 ; then 
   true 
 else 
 




--------------------------------------------------------------------------------------
\graphicspath{{eps/}{tiff/}}
http://www.tex.ac.uk/cgi-bin/texfaq2html?label=graphicspath
\DeclareGraphicsExtensions{.png,.jpg,.eps,.ps} 
\includegraphics{fig} 
--------------------------------------------------------------------------------------
To: owner@bugs.debian.org (Debian Bug Tracking System)
Cc: Kapil Hari Paranjape <kapil@debian.org>
Subject: Bug#486482: tex4ht: a java program invoked from  
 oolatex goes in error
In-Reply-To: <handler.486482.D486482.122136489916562.ackdone@bugs.debian.org>
References: <E1KeiZn-0002ui-6I@ries.debian.org>
        <20080616111948.5356.58963.reportbug@boffi95.stru.polimi.it>
        <handler.486482.D486482.122136489916562.ackdone@bugs.debian.org>
X-Mailer: VM 7.04 under 21.1 (patch 14) "Cuyahoga Valley" XEmacs Lucid
--text follows this line--

An apology if this is a duplicate posting.  I tried to email earlier
a similar message I fail to find a trace of it.

--------


Kapil Hari Paranjape, Mon, 16 Jun 2008 20:00:02

> This (second) failure seems to the same error as is evident in bug
> report #471837, which is due to a problem with "default-jre" under
> Debian (see #474075). The problem is that the JRE which uses gij (or
> cacao or jamvm) is not entirely compatible with Sun's "java".
> 
> The original "xtpipes" java code and the resulting byte code works
> fine with Sun's "java". If you choose the Sun java (currently only in
> non-free under Debian) using "update-alternatives", then the
> conversion ought to work for now. 
> 
> We are still trying to fix this compatibility issue.

Eitan Gurari, Tue, 17 Jun 2008 00:48:31
> 
>  > We are still trying to fix this compatibility issue.
> 
> It will take some months until I'll confront the problem head on.  I
> intend first to learn the gij/gcj environment by using it to develop
> software for another project.  -eitan

The oolatex problem with gij/gcj is due to the problem exhibited in
the attached sample program.  The program produces the same error (of
a missing office.dtd file) under SUN Java and GNU Java.  When the
EntityResolver code is uncommented, the SUN Java program executes
smoothly.  On the other hand, the GNU Java still issues the error
message.  That is, GNU Java seems to have broken implementation for
the EntityResolver feature.

I ran the code with the commands

   javac Test.java ; java Test test.tmp 
   gcj -C Test.java ; gij Test test.tmp 

within the following environment.

   java version "1.5.0" 
   gij (GNU libgcj) version 4.2.4 (Ubuntu 4.2.4-1ubuntu3) 

By introducing an empty office.dtd file in the work directory, oolatex
would work also under gcj/gij. 
 
-eitan 
 
------------------- Test.java ---------------------- 

import javax.xml.parsers.*; 
import org.xml.sax.*; 
import java.io.*; 
class Test { 
  static public void main(String[] args) 
                         throws Exception { 
    SAXParserFactory factory =  SAXParserFactory.newInstance(); 
    SAXParser saxParser = factory.newSAXParser(); 
    XMLReader xmlReader = saxParser.getXMLReader(); 
    /* 
    xmlReader.setEntityResolver(new org.xml.sax.EntityResolver() { 
         public InputSource resolveEntity( 
                          String publicId, String systemId) { 
           StringReader strReader = new StringReader(""); 
           System.err.println(".......... " + systemId); 
           return new org.xml.sax.InputSource(strReader); 
         } 
    }); 
    */ 
    xmlReader.parse 
        ( new File(args[0]).toURI().toURL().toString() ); 
} } 

------------------- test.tmp -----------------------
  
<?xml version="1.0" encoding="UTF-8"?>  
<!DOCTYPE 
    office:document-content 
    PUBLIC "-//OpenOffice.org//DTD OfficeDocument 2.0//EN"  
    "office.dtd">  
<office:document-content> 
</office:document-content> 

----------------------------------------------------


--------------------------------------------------------------------------------------

javac Test.java ; java Test test.tmp
gcj -C Test.java ; gij Test test.tmp

java version "1.5.0"
gij (GNU libgcj) version 4.2.4 (Ubuntu 4.2.4-1ubuntu3)





import javax.xml.parsers.*;
import org.xml.sax.*;
import java.io.*;
class Test {
  static public void main(String[] args)
                         throws Exception {
    SAXParserFactory factory =  SAXParserFactory.newInstance();
    SAXParser saxParser = factory.newSAXParser();
    XMLReader xmlReader = saxParser.getXMLReader();
    /*
    xmlReader.setEntityResolver(new org.xml.sax.EntityResolver() {
         public InputSource resolveEntity(
                          String publicId, String systemId) {
           StringReader strReader = new StringReader("");
           System.err.println(".......... " + systemId);
           return new org.xml.sax.InputSource(strReader);
         }
    });
    */
    xmlReader.parse
        ( new File(args[0]).toURI().toURL().toString() );
} }

 
<?xml version="1.0" encoding="UTF-8"?> 
<!DOCTYPE
    office:document-content
    PUBLIC "-//OpenOffice.org//DTD OfficeDocument 2.0//EN" 
    "office.dtd"> 
<office:document-content>
</office:document-content>


        

------------------------------------------------------------------------------
 > tc:\Program Files\MiKTeX 2.7\fonts\! 
 > which it could not resolve. 
 > I now changed it to 
 > tc:\Progra~1\MiKTeX~2.7\fonts\! 
 > which makes it work now. 
--------------------------------------------------------------------------------------

As explained in section 2, TEX4ht needs external utilities for
creating the bitmaps used in the web-documents. The default set up is
a combination of dvips, gs (GhostScript) and ImageMagick�s convert
utility [4]. Unfortunately this is a rather slow process, and not a
very good one when it comes to antialiasing, and it�s definetely not
suited for typesetting uses. However, the dvi2bitmap tool available
freely from [3] is efficient and converts dvi files directly to png or
gif files. Dvi2bitmap works almost like a normal dvi-driver, but has
no support for so-called dvi-specials (such as included eps-figures
et.c.) Thus, a combination of the two is required: We use dvi2bitmap
whenever we encounter maths and special symbols as this is efficient
and produce good results, but use the aforementioned
dvips/gs/convert-process for items that dvi2bitmap cannot handle, such
as included PostScript figures.

--------------------------------------------------------------------------------------
http://schlosser.info/latexsystem-en.html#x1-200003.6
--------------------------------------------------------------------------------------
I forgot to mention about one thing. The installation script I posted
earlier does not install tex4ht.exe and t4ht.exe binaries. They come
with miktex-tex4ht-bin-2.7 package and should be installed first
(using miktex package manager). The rest of tex4ht is installed by the
script and it won't show up in miktex package manager (it have to be
maintained manually).

And a small correction - lines [51-54] are not needed and can be
deleted (but they don't do any harm, it's not a bug, just some dead
code).



>A few final remarks. It works on NT based systems only, so users of
>win98 and below are out of luck. Sorry. Shell scripting on NT is
>already traumatic enough for me. All lines are numbered in case they
>get wrapped in this post. Copy and paste them into a text editor and
>unwrap all lines not starting with a number. Remove the bracketed
>numbers and save it as 'install-tex4ht.bat'. If you are too lazy to do
>it manually save it as 'install-tex4ht.txt' and from the command line
>run (one line):
>(for /f "tokens=1* delims=[]" %K in (install-tex4ht.txt) do @echo.
>%L)>install-tex4ht.bat
>
>OK, without the further ado, here's the installation script.
>
>---------- INSTALL-TEX4HT.BAT
>[1]::--------------------------------------------------------------------::
>[2]::-             *** TEX4HT installation script
>***                   -::
>[3]::-
>-::
>[4]::-
>Usage:                                                           -::
>[5]::- *
>Installation:                                                  -::
>[6]::-      install-
>tex4ht.bat                                          -::
>[7]::-      install-tex4ht.bat
>install                                  -::
>[8]::- *
>Uninstallation:                                                -::
>[9]::-      install-tex4ht.bat
>uninstall                                -::
>[10]::-
>-::
>[11]::-
>Remaks:                                                          -::
>[12]::- * Installation script will prompt you for locations of
>unzipped  -::
>[13]::-   tex4ht.zip and newt4ht.zip files; you can to avoid that
>by     -::
>[14]::-   defining them in the configuration section of the
>script       -::
>[15]::- * After installation created TDS (TeX Directory Structure)
>tree  -::
>[16]::-   has to be added to MiKTeX's roots (use MiKTeX's
>Settings       -::
>[17]::-   utility mo.exe). Alternatively, you can install to
>MiKTeX's    -::
>[18]::-   intallation directory but this is not recommended.
>Refresh     -::
>[19]::-   FNDB after
>installation.                                       -::
>[20]::- * Uninstallation will leave behind created TDS
>tree;             -::
>[21]::-   delete it manually if
>desired                                  -::
>[22]::--------------------------------------------------------------------::
>[23]
>[24]@echo off
>[25]setlocal
>[26]
>[27]:: ******************** Configuration section
>*********************** ::
>[28]::unzipped tex4ht location
>[29]set tex4ht_unzipped_dir=.\tex4ht-20070708
>[30]::unzipped newt4ht location
>[31]set newt4ht_unzipped_dir=.\newt4ht
>[32]::tex4ht installation dir
>[33]set tex4ht_install_dir=C:\texmf-local
>[34]::change (Y) or not (N) the default <oo> script to <oo-alt>
>[35]set use_oo-alt=Y
>[36]::rollback if installation failed (Y/N)
>[37]::(removes only texh4ht and leaves created TDS tree behind)
>[38]set rollback=N
>[39]::
>****************************************************************** ::
>[40]
>[41]if not "%OS%"=="Windows_NT" goto requireNT
>[42]for %%G in (xcopy.exe) do if "%%~$PATH:G"=="" goto :noxcopy
>[43]
>[44]call :help
>[45]for %%G in (tex4ht_unzipped_dir newt4ht_unzipped_dir
>tex4ht_install_dir) do (
>[46]   call :setupdirs %%G
>[47]   if not defined %%G goto :abort
>[48])
>[49]goto %1 :install
>[50]
>[51]for %%G in (%tex4ht_install_dir%\tex4ht\base\win32\tex4ht.env
>%tex4ht_install_dir%\scripts\tex4ht\bat\*.bat) do (
>[52]   call %% > nul
>[53]   REM && move /Y %%G.tmp %%G || goto :copyerror
>[54])
>[55]
>[56]:install
>[57]set "copyopt=/I /S /Y /D"
>[58]xcopy.exe %newt4ht_unzipped_dir%\texmf\tex4ht %tex4ht_install_dir%
>\tex4ht %copyopt% || goto :copyerror
>[59]xcopy.exe %newt4ht_unzipped_dir%\texmf\tex\generic\tex4ht
>%tex4ht_install_dir%\tex\generic\tex4ht %copyopt% || goto :copyerror
>[60]xcopy.exe %tex4ht_unzipped_dir%\texmf\tex4ht\ht-fonts
>%tex4ht_install_dir%\tex4ht\ht-fonts %copyopt% || goto :copyerror
>[61]xcopy.exe %tex4ht_unzipped_dir%\bin\win32\*.bat %tex4ht_install_dir
>%\scripts\tex4ht\bat %copyopt% || goto :copyerror
>[62]echo Search and replace "c:\tex4ht\texmf\" with
>"%tex4ht_install_dir%"
>[63]for %%G in (%tex4ht_install_dir%tex4ht\base\win32\tex4ht.env
>%tex4ht_install_dir%scripts\tex4ht\bat\*.bat) do (
>[64]   call :replstr %%G "c:\tex4ht\texmf\" "%tex4ht_install_dir%"
>[65])
>[66]if /i "%use_oo-alt%"=="Y" (
>[67]   echo Search and replace "oo>" with "oo-bak>"
>[68]   call :replstr %tex4ht_install_dir%tex4ht\base\win32\tex4ht.env
>"oo>" "oo-bak>"
>[69]   echo Search and replace "oo-alt>" with "oo>"
>[70]   call :replstr %tex4ht_install_dir%tex4ht\base\win32\tex4ht.env
>"oo-alt>" "oo>"
>[71])
>[72]echo To finish installation add "%tex4ht_install_dir%" to MiKTeX
>roots and refresh FNDB
>[73]pause
>[74]goto :eof
>[75]
>[76]
>[77]:uninstall
>[78]echo Uninstalling TEH4HT...
>[79]for %%H in (%tex4ht_install_dir%tex4ht %tex4ht_install_dir%tex
>\generic\tex4ht %tex4ht_install_dir%scripts\tex4ht) do (
>[80]   if exist "%%~H\" (
>[81]           echo Removing "%%~H\"
>[82]           rmdir /S /Q "%%~H\"
>[83]   )
>[84])
>[85]echo Uninstallation complete
>[86]pause
>[87]goto :eof
>[88]
>[89]:help
>[90]findstr "^::-" %~sf0
>[91]goto :eof
>[92]
>[93]:setupdirs
>[94]::let's make sure that dirs are in short format
>[95]for /f "tokens=1* delims==" %%H in ('set %1 2^>^&1') do set
>"$cratchVar=%%~sfI"
>[96]if defined $cratchVar set "$cratchVar=%$cratchVar%\"
>[97]if not exist "%$cratchVar%" (set /p $cratchVar=%1=)
>[98]if not exist "%$cratchVar%" (
>[99]   set "$cratchVar="
>[100]  set /p $cratchVar=Specified directory does not exist. Retry? [y/
>n]:
>[101]  set $cratchVar | findstr /i "=y\>" >nul && goto :setupdirs ||
>set "$cratchVar="
>[102])
>[103]set %1=%$cratchVar%
>[104]goto :eof
>[105]
>[106]:replstr <file name> <search string> <replace string>
>[107]setlocal
>[108]echo.%1
>[109]for /f "tokens=1* delims=[]" %%K in ('find /n /v "" "%~1"') do (
>[110]  set lineNo=0000%%K
>[111]  set "txtline==%%L"
>[112]  REM This may break if txtline contains poison characters
>[113]  REM together with quotation characters
>[114]  call set "txtline.%%lineNo:~-4%%=%%txtline:%~2=%~3%%"
>[115])
>[116]::echo from for command so we don't get screwed by poison
>characters
>[117]>"%~1" (for /f "tokens=1* delims==" %%K in ('set txtline.') do
>echo.%%L)
>[118]goto :eof
>[119]
>[120]:copyerror
>[121]>&2 echo There were some errors while copying files. Make sure
>that correct installation directories are specified.
>[122]>&2 echo Aborting installation.
>[123]if /i "%rollback%"=="Y" call :uninstall
>[124]pause
>[125]exit /b 1
>[126]
>[127]:abort
>[128]>&2 echo Aborting installation.
>[129]pause
>[130]exit /b 1
>[131]
>[132]:requireNT
>[133]>&2 echo This script requires Windows NT based system. Sorry.
>[134]pause
>[135]exit /b 1
>[136]
>[137]:noxcopy
>[138]>&2 echo No xcopy command? Aborting installation, please don't
>hate me.
>[139]pause
>[140]exit /b 1

--------------------------------------------------------------------------------------
http://groups.google.com/group/comp.text.tex/browse_thread/thread/cd12489666d4d15f/2662902be1a6b276?lnk=gst&q=tex4ht+2008+pdf#2662902be1a6b276
--------------------------------------------------------------------------------------

comp.text.tex #397221 (0 + 9 more)                                            (1)+-(1)
From: Turgut Durduran <ugdc@ugdc.org>                                            \-(1)--[1]
[1] Re: tex4ht problems
Date: Thu Feb 28 17:31:52 EST 2008
Lines: 48

On 2008-02-21, Eitan M Gurari <gurari@cse.ohio-state.edu> wrote:
>
>>LaTeX document to either open document format, or to html that I would 
>
>>I am using text4ht from Ubuntu Gutsy Gibbon repositories. Nothing 
>>customized.
>>
>>The files I am using are here: 
>>http://www.stwing.upenn.edu/~durduran/tmp/distr/
>
>>The output is  broken -- Firefox can't open it with the following error. 
>
> Try upgrading the tex4ht instalation with the files from 
>
>     http://www.cse.ohio-state.edu/~gurari/TeX4ht/bugfixes.html
>
>
>>(2) references are listed but appear as:

From: Turgut Durduran <ugdc@ugdc.org>                                            \-(1)--(1)
[1] Re: tex4ht problems
Date: Thu Feb 28 17:31:52 EST 2008
Lines: 48

On 2008-02-21, Eitan M Gurari <gurari@cse.ohio-state.edu> wrote:
>
>>LaTeX document to either open document format, or to html that I would 
>
>>I am using text4ht from Ubuntu Gutsy Gibbon repositories. Nothing 
>>customized.
>>
>>The files I am using are here: 
>>http://www.stwing.upenn.edu/~durduran/tmp/distr/
>
>>The output is  broken -- Firefox can't open it with the following error. 
>
> Try upgrading the tex4ht instalation with the files from 
>
>     http://www.cse.ohio-state.edu/~gurari/TeX4ht/bugfixes.html
>
>
>>(2) references are listed but appear as: ``�:mbox:1,�:mbox:2'' instead of 
>>supersciprt 1,2 etc.
>
> I modified the bugfixes distribution to take care of the problem along the
> approach suggested by Dan (thanks Dan!).
>
> -eitan
>



I thank Dan and Eitan for their help.

Eitan, under the instructions for an ``upgrade'' (which I assume bugfixes 
should be applied as such?), it says:

``[Invocation Scripts] Find where htlatex resides (e.g., \u2018which 
htlatex\u2019). Save the files htlatex, httex, httexi, and htcontext 
elsewhere and get new scripts from
bin/linux/
for a replacement.''


 but there is no such thing in 
http://www.cse.ohio-state.edu/~gurari/TeX4ht/fix/tex4ht-1.0.2008_02_26_0908.tar.gz

should it read as bin/unix instead? (I tried that, seems fine)

I am checking if this fixed my problem yet.


--------------------------------------------------------------------------------------
        ---- ---------------------- --------------------------------------------------
From: Richard Koch <koch@math.uoregon.edu>
Sender: macosx-tex-bounces@email.esm.psu.edu
To: emalito@uchicago.edu,
        TeX on Mac OS X Mailing List <macosx-tex@email.esm.psu.edu>
Cc: 
Subject: Re: [OS X TeX] convert .tex file to .html
Date: Sat, 17 Nov 2007 19:25:47 -0800

Enrico,

What TeX distribution are you using? If it is TeX Live or gwTeX, then  
TeX4ht should be already installed in the distribution.

The latest version of TeXShop has an "engine" file for \ this program.  
It is in

        ~/Library/TeXShop/Engines/Inactive

If you don't find it there, but have TeXShop 2.14, then move the  
entire folder ~/Library/TeXShop/Engines to your desktop. The next time  
TeXShop starts, it will create the default folder, which will include  
a TeX4ht folder and htlatex.engine. To use it, move htlatex.engine  
from the inactive folder to  ~/Library/TeXShop/Engines.

To use this program, enter standard latex source in the Source Window.  
In the pulldown menu next to the Typeset button, select htlatex. When  
you push the Typeset button, your latex source will be typeset and  
then converted to an html file, and this html file will open in Safari.

You can make this happen automatically (without selecting htlatex) by  
entering

        %!TEX TS-program = htlatex

near the top of your source file.

Dick Koch
koch@math.uoregon.edu
====================
From: Thomas Rike <tricycle222@earthlink.net> 
Sender: macosx-tex-bounces@email.esm.psu.edu 
To: emalito@uchicago.edu, 
        TeX on Mac OS X Mailing List <macosx-tex@email.esm.psu.edu> 
Cc:  
Subject: Re: [OS X TeX] convert .tex file to .html 
Date: Sat, 17 Nov 2007 19:37:15 -0800 
 
Enrico, 
 
I am no expert, but on the command line I first change my directory to   
the directory containing the latex file (Just drag and drop the folder   
works fine). Then type htlatex nameofyourfile.tex 
 
For graphics I think you will need ghostscript and Imagic which I got   
with the i-Intaller. 
 
If you are not working on the command line then a very effective gui   
is SimpleTeX4ht (version 1.8). Just download, install and run. Select   
the button ``Convert'' and you can browse your computer for the file you   
want to convert to html. 
 
http://www.apple.com/downloads/macosx/unix_open_source/simpletex4ht.html 
 
HTH, Tom Rike 
 
--------------------------------------------------------------------------------------


--------------------------------------------------------------------------------------
http://www.sun.com/software/star/odf_plugin/

--------------------------------------------------------------------------------------
--------------------------------------------------------------------------------------
 > > The problem was in missing hypertext fonts, as indicated by the 
 
 > I think that's only one of two errors. Tex4ht should produce valid mml 
 > files even if the .htf files are missing, as TeX allows users to define 
 > their own fonts. Unknown Characters could be replaced by a square or a 
 > question mark as it is used in other contexts. 
 
I believe it will be a wrong approach, as users might get improper 
content and in some cases are unlikely to notice it.  Moreover, the 
philosophy of OpenOffice, and of quite a few advocates of XML, is to 
maintain zero tolerance for errors. So not allowing improper data is 
consistent with that philosophy. 
 
The squares or question marks OpenOffice and other browsers display 
are not due to errors in the exhibited files. They point to missing 
fonts in the platforms on which the browsers run. 
 
A solution to a missing font problem should not be that difficult to 
achieve by replacing fonts.  For instance  dvi2dvi seems to be 
a utility designed for that purpose (http://packages.debian.org/sid/dvi2dvi). 
 
Technically, it probably will be easy to add an option for tex4ht to 
output, say, a question mark where characters are missing.  In case 
you strongly like to have such an option, I'll provide it. 
 
-eitan 

--------------------------------------------------------------------------------------

biblatex
--------------------------------------------------------------------------------------

FWIW, I mirror Eitan's web pages (and distributions) at 
ftp://tug.org/mirror/www.cse.ohio-state.edu/~gurari/TeX4ht.  Maybe CTAN 
could easily mirror that and hence be up to date?  In particular, the 
file 
ftp://tug.org/mirror/www.cse.ohio-state.edu/~gurari/TeX4ht/fix/tex4ht.tar.gz 
is the latest bug fix release, I believe. 

--------------------------------------------------------------------------------------
 
 

 
 

--------------------------------------------------------------------------------------
MacTeX installs the full TeX Live 2007, with virtually no   
changes. Indeed, you can go to 
 
        http://www.tug.org/mactex/whatgetsinstalledwhere.html 
 
to see exactly how we configure TeX Live. 

--------------------------------------------------------------------------------------


I've done this with MetaPost graphics.  The first step is to get
yourself a PDF output as you suggested in (b).  I've tried various
techniques, but I've decided for myself that this is the "best"
solution (quality wise).  You'll need the following tools:

1) pstoedit --> http://www.pstoedit.net/pstoedit
2) skencil --> http://www.skencil.org
3) inkscape --> http://www.inkscape.org

In Gentoo Linux I needed Skencil in order for Inkscape to have SKencil/
SKetch support; however, pstoedit and Inkscape are the only two
utilities *directly* used.  If your version of Inkscape has SK support
without installing SKencil, then that's good.

Let's assume your PDF is foo.pdf.  Then do the following:

pstoedit -page 1 -rgb -dt -psarg "-r9600x9600" -f sk foo.pdf foo.sk

This will produce a SKencil/SKetch format of page 1 of your document.
-rgb uses RGB colors and -dt tells pstoedit to trace your fonts.  This
tracing gives a larger filesize, but it ensures that your output text
looks exactly like your text in your PDF.  The -psarg "-r9600x9600"
tells GhostScript to process stuff at 9600dpi.

Note that pstoedit can export to SVG directly, but I've found that the
results aren't as good as going through SK and using Inkscape to
convert the SK to SVG.

After you get foo.sk, you can then do

inkscape -z -f foo.sk -l foo.svg

Hopefully this will be to the quality and filesize that you want.


--------------------------------------------------------------------------------------
http://ooolatex.sourceforge.net/
--------------------------------------------------------------------------------------

To: =?iso-8859-2?B?QmFyYm9yYSBIYXbt+G924Q==?= <bhavirova@seznam.cz> 
Subject: TeX4ht for LaTeX to DocBook 
In-Reply-To: <000701c73441$e95d8f90$7601a8c0@barbora> 
References: <000701c73441$e95d8f90$7601a8c0@barbora> 
X-Mailer: VM 7.04 under 21.1 (patch 14) ``Cuyahoga Valley'' XEmacs Lucid 
--text follows this line-- 
 
 
Hi Barbora, 
 
Would the following (unsafe) configuration do the job for you. 
 
  \newtoks\toks  
  \Configure{$}{}{}{\getMath}    
  \def\getMath#1${\relax$\fi  
                  \expandafter\toks\expandafter{\gobble#1}%  
                  \HCode{<mathphrase>\the\toks</mathphrase>}}  
  \def\gobble#1{}  
 
-eitan 
 
 > I need to let all the mathematics as it is and only wrap 
 > it in <mathphrase></mathphrase>.  
 >  
 > I.e. $\sqrt{a^b}$ will be <mathphrase>\sqrt{a^b}</mathphrase> 
 >  
 > Is it possible?  
 >  
 > I am able to wrap it with  
 >  
 > \Preamble{xhtml}  
 > \begin{document}  
 > \Configure{$}{\HCode{<mathphrase>}}{\HCode{</mathphrase>}}{}  
 > \EndPreamble   
 >  
 > in my configuration file, but I don't know how to forbid processing of the 
 > math code. 

Barbora, 
 
Another (unsafe) option: 
  
  \newtoks\toks  
  \Configure{$}{}{}{\getMath}    
  \def\getMath#1${\relax$\fi  
                  \expandafter\toks\expandafter{\gobble#1}%  
                  \HCode{<mathphrase>}%  
                  \bgroup  
                      \noindent  
                      \Configure{HtmlPar}{}{}{}{}  
                      \ConfigureEnv{verbatim}{}{}{}{}  
                      \immediate\openout15=tempmatheq.tex  
                      \immediate\write15{\string\begin{verbatim}\the\toks  
                                         \string\end{verbatim}}%  
                      \immediate\closeout15  
                      \input tempmatheq.tex \noindent  
                  \egroup  
                  \HCode{</mathphrase>}}  
  \def\gobble#1{}  
 
-eitan 
 
 > I found a problem, because the symbols < and > 
 > can't be in the mathematic text as they are used for tags. I hope that 
 > changing it into &lt; and &rt; will help.  
 
 

Barbora, Try the following variant. -eitan

  \newtoks\toks 
  \Configure{$}{}{}{\getMath}   
  \def\getMath#1${\relax$\fi 
                  \expandafter\toks\expandafter{\gobble#1}% 
                  \HCode{<mathphrase>}% 
                  \bgroup 
                      \noindent 
                      \Configure{HtmlPar}{}{}{}{} 
                      \ConfigureEnv{verbatim}{}{}{}{} 
                      \Configure{verbatim}{ }{ } 
                      \immediate\openout15=tempmatheq.tex 
                      \immediate\write15{\string\begin{verbatim}\the\toks 
                                         \string\end{verbatim}}% 
                      \immediate\closeout15 
                      \input tempmatheq.tex \noindent 
                  \egroup 
                  \HCode{</mathphrase>}} 
  \def\gobble#1{} 


 > It works. But when there is a space in math, it is replaced by &#x00A0; 

--------------------------------------------------------------------------------------

http://www.alanwood.net/unicode/

Alan Wood�s Unicode Resources
Unicode and Multilingual Support in HTML, Fonts, Web Browsers and Other Applications
--------------------------------------------------------------------------------------
 
 > > How about a command similar to the following one? 
 > > 
 > >      t4ht try  -d'~/WWW/temp/foo\ foo/' 
 > > 
 >  
 > To get to that requires changing the htlatex script to quote the   
 > argument to t4ht.   
 
I don't think we can go for htlatex modifications, as the outcome 
might also depend on the operating system in use.  In any case, I 
modified t4ht.c to allow for FULLY quoted arguments such as 
 
      t4ht try  '-d~/WWW/temp/foo\ foo/' 
      t4ht try  "-d~/WWW/temp/foo foo/"

and in the case of htlatex 
 
  htlatex file " "   ``'-d~/WWW/temp/foo foo/''' 

--------------------------------------------------------------------------------------
--------------------------------------------------------------------------------------

 
>Ok. I think I can fix my problem editing the tex4ht.env file. 
>I can read in the file /usr/share/doc/tex4ht/README.Debian 
>that TeX4ht can use three different way to convert images: 
> 
>DEFAULT: with dvipng. I think I'm using it right now. 
>UPSTREAM: using ImageMagick suite, so using ``convert'' 
>NETPBM: with the netpbm suite. Never heard before. 
> 
>One of them is active, the other scripts are commented 
>(how? with spaces?) 
 
My guess the relevant parts in the Debian environment file have the 
following outline.   
 
    <convert>  
     G.png  
     Gdvips ... 
     Ggs ... 
    G.svg  
    Gdvips ... 
    Gpstoedit ... 
    G.  
    Gdvips ... 
    Gconvert ... 
    </convert>  
     
    <netpbm>  
    ....... 
    </netpbm>  
     
     <dvipng>  
    G.png  
    Gdvipng ... 
    G.gif  
    Gdvipng ... 
    G.  
    Gdvips ... 
    Gconvert ... 
     </dvipng>  
 
The dvipng segment is the active one as its enclosing tags are 
commented out, with spaces before the opening tag <dvipng> and the 
closing tag </dvipng>.  The dvipng segment consists of three 
parts:  
 
* A G.png and a G.gif parts invoking the dvipng 
  utility to satisfy requests for png and gif bitmaps 
 
* A G. part invoking dvips+convert to handle requests for other 
  formats of bitmaps  
 
Spaces before the records of the first two parts will comment out the 
dvipng subscripts, and forward all the requests to the subscript in 
the third part. 
 
Alternatively, remove the spaces before the <dvipng> and </dvipng> tags 
to deactivate the script they enclose, and place spaces before the 
<convert> and </convert> tags to activate the enclosed script. 
 
>In which language is written the tex4ht.env file? Python? Perl? 
>Can somebody show me how to edit the tex4ht.env file? 
 
The file cosists of a specially designed script to be interpreted by 
the tex4ht utilities. The tex4ht.env entry in the index of 
 
  http://www.cse.ohio-state.edu/~gurari/TeX4ht/mn.html 
 
offers some clues to its content.  Modifications to the file can be 
done with editors capable of working with plain text. 
 
-eitan 

--------------------------------------------------------------------------------------
OS-X:
                    (/usr/local/teTeX/share/texmf.local/tex4ht/base/tex4ht.env) 
                    (/usr/local/teTeX/share/texmf.tetex/fonts/tfm/public/cm/cmr10.tfm) 
                    --- warning --- Couldn't find font `cmr10.htf' (char codes:   0--127) 
          Can you show me the messages issued� for 
                
 
          � � tex4ht� -hF -hv� test 
 
That gave me a bit more to work with, I added� 
 
 
 
TEX4HTFONTSET=alias,iso8859�� 
 
TEX4HTINPUTS=.;$TEXMF/tex4ht/base//;$TEXMF/tex4ht/ht-fonts/{$TEX4HTFONTSET}//�� 
 
T4HTINPUTS=.;$TEXMF/tex4ht/base//�� 



tex4ht.c (2006-07-14-10:23 kpathsea) 
 
tex4ht -hF 
 
� -hv 
 
� test 
 
(/usr/local/teTeX/share/texmf.local/tex4ht/base/tex4ht.env) 
 
given TEX4HTFONTSET = alias,iso8859 
 
setting TEX4HTFONTSET={iso8859/1,ascii,alias,mozilla,unicode,alias,iso8859} 
 
texmf.cnf = /usr/local/teTeX/texmf.cnf 
 
TEX4HTINPUTS = 
.:{/Users/kai/Library/texmf,!!/usr/local/teTeX/share/texmf.local,!!/usr/local/teTeX/share/te\
xmf.gwtex,!!/usr/local/teTeX/share/texmf.tetex,!!/usr/local/teTeX/share/texmf}/tex4ht/base//\
:{/Users/kai/Library/texmf,!!/usr/local/teTeX/share/texmf.local,!!/usr/local/teTeX/share/tex\
mf.gwtex,!!/usr/local/teTeX/share/texmf.tetex,!!/usr/local/teTeX/share/texmf}/tex4ht/ht-font\
s/{iso8859/1,ascii,alias,mozilla,unicode,alias,iso8859}// 
 
(/usr/local/teTeX/share/texmf.local/tex4ht/ht-fonts/iso8859/1/charset/unicode.4hf) 


--------------------------------------------------------------------------------------
 
 > I downloaded the recent release tex4ht-1.0.2006_07_23_0130, but where  
 > can I find the documentation for installing it? 
  
I should wrie such instructions... The idea is just to replace the old 
files/directories with the new ones: 
 
   bin/ht/perl/mk4ht.perl 
   bin/ht/unix/ht* 
   bin/mac/* 
   texmf/* 
 

 
and refresh lR-s, e.g., with the texhash command. 

 
--------------------------------------------------------------------------------------

The problem has vanished.  It
might be connected with the update of tetex-bin that I just made on the
sid box.

--------------------------------------------------------------------------------------

a side note: recently I released TeXML 2.0 beta1. I expect no issues,
and I'm  going to announce TeXML 2.0 in the middle of July. The new
TeXML site is: http://getfo.org/texml/ .

conrad.ammon@gmail.com wrote:
> Has anyone taken a look at using XSL to give needed functionality to
> TeXML?
>
> TeXML on the level of TeX... on its own its not too useful.  However,
> XSL was specifically designed for doing markup, similar to the macros
> of LaTeX.  Is there a good set of XSL scripts that anyone has written?

There are a lot of them, but they are private. The only public script
I'm aware of is XML2TeXML by Wolfgang Jeltsch
(http://xml2texml.sourceforge.net/ ). And I'm starting working on a
DocBook to TeXML converter in the near future.

--------------------------------------------------------------------------------------
To: gurari.1@osu.edu 
Cc: Barbara Beeton <bnb@ams.org>, Anna Hattoy <amh@ams.org> 
Subject: documentation for dratex (fwd) 
Date: Fri, 02 Jun 2006 10:26:22 -0400 (EDT) 
 
hi, eitan, 
when i tried to send this message to you, using 
the e-mail address i got from the tug office, 
it was returned.  i found the address i'm using 
now by searching the osu.edu web site.  it seems 
that the university has made your previous 
incarnation unreachable in a lot of places ... 
                                                        -- bb 
 
---------- Forwarded message ---------- 
Date: Fri, 2 Jun 2006 10:20:23 -0400 (EDT) 
From: Barbara Beeton <bnb@ams.org> 
To: gurari@cis.ohio-state.edu 
Cc: Barbara Beeton <bnb@ams.org>, Anna Hattoy <amh@ams.org> 
Subject: documentation for dratex 
 
hello, eitan, 
we've received a manuscript for publication in 
one of our journals that calls for 
\usepackage{DraTex}. 
 
before we can decide whether this should be 
installed here for production, we need to look 
at the documentation.  but the documentation 
isn't at ctan -- the readme file says that it 
can be found at 
   http://www.cis.ohio-state.edu/~gurari/systems.html 
 
an attempt to reach that url results in 
   www.cis.ohio-state.edu cannot be found.  please 
   check the name and try again. 
 
what is the current location, please?  and 
please update the readme file for this and any 
other material at ctan that has this reference. 
 
thanks. 
                                            
--------------------------------------------------------------------------------------

mk4ht xhlatex convtest "html,jsmath"

How about trying the `jslatex' option.  It doesn't preserve the 
original latex math. Instead, it produces html with normalized latex 
math.   
 
-eitan 
 
 > Thanks for the tip, I had already tried something like that, but Word  
 > doesn't like it. Also, going through OpenDocument doesn't quite get the  
 > formulae right.. 
 >  
 > I am now trying to edit the mathml configuration so that it pastes the  
 > LaTeX 


--------------------------------------------------------------------------------------

 > > There is an option to send output files (.html,.css,.png) files to
 > > a directory other than the current directory. For example
 > >
 > >    htlatex sample2e "" "" "-d~/home_page/test"
 > >
 > > will send the output files to ~/home_page/test.

 > Well, it would be nice if the temporary directory was automatically
 > created if it doesn't exist.  But what I'd appreciate even more is if it
 > would (or is?) possible to permanently set the -d switch, so that I
 > don't need to type it every time.

That can be done by adding in the htlatex-like script files the -d
switch to the t4ht records, e.g.,

 t4ht ... -d~/WWW/temp/

Similarly, within mk4ht add the -d switch to the trailing quoted "..."
segments, e.g.,

  "ht",   "htlatex",     "latex",      "", "", "-cvalidatehtml -d~/WWW/temp/", 

It doesn't seem appropriate to introduce such a switch permanently
as different users might prefer different directories for their
output.  

I have the switch -d~/WWW/temp/ set in my private scripts. I insert
the switch -d./ in the command line, when I want to overwrite that
switch to have the outcome stay in the work directory.

 > I've read the documentation about configuration files, but it seems this
 > only relates to (La)TeX code, not to a configuration file for
 > mk4ht/htlatex/...  Did I miss something?

Try the command `t4ht' without arguments.  It will show the available
options.  I'll modify the documentation to deal with issue.

--------------------------------------------------------------------------------------

gs -sDEVICE=png256 -sOutputFile=figure1.png -dEPSCrop -r600 -dBATCH
     -dNOPAUSE figure1.eps

--------------------------------------------------------------------------------------

 > As far as I saw, the pictures inserted in the OO document, have normally 
 > a size of appr. 1cm without recognizing the original ratio. Is it 
 > possible to recognize either the size or the ratio (then resizing is 
 > very simple) for pictures ? I normally use \includegraphics, most of the 
 > time with the option [width=1.0\textwidth] to insert my pictures. If it 
 > helps I could send you a sample file.  
 
TeX4ht specifies sizes only for bounding boxes, when it gets them from 
latex.  I don't know how to provide sizes in general, as dimensions 
for paper output quite often result in distorted dimensions within XML 
presentations.  I don't understand why OpenOffice uses 1cm dimensions 
for figures that are offered without requested dimensions---I would 
expect the natural sizes of the figures to be preferable. 


 > You're right. The converted pictures are all in a good size. If I click 
 > in OO on one picture and change it to  
--------------------------------------------------------------------------------------
comp.text.tex #353104 (0 + 4 more)                                                      [1]
Date: Thu Mar 23 18:29:21 EST 2006
From: USENET-news <usenet@marlowa.plus.com>
[1] latex2html and wiki
Lines: 19

Hello Texnicians,

I have found that wiki is a great way to publish a doc so 
that the other developers can collaborate on it, but I 
prefer to start that document in LaTeX initially. So I used 
latex2html to convert it to HTML and used copy-n-paste to 
copy from the HTML to a WIKI page. This gave me a document 
that the other developers (who are not TeX aware) can work 
with. But this was rather tedious to do.

I have seen several attempts to write latex2wiki but they 
all seem quite poor. It occurs to me that it might be better 
to take latex2html and add a mode that would allow WIKI 
output as an option. What do you guys think? Does this sound 
reasonable? Is latex2html still supported/maintained/developed?

Regards,

Andrew Marlow

> Most probably it is well-defined. Most probably he assumes some
> particular wiki engine with some fixed wikitext syntax.

Yes, I assume mediaWiki.

--------------------------------------------------------------------------------------
 
 > >The problem is in missing fonts for the browser to display 
 > >the characters &#x2329; and  &#x232A;. 
 >  
 > These characters, and a bunch of others from what we can see here 
 > are sortof at the ragged edge of font world as far as default browsers 
 > are concerned.  We have the default installs of IE6 and Firefox 1.5 
 > loaded here, to emulate what we would expect the average user to use. 
 > >From what we can see here, these characters do not come loaded by default 
 > with 
 > the ability to render these characters...  Or are off base here somehow? 
 
Look at the messages issued by tex4ht.c for the location of the active 
unicode.4hf file.  Copy that file to your work directory. You can 
redefine any character to your liking. For instance, an entry 
 
    '&#x2329;' ''    'foo' ''  
 
will substitute the character &#x2329; with `foo' everywhere.  On the 
other hand, an entry asks for a pictorial substitution where allowed 
and `foo' where pictures are not allowed. 
 
    '&#x2329;' ''    'foo' '1'  
 
-eitan 










--------------------------------------------------------------------------------------
http://www.simpletex4ht.free.fr/
--------------------------------------------------------------------------------------

http://fedoraproject.org/extras/4/i386/repodata/repoview/__nogroup__.group.html
--------------------------------------------------------------------------------------

   #help www-math@w3.org archives

   W3C home > Mailing lists > Public > www-math@w3.org > January 2006

Re: mathml applications

     * This message: [ Message body ] [ Respond ] [ More options ]
     * Related messages: [ Previous message ] [ In reply to ]


    From: Bruce Miller <bruce.miller@nist.gov>
    Date: Mon, 09 Jan 2006 10:54:41 -0500
    Message-ID: <43C28741.8000906@nist.gov>
    To: Neil Soiffer <neils@dessci.com>
    CC: Public MathML mailing list <www-math@w3.org>


Neil Soiffer wrote:
> This week, I will be giving an update to a talk I gave last year about
> MathML applications.  If anyone has some new applications not listed on
> http://www.w3.org/Math/Software/ or if you have an update to info listed
> there, please send that info so I can incorporate it into the talk.

There's my LaTeXML, which I've been meaning to submit to the software
list, once I get a breath.
See  http://dlmf.nist.gov/LaTeXML/
for description, documentation and a (somewhat out-of-date) example from DLMF.

A brief description (and apologies, in advance, for beating my own drum):

LaTeXML attempts to mimic TeX's behaviour as fully as possible, but interprets
TeX via a Perl program, rather than using TeX's engine itself.  Thus extension
and customization can be done using perl modules, as well as TeX code.

It processes whole documents (although it can be induced to process fragments,
such as formula), converting the document to a LaTeXML DocType that corresponds
to LaTeX's constructs.  This avoids the information lost by direct conversion
to html. This output can then be converted to html (w/images for math),
xhtml (w/MathML) or other formats using XSL stylesheets (included for html/xhtml).

[In principle, but not practically, it could also be made to directly generate
 other XML formats: Ie. the processing engine is (almost) separated from the
 document generation. However, since the vocabularies are so large it's nontrivial
 to replace the generation definitions.]

As for MathML: the program attempts to parse the formula into, at least,
a parse tree.  This is necessary to generate "Good" presentation MathML.
It is close to enough to generate content MathML, as well, although
some major sticking points and ambiguities, such as the meaning of superscripts
and the handling of unknown symbols, are the subject of further development.
To increase the quality of the parsing, an author can either use special
markup to disambiguate the notations, or use document-specific external declarations.


> The audience will likely have a strong TeX background.  If you have had
> any experience (pro or con) with the various TeX-to-XHTML+MathML
> converters in the last year, I'd like to hear about that so I can pass
> it along.

I've been using it pretty heavily in the DLMF (Digital Library of Mathematical
Functions) project.  I've processed 26 chapters (must be equivalent to
6-700 pages) so far; we're preparing a draft of the site for evaluation
(alas, the editors are keeping it close to the chest, however).
So, I'd have to say my experience is positive :>

Michael Kohlhase has also been using LaTeXML in developing his STeX system.

> Thanks,
>
> Neil Soiffer
> Senior Scientist
> Design Science, Inc.
> www.dessci.com <http://www.dessci.com>
> ~ Makers of Equation Editor, MathType, MathPlayer and MathFlow ~


--
bruce.miller@nist.gov
http://math.nist.gov/~BMiller/

   Received on Monday, 9 January 2006 15:54:31 GMT

     * This message: [ Message body ]
     * Previous message: Paul Libbrecht: "Re: mathml applications"
     * In reply to: Neil Soiffer: "mathml applications"

     * Mail actions: [ respond to this message ] [ mail a new topic ]
     * Contemporary messages sorted: [ by date ] [ by thread ] [ by subject ] [ by
       author ]
     * Help: [ How to use the archives ] [ Search in the archives ]

   This archive was generated by hypermail 2.2.0+W3C0.50 : Monday, 9 January 2006
   15:54:32 GMT 

--------------------------------------------------------------------------------------
http://www.activemath.org/amwiki/index.php/Conversion-efforts-to-content-math
--------------------------------------------------------------------------------------
omp.text.tex #347601 (0 + 2 more)                                  (1)--[1]
From: Troy Henderson <thenders@gmail.com>
[1] Re: eps to svg
Lines: 15
Date: Sun Jan 01 12:59:14 EST 2006

Convert it to PDF using, say, epstopdf.  After that, use my pdf2svg
converter located at

http://www.tlhiv.org/MetaPost/tools/mptosvg/

If you have any trouble, let me know.

Troy Henderson
Assistant Professor
Department of Mathematical Sciences
252 Thayer Hall
United States Military Academy
West Point, NY  10996
(845) 938-5649
http://www.tlhiv.org

--------------------------------------------------------------------------------------
Clarify:

9. \Configure{CutAt} {unit} {before-button} {after-button}

10. \Configure{+CutAt} {unit} {before-button} {after-button} 
--------------------------------------------------------------------------------------


   One more request. 
   Would be so kind as to generate the SINGLE page HTML 
   from http://www.cse.ohio-state.edu/~gurari/TeX4ht/                
   It will make searching much easier. 
   Simple `find' it will suffice. Currently 
   I keep all the pages on my hard disk and use                      
   grep instead. 
 
From: Wlodek Bzyl <matwb@univ.gda.pl> 
To: Eitan Gurari <gurari@cse.ohio-state.edu> 
Subject: Re: TeX4ht: SVG: Wrong walue of the TYPE attribute in the generated 
 OBJECT tag 
Date: Mon, 24 Oct 2005 21:45:42 +0200 
 
--------------------------------------------------------------------------------------

One suggestion though is, that you make it more obvious that one should  
really study the .log file. (I normally don't do this.) I have seen that  
you have put somewhere a hint to the mktex4ht.4ht index, but I cannot  
remember that I have seen that index on your website. That would be very  
much welcome. In fact, but the sources as html online. Then one could  
more easily refer to it. 

--------------------------------------------------------------------------------------

The (behind the scenes) process of converting the PDF to SVG above uses the following three external programs:

   1. pstoedit to convert the PDF to SKencil/SKetch format
   2. skconvert to convert the SKencil/SKetch file to SVG
   3. gzip to compress the SVG

The actual commands that are executed are:

    * pstoedit -page 1 -dt -psarg 

http://www.nongnu.org/skencil/
--------------------------------------------------------------------------------------

 > http://baruch.ev-en.org/proj/chktex/


--------------------------------------------------------------------------------------

http://www.dcs.fmph.uniba.sk/~emt/


There is absolutely no reason why this should be, given that an excellent
implementation of direct one-click real-time lossless translation of TeX 
math to SGML and back again has existed for a decade (in the EuroMath 
editor). 


--------------------------------------------------------------------------------------
From: Herbert Schulz <herbs@wideopenwest.com>
Sender: <MacOSX-TeX@email.esm.psu.edu>
To: "TeX on Mac OS X Mailing List" <MacOSX-TeX@email.esm.psu.edu>
Subject: Re: [OS X TeX] TeX to word on Mac
Date: Sun, 26 Jun 2005 09:04:44 -0500


On Jun 25, 2005, at 11:49 PM, George Ghio wrote:

>
> Open terminal and type;
>
> Welcome to Darwin!
> George-Ghios-Computer:~ georgeghio$ cd /Users/georgeghio/Desktop/TeX
>
> Hit return & type
>
> George-Ghios-Computer:~/Desktop/TeX georgeghio$ htlatex /Users/ 
> georgeghio/Desktop/TeX/TheFlight.tex
>
> Hit return I then get the html file,
>
> I then open the html file with Explorer,
>
> Select all,
>
> Copy,
>
> Paste into word.
>
> If I try to open the html file with word  MS does what it does best  
> and crashes.
>
> Oh well, it works and I end up with a word document.
>
> Thank you and everyone else for taking the time to keep me from  
> becoming a basket case.
>
> George L Ghio
>

Howdy,

Now that you know how to do it in terminal, here's a way to do it  
from within TeXShop.

Go to the ~/Library/TeXShop/Engines/ folder (~ is your HOME  
directory) and duplicate one of the .engine files there. Rename it  
htlatex.engine and open it up (double clicking will open it up in  
TeXShop). Remove the contents of the file and put the following lines  
into that file:

#!/bin/bash
htlatex "$1"

and save the file (again in ~/Library/TeXShop/engines and with the  
name htlatex.engine).

At the beginning of any .tex file you want to be processed with  
htlatex put the line

%!TEX TS-program = htlatex

and then process the file with Cmd-T (Typeset). As an alternative you  
can simply choose the htlatex engine in the dropdown menu in the  
toolbar.

Good Luck,

Herb Schulz
(herbs@wideopenwest.com)


--------------------- Info ---------------------
Mac-TeX Website: http://www.esm.psu.edu/mac-tex/
           & FAQ: http://latex.yauh.de/faq/
TeX FAQ: http://www.tex.ac.uk/faq
List Post: <mailto:MacOSX-TeX@email.esm.psu.edu>

--------------------------------------------------------------------------------------
SVG :


==> pstoedit : 
I tried pstoedit -f svg-plot file.eps file.svg, but that did not work well.
The greek letters were omitted and the non-greek letter did overlap.


==> Inkscape (http://www.inkscape.org/download.php) 

==> TLC2 mentiones dvi2svg <URL:http://www.activemath.org/~adrianf/dvi2svg/>,
pstoedit and an article in TUGboat:

Michel Goossens and Vesa Sivunen
LaTeX, SVG, Fonts
TUGboat 22(4) 269-279, 2001


  http://www.tug.org/TUGboat/Articles/tb22-4/tb72goos.pdf

==> http://dvisvg.sourceforge.net/

-----------------------------------------------------------------------------------
---

gcc -Wall -o tex4ht tex4ht.c -I/usr/local/teTeX/include -L/usr/local/teTeX/lib -DKPATHSEA -DHAVE_DIRENT_H -DHAVE_STRING_H -lkpathsea

su
rm -R /usr/local/teTeX/share/texmf.local/tex4ht/ht-fonts
mv -R /Users/eitangurari/Desktop/downloads/ht-fonts  /usr/local/teTeX/share/texmf.local/tex4ht/.
mv /Users/eitangurari/Desktop/downloads/newt4ht/bin/mac/tex4ht /usr/local/teTeX/bin/powerpc-apple-darwin-current/.
cp /Users/eitangurari/Desktop/downloads/newt4ht/texmf/tex4ht/base/unix/tex4ht.env /usr/local/teTeX/share/texmf.local/tex4ht/base/.
texhash
exit

--------------------------------------------------------------------------------------

Use latex (with preview.sty) and dvipng.  dvipng will even output the
ascender and descender values so that inline math can be displayed
with correct vertical positioning.

--------------------------------------------------------------------------------------
add the outcome of javahelp
--------------------------------------------------------------------------------------

WordML2LaTeX is a meeting point between two titans in word processing: 
Microsoft Word 2003 and LaTeX2e.

It is a XSL stylesheet that transforms a Word document (WordML) in a 
LaTeX2e source. With it You can use Word as a front end for LaTeX.

Location on CTAN: /support/WordML2LaTeX

--------------------------------------------------------------------------------------
--------------------------------------------------------------------------------------

On mac OS X:

%% t4ht utility, sharing files with TeX4ht
TEX4HTFONTSET=alias,iso8859
TEX4HTINPUTS = .;$TEXMF/tex4ht/base//;$TEXMF/tex4ht/ht-fonts/{$TEX4HTFONTSET}//
T4HTINPUTS=                     .;$TEXMF/tex4ht/base//

--------------------------------------------------------------------------------------
 Using fonts of (La)TeX in stylesheets of WWW-pages
Date: Wed Aug 25 06:42:40 EDT 2004
Lines: 20



Yes! It is possible:

http://iki.fi/juhtolv/pelle.html
http://iki.fi/juhtolv/css-download/
http://iki.fi/juhtolv/css-download/readme.txt

You'd better use Mozilla, Mozilla Firefox, Galeon, Konqueror or Opera,
because they support alternative stylesheets and MSIE don't.

I'd like to support cm-super, too, but it is fscking huge package and it
is real pain in the ass to find right font from those gazillion files,
because they have almost meaningless fontnames.

--------------------------------------------------------------------------------------

m-tex4ht.tex

--------------------------------------------------------------------------------------
MS:

 >  run
 >
 >      htlatex  analysis.tex >printout.txt 2>&1
 >
 > to get both output and stderr to file
 >
--------------------------------------------------------------------------------------
To: Maarten Wisse <Maarten.Wisse@urz.uni-heidelberg.de>
Subject: Gentoo ebuild, or, Pay my debt :-)
In-Reply-To: <200404232220.49246.Maarten.Wisse@urz.uni-heidelberg.de>
References: <200404232220.49246.Maarten.Wisse@urz.uni-heidelberg.de>
X-Mailer: VM 7.04 under 21.1 (patch 14) ``Cuyahoga Valley'' XEmacs Lucid
--text follows this line--

Maarten,

You owe me nothing! It is a pleasure for me to work with you, and that
is more than enough for me.

The pointer regarding Gentoo might become useful to me.  I hope to
find a few days in the coming months to set up a server running Linux
and I was wondering which distribution to use.  So from your
experience Gentoo seems to be a good choice.

I placed in the bug fixes page a pointer to the zipped file renamed
to tex4ht-20040211.zip, and will fix the name in the main page next
year when I'll revise the home page.

Thanks, -eitan



 > Given that you so gratiously helped me, I asked myself how I could do
 > something back and also, to help others with this great tool, especially
 > those who use my favourite Linux distribution, Gentoo Linux, you probably
 > know, the distro which is entirely managed from source with the intelligent
 > Portage system. I like it very much wont switch to another distro so easily
 > as I did with RedHat or SuSE. But, this is still a matter of fact:
 >
 > http://bugs.gentoo.org/show_bug.cgi?id=33259
 >
 > no ebuild, so no automated install of TeX4ht is available yet.
 >
 > I'm not a very able ebuild writer, but it shouldn't be difficult to get this\

 > going. However, one thing which immediately came to my mind: you don't
 > release the source distribution of TeX4ht with a name indicating its version\
.
 > That's inconvenient for Gentoo, because ideally it spreads the files on top
 > of the original source location, so it uses its mirror system and when that
 > doesn't contain the file, it falls back to the root source location. When
 > that location just contains `tex4ht.zip' then Gentoo has no way to know whic\
h
 > version that is. So could you rename it to: tex4ht-20040211.zip or the like?\

 > Then I can make an ebuild tex4ht-20040211.ebuild which then automatically
 > downloads that version. When an upgrade appears, we check the ebuild, bump
 > the version number and it uses that one. So we can cater for stable and
 > unstable ebuilds too.
 >
 > PS: as an expression of debt, I mentioned your name in the news item on our
 > website, called `New in volume 4'. You deserve it.

--------------------------------------------------------------------------------------
comp.text.tex #305453 (0 + 13 more)                                        [1]
From: Torsten Bronger <bronger@physik.rwth-aachen.de>
Newsgroups: comp.text.xml,comp.text.tex
[1] [ANN] tbook 1.5.2
Date: Sat Apr 17 12:56:34 EDT 2004
Lines: 43

Halloechen!

tbook 1.5.2 is available for download at
http://sourceforge.net/projects/tbookdtd/

Its homepage with documentation and a demo is available at
http://tbookdtd.sourceforge.net/


Version 1.5.2 is a major bugfix release by and large.  New features:

* Full XHTML compliance if you want (however with MathML you don't
  want it ;)

* New <blockquote> und <quote> elements in the DTD with tha same
  meaning as in DocBook and HTML.

* Many additionally supported Unicode symbols.


From the Sourceforge summary:

   tbook is a system that typesets XML documents with high-level LaTeX
   while HTML, XHTML+MathML and DocBook output are equally
   possible. It bases on the LaTeX-like tbook DTD developed for this
   project, XSLT transformations and further tools.

All transformations work with bibliography, index, formulae, and
graphics.  Therefore the transformation processes include some complex
tasks (e.g. generating of bitmaps for all formulae) which are greatly
simplified by shell scripts that are automatically generated.

The LaTeX output is also suitable for pdfLaTeX and uses its features.

For equations, you may use an almost-LaTeX syntax; except for complex
cases, you don't have to type MathML directly.  (Instead, it is
transformed to that eventually.)

Tschoe,
Torsten.

-------------------------------------------------------------------------------------
http://freshmeat.net/projects/flpsed/
--------------------------------------------------------------------------------------


Wlodek,

I suspect the problem is in the tex4ht.env within a segment 
similar to the following one that takes precedence over the values in
the variables of texmf.cnf

  <default>
  i~/tex4ht.dir/texmf/tex4ht/ht-fonts/iso8859/1/!
  i~/tex4ht.dir/texmf/tex4ht/ht-fonts/ascii/!
  i~/tex4ht.dir/texmf/tex4ht/ht-fonts/alias/!
  i~/tex4ht.dir/texmf/tex4ht/ht-fonts/mozilla/!
  i~/tex4ht.dir/texmf/tex4ht/ht-fonts/unicode/!
  </default>
  
Try removing the i-records or modifying their content.

-eitan


 > When compiling a simple (iso8859-2 encoded) LaTeX file which uses 
 > the Polish Computer Modern family of fonts (pl*.tfm fonts), the
 > tex4ht program searches and uses htf fonts (plr.htf) from 
 > iso8859/1 subdirectory instead from iso8859/2.
 > 
 > I am using tex4ht under Linux with kpathsea support
 > and texmf.cnf which contains:
 > 
 > TEX4HTFONTSET=iso8859/2,iso8859/1,alias
 > TEX4HTINPUTS=.,$TEXMF/tex4ht/ht-fonts/{$TEX4HTFONTSET}//
 > 
 > Below I include the output of tex4ht executed with -hF option:
 > 
 > --------------------------------------------
 > tex4ht.c (2004-02-05-19:30 kpathsea)
 > tex4ht -f/01 
 >   -hF 
 > (tex4ht.env)
 > given TEX4HTFONTSET = iso8859/2,iso8859/1,alias
 > setting TEX4HTFONTSET={iso8859/2,iso8859/1,alias}
 > TEX4HTINPUTS =
 > .,{texmf,../texmf,/home/hold/texmf,!!/usr/local/texmf-local,!!/usr/local/texmf-var,!!/usr/local/texmf}/tex4ht/ht-fonts/{iso8859/2,iso8859/1,alias}//
 > (/usr/local/texmf/fonts/tfm/public/pl/plsy10.tfm)
 > (/usr/local/texmf-local/tex4ht/ht-fonts/alias/pl/plsy.htf)
 ....
-------------------------------------------
By the way, it would be
>    convenient to have all TeX4ht documentation formated entirely
>    as _one_ web page. (I have no idea if I had read 10% or 90% available
>    documentation.)
-------------------------------------------
http://www.tmit.ac.jp/~esl/soft/t4hc-intro.html

Two-steps picture conversion
The default bitmap generation script for Unix-like environment is three-step procedure calling dvips, ghostscript, and ImageMagick sequentially. But the author prefers to two-step procedure calling dvipdf and ImageMagick, which is written as follows:

Ghtcopyfile %%1 _t4hcache.dvi
Gdvipdfm -s %%2-%%2 _t4hcache
Gc:/progra~1/picturem~1.8-q/convert.exe pdf:_t4hcache.pdf -crop 0x0 +repage -density 220x220 -geometry 50% -transparent '#FFFFFF'  %%3
Ghterasefile _t4hcache.dvi
Ghterasefile _t4hcache.pdf

Most of recent Web2C systems such as Kakuto's win32 package contain dvipdf (with the filename dvipdfm), and resultant PDF file is properly processed by ImageMagick.


-------------------------------------------
-------------------------------------------
-------------------------------------------

-------------------------------------------
-------------------------------------------
-------------------------------------------


As of (at least) miktex 2.4, miktex will automatically install tex4ht. The
problem is, as best I can see, it does not do the fairly extensive editing
of the batch files (and tex4ht.env) needed to tell tex4ht where the various
files are. I've just added a note to my instructions recommending that
miktex users uninstall miktex's version and use mine instead.

I just want to arert you to what can happen: if users don't do this, then
the locations pointed to by eg the -e switch in the batch files, though it
may *look* correct, in fact point to the wrong locations. So if you get a
request for help from a miktex user, this is the first thing you should
probably have them check.
-------------------------------------------

Mention the dates of the files, simlarly to the eay it is done 
in the bug fixes page.


-------------------------------------------
http://bugs.debian.org/cgi-bin/pkgreport.cgi?pkg=tex4ht

-------------------------------------------

http://www.perce.de/LaTeX/math/Mathmode-TeX.pdf
http://www.ntg.nl/doc/wilkins/pllong.pdf
http://sarovar.org/download.php/120/ltxprimer-1.0.pdf
http://www.tug.org/utilities/plain/cseq.html
http://www.tug.org.in/tutorials.html

-------------------------------------------

 support/TeX4ht/tex4ht-mn.zip
>
> which is probably what eitan was talking about.

I thought, this was only the manual. ``mn'' seems to mean ``manual''?AFAICS, it
is used in this meaning in the names of the HTML files in the archive.
Strange that the actual TeX4ht ZIP archive is included as a file in the mn
archive.

tex4ht-all in CTAN!!!!

-------------------------------------------
http://http.cs.berkeley.edu/~phelps/Multivalent/
-------------------------------------------

> there's also a tex4ht distribution embedded in the texlive
> distribution.  the difference from eitan's original is it's tailored
> to work with the kpathsea libraries; i'm investigating that as a route
> to a working linux version.  (i'll probably end up by installing the
> whole of texlive, but i don't want to rush it since there's quite a
> bit of configuration gone into the tetex setup i currently give my
> users.)

I think for kpathsea the main differences are that the compilation of
the C programs require a special switch:

   http://www.cse.ohio-state.edu/~gurari/TeX4ht/mn.html
   => Unix
   => (c)
   => alternative

and variable settings similar to the following ones in texmf.cnf

 TEX4HTINPUTS=.;$TEXMF/tex4ht/base//;$TEXMF/tex4ht/ht-fonts/{$TEX4HTFONTSET}//
 TEX4HTFONTSET=iso8859/1,alias

-------------------------------------------

OK, but how about putting a readme file in CTAN:support/TeX4ht telling
people to check out http://www.cse.ohio-state.edu/~gurari/TeX4ht/bugfixes.html


-------------------------------------------


  http://netpbm.sourceforge.net/
  http://sourceforge.net/projects/netpbm/
  (latest version: Netpbm-10.19 update at 11/15 2003)

-------------------------------------------
From zierke@dante.de Wed Nov  5 05:18:49 EST 2003
Article: 292426 of comp.text.tex
Path: news.cse.ohio-state.edu!news.maxwell.syr.edu!news-han1.dfn.de!news-ham1.dfn.de!news.uni-hamburg.de!dante.de!owner-ctan-ann
From: zierke@dante.de (Reinhard Zierke)
Newsgroups: comp.text.tex
Subject: CTAN Upload - BaKoMa TeX 6.10 - Animated GIF, Sound for SVG
Date: 5 Nov 2003 09:32:48 +0100
Organization: University of Hamburg -- Germany
Lines: 264
Approved: yes
Message-ID: <20031105083010.GA22304@dante.de>
NNTP-Posting-Host: sun.dante.de
Mime-Version: 1.0
Content-Type: text/plain; charset=us-ascii
X-Trace: rzsun03.rrz.uni-hamburg.de 1068021168 23174 134.100.9.52 (5 Nov 2003 08:32:48 GMT)
X-Complaints-To: usenet@news.uni-hamburg.de
NNTP-Posting-Date: 5 Nov 2003 08:32:48 GMT
To: ctan-ann@dante.de
Organisation: none
Xref: news.cse.ohio-state.edu comp.text.tex:292426

----- Forwarded message from "Basil K. Malyshev" -----
I have uploaded to `ftp.dante.de/incoming/bakoma' directory
update for BaKoMa TeX 6.10 (issued 1-Nov-2003).

Please put the files to 'CTAN:nonfree/systems/win32/bakoma' directory.

Most important improvements in BaKoMa TeX 6.10 are:

SVG Generation:
  * Import of animated GIF into SVG as animated image.
  * Import multi-page Postscript into SVG as animated
    graphics - every page is translated as frame.
  * Sound effects are supported by SVG.STY via \sound command.
  * More customization in Navigation Bar added at SVG generation: size, colors.
 
  More details about SVG Generation features are available in 
  <ftp://ftp.dante.de/tex-archive/systems/win32/bakoma/programs/svgwrite.html>

Graphics (Supported for View, Print, SVG, PDF, Postscript)
  * Support of transparent background in imported GIF files.
  * Support of Color Key Masked Images - Type 4 (Postscript Level 3)

BibEdit Improvements:
  * Refined disappering comment subwindow 
    when adding new fields and resizing window.
  * Added Rename command for changing entry key and field name.

More information about changes is at:
  <ftp://ftp.dante.de/tex-archive/systems/win32/bakoma/programs/changes5.html>

Bye.
Basil K. Malyshev
4-Nov-2003

PS: Please send announce to appropriate mailing lists and news groups.

--------------- About BaKoMa TeX 6.10 ----------------------



-------------------------------------------
From: The Thanh Han <hanthethanh@gmx.net>
To: gurari@cse.ohio-state.edu
Cc: Fabrice Popineau <Fabrice.Popineau@supelec.fr>
Subject: tex4ht question
Date: Fri, 8 Aug 2003 08:22:24 +0700

Hi,

can you please give me some hints to the following: I was trying to use
tex4ht to convert a latex file to html. The latex source is in
Vietnamese, and I would like to have the html output in Unicode. I
looked at the tex4ht files in texlive7 cd, and created a set of htf
fonts for vietnamese. It nearly works, apart from an only oddity: that
all glyphs with a dot below (like adotbelow, edotbelow, idotbelow, ...) is
converted to their ascii counterpart (a, e, i, ...). The latex file
itself works ok with latex; ie when I latexed it and viewed the dvi ouput
using xdvi, everything seems to be ok. I was digging a bit but couldn't
fix it, so if you have time, can you please take a look at it? Apart
from files on texlive7 cd, I used the following:

1) a htf font for vietnamese, which is attachted in this mail.

2) vietnamese tex support: vntex from http://vinux.sourceforge.net/vntex

The test file I used is included in vntex as well; it's in
doc/generic/vntex/vntest1.tex. I copied those tex files to a temporary
directory and run:

uhtlatex vntest1
-------------------------------------------
http://lists.w3.org/Archives/Public/www-math/2003Sep/0023.html
-------------------------------------------


\HCode{<OBJECT height="20\%" width="100\%" border="0"
         type="text/html"
         data="http://www.cse.ohio-state.edu/\string
               ~gurari/course/cis788/cis788.html"
      ><A class="navigation" 
        href="http://www.cse.ohio-state.edu/\string
                ~gurari/course/cis788/cis788.html" target="testwindow"
       >Test</A></OBJECT>
}


-------------------------------------------



CTAN: to promote latex, we need to promote standarization.  We can
have committees.  We can also promote certifications: in ctan put a
comment around each contribution on the level of compatibility the
contribution provides!! This will encourage contributures to produce
more fitting parts.  Also, had a discussion group for contributions,
for pre discussing contributions--i.e. cooling system for cross
contributions.


 > \begin{list}{\textbullet}{}
 > \item Bla bla bla...
 > \end{list}
 > 
 > \begin{itemize}
 > \item Bla bla bla...
 > \end{itemize}

Latex and pdf deal with visual layout of the lists. TeX4ht needs to
map the lists to html.  The default configurations map the first
`list' list into a `description' (dl) list of html, and the second
into an itemized one (ul).  I knowledge that the first list list
behaves like an itemized list is not at all clear without inspecting
the visual output.  


$$\begin{minipage}{\textwidth}
\small\itshape The first section \textbullet\ The socond one  \textbullet\
The third one {\upshape(A subsection in it, 1. Another, 2)} \textbullet\ A
fourth section {\upshape(A subsection in it, 1. Another, 2)}.
\end{minipage}$$

http://omega.enstb.org/eurotex2003/


> >A compilation with the command line option `mathml-'  provides a ``correct''
> >mathml that doesn't check for balanced parentheses.
>
>    Oh, is that the only difference? I was worried I'd be disabling some
> functionality in the output with that option.

MathML looses the knowledge of the parenthesis being delimiters, and
so the output code provides reduced amount of clues for postprocessing
(and displaying). Personally, when I think authors should avoid this
option, anf fix their source files. On the other hand, it is too much
of a burdun for third parties to do so.


\Css{@media print {div.crosslinks {visibility:hidden;}}}



\addcontentsline{toc}{slide}{#1}

Susan's work

Review participants & program

note: The index doesn't include math, figs fragments


http://www.activemath.org/~adrianf/dvi2svg/


http://www-2.cs.cmu.edu/~kohlhase/talks/om-tutorial/
http://www.dessci.com/en/reference/webmath/status/default.htm

OpenOffice for docbook: http://xml.openoffice.org/xmerge/docbook/


What makes latex difficult (What You Ask Is not What You Get):

   You ask for

      \begin{split}
      \end{split}

   end you get instead

      \begin{split}
         \begin{split}
         \end{split}
      \end

Similarly 

    \begin{document}
    \end{document}
    \begin

    \end{document}




In open office:

   --------
    ------
   --------
    ------
   --------
    
   \begin{center}
   ....

   ....
   \end{center}

is implemenmted as three paragraphs:

    <p>...</p>
    <p/>
    <p>...</p>




The importance of having standards (like amsmath) in latex, and having
them stables:   e.g.




Why latex is ugly: check the sample ams page!!!

Why mathml is ugly: check the specification web page!!!




-------------------------------------------

%%%%%%%%%%%%%%%%%%%%%%
\DocPart{Limitations}
%%%%%%%%%%%%%%%%%%%%%%


\List{*}
\item {Superscripts or Subscripts}

 Commands within scripts should have their parameters explicitly
 included within the scripts. For instance, use \`'a^{\sqrt b}'---not
 \`'a^\sqrt b'.

 [{\sl MathML}]:
 Empty bases in many contexts should be represented by a code similar
 to \`'{\csname HCode\endcsname{}}'. Otherwise, the the content
 preceding the empty bases will be assumed for the bases.

 Subscripts and superscripts within a preamble (that is, before
 \`'\begin{document}') should be accessed through the \''\sb' and
 \''\sp' commands instead of the \`'_' and \`'^'
 operators. Alternatively, in many cases the \`'early_' and
 \`'\early^' options might be used as an alternative to allow
 references to the latter operations. On the other hand, within loaded
 files the \`'\sb' and \`'\sp' commands should be employed.

\item {Parentheses}

 [{\sl MathML}]: Formulas without proper nesting of delimiters are
 considered harmful.  It is possible to relax this constraint on a
 global level through the \`'mathml-' command line option, but it is
 highly not recommended to use this feature.  Instead, localized
 approaches similar to the following may be employed to downgrade the
 meaning of delimiters.

\List{}
\item {} \''\left|...\right|'
\item {} \''\bigl\{...\bigm|...\bigr\}'
\item {} \''\begin{array} ...\mathord(... & ...\mathord)...\end{array}'
\EndList


\item{Lists}

\TeX4ht doesn't sense changes to default appearances of lists in
native \LaTeX.  To reflect in the hypertext output on  such changes, the
user should adjust the given configurations of \TeX4ht.

\EndList

-------------------------------------------


> >A compilation with the command line option `mathml-'  provides a ``correct''
> >mathml that doesn't check for balanced parentheses.
>
>    Oh, is that the only difference? I was worried I'd be disabling some
> functionality in the output with that option.

MathML looses the knowledge of the parenthesis being delimiters, and
so the output code provides reduced amount of clues for postprocessing
(and displaying). Personally, when I think authors should avoid this
option, anf fix their source files. On the other hand, it is too much
of a burdun for third parties to do so.


>    Is there anything wrong with doing what I did -- ie, downloading the
> tar.gz from the mathzilla page, and then refreshing the individual files for
> the install?

I think that is fine. Just need to be careful to consult the paths in
old the old versions when replacing the tex4ht.env and htlatex-like
files with new ones.


-------------------------------------------


>    Please forgive my ignorance, but how is one actually supposed to obtain
> and install a distribution, say on linux? The mathzilla page just had a
> tar.gz file to download, untar/gzip, and run localInstall.

The easiest way to install tex4ht would be by installing any
distribution that contains tex4ht, and then refresh the different
files

  1. tex4ht.sty
  2. *.4ht
  3. Executables of tex4ht.c and t4ht.c

and may be

  4. tex4ht.env
  5. *htf

I can guide you through this path if you choose to take it.

-------------------------------------------
http://www.texmacs.org/
http://makingtexwork.sourceforge.net/mtw/
-------------------------------------------
-------------------------------------------


 > 7 days trial version of Vector Eye for Windows has been released
 > due to Siame & Celinea. This is a quick converter of raster to vector graphi\
cs
 > giving, in particular, SVG output. THEORETICALLY it seems to create an
 > opportunity
 > for you to make the first experiments with svg graphic files included in
 > the xml document, as we dreamed about few months ago. I had a problem
 > with installing Vector Eye as it requires a high resolution screen. Furtherm\
ore
 > something was changed in Mozilla svg system and lost (I hope temporarily)
 > a chance to use Mozilla for svg viewing. Out of this the real capabilities
 > of this
 > software remain unknown for me.
 >
 > -Piotr
 >
 >
 > http://www.celinea.com/

-------------------------------------------

http://www.pianotype.net/
http://www.pianotype.net/eBook/index.html

- Latex
- WEB (Literate Programing)
- html, xhtml, xml : dtd DocBook and TEI
- xslt
- pdf, lit... &

Patrice :-)

Reply-To: ``Patrice GUERPILLON'' <pguerpil@club-internet.fr>
From: ``Patrice GUERPILLON'' <pguerpil@club-internet.fr>
[1] A new site (La)TeX / XML & C� from France
Date: Sun May 11 16:32:36 EDT 2003

-------------------------------------------


 > I'd also love to see a complete list of tex4ht parameters without having to
 > look through the .log file although there is lots of information there.

The options depend on the file styles being loaded, so a complete list
would be difficult to compile.  However, I agree it would be desirable
to have a list of major options.  I have none currently :-(


-------------------------------------------
converts troff macros to XML:
 http://catb.org/~esr/doclifter/
-------------------------------------------

\Verbatim

Refresh the filename database:  in the MikTeX options wizard, or
initexmf -u from the command line.

\EndVerbatim
-------------------------------------------
can be invoked with a command of the form \`'ht tex filename' (\HPage{example}
\rightline{\ExitHPage{}}
\SubSection{An Example for Invoking ht}
A system invocation of the form \`'ht tex foo' for a file
\''foo.tex' of the following form.

\Verbatim
\input amstex
\documentstyle{amsppt} 

\Preamble{xhtml,html4.4ht,unicode.4ht,mathml.4ht}
\EndPreamble

\document
.....
\enddocument
\EndVerbatim\EndHPage{}).
-------------------------------------------

> How do you feel about the described approaches?  My preference is for
> the first one.

Many thanks for this.  I tried Approach A and found that the width=75%
statement did not get carried through to to the html for the .pstex figures
from xfig.  It worked for the .eps figures.

I therefore moved over to Approach B.  I created a script called
'convertScale' (note: the grep picks up .pstex files also since there is a
BeginDocument:.*.eps line in all of them):

#-----------------------------------------------------------------------
#!/bin/sh
#

ISEPS=`grep -E ``BeginDocument:.*.eps'' zz${1}.ps`
if [ ``$ISEPS'' == " ] ; then
    SCALE=''100%''
else
    SCALE=''50%''
fi
convert zz.bmp -crop 0x0 +repage -transparent '#FFFFFF' -scale $SCALE  zz${1}.gif
#-----------------------------------------------------------------------

and then in the env file:
GconvertScale %%4

This worked well so I have adopted this approach.

-------------------------------------------
http://www.uoregon.edu/~koch/texshop/texshop.html
-------------------------------------------

converteters to SVG

http://www.w3.org/Graphics/SVG/SVG-Implementations.htm8
-------------------------------------------
http://www.bloodyeck.com/wwwis/

---------------
http://www.w3c.org/Style/XSL/



The web page

  http://dmoz.org/Computers/Data_Formats/
  Markup_Languages/XML/Style_Sheets/XSL/Implementations/

lists many processors.  (The two lines should be merged in
order to get the address.)

----



Tex4ht seems to strain TeX somewhat. To keep from getting messages
that say ``TeX capacity exceeded, sorry'' it may be necessary to
increase TeX's capacity by placing a configuration file texmf.cnf in
the document directory and setting the environment variable TEXMFCNF
to ``dir::'' where dir is the path to the document directory. This
only works for TeXs that use kpathsea.
-------------------------------------------
-------------------------------------------
Gc:\yandy\Dvipsone\dvipsone  -B=%%2 -E=%%2 -d=zz%%4.ps  %%1
Gc:\ImageMagick\convert zz%%4.ps -crop 0x0 +repage -density 110x110 -transparent #FFFFFF %%3

-------------------------------------------
comp.text.tex #250583 (0 + 4 more)                                         [1]
From: Ralph Furmaniak <sugaku@sympatico.ca>
[1] Announcing gotex
Lines: 35
Date: Tue May 21 17:50:29 EDT 2002

Gotex is a program that I made, which is basically a preprocessor and
(albeit small) set of macros for LaTeX, to give it extra functionality
for presenting files and web sites.  It translates the `got' file into a
`tex' file, which it then sends on to `tex4ht'.  A full description is
on my website at http://sugaku.homeunix.org

-------------------------------------------


Like all automatic HTML generators, Tex4ht does not always produce
clean (or even correct) HTML. HTML Tidy is an incredibly useful tool
for finding (and often, automatically fixing) problems in HTML
files. WWWis is a useful utility for making image size annotations in
web pages agree with actual image sizes (it's surprising how easy it
is to get these annotations wrong or to let them get out of
date). Finally, always look over your web pages using several
different browsers -- HTML generators seem to have a knack for
emitting constructs that render differently in different browsers.

-------------------------------------------

align=\string''absbottom\string''


 >           { class=\string''math\string''
 >            align=\string''absbottom\string''
 >            }
-------------------------------------------
-------------------------------------------
From: Patrice GUERPILLON <pguerpil@club-internet.fr>
Sender: pat@club-internet.fr
To: gurari@cse.ohio-state.edu
Subject: TeXht and dtd Tei/Docbook
Date: Sun, 16 Dec 2001 16:05:02 +0100

Dear M. Gurari,

i have computed my LaTeX document with TeXht to get Tei and Docbook
code.

Its OK. (perfect after some coorections of the code)

The result is present on my page :
http://www.club-internet.fr/perso/pguerpil

-------------------------------------------
Probably you can simply tell tex4ht to look in
  c:\progra~1\texmf
which is a non-spaced alias for
  c:\program files\texmf
-------------------------------------------
 > 2. note that if the batch files contain unix line endings (LF, rather than
 > dos's CR LF) you can crash win95/98/ME (NT and win2k are ok). There was a
 > period during the autumn when some of the batch files may have had this
 > problem. (Either that, or someone managed to copy the wrong files). Anyway,
 > this is worth watching out for, especially if someone reports an otherwise
 > incomprehensible failure.
-------------------------------------------
-------------------------------------------

> > Does any know how to use 'jpeg' images with LaTex?  Including the
> > environment setting and the command ....

This can be done on the fly with psgraphicx by using, eg
  \DeclareGraphicsRule{.jpg}{eps}{.jpg.bb}{`convert jpeg:#1 eps:-}

and creating a bounding box file called xxx.jpg.bb (if your image is
xxx.jpg) by using
  convert xxx.jpg eps:- | grep BoundingBox > xxx.jpg.bb

See the file `Using EPS Graphics in LaTeX2e Documents', available
from ctan in /tex-archive/info/ as epslatex.ps or epslatex.pdf.
------------------------------------------- 


>I'm using tetex on RH 7.3.  In the /usr/share/texmf/web2c/texmf.cnf file,
>I have:
>
>TEXMFMAIN = /usr/share/texmf
>TEXMFLOCAL = /usr/local/share/texmf
>HOMETEXMF = $HOME/texmf
>TEXMF = {$HOMETEXMF,$TEXMFLOCAL,!!$TEXMFMAIN}
>
>and it used to work fine with my extra style files, bib files, etc. stored in
>~/texmf.  But ever since one rainy day, something is wrong:
>

I've had success using:

 env KPATHSEA_DEBUG=-11 latex 

say, to debug issues with TEXMF and other path-related elements.  I found
that I'd stuck some things in non-intuitive places...

>> I'm using tetex on RH 7.3.  In the /usr/share/texmf/web2c/texmf.cnf file,
>> I have:
>>
>> TEXMFMAIN = /usr/share/texmf
>> TEXMFLOCAL = /usr/local/share/texmf
>> HOMETEXMF = $HOME/texmf
>> TEXMF = {$HOMETEXMF,$TEXMFLOCAL,!!$TEXMFMAIN}
>>
>> and it used to work fine with my extra style files, bib files, etc. stored in
>> ~/texmf.  But ever since one rainy day, something is wrong:
>>
>> nali$ kpsewhich --expand-path='$HOMETEXMF'
>> /home/nali/texmf
>> nali$ kpsewhich --expand-path='$TEXMF'
>> /usr/share/texmf
>>
>> I didn't do anything to the texmf.cnf file. What could be affecting
>> the value of TEXMF?

David> Perhaps the value of TEXMF?  What does
David> echo $TEXMF
David> display?

echo $TEXMF returns /usr/share/texmf.  I don't know why I defined it in
/etc/profile.  Removing that fixed the problem.  Thanks.


----


http://john.regehr.org/latex/:

Tex4ht seems to strain TeX somewhat. To keep from getting messages
that say ``TeX capacity exceeded, sorry'' it may be necessary to
increase TeX's capacity by placing a configuration file texmf.cnf 

  hash_extra = 25000
  pool_size = 750000
  string_vacancies = 45000
  max_strings = 55000
  pool_free = 47500
  save_size = 8000
  
in the document directory and setting the environment variable
TEXMFCNF to ``dir::'' where dir is the path to the document directory.
This only works for TeXs that use kpathsea.
This only works for TeXs that use kpathsea.
%--------------------------------------------------------------
 > The following approach of Sebastian allows to direct the kpathsea
 > search to a subdirectory of the ht-fonts tree.
 > 
 >      TEX4HTFONTSET={alias,iso8859}
 >      T4HTINPUTS = .;$TEXMF/tex4ht/base//;$TEXMF/tex4ht/ht-fonts/{$TEX4HTFONTSET}//
 >      TEX4HTINPUTS= .;$TEXMF/tex4ht/base//
 >     
 >     in texmf.cnf, and the htlatex script says
 >     
 >      test -z "$3" || TEX4HTFONTSET=$3;export TEX4HTFONTSET
 > 
 > On a second thought, the latter approach might not work for the
 > mozilla script. That is so because for mozilla we want the search to
 > be conducted under two subdirectories: first under ht-fonts/mozilla
 > and then under ht-fonts/unicode.
just
  TEX4HTFONTSET={mozilla,unicode}
surely?
-------------------------------------------
#!/bin/sh

TEXINPUTS=$XMLCDROM/texmf/tex4ht:
export TEXINPUTS

$XMLCDROM/bin/i386/tex4ht.exe $1 \
  -e$XMLCDROM/texmf/tex4ht/base/tex4ht.env \
  -i$XMLCDROM/texmf/tex4ht/ht-fonts/unicode/\!

Note the \  before the !
-------------------------------------------
OK, in texmf.cnf I say

TEX4HTFONTSET={alias,iso8859}
T4HTINPUTS = .;$TEXMF/tex4ht/base//;$TEXMF/tex4ht/ht-fonts/{$TEX4HTFONTS}//
TEX4HTINPUTS= .;$TEXMF/tex4ht/base//;$TEXMF/tex4ht/ht-fonts/{$TEX4HTFONTS}//

and then in eg htlatex I can just do

TEX4HTFONTSET=$3
export TEX4HTFONTSET
tex4ht $1
t4ht $1 $4 


make sense?

sebastian

Eitan Gurari writes:
OK,

Now I have

 TEX4HTFONTSET={alias,iso8859}
 T4HTINPUTS = .;$TEXMF/tex4ht/base//;$TEXMF/tex4ht/ht-fonts/{$TEX4HTFONTSET}//
 TEX4HTINPUTS= .;$TEXMF/tex4ht/base//

in texmf.cnf, and the htlatex script says

 test -z "$3" || TEX4HTFONTSET=$3;export TEX4HTFONTSET

and this seems to do the job

 > Would that guarantee searching the .../ht-fonts/iso8859/... branch
 > before the other brances, since alphabetically iso8859 is the first
 > one among the subdirectories?

kpathsea has no guarentee of search order. it may be alphebetical, but 
maybe not

 > If not, can we inform kpathsea about the directories
 > .../ht-fonts/iso8859/...  and .../ht-fonts/alias/..., without
 > reporting to it the other directories? 


yes, if it works properly, my default 

 TEX4HTFONTSET={alias,iso8859}

does the job

 > Did you sort out the problem with testa?  I can't recreate it and
 > wonder what happened there.
still testing....

sebastian
From: Sebastian Rahtz <sebastian.rahtz@computing-services.oxford.ac.uk>
To: gurari@cse.ohio-state.edu
Subject: Re: ht-fonts
Date: Tue, 14 Mar 2000 16:08:00 +0000 (GMT)

Eitan Gurari writes:
 > 
 > also the /! is needed.
 > 
 >  > htlatex filename "html,html.4ht,picmath4.4ht"    "symbol" 
 > 
 > htlatex filename "html,html4.4ht,picmath4.4ht"    "symbol/!" 

no. not needed! kpathsea takes care of that.

 > xhtml should be first:
 > 
 >  > htlatex filename "html4.4ht,unicode.4ht,mathml.4ht,xhtml" "unicode" 
 > 
 > htlatex filename "xhtml,html4.4ht,unicode.4ht,mathml.4ht" "unicode/!"
 >              ^^^^^^                                          ^^

ah, now it works fine, and i see mathml in the output.
excellent! now I can attempt to make these german Web Companion
examples work again

sebastian
-------------------------------------------
http://cauchy.math.missouri.edu/~stephen/cgi-bin/naturalmath.cgi
http://ftp.geophysik.uni-frankfurt.de/~ruedas/links-latex.html
http://math.albany.edu:8010/g/Math/MathComp/mathwww/
http://www.astro.gla.ac.uk/users/norman/bookmarks/lists-maths.html
http://www.dcs.fmph.uniba.sk/~emt/
http://www.math.berkeley.edu/~tinsel/mathunix/webmath.html
http://www.mathtools.net/LaTeX/Other_Links/index.html
http://www.openebook.org/
http://www.w3.org/Math/
http://www.webeq.com/mathml/resources.html
http://www.zib.de/Visual/software/doc++/
http://www.w3.org/Tools/
--------------------------------------------------------
 \Configure{tabbing}[1.6]{}{}{}{}
---------------------------------------
service in India -- River Valley Technologies, http://www.river-valley.com --
that creates ``a variety of outputs like screen pdf, regular PDF, dvi,
PostScript,
SGML, XML and MathML from a single TeX source document''.

------------------------------------------------------

From: plaice@cse.unsw.edu.au (John Plaice)
To: rey@ams.org (Ralph Youngen), mjd@ams.org (Michael Downes),
        ion@ihes.fr (Patrick Ion), sutor@us.ibm.com (Robert Sutor),
        aldiaz@us.ibm.com (Angel Diaz),
        s.rahtz@elsevier.co.uk (Sebastian Rahtz),
        gurari@cse.ohio-state.edu (Eitan Gurari)
Date: Thu, 19 Nov 1998 18:18:19 +1100 (EST)
Message-Id:  <981119071819.7482@cse.unsw.edu.au>
Subject: TeX to MathML converter
Cc: plaice@cse.unsw.edu.au (John Plaice)
X-Mailer: ELM [version 2.4 PL25]
MIME-Version: 1.0
Content-Transfer-Encoding: 7bit


Hello everyone,

The MathML translator is now finished, and it works wonderfully.
It will read a TeX or Omega document and transform the math part
into a MathML file (suffix .mml).  This is done by replacing the
mlist_to_hlist part of TeX with an SGML pretty-printer, along
with some nifty stuff along the way to recognize parenthesized
expressions and the like.

In addition to the driver work, Omega will now, upon loading a
font, look for a .onm file corresponding to the font (cmr.onm
for cmr10.tfm or cmr7.tfm, ecrm.onm for ecrm10.tfm, etc.).
This file contains a list of declarations such as
        \SGMLFontEntity{eufb}{"41}{\SGMLname{Afr}}{mi}{\SGMLbold}

which states that position "41 of Font sort eufb (Euler Fraktur Bold)
is entity "&Afr;".  If used in Math mode, it will produce
        <mi fontweight="bold"> &Afr; </mi>

If used in text mode, it will produce
        &Afr;

All of the standard math fonts have corresponding .onm files,
and most of the cm family of fonts do as well.  I can easily
generate more if necessary.

In addition, there are new Omega primitives, in addition
to \SGMLFontEntity, to produce tags and entities as desired
in the text.

\SGMLstarttag{tagname}  will start a new TeX level, and
will ultimately place <tagname> in the output file.

\SGMLendtag{tagname} will end a TeX level, and must match
a preceding \SGMLstarttag{tagname}.  It will place
</tagname> in the output file.

\SGMLattribute{attrname}{value} will place in the start tag
the text attrname="value".  It allows one to define extra
attributes for tags.

\SGMLemptytag{tagname}{attributes} will place an empty tag,
possibly with attributes, as in <tagname/>.  The attributes
are defined using \SGMLattribute.

The translator as it now stands seems to be able to do
all of normal LaTeX math, as well as AMS-LaTeX math.  It
will also do a fair amount of TeX stuff.  The most important
TeX thing that it cannot work with is \halign, which is
OK if one is doing LaTeX, since \halign is not normally
called in LaTeX.

Unlike LaTeXtoHTML, this translator is capable of interpreting
and taking advantage of TeX's macro language.  As a result,
it should be able to interpret a far greater proportion of
TeX files than can any Perl script.  In addition, the translator
is easily adapted:  if you do not like the interpretation,
then just change it.

I will place the appropriate files on my FTP site tomorrow.

Best,

John Plaice


%%%%%%%%%%%%%%%%%%%%%%%%%%%%%%%%%%%%%%%%%%%%%%%%%%%%%%%%%%%%%
%%%%%%%%%%%%%%%%%%%%%%%%%%%%%%%%%%%%%%%%%%%%%%%%%%%%%%%%%%%%%
/ENDSKIP ///////////////+/
%%%%%%%%%%%%%%%%%%%%%%%%%%%%%%%%%%%%%%%%%%%%%%%%%%%%%%%%%%%%%
% START
%%%%%%%%%%%%%%%%%%%%%%%%%%%%%%%%%%%%%%%%%%%%%%%%%%%%%%%%%%%%%
%%%%%%%%%%%%%%%%%%%%%%%%%%%%%%%%%%%%%%%%%%%%%%%%%%%%%%%%%%%%%

\def\temp{htm} \ifx\switch\temp  \def\htm{htm}\fi




%%%%%%%%%  
%%%%%%%%%%%%%%%%%%%%%%%%%%    \def\htm{htm}             
\ifx \ROOT \UnDef
   \immediate\write16{....................Undef ROOT--using temp/TeX4ht/}
   \def\ROOT{temp/TeX4ht/}   
\fi
\ifx \htm  \UnDef
    \def\htm{html}             
    \def\CLASS{class}
    \let\DOS=\empty
    \let\ARG=\empty
\else
    \def\CLASS{class}  %%%     \def\CLASS{cla}  fails
    \def\DOS{\noexpand\let\noexpand\DOS=Y}
    \def\ARG{[htm]}
\fi

% \def\unzip{\HCode{<sup>}[\UNZIP]\HCode{</sup>}}
\def\UNZIP{%\Link[http://www.info-zip.org/pub/infozip/]{}{}
           unzip%
          %\EndLink
}



\openin15=tex4ht.exe \ifeof15
     \immediate\write16{=============  tex4ht.exe missing=========}
\fi  \closein15
\openin15=t4ht.exe \ifeof15
     \immediate\write16{=============  t4ht.exe missing===========}
\fi  \closein15



\immediate\write16{....  ftp  /tex-archive/support/TeX4ht ... anonymous .. gurari@cse.ohio-state.edu}
\immediate\write16{....NEEDS        11. wfiles.zip}
\immediate\write16{....NEEDS        12. bcc32 -BCC32 tex4ht.c}
\immediate\write16{....NEEDS        13. bcc32 -BCC32 t4ht.c}
\immediate\write16{...KEEP dosports}
\immediate\write16{...zip -r foo *.*}


\input DraTex.sty
\input AlDraTex.sty
\input  tex4ht.sty





\def\GOBBLE#1{}
\ifx  \SysNeeds\GOBBLE
   \Configure{DOCTYPE}{\Tg<?xml version="1.0"?> 
   \Tg<!DOCTYPE html PUBLIC "-//W3C//DTD XHTML 1.0 
          transitional//EN" 
    "xhtml.dtd">}
\else
   \def\SysNeeds{\Needs}
\fi

\Preamble{xhtml,th4,index}
\EndPreamble


\input ProTex.sty
\AlProTex{llo,<<<>>>,?}

\def\Template{\IgnorePar\EndP\bgroup \tt \HCode{<div class="Template">}
   \def\cr{\HCode{<br />}\hfil\break}
   }
\def\EndTemplate{\IgnorePar\EndP\HCode{</div>}\egroup}



\ScriptCommand{\LikeVerbatim}{\NoFonts}{\EndNoFonts}

\def\BULLET{$\bullet$}

%%%%%%%%%%%%%%%%%%%%%%%%%% from latex %%%%%%%%%%%%%%%%%%%%%%%%%
%%%%%%%%%%%%%%%%%%%%%%%%%%%%%%%%%%%%%%%%%%%%%%%%%%%%%%%%%%%%%%%
\let\'=\Verb
\def\`{\expandafter\expandafter\expandafter\qts\Verb}
\def\qts#1{`#1\aftergroup'}



\def\LaTeX{La\TeX}
\def\TeX{TeX}

\def\INDEX#1#2{\--\expandafter\EAT\string#2//{{\tt\char 92}}/#1#2}

\def\IDX--#1/#2/#3/#4/{{%
  \Configure{--}
      {}
      {\string\csname\space :gobble\string\endcsname{\csname
                                            html:lbl\endcsname}%
        \string\Link[#4]{}{}\folio\string\EndLink  }%
   \--#1/#2/#3/}}


% Examples:
%         \--Fonts/htf\string\empty//%
%        \--Fonts/htf\relax,\string~configuring//%



\def\EAT#1{}

\NewSection\TroubleSec{}{}
\Configure{TroubleSec}
  {} {} 
  {\par \IgnorePar\EndP\HCode{<p align="right" class="rightline">}\bgroup\bf}
  {\egroup\HCode{</p>}\ShowPar\IgnoreIndent\par}
\ConfigureToc{TroubleSec} 
  {} {$\bullet$\HChar{160}}  {}  { }






\NewSection\DocChapter{}{}
\Configure{DocChapter}
  {\IgnorePar} {} 
  {\IgnorePar\EndP\HCode{<h2 class="ChapterHead">}}
  {\HCode{</h2>}\ShowPar\IgnoreIndent}
\ConfigureToc{DocChapter} 
  {} {\HCode{<span class="DocChapter">}$\bullet$\HChar{160}}  
  {}  {\HCode{</span>} }


\HAssign\OddDoc=1

\NewSection\DocPart{}{}
\Configure{DocPart}
  {\IgnorePar} {}
  {\IgnorePar\EndP\HCode{<h2 class="ChapterHead">}}
  {\HCode{</h2>}\ShowPar\IgnoreIndent}
\ConfigureToc{DocPart} 
     {} {$\bullet$\HChar{160}}  {}  { }

\TocAt{DocPart,DocChapter}

\NewSection\DocSection{}{}
\Configure{DocSection}
  {\IgnorePar} {} 
  {\EndP\HCode{<h2 class="SectionHead">}} {\HCode{</h2>}\IgnorePar}
\ConfigureToc{DocSection} 
     {} {$\bullet$\HChar{160}}  {}  { }


\NewSection\DocSubSection{}{}
\Configure{DocSubSection}
  {\IgnorePar} {} 
  {\EndP\HCode{<h3 class="SectionHead">}} {\HCode{</h3>}\IgnorePar}





\NewSection\InstallSection{}{}
\Configure{InstallSection}
  {\IgnorePar} {} 
  {\EndP\HCode{<h3 class="SubSectionHead">}} {\HCode{</h3>}\IgnorePar}
\ConfigureToc{InstallSection} 
     {} {$\bullet$\HChar{160}}  {}  { }






\NewSection\QAChapter{}{}
\Configure{QAChapter}
  {\IgnorePar} {} 
  {\IgnorePar\EndP\HCode{<h3 class="QAChapterHead">}}
  {\HCode{</h3>}\ShowPar\IgnoreIndent}
\ConfigureToc{QAChapter} 
  {} {\HCode{<span class="QAChapter">}}
  {} {\HCode{</span><br />} }

\Css{span.QAChapter{font-size: 125\%;}}                  

\NewSection\QASection{}{}
\Configure{QASection}
  {\IgnorePar} {} 
  {\IgnorePar\EndP\HCode{<h4 class="QASectionHead">}}
  {\HCode{</h4>}\ShowPar\IgnoreIndent}
\ConfigureToc{QASection} 
  {} {\HCode{<span class="QASection">}$\bullet$\HChar{160}}  
  {} {\HCode{</span><br />} }









\def\Example{\IgnorePar\EndP\IgnoreIndent 
  \HCode{<div class="EXAMPL">}{\bf Example}}
\def\EndExample{\IgnorePar\EndP\HCode{</div>}}


\def\Notes{\IgnorePar\EndP\IgnoreIndent 
  \HCode{<div class="Notes">}{\bf Notes}}
\def\EndNotes{\IgnorePar\EndP\HCode{</div>}}



\def\Sign#1#2{\aSign#1\bSign#2\cSign}


\NewConfigure{Sign}[3]{\def\aSign{#1}\def\bSign{#2}\def\cSign{#3}}



% 
% \Configure{buttonList}
%    {\IgnorePar\EndP\HCode{<table class="button-list">}}  
%    {\HCode{</table>}\par\ShowPar}
%    {\HCode{<tr valign="top" class="button-list"><td class="button-list">}}
%    {\HCode{</td><td\Hnewline
%           class="button-list">}#1\HCode{</td></tr>}}  
%    {\ListCounter}


\Configure{buttonList}
   {\IgnorePar\par\EndP\HCode{<table class="button-list">}}  
   {\HCode{</table>}\par\ShowPar}
   {\HCode{<tr valign="top" class="button-list"><td class="button-list">}}
   {\HCode{</td><td\Hnewline class="button-list">}\ShowPar
        #1\IgnorePar\EndP\HCode{</td></tr>}}  
   {\ListCounter}



\def\ResourceList{\begingroup
%
  \def\ListSepr{\gdef\ListSepr{, }}
  \Configure{buttonList}
   {\IgnorePar\HCode{<div class="converters">}}  
   {\HCode{</div>}\IgnorePar}
   {\long\def\author####1(####2){}\ListSepr
     ##1\IgnorePar\HCode{<sup>}\let\author=\space}
   {\HCode{</sup>}}
   {\ListCounter}
%
   \List{button}}

\def\EndResourceList{\EndList \endgroup}

\def\ListSepr{\gdef\ListSepr{, }}


\def\setup#1{\rightline{{\bf #1}}}


\long\def\Warning#1{\IgnorePar\Tg<div class="warning">{\IgnoreIndent
    \par\sl#1}\Tg</div>}

\def\frac#1#2{{#1\over#2}}
\def\temp#1#2{\hbox{\Picture*{ 
   align="middle"}$#1\over#2$\EndPicture}}
\HLet\frac=\temp

\def\EXAMPLE{\HTable/[/class="output"]\HCode{<div class="output">}}
\def\ContEXAMPLE{\HCode{</div>}\&\HCode{<div class="source">}}
\def\EndEXAMPLE{\HCode{</div>}\EndHTable}

\ScriptCommand{\Same}{\NoFonts}{\EndNoFonts} 

\def\sl{\it}   

%%%%%%%%%%%%%%%%%%%%%%%%%%%%%%%%%%%%%%%%%%%%%%%%%%%%%%%%%%%%%%%%%%

\Css{h1, h2, h3 { font-family: comic; }} %  sans MS
\Css{body { background: white; }}

\Configure{Sign}
  {\IgnorePar\EndP\HCode{<div class="SIGN">}\bgroup \it}
  {\HCode{<br />}}
  {\egroup\HCode{</div>}}
\Css{.SIGN{text-align:right;}}

\Css{.Template { margin-left:3em; }}

\Configure{ShowCode}
   {\IgnorePar\EndP\HCode{<div class="ShowCode">}}  
   {\HCode{</div>}}
   {\HCode{<br />\string&nbsp;}}
   {\HCode{<i>}}
   {\HCode{</i>}}
   {\HCode{\string&nbsp;}}

\def\.#1{\def\temp##1#1{\HCode{<span 
   class="showcode">}##1\HCode{</span>}}\temp}

\Css{div.ShowCode{
   font-family:monospace;
   white-space:nowrap; 
}}
\Css{span.showcode { color: green; }}


\Css{div.TableOfContents {  margin-top: 1em;
         margin-right: 1em; margin-left:1em;}}
\Css{div.Warning{ margin-right: 8\%; margin-left:8\%; 
   text-align:justify; }}

\Css{div.Warning div.Warning{ font-weight: bold;
   text-align:justify; border:solid 1px;  color:red;
}}


\Css{.tabular, div.output  { background-color:\#FFFFCC;  }}
\Css{div.output  { margin-left:3\%;  }}

\Css{.button-list P{
%%%%%%  margin-left:2em; text-indent:-2em;   IE is totally
%%%%%     broken here. It  stalls hypertext links, and
%%%%%     quits upon narrowing the window.
    margin-top:0em;  margin-bottom:0;  }}
\Css{div.button-list { margin-left:3\%; margin-right:3\%; }}


\Css{div.ShowCode {
   margin-left:3\%; margin-right:3\%;
   border-top: solid \#CCCCCC 1pt;
   border-bottom: solid \#CCCCCC 1pt;
}}

\Css{div.source {
   border-top: solid \#CCCCCC 1pt;
   border-bottom: solid \#CCCCCC 1pt;}}

% \def\EndDocChapter{%
%       \par\ShowPar\IgnoreIndent[\Link[\UpFilename]{}{}up\EndLink]\EndHPage{}%
%       \let\DocPart=\oldDocPart
%       \TableOfContents[DocChapter]}
% 
% \let\oldDocChapter=\DocChapter
% \def\DocChapter{%
%    \let\oldDocPart=\DocPart
%    \edef\UpFilename{\FileName}%
%    \HPage{}%
%    [\Link[\UpFilename]{}{}up\EndLink]
%    \Part{Features of \TeX4ht}%
%    \TableOfContents[DocChapter]     
%    \def\DocPart{%
%       \EndDocChapter
%       \DocPart}
%    \let\DocChapter=\oldDocChapter
%    \DocChapter}




\def\BR{\HCode{<br />}}

\gdef\SysVar#1{\Link{x-#1}{#1}#1\EndLink}
%%%%%%%%%%%%%%%%%%%%%%%%%%%%%%%%%%%%%%%%%%%%%%%%%%%%%%%%%%%%%%%%%%
%         DIRECTORIES
%%%%%%%%%%%%%%%%%%%%%%%%%%%%%%%%%%%%%%%%%%%%%%%%%%%%%%%%%%%%%%%%%%

\def\GURARI{http://www.cse.ohio-state.edu/\string~gurari/}
\def\GOLD{/n/gold/5/gurari/}

\def\WWW{\GURARI \ROOT}
\def\DIR{\GOLD WWW/\ROOT}
\def\TempDIR{\DIR temp/}

\def\HOME{\GURARI TeX4ht/}   

\def\SYSTEMS{\GURARI systems.html}  % ProTex AlProTex

\def\tugHOME{http://www.tug.org/applications/tex4ht/}







%\def\shareFTP{\WWW share/}       
%     \def\shareDIR{\DIR share/}
%     \def\shareTempDIR{\TempDIR share/}
%     \def\shareHOME{\HOME share/}
% \def\dosFTP{\WWW dos/}           
%      \def\dosDIR{\DIR dos/}
%     \def\dosHOME{\HOME dos/}    
% \def\winFTP{\WWW win95/}         
%     \def\winDIR{\DIR win32/}
%     \def\winTempDIR{\TempDIR win32/}
%     \def\winHOME{\HOME win32/}  

%\def\temp{TeX4ht} \ifx\switch\temp \else
%     \def\shareFTP{share/}
%     \def\shareHOME{share/}
%     \def\dosFTP{dos/}
%     \def\dosHOME{dos/}
%     \def\winFTP{win32/}
%     \def\winHOME{win32/}
% \fi


\def\WWWxml{http://www.tug.org/applications/tex4ht/mml/}
\def\dosports{\GURARI dosports/readme.html}






                        % can't clean the full directory because
                        % the new html stuff is already there
  %%%%%%%%%%%%%%% FIRST      %%%%%%%%%%%%%%%%%%%%%
\csname SysNeeds\endcsname{"rm ht-unix/*"}
\csname SysNeeds\endcsname{"rm ht-win32/*"}
\csname SysNeeds\endcsname{"chmod 711 \DIR"}
%foo \csname SysNeeds\endcsname{"chmod 711 \TempDIR"}
                                    %%%%%%%%%%%%%%%%%%%%%

%foo \csname SysNeeds\endcsname{"mkdir \TempDIR tex4ht"}   
%foo \csname SysNeeds\endcsname{"chmod 755 \TempDIR tex4ht"}
%foo \csname SysNeeds\endcsname{"mkdir \TempDIR tex4ht/temp"}   
%foo \csname SysNeeds\endcsname{"chmod 755 \TempDIR tex4ht/temp"}
%foo \csname SysNeeds\endcsname{"mkdir \TempDIR tex4ht/src"}   
%foo \csname SysNeeds\endcsname{"chmod 755 \TempDIR tex4ht/src"}
%foo \csname SysNeeds\endcsname{"mkdir \TempDIR tex4ht/bin"}   
%foo \csname SysNeeds\endcsname{"mkdir \TempDIR tex4ht/bin/unix"}   
%foo \csname SysNeeds\endcsname{"mkdir \TempDIR tex4ht/bin/win32"}   
%foo \csname SysNeeds\endcsname{"chmod 755 \TempDIR tex4ht/bin"}
%foo \csname SysNeeds\endcsname{"chmod 755 \TempDIR tex4ht/bin/unix"}
%foo \csname SysNeeds\endcsname{"chmod 755 \TempDIR tex4ht/bin/win32"}
%foo \csname SysNeeds\endcsname{"mkdir \TempDIR tex4ht/texmf"}   
%foo \csname SysNeeds\endcsname{"mkdir \TempDIR tex4ht/texmf/tex"}   
%foo \csname SysNeeds\endcsname{"mkdir \TempDIR tex4ht/texmf/tex/generic"}   
%foo \csname SysNeeds\endcsname{"mkdir \TempDIR tex4ht/texmf/tex/generic/tex4ht"}   
%foo \csname SysNeeds\endcsname{"mkdir \TempDIR tex4ht/texmf/tex4ht"}   
%foo \csname SysNeeds\endcsname{"mkdir \TempDIR tex4ht/texmf/tex4ht/ht-fonts"}   
%foo \csname SysNeeds\endcsname{"mkdir \TempDIR tex4ht/texmf/tex4ht/base"}   
%foo \csname SysNeeds\endcsname{"mkdir \TempDIR tex4ht/texmf/tex4ht/base/bin"}   
%foo \csname SysNeeds\endcsname{"chmod 755 \TempDIR tex4ht/texmf"}   
%foo \csname SysNeeds\endcsname{"chmod 755 \TempDIR tex4ht/texmf/tex"}   
%foo \csname SysNeeds\endcsname{"chmod 755 \TempDIR tex4ht/texmf/tex/generic"}   
%foo \csname SysNeeds\endcsname{"chmod 755 \TempDIR tex4ht/texmf/tex/generic/tex4ht"}   
%foo \csname SysNeeds\endcsname{"chmod 755 \TempDIR tex4ht/texmf/tex4ht"}   
%foo \csname SysNeeds\endcsname{"chmod 755 \TempDIR tex4ht/texmf/tex4ht/ht-fonts"}
%foo \csname SysNeeds\endcsname{"chmod 755 \TempDIR tex4ht/texmf/tex4ht/base"}
%foo \csname SysNeeds\endcsname{"chmod 755 \TempDIR tex4ht/texmf/tex4ht/base/bin"}

%foo \csname SysNeeds\endcsname{"mkdir \TempDIR tex4ht/texmf/tex4ht/base/unix"}   
%foo \csname SysNeeds\endcsname{"mkdir \TempDIR tex4ht/texmf/tex4ht/base/win32"}   
%foo \csname SysNeeds\endcsname{"chmod 755 \TempDIR tex4ht/texmf/tex4ht/base/unix"}
%foo \csname SysNeeds\endcsname{"chmod 755 \TempDIR tex4ht/texmf/tex4ht/base/win32"}


%%%%%%%%%%%%%%%%%%%%%%%%%%%%%%%%%%%%%%%%%%%%%%%%%%%%%%%%%%%%%%%%%%
%         END DIRECTORIES
%%%%%%%%%%%%%%%%%%%%%%%%%%%%%%%%%%%%%%%%%%%%%%%%%%%%%%%%%%%%%%%%%%

%%%%%%%%%%%%%%%%%%%%%%%%%%%%%%%%%%%%%%%%%%%%%%%%%%%%




\csname UnderRevision\endcsname

\Part{TeX4ht: LaTeX and TeX for Hypertext}


\TeX4ht is a highly configurable \TeX-based authoring system 
dedicated mainly to 
produce hypertext. It interacts with \TeX-based applications through
style files and postprocessors, leaving the processing of the source
files to the native \TeX{} compiler. Consequently, \TeX4ht can handle
the features of \TeX-based systems in general, and of the \LaTeX{} and
AMS style files in particular.

Pre-tailored configurations are offered for different
\Link[\jobname-commands.html]{}{}output formats\EndLink{}, including
(X)HTML, MathML, OpenDocument, and DocBook.  This document and the utility code
are available for downloading in \Link[tex4ht-all.zip]{}{}zipped
format\EndLink{}.

% (alternative sites: \Link[\WWW
% tex4ht-all.zip]{}{}osu\EndLink \tugOnOff{, \Link[\tugHOME
%  tex4ht-all.zip]{}{}tug\EndLink}).

 \TableOfContents[DocPart]

\HPage<toc>{}
%%%%%%%%%%%%%%%%%%%%%%%%%%%%%%%%%%%%%%%%%%%%%%%%%%%%%%%%%%%%%%%%%%%%%
\DocPart{Table of Contents}
%%%%%%%%%%%%%%%%%%%%%%%%%%%%%%%%%%%%%%%%%%%%%%%%%%%%%%%%%%%%%%%%%%%%%
\bgroup

\ConfigureToc{DocPart} 
  {} {\HCode{<div class="DocPart" align="center">}}  
  {} {\HCode{</div>} }

\TableOfContents[DocPart,DocChapter,DocSection,DocSubSection,InstallSection]


\Css{div.TableOfContents div.DocPart{margin-top:1em; margin-bottom:1em;
        font-size:110\%;}}


% \TableOfContents[SubSection]

% \TableOfContents[TroubleSec]

% \TableOfContents[QAChapter,QASection]

\egroup
\EndHPage{toc}







%%%%%%%%%%%%%%%%%%%%%%%%%%%%%%%%%%%%%%%%%%%%%%%%%%%%%%%%%%%%%%%%%%%%%%%%%%%%%%%%

\Css{.bugfixes{background: yellow;
%                padding:0.5em;
%               margin-left:70\%;
%              text-align:center;
 font-style: italic; 
font-weight: bold;
}}



 \HPage{trouble shooting}\ExitHPage{}

\Link{}{trbl-sht}\EndLink



\InstallSection{Trouble Shooting}

\TableOfContents [TroubleSec]

\TroubleSec{Bitmaps}

\let\svTroubleSec=\TroubleSec
\def\TroubleSec#1{\EndList \svTroubleSec{#1}
   \List{}
   \let\SV=\item
   \def\item##1{\SV{\Tg<span class="underline">##1\Tg</span>}}%
   }

\List{}
\let\SV=\item
\def\item#1{\SV{\Tg<span class="underline">#1\Tg</span>}}
\item{Too small math fonts in images}

Use instructions of the form

\centerline{\tt \string\DeclareMathSizes
   \string{{\it surrounding text size}\string}
   \string{{\it base math}\string}
   \string{{\it subscript math}\string}
   \string{{\it 2nd order subscript math}\string}}

to request math font dimensions for formulas embedded within text of
specified font dimensions.  Use sizes of magnitude $10*(1.2)^i$.

{\bf Example:}

\Verbatim
\documentclass{article}
   \DeclareMathSizes{10}{24.88}{20.74}{17.28}
\begin{document}
   Test in 10pt. \( {base 24.88}_{script 20.74} \)
\end{document}
\EndVerbatim

\item{Bad Quality of Pictures}

Increase the  density (number of dots per inch) when converting the
pictures, and then sub-sample the picture. Specifically, 
replace   in \`'tex4ht.env' the switch

        \centerline{-density 110x100}

\noindent   with the switch

        \centerline{-density 220x220 -geometry 50\%}

\noindent or with another switch of the form

        \centerline{-density ...x... -geometry ...\%}

The  \''-density' switch increases the number of pixels  per inch, 
and the \''-geometry' option reduces that number. The process
 smoothes (anti-aliases) the edges.
 Typically screens display about 72 to 100 dots per inch.

The type of fonts in use may also affect the quality of the output.
In particular,
\Link[http://www.ams.org/index/tex/type1-cm-fonts.html]{}{}Type
1\EndLink{} (scalable outlines) fonts offer better outcome than Type 3
(bitmapped) fonts.

\item{Direct Translations of EPS Figures}

The quality of the bitmaps of EPS figures may be improved by
converting the figures directly, without transmitting them through the
dvi code.  For instance, the \''\includegraphics' command in the
presence of the following configuration provides such a route.

\Verbatim
       \Configure{graphics*}
         {eps}
         {\Needs{"convert \csname Gin@base\endcsname.eps
                               \csname Gin@base\endcsname.png"}%
          \Picture[pict]{\csname Gin@base\endcsname.png}%
         }
\EndVerbatim

\--Bitmaps and graphics/eps//%
The following script provides a more efficient and general approach.

\Verbatim
       \Configure{graphics*}
         {eps}
         {\openin15=\csname Gin@base\endcsname.\PictExt\relax
          \ifeof15
             \Needs{"convert \csname Gin@base\endcsname.eps
                             \csname Gin@base\endcsname.\PictExt"}%
          \fi
          \closein15
          \Picture[pict]{\csname Gin@base\endcsname.\PictExt}%
         }
\EndVerbatim


\item{Truncated and empty png files}

Such a behavior might result from pictures which end up off the
dvi page limits.
A  larger paper size may be requested from dvips
 through the switch `-T offset' (e.g., -T 14in,14in).
The dvips command is activated from tex4ht.env.


\item{Truncated and empty png files for Xfig pictures}
  
  Some Xfig files are made up of overlapping picture environments,
  with \TeX4ht viewing the components as defining independent figures.
  The problem can be solved by importing the Xfig files into 
  pictorial environments of \TeX4ht.
 
\Verbatim
   \newenvironment{mypic}{\Picture*{}}{\EndPicture}
   \begin{mypic} \input{xfig-file}   \end{mypic}
\EndVerbatim 

Alternatively, 
%\Link{pdf-based}{}
PDF-based
%\EndLink{}
 translations into bitmaps might also 
offer a solution.

% > xfig to draw, 
% > fig2dev to convert 
% >   - text to TeX picture
% >   - graphics to eps
% > This way we have full TeX fonts/features in all our pictures.
\TroubleSec{\LaTeX}


\item{Unable to find a newly installed file}

The \TeX{} engine might require an update of a search directory: the
ls-R database for installations employing kpathsea (run texhash), the
data base directory in the case of MiK\TeX{} (select \'Start -> Programs -> MiKTeX -> Refresh', or run \`'initexmf -u' from
a DOS session, to update it).



  \TroubleSec{Fonts}

\item{Can't find/open file foo.tfm}

\List{*}
\item \--Fonts/tfm//%
Locate the directory where \LaTeX/\TeX{} finds font foo.tfm,
and add to \''tex4ht.env' a \`'t' record pointing to that directory
(e.g., \`'tc:\localtexmf\fonts\tfm!').
\item
Insert the record at the start of the line. \TeX4ht considers the first
character in each line to be a code describing the type of the record in
hand.  Lines which start with unrecognized character codes, spaces
included, are  ignored.
\item
Note also that \LaTeX/\TeX{} may create fonts on the fly and put them
in a temporary directory (e.g., \`'/var/tmp/texfonts/tfm/').
\EndList

\item{Linux, Netscape, and the SYMBOL font}


 To display the  SYMBOL fonts in Netscape  on  Linux
add

\centerline{\tt Netscape*documentFonts.charset*adobe-fontspecific:   iso-8859-1}

\noindent to the \''~/.Xdefaults' file

%%%%%%%%%%%%%%%%%%%%%%%%%%%%%%%%%%%%%%%%%%%%
\TroubleSec{C programs}

\item{\--tex4ht.c/Compiling//tex4ht.c doesn't compile ... ERROR:3396: `DIR'
undeclared (first use this function)... }

Consider adding the switch \`'-DHAVE_DIRENT_H' to the command line.
For instance,

\'+gcc -o tex4ht tex4ht.c -DENVFILE='"path/tex4ht.dir/texmf/tex4ht/base/unix/tex4ht.env"'
 -DHAVE_DIRENT_H+







%%%%%%%%%%%%%%%%%%%%%%%%%%%%%%%%%%%%%%%%%%%%
\TroubleSec{Scripts}

\item{Bad end-of-line characters in htlatex.bat/httex.bat/htexi.bat/ht.bat}


Remove undesirable trailing characters in the 
lines of the scripts, introduced by the utilities which download the files.

DOS/WINDOW platforms use an endline pair of characters: a carriage
return and a line feed  (0Dx,0Ax hexadecimal; 13,10
decimal). MAC platforms use only a single carriage feed character
(0Dx; 13dec).  UNIX platforms use only a single line feed character
(0Ax; 10 dec).

\item{Can't find/open file `xxx.dvi' or  `xxx.lg'}
  
  In some platforms, the operating systems pass on the quotes of the
  parameters of \''htlatex', \''httex', and \''httexi' to the
  utilities \''tex4ht' and \''t4ht'. In such cases, the utilities issue
  complaints of the above nature for file names \`'xxx' other than
  those being compiled.  The problem can be resolved by 
\HPage{installing a filter}
%%%%
\ExitHPage{}

\SubSection{htcmd}


\--htcmd.c///%
The source is available at 
\''bin/temp/htcmd.c', and executable for MS Win 95/98/NT is
available at \''bin/win32/htcmd.exe'.



%%%%%
\EndHPage{}  named \`'htcmd',
and submitting
  the \''tex4ht' and \''t4ht'  commands to the filter. Backslash
  characters \`'\' might need replacements with double backslash
  characters \`'\\'  or forward slash characters \`'/'.

For instance, if \''htlatex.bat' contains a command line of the form
 \`'C:\tex4ht\t4ht %1 %4', then after introducing the filter the 
modified command line will take the form
 \`'C:\\tex4ht\\htcmd C:\\tex4ht\\t4ht %1 %4'.



%%%%%%%%%%%%%%%%%%%%%%%%%%%%%%%%%%%%%%%%%%%%
   \TroubleSec{Environment File}
\item
{Can't find/open file `tex4ht.env'}

\--tex4ht.env///%
The switch  \''-hV' on the calling \''tex4ht' command shows where
the file is being searched. The following are possible solutions to
the problem.  
\List{*}
\item Set an environment variable TEX4HTENV to the address of the file
\item Add the switch \`'-e...address-of-tex4ht.env...' to the command 
lines of t4ht
 and tex4ht within the  htlatex, httex, 
httexi, and ht (or  htlatex.bat, httex.bat,
httexi.bat,
 and ht.bat) scripts.
\item If tex4ht and t4ht are compiled for kpathsea check that the
texmf.cnf contains records similar to the following ones.

\Verbatim
  TEX4HTFONTSET=alias,iso8859
  TEX4HTINPUTS=.;$TEXMF/tex4ht/base//;$TEXMF/tex4ht/ht-fonts/{$TEX4HTFONTSET}//
  T4HTINPUTS=.;$TEXMF/tex4ht/base//
\EndVerbatim

\item Make the location of
the environment file known to the programs in another manner
(see the pointers from the entry
 \`'tex4ht.env' in the \Link{Index}{}index\EndLink).
\EndList

\item{No permission for system call: ...}

 Make sure that the \`'S' records in \''tex4ht.env' don't
end with invisible spaces, and that the file ends
with the record \`'% end of
file'.

\item{Problems with argument {\tt -d...} of t4ht}

The specified directory must be augmented by a slash character \''/'.




\item{Problems with the convert utility}

\List{*}
\item Make sure the convert program of ImageMagick is called and
not another convert system in your directories.  In the latter case,
insert the full address of convert into the following command within tex4ht.env.

\'+Gconvert zz%%4.ps -trim -density 110x110 -transparent '#FFFFFF' %%3+


For instance, use

\''Gc:\TeX\Imagick\convert'

instead of 

\''Gconvert'
\item You might need to remove the quotes from \'+'#FFFFFF'+
\EndList







%%%%%%%%%%%%%%%%%%%%%%%%%%%%%%%%%%%%%%%%%%%%


\TroubleSec{tex4ht.sty / *.4ht}

\item{Foreign content in {\tt <title>...</title>} elements}
  
  Use the {\tt \string\Configure\string{@TITLE\string}\string
    {...\string}} command to redefine for the these elements the
  harmful macros that appear in headers of logical units like
  \''\title' and \''\chapter'. For instance, the definition
   \''\Configure{@TITLE}{\def\LaTeX{LaTeX}}' for the source
\''\title{with \LaTeX}'.

\item{Problematic commands in titles of logical divisions}

Commands within titles of divisions might need protection when
transported by tex4ht to other locations.  The command
\''\Configure{NoSection}{..before...}{...after...}'
may be used for such a purpose.



%%%%%%%%%%%%%%%%%%%%%%%
   \TroubleSec{DVI Code}


\item{XDVI/DVIWINDO/YAP... hang on {\tt\char92special} command}
 
 The dvi code produced by tex4ht is not valid for use with other
 utilities.  Recompile the source file without the presence of tex4ht,
 to provide a proper code to your dvi viewer.

%%%%%%%%%%%%%%%%%%%%%%%
   \TroubleSec{JavaScript}


\item{`{\tt onovermouse}' package option}

This option currently relies on
the JavaScript utility  
      \Link[http://www.bosrup.com/web/overlib/]{}{}overlib.js\EndLink{}
of Erik Bosrup.  The file might need to be fixed at line 234, by
introducing \`'return "";' instead of \`'return;' (an already 
\Link[http://www.egroups.com/message/overlib/133]{}{}reported\EndLink{}  problem).


%%%%%%%%%%%%%%%%%%%%%%%
   \TroubleSec{Subscripts and Superscripts}


\item{Loss of structural information}

\--Subscripts and superscripts///%
%
 Subscripts and superscripts are among the weakest points of tex4ht.
 In order to recognize them for non-bitmap representations, \TeX4ht changes the category codes of
 \`'^' and \`'_', upon reaching the \''\begin{document}' instruction,
and ignores the operations if introduced earlier.


The ideal solution would have been to get the superscript and
subscript operations, as well as the empty bases \`'{}', marked upon
request by the native compilers in the dvi code. Currently, that is
just a good night dream.

The followings are possible ways to overcome the problem.

\List{1}

\item Prepare TeX4ht configuration files containing redefinitions
for the the sensitive macros.  For instance, the configuration file
\''test.cfg'

\Verbatim
\Preamble{}
          
\renewcommand{\FQED}[2]{F_{#1#2}}
\renewcommand{\Mlones}{M^2}
\renewcommand{\bzms}{M^2_{_0}}
          
\begin{document}
\EndPreamble
\EndVerbatim


for a source \''test.tex', and a compilation invoked through the
command \`'mzlatex test "test"'.  

\item Use \''\sb' and \''\sp',
instead of \`'_' and \`'^', in auxiliary files and preambles of files,
or push the definitions to after the \''\begin{document}' statement.
The danger in this approach is that occasionally
   users provide new meanings to existing control sequences,
   without tex4ht taking it into account.

\item
  Use the command line options \''early_' and \''early^'. For instance, 
  \''htlatex file "html,early_"'.





\EndList




\EndList
\EndHPage{}
%
{\tt |}
%
\noindent \HPage<QA>{Q/A}\ExitHPage{}


\InstallSection{Q/A}

\TableOfContents [QAChapter,QASection]


\noindent {\bf Note:} Please also check the .log files of the compilations
for hints provided by tex4ht.


%%%%%%%%%%%%%%%%%%%%%%%%%%%%%%%%%%%%%%%%
\QAChapter{Addresses of Bitmap Files}
%%%%%%%%%%%%%%%%%%%%%%%%%%%%%%%%%%%%%%%%

%%%%%%%%%%%%%%%%%%%%%%%%%%%%%%%%%%%%%%%%
\QASection{How Bitmap Files Are Created for  Pictures}
%%%%%%%%%%%%%%%%%%%%%%%%%%%%%%%%%%%%%%%%


There are two kinds of pictures: those created for letters, and those
requested directly or indirectly for figures in the source files.  The
tex4ht.c utility extracts the figures from the foo.dvi file into a
foo.idv file, and places in foo.lg messages of how to extract the
figures.  The t4ht.c utility, uses the requests listed at foo.lg, to
activate the G-script from tex4ht.env for the different figures.
However, if a F-script is available,  it is used instead for the letters.

%%%%%%%%%%%%%%%%%%%%%%%%%%%%%%%%%%%%%%%%
\QASection{References to Bitmap Files of General Pictures}
%%%%%%%%%%%%%%%%%%%%%%%%%%%%%%%%%%%%%%%%

   The command
   
\''\Configure{IMG}{<img\Hnewline
   src="foo.dir/}{" alt="}{" }{ />}'
   
   in the configuration file adds the prefix
   
     \''foo.dir/'

   to the file names.
   
   The default setting uses the configuration
      
\''\Configure{IMG}{<img\Hnewline
   src="}{" alt="}{" }{ />}'

%%%%%%%%%%%%%%%%%%%%%%%%%%%%%%%%%%%%%%%%
\QASection{Placement of Bitmap Filess of General Pictures}
%%%%%%%%%%%%%%%%%%%%%%%%%%%%%%%%%%%%%%%%

Add to the \`'G' script in \''tex4ht.env'
a command to move the bitmap files to the desired
destination (e.g., \`'Gmv %%3 foo.dir/.').

%%%%%%%%%%%%%%%%%%%%%%%%%%%%%%%%%%%%%%%%
\QASection{References to  Bitmap Files  of HTF Symbols}
%%%%%%%%%%%%%%%%%%%%%%%%%%%%%%%%%%%%%%%%

Add the following commands to  the configuration file.
   
\Verbatim
\Configure{htf}{1}{+}{<img\Hnewline
   src="foo.dir/}{" alt="}{" class="}{\%s}{-\%d}{x-x-\%x}{" />}
\Configure{htf}{3}{+}{<img\Hnewline
   src="foo.dir/}{" alt="}{" class="\%s-}{\%s}{-\%d}{x-x-\%x}%
   {" align="middle" />}
\EndVerbatim
   
%%%%%%%%%%%%%%%%%%%%%%%%%%%%%%%%%%%%%%%%
\QASection{Placement of Bitmap Files of HTF Symbols}
%%%%%%%%%%%%%%%%%%%%%%%%%%%%%%%%%%%%%%%%


If \`'foo.dir/' is not the directory used in the `G' script, introduce into
tex4ht.env a similar  `F' script for the bitmaps of character, e.g.,

\Verbatim
Fdvips -T 14in,14in -Ppdf -mode ibmvga -D 110 -f %%1 -pp %%2  > zz%%4.ps
Fconvert zz%%4.ps -trim -density 110x110 -transparent '#FFFFFF' %%3
Fmv %%3 foo.dir/.
\EndVerbatim



%%%%%%%%%%%%%%%%%%%%%%%%%%%%%%%%%%%%%%%%%%%%%%%%%%%%%%%%%%%%%%%%%%%%%%%%
\QAChapter{Representation of formulas}
%%%%%%%%%%%%%%%%%%%%%%%%%%%%%%%%%%%%%%%%%

%%%%%%%%%%%%%%%%%%%%%%%%%%%%%%%%%%%%%%%%%%%%%%%%%%%%%%%%%%%%%%%%%%%%%%%%
\QASection{Make picture's alternative text to show the original tex code}
%%%%%%%%%%%%%%%%%%%%%%%%%%%%%%%%%%%%%%%%%

Explicit requests:

\Verbatim
\newtoks\eqtoks
\def\ImageAlt{\afterassignment\setimg\eqtoks}
\def\setimg{\Picture*[\HCode{\the\eqtoks}]{}\the\eqtoks\EndPicture}

\ImageAlt{$\alpha + \beta$}

\ImageAlt{$$
  \begin{array}{cc}
    a & b \\
    c & d
  \end{array}
$$}
\EndVerbatim

Implicit requests may also be tailored along the following lines, but
they should be used carefully because they are not safe.

\Verbatim
\newtoks\eqtoks
\def\AltMath#1${\eqtoks{#1}%
   \Picture*[\HCode{\the\eqtoks}]{ align="middle"}$#1$\EndPicture$}
\Configure{$}{}{}{\expandafter\AltMath}
\EndVerbatim


%%%%%%%%%%%%%%%%%%%%%%%%%%%%%%%%%%%%%%%%%%%%%%%%%%%%%%%%%%%%%%%%%%%%%%%%
\QASection{Avoiding bitmaps within formulas}
%%%%%%%%%%%%%%%%%%%%%%%%%%%%%%%%%%%%%%%%%


\--Bitmaps and graphics/Requests//%
{\it Q. I found that some equations have smaller fonts than the
others. I wonder why the size of the equations changes along the
document.}

Some equations and symbols are converted to bitmaps--I guess they are
the parts you are concerned about.

If you try the command \`' mzlatex filename', provided you have the
xhlatex-like script of mzlatex configured, you'll get the bitmaps
removed in favor of unicode symbols and mathml formulas. In such a
case, one will need a mathml-enabled browser to view the outcome.

If you don't care using the MS SYMBOL font, which non-pc browsers are
likely not to recognize, try the command \`' htlatex filename "" "symbol/!" '.
It will resolve the bitmap problem for of the characters.

%%%%%%%%%%%%%%%%%%%%%%%%
\QASection{Setting bitmaps within formulas}
%%%%%%%%%%%%%%%%%%%%%%%%


A code similar to the following one may
be used to configure instructions like \`+\int_{xxx}^{yyy}+ and
\`+\sum_{xxx}^{yyy}+
 to produce pictorial representations within non-pictorial 
formulas.

\Verbatim
\def\SubSupOp#1{%
   \edef\temp{\expandafter\gobble\string#1}%
   \expandafter\let\csname old\temp\endcsname=#1
   \edef\temp{\noexpand\SUBSUPOP{\expandafter\noexpand
                   \csname old\temp\endcsname}}%
   \HLet#1=\temp
}
\def\gobble#1{}
\def\SUBSUPOP#1{\let\curOP=#1%
   \let\next=\putOP \let\OPsub=\empty \let\OPsup=\empty
   \futurelet\nextop\getOP}
\def\getOP{%
  \ifx _\nextop \let\next=\getsub
     \else\ifx ^\nextop \let\next=\getsup\fi\fi \next}
\def\getsub#1#2{\def\OPsub{#2}\let\next=\putOP
   \futurelet\nextop\getOP}
\def\getsup#1#2{\def\OPsup{#2}\let\next=\putOP
   \futurelet\nextop\getOP}
\def\putOP{\Picture+{  align="middle"}{\curOP_{\OPsub}^{\OPsup}}\EndPicture}

\SubSupOp\sum
\SubSupOp\int

\EndVerbatim



%%%%%%%%%%%%%%%%%%%%%%%%%%%%%%%%%%%%%%%%%%%%%%%%%%%%%%%%%%%%%%%%%%%%%%%%
\QAChapter{{\tt \char92 includegraphics}}
%%%%%%%%%%%%%%%%%%%%%%%%%%%%%%%%%%%%%%%%%%%%%%%%%%%%%%%%%%%%%%%%%%%%%%%%

%%%%%%%%%%%%%%%%%%%%%%%%%%%%%%%%%%%%%%%%%%%%%%%%%%%%%%%%%%%%%%%%%%%%%%%%
\QASection{Conditional bitmap conversion for imported graphic files}
%%%%%%%%%%%%%%%%%%%%%%%%%%%%%%%%%%%%%%%%%%%%%%%%%%%%%%%%%%%%%%%%%%%%%%%%

{\it Q. How to avoid the conversion of eps files to PNG's,
included through the \''\includegraphics{...}' command', each time
the source file is run across tex4ht.}

        
Compile your source with the command line \`' htlatex filename "html,info" ',
and check the log file for the information provided there.  In
particular, the \`'\Configure{graphics*} {wmf} ...' example may be
adapted for dealing with eps files, where a conditional conversion is
requested within the \`'\Needs{"..."}' command (possibly indirectly
through a call to a script for doing he job).



%%%%%%%%%%%%%%%%%%%%%%%%%%%%%%%%%%%%%%%%%%%%%%%%%%%%%%%%%%%%%%%%%%%%%%%%
\QASection{Clickable Thumbnail Images}
%%%%%%%%%%%%%%%%%%%%%%%%%%%%%%%%%%%%%%%%%%%%%%%%%%%%%%%%%%%%%%%%%%%%%%%%

{\it
\--Pictures/Thumbnail//%
 Q. 
How an image can be made a thumbnail to be  clicked on for 
 bringing up the full size image.}

\Verbatim
     \Configure{graphics*}  
        {png}  
        {\Link[\csname Gin@base\endcsname .png]{}{}% 
           \Picture[pict]{\csname Gin@base\endcsname .png 
              \space width="40px" height="40px" }%
         \EndLink
         }  
\EndVerbatim



\Verbatim
     \Configure{graphics*}
        {eps}
        {\openin15=\csname Gin@base\endcsname\PictExt\relax
         \ifeof15
            \Needs{"convert \csname Gin@base\endcsname.eps
                            \csname Gin@base\endcsname\PictExt"}%
         \fi
         \closein15
         \Link[\csname Gin@base\endcsname\PictExt]{}{}
            \Picture[pict]{\csname Gin@base\endcsname\PictExt
                        \space width="40px" height="40px" }%
         \EndLink
        }
\EndVerbatim

%%%%%%%%%%%%%%%%%%%%%%%%%%%%%%%%%%%%%%%%%%%%%%%%%%%%%%%%%%%%%%%%%%%%%%%%
\QAChapter{Other}
%%%%%%%%%%%%%%%%%%%%%%%%%%%%%%%%%%%%%%%%%%%%%%%%%%%%%%%%%%%%%%%%%%%%%%%%

%%%%%%%%%%%%%%%%%%%%%%%%%%%%%%%%%%%%%%%%%%%%%%%%%%%%%%%%%%%%%%%%%%%%%%%%
\QASection{Missing Configurations}
%%%%%%%%%%%%%%%%%%%%%%%%%%%%%%%%%%%%%%%%%%%%%%%%%%%%%%%%%%%%%%%%%%%%%%%%

{\it Q. I am attempting to convert a Ph.D. thesis written in LaTeX,
using a locally developed class ths.cls based on the report class, which
redefines some elements of the class for local needs.  The output is
lacking the Table of Contents and the Lists of Figures and Tables, and
the bibliography at the end is only minimally formatted.}

The problems result from a lack of a configuration file for the thesis
class. Try introducing a configuration file similar to the following
one.

\Verbatim
   % file name: ths.4ht
   \input report.4ht
\EndVerbatim



\EndHPage{QA}
%
{\tt |}
%
%
%
% \HCode{<span class="bugfixes">}\Link[http://www.cse.ohio-state.edu/\string
%     ~gurari/TeX4ht/bugfixes.html]{}{}bug fixes\EndLink
% \HCode{</span>}
%
\NextFile{mml.html}%
\HPage{common problems for MathML}
\--MathML/Do and Don't//%
\rightline{\ExitHPage{up}}
\Link{}{mml-prob}{}%
\NextFile{mml-issues.html}%
%%%%%%%%%%%%%%%%%%%%%%%%%%%%%%%%%%%%%%%%%%%%%%%%
\SubSection{Some Sources of Problems for MathML}
%%%%%%%%%%%%%%%%%%%%%%%%%%%%%%%%%%%%%%%%%%%%%%

\List{a}
\item
Broken math formulas, such as \`'$R=\{x|x$ is real $\}$'
instead of         \`'$R=\{x|x \mbox{ is real } \}$'. 
% 
\item
  Math mode employment for presenting nonmath content,
such as \`'$$\vbox{...}$$ ' instead of
         \`'\begin{center}\vbox{...}\end{center}'.          
%
\item
 Unmatched parentheses within entries of arrays, 
such as \`'\begin{array}...(... & ...)...\end{array}'
instead of 
\`'\begin{array}...\left(... \right.&
                    \left. ...\right)...\end{array}'.
%
\item
Incorrect annotation of  delimiters, such as \`'\bigl\{...\bigr|...\bigr\}'
instead of
       \`'\bigl\{...\bigg|...\bigr\}'
%
\item
%%\relax\relax\--\string_//\string\tt/%
\relax\relax\--!//{\string\tt\char \space 95}\string\csname\space :gobble\string\endcsname/%
%%\relax\relax\--\string^///%
\relax\relax\--!//{\string\tt\char \space 94}\string\csname\space :gobble\string\endcsname/%
 Empty bases for subscripts and superscripts, 
such as \`'{}^{...},  {}_{...}' instead of
\`'{\csname HCode\endcsname{}}^{...},  {\csname HCode\endcsname{}}_{...}'.
%

\item
Missing grouping for bases of subscripts and superscripts, such as
\`'10^6' instead of \`'{10}^6'. (Some fixing of the problem can be requested through a `-cxhtmml' command line option for t4ht.c.  For instance,
`{\tt mzlatex file "" "" "-cxhtmml"}'.)

\item Missing grouping for subscripts and suprscripts, such as
\`'A_\mathit{...}' instead of \`'A_{\mathit{...}}'.  (\LaTeX{} allows
indirect access to the content of the operands; \TeX4ht requires 
 direct access.)



\item
Use of the operators \`'^' and \`'_', instead of \`'\sb' and \`'\sp',
outside the presence of \TeX4ht.
    (\TeX4ht becomes active only at the \`'\begin{document}' command.
The \`'early_' and \`'early^' options extend this awareness to the preambles of the source latex documents.)

\item
If math environments nested within tabular environments cause \HPage{problems}

\noindent{\bf  Any explanation for the following problem?}
The native LaTeX compiler has no problem with examples  1 and 2. On example 3 it
complains: \`'Argument of \xhalign has an extra }'. When compiled in isolation, examples 2 and 3 produce the same tracing until \''\xhalign' is invoked.



\Verbatim
  \tracingcommands=2 
  \tracingmacros=2 
\documentclass{article} 
\begin{document} 

%%%%%%%%%%%%%%%%%%%%%%%%%%%%%%%%%%%%%%%%%%%%%%%%%%%%%%%%%%%%%%%%%%%%%%%
\catcode`\@=11

\def\eqnarray{% 
   \stepcounter{equation}% 
   \def\@currentlabel{\p@equation\theequation}% 
   \global\@eqnswtrue 
   \m@th 
   \global\@eqcnt\z@ 
   \tabskip\@centering 
   \let\\\@eqncr 
   $$\everycr{}\xhalign  \cr 
} 

\def\xhalign#1\cr{%
   \halign to\displaywidth\bgroup 
       \hskip\@centering$\displaystyle\tabskip\z@skip{##}$\@eqnsel 
      &\global\@eqcnt\@ne\hskip \tw@\arraycolsep \hfil${##}$\hfil 
      &\global\@eqcnt\tw@ \hskip \tw@\arraycolsep 
         $\displaystyle{##}$\hfil\tabskip\@centering 
      &\global\@eqcnt\thr@@ \hb@xt@\z@\bgroup\hss##\egroup 
         \tabskip\z@skip 
      \cr 
}
%%%%%%%%%%%%%%%%%%%%%%%%%%%%%%%%%%%%%%%%%%%%%%%%%%%%%%%%%%%%%%%%%%%%%%%

  \vbox\bgroup                                      %% example 1 %%
    \begin{eqnarray}  x \end{eqnarray}%
  \egroup

  \begin{tabular}{c}                                %% example 2 %%
    \vbox{%
      \begin{eqnarray}  x \end{eqnarray}%
    }%
  \end{tabular} 

  \begin{tabular}{c}                                %% example 3 %%
    \vbox\bgroup   
      \begin{eqnarray}  x \end{eqnarray}%
    \egroup
  \end{tabular} 

\end{document} 
\EndVerbatim







\EndHPage{}
 enclose them 
within braces (for instance,
\`'\begin{tabular}{c} 
  \begin{minipage}{4in} 
 { \begin{eqnarray} 
  x & = & y 
  \end{eqnarray} }
  \end{minipage} 
  \end{tabular}').


\EndList
\EndHPage{}
%%%%%%%%%%%%%%%%%%%%%%%%%%%%%%%%%%%%%%%
%{\tt |}  .log files of the compilations


\NextFile{bugfixes.html}\HPage{}

%This is an incorrect site for the bug fixes home page. 

Please proceed
to \Link[\WWW bugfixes.html]{}{}osu\EndLink{} or
\Link[\tugHOME bugfixes.html]{}{}tug\EndLink.
\EndHPage{}


%%%%%%%%%%%%%%%%%%%%%%%%%%%%%%%%%%%%%%%%%%%%%%%%%%%%%%%%%%%%%%%%%%%%%
\DocPart{Using the System}
%%%%%%%%%%%%%%%%%%%%%%%%%%%%%%%%%%%%%%%%%%%%%%%%%%%%%%%%%%%%%%%%%%%%%

Typical \LaTeX{} source files can be compiled into standard HTML and
XML formats in a manner similar to the way they are compiled into
print formats, namely, through variations of the command
\--htlatex///`{\tt htlatex {\it filename} "{\it options1}" "{\it
    option2}" "{\it options3}" "{\it options4}"}'.  For instance,

\Verbatim
htlatex  filename
htlatex  filename "html,2,info"  "dbcs/!"
htlatex  filename "foo,frames"   ""            "-p"
htlatex  filename ""             " -ciso2htf"  ""   "-translate-file=il2-pl"
xhlatex  filename 
mzlatex  filename 
oolatex  filename 
\EndVerbatim

In some platforms the double quotes should be replaced with  single
right-quotes, and in some cases they might be omitted.

For details, visit the
\NextFile{\jobname-commands.html}\HPage<hts>{calling commands}

\rightline{\ExitHPage{up}}

\DocPart{Calling Commands}

% \TableOfContents[DocChapter]

%%%%%%%%%%%%%%%%%%%%%%%%%%%%%%%%%%%%%%
\DocChapter{From LaTeX\space to HTML}
%%%%%%%%%%%%%%%%%%%%%%%%%%%%%%%%%%%%%%


\leavevmode\Link{}{features}\EndLink
The translation of a \LaTeX{} source file into HTML involves of
loading  \''tex4ht.sty' and  *.4ht  style files,
choosing the desirable options for the translation, compiling the
source into \''dvi' code with the native \LaTeX{} engine, and
postprocessing the outcome with the \''tex4ht' and \''t4ht'  programs
(see \Link[\RefFile{overview}]{}{}overview\EndLink{}).


\--htlatex///%
The  \''htlatex' command loads a script which takes on itself to invoke
the different steps of the process, without user intervention.  The
command assumes the form

\centerline{{\tt htlatex filename "{\it options1}" 
   "{\it option2}" "{\it options3}"  "{\it options4}"}} 


\noindent 
where the first set of options is for the
\''tex4ht.sty' and \''*.4ht' style files, the second set is for the
\''tex4ht'  
postprocessor,  the third for the \''t4ht' postprocessor, and 
the last one is for the \LaTeX{} compiler. 
 For instance,







\List{}
\item{{\''htlatex filename'}}This command requests a translation according
to the default conditions, which are set to produce  HTML transitional
4.0 code. 


\item{{\''htlatex filename "html,2,info"'}}This command is 
equivalent to the previous one, specifying explicitly the option 
\''html' for \''tex4ht.sty' instead of doing so implicitly. 

In addition,
the command requests a break up of the output into separate web pages, 
in accordance to the two top sectioning levels of the document.  

Moreover,
it asks for a listing in the log file of the information available for
the style files in use. That information, among other things, also 
introduces additional values available for the first list of options.

%
\item{{\''htlatex filename "" "dbcs/!"'}}This command requests the
loading
of the dbcs branch of Chinese hypertext fonts (on top of those already requested by
the default setting).

%
\item{{%
\--Bitmaps and graphics/Requests//%
\--Frames///%
\--Chinese///%
\--Fonts/Chinese//%
\--t4ht.c/-p//%
\''htlatex filename "foo,frames" "" "-p"'}}This command requests \LaTeX{}
to load a private
configuration file, named \''foo.cfg',
and to  place the content and table of contents
in separate frames.
In addition, it  asks
                   \''t4ht' not to produce bitmaps for pictures.



\item{{\''htlatex filename "" " -ciso2htf" ""
"-translate-file=il2-pl"'}}This command invokes the \LaTeX{} compiler
with the instruction \`'latex -translate-file=il2-plfilename'.

\EndList


%%%%%%%%%%%%%%%%%%%%%%%%%%%%%%%%%%
\DocChapter{Available Values for the Options}
%%%%%%%%%%%%%%%%%%%%%%%%%%%%%%%%%%

\--htlatex/available options//%
The fields of {\it option1} should be separated by commas.  An
\`'info' field requests a listing in the .log file of many of the
the available values.  If the list is not empty, it must start with
the entry \`'html', \`'xhtml', or a name of a private configuration
file.

The fields of {\it option2} and {\it options3} should be separated by
spaces.  The available values can be listed by executing the
postprocessors \''tex4ht.c' and \''t4ht.c', respectively, without
arguments (or with wrong sets of arguments).

The first field of {\it option2} should be empty or a subdirectory of
\''ht-fonts' (typically augmented with an exclamation mark \`'!'). A
space should separate the first field from the second one, also when
the first field is empty.

The underlying output formats of available \''htlatex'-like commands
are tailored into the commands through fields of {\it option1}. The
names of these fields are defined in \''tex4ht.4ht' and \''tex4ht.usr'
(see \Link{confFiles}{}\Ref{confFiles}\EndLink). These values should
be of little interest to most users.



\--htlatex/mk4ht//% 
\--mk4ht///% 
Different variants of the \''htlatex'
command may be invoked by introducing the commands as arguments to a
driver named \''mk4ht'.  When provided without arguments, the driver
lists the commands it recognizes.

\bgroup \tt
\HTable ^/
[/style="white-space: nowrap; margin-right:2em;"]   mk4ht mzlatex filename "html,3" \&   (htlatex filename "html,3,xhtml,mozilla" "~-cmozhtf") \CR
[/style="white-space: nowrap; margin-right:2em;"]   mk4ht oolatex filename           \& (htlatex filename "xhtml,ooffice" "ooffice/!~-cmozhtf" "-coo")
\EndHTable
\egroup

\--htlatex/mkht-scripts.4ht//%
Alternatively, a compilation 
\`'latex mkht-scripts.4ht'
produces different named scripts of similar functionality.

%%%%%%%%%%%%%%%%%%%%%%%%%%%%%%%%%%
\DocChapter{XHTML and Unicode}
%%%%%%%%%%%%%%%%%%%%%%%%%%%%%%%%%%
\--XHTML///%'
\--Unicode///%'
\--tex4ht.c/-cunihtf//%
\--Unicode/-utf8//%
The \`'xhlatex' command is a variant of the \`'htlatex' command
requesting XHTML output.  It consists just of a call to \`'htlatex'
with the entry \`'xhtml' in the first list of options 
and \`'-cvalidate' in the third list. For instance,
\`'xhlatex filename' or \`'htlatex filename "xhtml"'.

To request a Unicode representation of symbols, the first list of
options should include the \`'uni-html4' entry, and the second list
should include the \`'-cunihtf' entry preceded by space.  For
instance, \`'xhlatex filename "xhtml,uni-html4" " -cunihtf"'.


Unicode representations of symbols in UTF-8 encoding may be requested
with the entry  \`'-utf8' added to the second list.
 For
instance, \`'xhlatex filename "xhtml,charset=utf-8" " -cunihtf -utf8"'.



To request expanded  usage of unicode values in iso-8859-1 output 
employ commands
similar to

      \centerline{\tt htlatex file "" "iso8859/1/charset/uni/!"}

\noindent or introduce a similar charset path in tex4ht.env.  Otherwise, non
iso-8859-1 characters might obtain bitmap representations.



%%%%%%%%%%%%%%%%%%%%%%%%%%%%%%%%%%
\DocChapter{XHTML with MathML}
%%%%%%%%%%%%%%%%%%%%%%%%%%%%%%%%%%

\--MathML///%
\--MathML/MathPlayer//%
\--mzlatex///%'
\--Scripts/mzlatex//%
\--xhmlatex///%'
\--Scripts/xhmlatex//%
\TeX4ht has different configurations for different modes of output. It
is distributed with pre-tailored base configurations for translating
\LaTeX{} math into MathML, and extra configurations for adjusting the
outcome to Mozilla,
 MathPlayer, 
and PMathML CSS. Only presentational 
MathML is supported.

\Verbatim
     mzlatex filename
     mzlatex filename "html,pmathml"
     mzlatex filename "html,mathml-"
     mzlatex filename "html,mathplayer"
     xhmlatex filename
\EndVerbatim



 For XHTML+MathML to be served both by Mozilla and
MSIE+MathPlayer use the command line option `mathplayer'.  


The \''mzlatex' command is a short cut representation for the command
\`'htlatex filename "xhtml,mozilla" " -cmozhtf"  "-cvalidate"'. It take into account special
needs of browsers.  The \''xhmlatex' command is a short cut
representation for the command 
\`'htlatex filename "xhtml,mathml" " -cunihtf" "-cvalidate"';
it does not make any compromizes toward browsers.


 It might be worthwhile to notice some of the
more  \Link{mml-prob}{}common sources of problems\EndLink{}  for MathML.
\--mzlatex/-mathml//%'
The \`'mathml-' options asks for a degraded
 MathML output that sidetracks some of 
the problems.

%%%%%%%%%%%%%%%%%%%%%%%%%%%%%%%%%%
\DocChapter{OpenDocument, OpenOffice,  and MS Word}
%%%%%%%%%%%%%%%%%%%%%%%%%%%%%%%%%%

\--OpenOffice///%
\--OpenDocument///%
\--Scripts/oolatex//%
\--oolatex///%
\--Word, MicroSoft///%
\--Scripts/MicroSoft Word//%
A translation for an OpenDocument
 format can
be requested by the \`'\oolatex' command. The command is a variant of
 \''htlatex'  in which the first list of options holds the entries
\`'xhtml,ooffice', the second list holds the entry \`'-cmozhtf'
preceded by a space, and the third list contains \`'-coo'
({\tt
htlatex filename "xhtml,ooffice" "ooffice/!~-cmozhtf" "-coo -cvalidate"}). The output
of a command \`'oolatex filename' is a zipped file named with  a
\`'.odt' extension. 




The OpenDocument code employs MathML for formulas, and XSL-FO for
formatting. It can be viewed by the 
\Link[http://www.openoffice.org/]{}{}OpenOffice\EndLink{}
 word processor which,
in turn, can export RTF and other MicroSoft-based formats (see also,
Maarten Wisse, ``Hacking TeX4ht for XML Output: The Road toward a TeX
to Word Convertor'',
\Link[http://www.ntg.nl/maps/electromaps.html]{}{}MAPS\EndLink~28~(2002),
pp.~28-35).



A command of the form \`'htlatex filename "html,word" "symbol/!"
"-cvalidate"' asks for HTML output tuned toward MicroSoft Word.  Such
a format, however, relies on bitmaps for mathematical formulas.


%%%%%%%%%%%%%%%%%%%%%%%%%%%%%%%%%%
\DocChapter{DocBook and TEI}
%%%%%%%%%%%%%%%%%%%%%%%%%%%%%%%%%%

\--DocBook///%
\--Scripts/dblatex//%
\--Scripts/dbmlatex//%
\--TEI///%
\--Scripts/teilatex//%
\--Scripts/teimlatex//%
\--dblatex///%
\--dbmlatex///%
\--teilatex///%
\--teimlatex///%
The following
commands may be used for requesting DocBook and TEI output.

\bgroup \tt
\HTable /
dbmlatex: \&
htlatex {\it filename}  "xhtml,docbook-mml"    " -cunihtf" "-cdocbk"\CR
dblatex:\&
htlatex {\it filename}  "xhtml,docbook"    " -cunihtf"  "-cdocbk"\CR
teimlatex:\&
htlatex {\it filename}  "xhtml,tei-mml"    " -cunihtf"  "-cdocbk"\CR
teilatex:\&
htlatex {\it filename}  "xhtml,tei"    " -cunihtf"  "-cdocbk"
\EndHTable
\egroup


%%%%%%%%%%%%%%%%%%%%%%%%%%%%%%%%%%
\DocChapter{JavaHelp}
%%%%%%%%%%%%%%%%%%%%%%%%%%%%%%%%%%

\--Unicode/-u10//% 
\--JavaHelp///%
\Link[http://java.sun.com/products/javahelp/index.jsp]{}{}JavaHelp\EndLink{}
is an online documentation system for use by Java-based applications.
Such documents can be produced from \LaTeX{} files through commands
similar to `{\tt jhlatex {\it filename}}' for JavaHelp version 2.0.

The above invocation stands for
`{\tt htlatex {\it filename} "html,3.2,xml,javahelp,unicode" " -cmozhtf
-u10" "-dfilename-doc/ -cjavahelp"}'. The \`'-u10' asks for entity
references in base 10---JavaHelp doesn't seem to support hexadecimal
representations. The \Verb=-cjavahelp=
invokes the JavaHelp indexer to produce the search
database at `jobname-doc/jobname-jhs' with a command
 similar to `{\tt java -jar
   \$\string{HOME\string}/jh2.0/javahelp/bin/jhindexer.jar -db
   jobname-doc/jobname-jhs  
jobname*.html}'.

The Java program is to be distributed with the
jobname-doc directory.


\Code\jhsample{XXXX}<<<
\documentclass{article}

  \usepackage{url}  
  \Configure{ProTex}{java,<<<>?empty>>,title,`,list,[[]]}

  \usepackage{makeidx}  
  \makeindex

\begin{document}

\tableofcontents

%%%%%%%%%%%%%%%%%%%%%%%%%%%%%%%%%%%%
\section{Compilation Instructions}
%%%%%%%%%%%%%%%%%%%%%%%%%%%%%%%%%%%%

\begin{itemize}
\item
Compile this \LaTeX{} file with the 
\index{jhlatex}%
`{\tt jhlatex 
jobname "html,3"}'
command.
\item
\index{JavaHelp URL}%
The URL into the javahelp file, as provided by tex4ht, is
`{\tt jobname-doc/jobname-jh.xml}'.
\item
The java programs should be compiled with commands  similar to
\index{javac}%
`{\tt javac -classpath
  \$\string{HOME\string}/jh2.0/javahelp/lib/jh.jar program.java}'. 
\item
The programs should run with commands similar to
\index{java}%
`{\tt java -classpath
  \$\string{HOME\string}/jh2.0/javahelp/lib/jh.jar:. program}'. 
\end{itemize}




%%%%%%%%%%%%%%%%%%%%%%%%%%%%%%%%%%%%
\section{Sample Program}
%%%%%%%%%%%%%%%%%%%%%%%%%%%%%%%%%%%%

\<jhprog\><<<
import java.net.URL;
import javax.help.*;
import javax.swing.*;
public class jhprog {
   public static void main(String args[]) {
      JHelp helpViewer=null;
      try {
         ClassLoader cl = jhprog.class.getClassLoader();
         URL url = HelpSet.findHelpSet(cl, 
                              "`jobname-doc/`jobname.hs");
         helpViewer = new JHelp(new HelpSet(cl, url));
      } catch (Exception e) { System.out.println("error");
      }
      JFrame frame = new JFrame();
      frame.setSize(500,500);
      frame.getContentPane().add(helpViewer);
      frame.setDefaultCloseOperation(JFrame.DISPOSE_ON_CLOSE);
      frame.setVisible(true);
   }
}
>?empty>>

\OutputCode\<jhprog\>


%%%%%%%%%%%%%%%%%%%%%%%%%%%%%%%%%%%%
\section{Sample Script}
%%%%%%%%%%%%%%%%%%%%%%%%%%%%%%%%%%%%

A source file \edef\temp{\noexpand\url{jhsample.tex}}\temp{} of this
document  may be compiled for JavHelp 2.0 with the following script.

\Verbatim
jhlatex jhsample
tex "\def\filename{{$1}{idx}{4dx}{ind}} \input  idxmake.4ht"
makeindex -o jhsample.ind jhsample.4dx
jhlatex jhsample
\EndVerbatim

The compilation produces the following files.

\begin{verbatim}
jhprog.class 
jhsample-doc/ 
jhsample-doc/jhsample.html 
jhsample-doc/jhsample.jhm 
jhsample-doc/jhsample-jhi.xml 
jhsample-doc/jhsample.hs 
jhsample-doc/jhsample-jht.xml 
jhsample-doc/jhprog.java 
\end{verbatim}


The following instruction can be used for compiling the program.

\Verbatim
javac -classpath ${HOME}/jh2.0/javahelp/lib/jh.jar jhprog.java
\EndVerbatim


The outcome could be viewed through the following command.

\Verbatim
java -cp ${\home}/jh2.0/javahelp/lib/jh.jar:. jhprog
\EndVerbatim


\printindex

\end{document}
>>>   \OutputCode[tex]\jhsample




%    Tex4ht does the following for JavaHelp
% 
%      * Creates keys for the entries in the table of contents and the
%        index
%      * Maps the keys into th erl's of the toc and the index (jobname.jhm)
%      * Maps the keys to the entries referenced by the toc
%        (jobname-jht.xml) and to the entries referenced by the toc
%        (jobname-jhi.xml)
%      * Creates a manifest of the files (jhelpset.hs), under the
%        assumption that jhindexer set a search directory at jobname-jhs
%        `java -jar
%        /n/gold/5/gurari/javahelp.dir/jh2.0/javahelp/bin/jhindexer.jar
%        -db jobname-jhs *.html'
% 




%%%%%%%%%%%%%%%%%%%%%%%%%%%%%%%%%%%%%%%%%%%%%
\DocChapter{Private Configuration Files}
%%%%%%%%%%%%%%%%%%%%%%%%%%%%%%%%%%%%%%%%%%%%%

\--Files/Configuration//% 
\Link{}{prvCfgFiles}\EndLink
The leading entry, in the first list of options of the
\''htlatex'-like commands, can equal \`'html' or \`'xhtml'. If this is
not the case, the entry is assumed to be the name of a configuration
file.  The extension \`'cfg' is assumed for names of configuration
files that are listed without their extension.

A configuration file should take  the following form for \LaTeX{}
files.


\Template
...{\sl early definitions}...\cr
{\string\Preamble\string{{\sl options}\string}}\cr
...{\sl definitions}...\cr
{\string\begin\string{document\string}}\cr
...{\sl insertions into the header of the html file}...\cr
{\string\EndPreamble}
\EndTemplate\ShowPar


It is up to the user to decide the distribution of entries between the
\''\Preamble' and the \''htlatex'-like commands.


\Example{}
The command \`'htlatex myfile "mycfg,2"' requests the compilation of 
a file named myfile.tex, in the presence of a configuration file named
mycfg.cfg. The configuration file might  have the following content.

\Verbatim
\Preamble{html}
\begin{document}
  \Css{body { color : red; }}
\EndPreamble
\EndVerbatim

\EndExample


\Notes \List{*} \item Notice that for a \LaTeX{} file the
\''\begin{document}' instruction should be present both in the
configuration file and the source file.

\item Instructions defined within a source file may be redefined in a
configuration file.  Such a feature enables to keep source files
intact for compilation to different formats by different tools.


For instance, a definition of the form
\''\renewcommand\mycommand{...}'  within a configuration file provided
for the following \LaTeX{} source.

\Verbatim
      \documentclass{...}
        \newcommand\mycommand{...}
      \begin{document}
         Use \mycommand{...}
      \end{document}
\EndVerbatim



\EndList
\EndNotes




%%%%%%%%%%%%%%%%%%%%%%%%%%%%%%%%%%%%%%%%%%%%%
\DocChapter{Creating Private Command Lines}
%%%%%%%%%%%%%%%%%%%%%%%%%%%%%%%%%%%%%%%%%%%%%

\--mkht.4ht/Configuration//% 
\--Scripts/tailored//%

A htlatex-like script {\tt foolatex.bat} can be obtained from the
compilation under \LaTeX{} of a file similar to the following one.
  
\Verbatim
 \def\script{bat}     
 \input mkht.4ht     
 \one{,html,next,3}  
 \two{-ic:\tex4ht\texmf\tex4ht\ht-fonts\#1 
      -ic:\tex4ht\texmf\tex4ht\ht-fonts\symbol\!}   
 \three{#1 -dc:\my\dir}  
 \make{foo}    
 \end{document}   
\EndVerbatim
 
A call of the form 
 
\centerline{\tt foolatex filename}
 
is then equivalent to a call of the following form.
 
\centerline{\tt htlatex filename "html,next,3" "symbol/!"
                                 "-dc:\string\my\string\dir"}


Scripts obtained in such a manner can embed parameters within their
bodies instead of expecting the parameters to be provided in command lines.

Details regarding the available options can be found by compiling under \LaTeX{}
a file of the following form.

\centerline{\tt\string\input\space mkht.4ht \string\end\string{document\string}}


The compilation requires the {\tt ProTex.sty} and {\tt AlProTex.sty}
files available at \Link[http://www.cse.ohio-state.edu/\string
~gurari/systems.html.]{}{}{\tt http://www.cse.ohio-state.edu/\string
~gurari/systems.html}\EndLink.








%%%%%%%%%%%%%%%%%%%%
\DocChapter{An Insight into the Commands}
%%%%%%%%%%%%%%%%%%%%



Given a \LaTeX{} file

  \Code\xxxx{XXXX}<<<
\documentclass{article}
\begin{document}
   ..................
\end{document}
>>>
\ShowCode-\xxxx\ShowPar

\noindent the `{\tt htlatex {\it filename}}' command produces a call
`{\tt latex {\it filename}}' to \LaTeX{} on an implicit file of the
following form. 


  \Code\xxxx{XXXX}<<<
\documentclass{article}
   ?.@\usepackage{tex4ht}@
\begin{document}
   ..................
\end{document}
>>>
\ShowCode-\xxxx\ShowPar

\noindent Similarly, the command `{\tt htlatex {\it filename}
{\tt"}{\it options}{\tt"}}' produces a call to a `{\tt latex {\it
filename}}' command on an implicit file of the following form.


  \Code\xxxx{XXXX}<<<
\documentclass{article}
   \usepackage[?.@options@]{tex4ht}
\begin{document}
   ..................
\end{document}
>>>
\ShowCode-\xxxx\ShowIndent


\--ht///%
The command `{\tt ht latex {\it filename} }' may be used, instead of
the `{\tt htlatex {\it filename} {\tt"}{\it options}{\tt"}}' command,
in cases that the \''\usepackage' instruction is explicitly introduced
into the source files.


%%%%%%%%%%%%%%%%%%%%
\DocChapter{A Deeper Insight}
%%%%%%%%%%%%%%%%%%%%

From the perspective of \TeX4ht, the \''htlatex'-like commands, and
the \''\usepackage', are indirect approaches for getting \LaTeX{}
files of the following form.
Such files can be explicitly provided for compilations 
 requested through the `{\tt ht latex {\it filename}}' command.

%%%%%%%%%%%%%%%%%%%%%%%%%%%%%%%%
  \Code\xxxx{XXXX}<<<
\documentclass{article}
   .....
   \input tex4ht.sty
   .....
\Preamble{?.@options@}
   .....
\begin{document}
   .....
\EndPreamble
   ..................
\end{document}
>>>
\ShowCode-\xxxx \ShowPar
%%%%%%%%%%%%%%%%%%%%%%%%%%%%%%%%




%%%%%%%%%%%%%%%%%%%%
\DocChapter{\TeX, Con\TeX t, and \TeX i}
%%%%%%%%%%%%%%%%%%%%

\--\TeX///% \--Scripts/\TeX//% \--\TeX i///% \--Scripts/\TeX i//%
\--ht///\--httex///%
\--Scripts/ht//%
Commands similar to those offered for \LaTeX{} are also offered for
\TeX{} (dbmtex, dbtex, ht, httex, mztex, ootex, t4ht, teimtex, teitex,
tex4ht, xhmtex, xhtex) and \TeX i (dbmtexi, dbtexi, httexi, mztexi,
ootexi, teimtexi, teitexi, xhmtexi, xhtexi).  In the case of \TeX{},
the fragment of code \`'\csname tex4ht\endcsname' should be introduced by the user
into the source file, after the preamble of the file where the
document definitions reside
  (\HPage{example}
\rightline{\ExitHPage{}}
\SubSection{An Example for Using httex}
A system invocation of the form \`'httex foo
"xhtml,html4.4ht,unicode.4ht,mathml.4ht" "unicode/!"' for a file
\''foo.tex' of the following form.

\Verbatim
\input amstex
\documentstyle{amsppt}  \csname tex4ht\endcsname
\document
.....
\enddocument
\EndVerbatim\EndHPage{}). 
 In the case of \TeX i, such a code
fragment is introduced implicitly.


The private configuration files are similar to those of \LaTeX{}, with
the instruction `{\tt \string\begin\string{document\string}}'
excluded. 

\Template
...\cr
{\string\Preamble\string{{\sl options}\string}}\cr
...\cr
{\string\begin\string{document\string}}\cr
...\cr
{\string\EndPreamble}\cr
...
\EndTemplate


The `{\tt ht tex {\it filename} }' and `{\tt ht texi {\it filename} }'
commands may apply for \TeX{} and \TeX i{} sources that embed such
code fragments in their body. The embeded code should replace the
\`'\csname tex4ht\endcsname' fragment in \TeX{} sources, be palces
at the strat of the files in \TeX i sources, and not include
the \Verb+\begin{document}+ instruction.



\--ConTeXt///%
\--htcontext///%
For \Link[http://www.pragma-ade.com/]{}{}Con\TeX t\EndLink{} similar 
instructions apply with suffixes `{\tt context}' instead of
`{\tt latex}', `{\tt tex}', or `{\tt texi}'. For instance, `{\tt htcontext'} .


%%%%%%%%%%%%%%%%%%%%%%%%%%%%%%%%%%
\DocChapter{Other Options}
%%%%%%%%%%%%%%%%%%%%%%%%%%%%%%%%%%



\List{*}
\item 
\--XwTeX///%
XeTeX files can be compiled with htlatex-like instructions
(e.g., htxelatex, htxetex, mzxelatex).  Currently only partial support is provided
and only TeX-based fonts are handled. 
\item  
\--JsMath///%
A \Link[http://www.math.union.edu/\string
  ~dpvc/jsMath/welcome.html]{}{}jsMath\EndLink{} mode of output may be requested with instructions similar to the following one.

      \centerline{\tt htlatex file "xhtml,jsmath" " -cmozhtf"}

\item 
\--Bitmaps and graphics/png//%
The dvipng utility might be activated for bitmap constructions
through a request `-cdvipng' in the third options list. For instance,

      \centerline{\tt htlatex file "" "" "-cdvipng"}

This utility is reported to produce fast high quality output with much
smaller files than other convertors.

\item 
\--Speech///%
TeX4ht offers also 
\Link[http://www.cse.ohio-state.edu/\string~gurari/laspeak]{}{}speech\EndLink{} output formats.

\EndList


%%%%%%%%%%%%%%%%%%%%%%%%%%%%%%%%%%
\DocChapter{Validation}
%%%%%%%%%%%%%%%%%%%%%%%%%%%%%%%%%%

\--Validation///%
The outcome of the translations should be checked by validators for
proper syntax.  Typically, with the presence of validators, errors are
easy to detect and correct, but they require human intervention.

\TeX4ht doesn't offer a built-in parser to verify the correctness of
the outcome.  However, external validator(s) can quite easily be 
integrated into the compilation process.


%%%%%%%%%%%%%%%%%%%%%%%%%%%%
\DocChapter{Recommendations}
%%%%%%%%%%%%%%%%%%%%%%%%%%%%

To keep with the spirit of \LaTeX{} and hypertext, in which style is
assumed to be separated from content, the users are encouraged to
avoid inserting \TeX4ht code into their source files.  Instead, they
should place their modifications, to the default settings, within
private configuration files to be loaded by htlatex-like
commands.

On the other hand, it should be noted that hypertext markings should
adhere to strict rules specified by different standards. Consequently,
it is strongly advised to check the output obtained from the default
configurations, before trying to tailor new ones.




\EndHPage{hts} section.


%%%%%%%%%%%%%%%%%%%%%%%%%%%%%%%%%%%%%%%%%%%%%%%%%%%%%%%%%%%%%%%%%%%%%
\SubSection{References}
%%%%%%%%%%%%%%%%%%%%%%%%%%%%%%%%%%%%%%%%%%%%%%%%%%%%%%%%%%%%%%%%%%%%%



The main features of \TeX4ht are described in:
\List{*}
\item 
% \Picture{http://www.awl.com/coverimage/0201433117.jpg align="right"}
 M.~Goosen and
 S.~Rahtz with E.~Gurari, R.~Moore, and R.~Sutor,
 {\sl The La\TeX{} Web Companion}, Addison-Wesley, 1999.
%
\item
Fabrice Popineau,
\Link[http://www.gutenberg.eu.org/pub/GUTenberg/publicationsPDF/37-popineau.pdf]{}{}{\sl Affichez vos documents \LaTeX{} sur le Web avec \TeX4ht}\EndLink,
Cahiers
              GUTenberg 37--38, December 2000, 5--43 (French, PDF).
%
\item
Gustavo Cevolani,
\Link[http://www.guit.sssup.it/guitmeeting/2005/articoli/cevolani.pdf]{}{}{\sl
Introduzione a \TeX4ht}\EndLink,
Proceedings of the 2004  Italian TUG meeting.
%
\item 
\HPage<here>{Within}

\rightline{\ExitHPage{up}}

%%%%%%%%%%%%%%%%%%%%%%%%%%%%%%%%%%%
\DocPart{Configurations}
%%%%%%%%%%%%%%%%%%%%%%%%%%%%%%%%%%%

% \TableOfContents[DocChapter]
\DocChapter{Background}  


TeX4ht handles correctly only macros whose logical meanings are
directly or indirectly declared in TeX4ht configurations.
For instance, without extra configurations, TeX4ht will provide
correct translation for

\Verbatim
     \divide{a}{b}
\EndVerbatim


under a user's definition of the form

\Verbatim
   \def\divide#1#2{{#1\over #2}}
\EndVerbatim

but not under a definition of the form

\Verbatim
   \def\divide#1#2{\vbox{\hbox{$#1$}\hrule\hbox{$#2$}}}
\EndVerbatim

%%%%%%%%%%%%%%%%%%%%%%%%%%%%%%%%%%%
\DocChapter{Recommendations}  
%%%%%%%%%%%%%%%%%%%%%%%%%%%%%%%%%%%

\--PostScript///% 
\--PDF///% 
It is highly recommended to leave source \LaTeX{} and \TeX{} files
intact, and not introduce \TeX4ht configurations there. The
configurations should be introduced indirectly in private
\Link{prvCfgFiles}{}configuration files\EndLink.  Source files
containing just native \LaTeX{} and \TeX{} code permit their
compilation to different output formats, including PostScript and PDF,
by \TeX4ht and other tools.


 Packages used by the general \LaTeX{} community typically provide
 better support than one can expect from tailoring private commands
 and configurations for such commands. It is also expected to take
 less effort to learn the features of existing packages than designing
 new ones.  Consequently, one is advised to investigate available
 resources before committing to work on private features.

%%%%%%%%%%%%%%%%%%%%%%%%%%%%%%%%%%%
\DocChapter{Low-Level Features}
%%%%%%%%%%%%%%%%%%%%%%%%%%%%%%%%%%%

The following are some of the more useful
underlying commands of \TeX4ht.

\List{button}
\Item {{\tt \--HCode//{{\tt\char 92}}/\string\HCode\string{...\string}}}%
\ContItem

%%\relax\relax\--#\relax///%
\relax\relax\--!//{\string\tt\char\space 35 }\string\csname\space :gobble\string\endcsname/%
This command allows only for the expansion of macros, before
sending its content to the output. The instruction \''\Hnewline'
may be introduced there for requesting line breaks, and the command \Verb'\#'
may be used for the sharp symbol \`'#'.

  \Code\xxxx{XXXX}<<<
Two lines of text      \HCode{<br />} 
separated by a horizontal line.

You probably don't want a `<br>'.
>>>
\ShowCode-\xxxx\ShowIndent

\Item {{\tt \--HPage//{{\tt\char 92}}/ \string\HPage\string{{\sl anchor}\string}{\sl content}\string
    \EndHPage\string{\string}}}%
\ContItem

This command dedicates a hypertext page for the specified content,
to be accessed through the given button.

\--ExitHPage//{{\tt\char 92}}/The \''\ExitHPage{...}' command may be employed within the content
to create exit buttons.

\Code\xxxx{XXXX}<<<
...... ............
\HPage{in}
   ................
   [\ExitHPage{out}]
   ................
\EndHPage{}
...................
>>>   \ShowCode-\xxxx\ShowIndent



\Item {\tt \--Link//{{\tt\char 92}}/\string\Link[{\it target-file 
  arguments}]\string{{\it target-loc}\string}\string{{\it
  cur-loc}\string}{\it anchor}\string\EndLink}%
\ContItem


This command requests an anchor that links to `{\it
target-file}\''#'{\it target-loc}', and marks the current 
location with the name `{\it cur-loc}'.


The component \`'[...]' is optional when it is empty, and the target
file need not be mentioned if it is created from the current source
file.

\Code\xxxx{XXXX}<<<
\HPage{}
   ......................
   \Link[http://www.tug.org/]{}{XX}\TeX{} Users Group 
                                    Home Page\EndLink
   ......................
\EndHPage{}
......................
\Link{XX}{}...\EndLink
>>>   \ShowCode-\xxxx\ShowIndent

%%%% \relax\relax\--\string~//\string\string/%
\relax\relax\--!//{\string\tt\char 126}\string\csname\space :gobble\string\endcsname/%
%%\relax\relax\--\string_//\string\tt\string\string/%
\relax\relax\--!//{\string\tt\char \space 95}\string\csname\space :gobble\string\endcsname/%
%\relax\relax\--!//\string\%\string\csname\space :gobble\string\endcsname/%
\relax\relax\--!//{\string\tt\char \space 37}\string\csname\space :gobble\string\endcsname/%
The characters \`'~', \`'_', and~\`'%' can be indirectly included
through the code~\`'\string ~',~\`'\string _', and~\`'\%',
respectively.

% \Item 
% {{\tt \--ifHTml//{{\tt\char 92}}/\string\ifHtml
%     ...tex4ht code...
%     \string\else ...non tex4ht code... \string\fi}}
% (or {{\tt \string\ifHtml\string\UnDef
%     ...non tex4ht code... \string\else ...tex4ht code... \string\fi}})
% \ContItem
% 
% This command provides the means to conditionally provide content
% to the  HTML  and non-HTML modes.  The \''\else' is optional when
% the false part  is empty.
% 

\Item {{\tt \--ifOption//{{\tt\char 92}}/\string\ifOption\string{...\string
 }\string{{\it true-part}\string }\string {{\it false-part}\string}}}%
\ContItem
\EndList

\DocChapter{Sectioning and Tables of Contents}

\--Sectioning///\--Tables of Contents///\--Options, package/1, 2, 3,
4//\--Options, package/next//\--Options, package/sections+//A
non-leading command line argument \`'1', `{\tt \HPageButton[]{2}}', \`'3',
or \`'4' asks for a tree-structured set of files,
reflecting on the sectioning of the document to the specified depth.
Sequential prev-next links within the hierarchy, instead of the
default hierarchical ones, can be requested with the \`'next'
parameter.  The parameter \`'sections+' creates titles for the
sectioning commands that link to the tables of contents.


\BeginHPage[]{2}

\ExitHPage{up}

   \Code\xxxx{XXXX}<<<
\documentclass{book}
   \usepackage[html,?..2.]{tex4ht}
\begin{document}
\chapter{...}  ......  \section{...}  ......  \section{...}  ......
\chapter{...}  ......  \section{...}  ......
\end{document}
>>>  \ShowCode-\xxxx\ShowIndent


\EndHPage{2}



Finer control is possible with the following commands.

\List{button}
%%%%%%%%%%%%%%%%%%%%%%%%%%%%%%%%%%%%%%%%%%%%%%%


\Item{\tt {\INDEX\string\CutAt\string{{\sl at-unit},{\sl 
     until-unit-1},{\sl until-unit-2},...\string}}}%
\ContItem

This directive asks the sectioning commands {\tt\char92{\sl at-unit}}
to place their units in separate hypertext pages. 
The pages are to terminate upon encountering any of the commands
in the list
{\tt\char92{\sl at-unit}},
{\tt\char92{\sl until-unit-1}},
{\tt\char92{\sl until-unit-2}},...


   \Code\xxxx{XXXX}<<<
?..\CutAt{section,chapter}.
\tableofcontents
\chapter{...}  ......  \section{...}  ......  \section{...}  ......
\chapter{...}  ......  \section{...}  ......
>>>  \ShowCode-\xxxx\ShowIndent


Within the \''\CutAt' instruction, the starred commands of \LaTeX{}
should be referenced with the prefix \`'like' instead of the postfix
of \`'*',    and appendices through the entry \`'appendix'.


A plus character \`'+', before the leading parameter, requests 
buttons that link to the hypertext pages; e.g., \`'\CutAt{+likesection}'.

The end points of  sections not specified within 
the \''\CutAt' commands can be made known with instructions
of the form `{\tt \string   \Configure\string {{\it endunit}\string}\string
{{\it unit},{\it unit},...\string}}'.



\Item{\tt\INDEX\string\tableofcontents[{\sl unit-1},{\sl unit-2},...]}
\ContItem

This variant of the \''\tableofcontents'
command specifies the kinds of entries that should be included in the
tables.


\Item{\tt\INDEX\string\TocAt\string{{\sl at-unit},{\sl unit-1},{\sl unit-2},...,{\sl
{\tt/}until-unit-1},{\sl {\tt/}until-unit-2},...\string}}
\ContItem

This directive asks for a local table of contents, at each division
created by the {\tt\char92{\sl at-unit}} command.  The tables should
include entries of types {\tt\char92{\sl unit-1}}, {\tt\char92{\sl
unit-2}},..., and terminate upon reaching any of the {\tt\char92{\sl
at-unit}}, {\tt\char92{\sl until-unit-1}}, {\tt\char92{\sl
until-unit-2}},... commands.

\--Sectioning/like//Within the \''\TocAt' instruction, the starred commands of \LaTeX{}
should be referenced with the prefix \`'like' instead of the postfix
of \`'*',    and appendices through the entry \`'appendix'.

   \Code\xxxx{XXXX}<<<
?..\TocAt{section,subsection,/likechapter}.
\CutAt{subsection,section,likechapter}

\section{...} ... \subsection{...} ...
\section{...} ... \subsection{...} ... \subsection{...} ...
\chapter*{...}
\section*{...} ...
>>>  \ShowCode-\xxxx\ShowIndent


A postfix \`'*' on \''\CutAt' asks the tables of contents
to appear after the preambles of the logical unit.


   \Code\xxxx{XXXX}<<<
?..\TocAt*{section,subsection,/likechapter}.
\CutAt{subsection,section,likechapter}

\section{...} ... \subsection{...} ...
\section{...} ... \subsection{...} ... \subsection{...} ...
\chapter*{...}
\section*{...} ...
>>>  \ShowCode-\xxxx\ShowIndent


A postfix \`'-' on \''\CutAt' asks to produce the local tables of contents
only on demand, through {\tt Cut{\it section-type}} commands.

The only \''\ConfigureToc' commands 
that count are those before 
the  \''\TocAt' instruction.  

\Item
{\tt
\INDEX\string\ConfigureToc
     \string{{\sl unit}\string}
     \string{{\sl before-mark}\string}
     \string{{\sl before-title}\string}
     \string{{\sl before-page-number}\string}
     \string{{\sl at-end}\string}
}%
\ContItem

This command determines how entries of the specified unit will appear
in the tables of contents. The entries include only fields
corresponding to nonempty fields in the \''\ConfigureToc'
commands.

   \Code\xxxx{XXXX}<<<
\ConfigureToc{section} {} {$\bullet$~}  {}  {~~ }
\tableofcontents[section]
>>>  \ShowCode-\xxxx\ShowIndent





\Item
{\tt 
\--Configure/tableofcontents/{{\tt\char92}}/%
\string\Configure
     \string{tableofcontents\string}
     \string{{\sl before-toc}\string}
     \string{{\sl end-of-toc}\string}
     \string{{\sl after-toc}\string}
     \string{{\sl before-nonindented-par}\string}
     \string{{\sl before-indented-par}\string}
}%
\ContItem

The {\sl end-of-toc} is inserted at the end of the internal
environment of the tables. The {\sl after-toc} is included after
leaving the internal environment.




\Item
{\tt
\--Configure/TocAt/{{\tt\char92}}/%
 \string\Configure
     \string{TocAt\string}
     \string{{\sl before-toc}\string}
     \string{{\sl after-toc}\string}
}%
\ContItem


\Item
{\tt
\--Configure/TocAt*/{{\tt\char92}}/%
 \string\Configure
     \string{TocAt*\string}
     \string{{\sl before-toc}\string}
     \string{{\sl after-toc}\string}
}%
\ContItem




\Item
{\tt
\--Configure/Sectioning units/{{\tt\char92}}/%
 \string\Configure
     \string{{\sl unit}\string}
     \string{{\sl top}\string}
     \string{{\sl bottom}\string}
     \string{{\sl before-title}\string}
     \string{{\sl after-title}\string}
}%
\ContItem

This command determines the content to be inserted at the mentioned
locations of the specified units.

   \Code\xxxx{XXXX}<<<
?..\Configure{.chapter?..}.
   ?..{.\HCode{<div class="chapter">}?..}.  ?..{.\HCode{</div>}?..}.
   ?..{.\HCode{<h2 class="chapterHead">}\chaptername
       ~\thechapter\HCode{<br />}?..}.
   ?..{.\HCode{</h2>}?..}.
\chapter{...} ....
\chapter{...} ....
>>>  \ShowCode-\xxxx\ShowIndent


\Item
{\tt
\--Configure/CutAt/{{\tt\char92}}/%
 \string\Configure
     \string{CutAt\string}
     \string{{\sl unit}\string}
     \string{{\sl before-button}\string}
     \string{{\sl after-button}\string}
}%
\ContItem


\Item
{\tt
\--Configure/+CutAt/{{\tt\char92}}/%
 \string\Configure
     \string{+CutAt\string}
     \string{{\sl unit}\string}
     \string{{\sl before-button}\string}
     \string{{\sl after-button}\string}
}%
\ContItem

\Item
{\tt \INDEX\string\NewSection
     {\char92}{\sl unit}
     \string{{\sl mark-for-toc}\string}
}%
\ContItem

This directive introduces a sectioning command {\tt\char 92{\sl unit}},
which submits {\sl mark-for-toc} to the tables of contents.  


      \Code\xxxx{XXXX}<<<
\newcounter{c}
\NewSection\X {\thec}
\Configure{X}
   {\addtocounter{c}{1}\HCode{<h2>}[\thec] }
   {\HCode{</h2>}}
   {}{}  
>>>  \ShowCode-\xxxx\ShowIndent





\Item{\tt
\--Configure/writetoc/{{\tt\char92}}/%
\string\Configure\string{writetoc\string}
   \string{{\sl definitions-for-the-writing-environment}\string}}
\ContItem




\TeX4ht expands and then writes the sectioning titles into an
auxiliary file, and it might encounter there problems from macros that
are not fit for such conditions or for inclusion in the table of
contents.  The current configuration instruction allows to locally
modify the behavior of macros for the writing phase.

For instance, the instruction \`'\section{Foo \\ bar}'
suggests a configuration similar to
\`'\Configure{writetoc}{\let\\\space}'.

% (or just  a \''\protect' on \''\flushright').









\Item{\tt
\--Configure/crosslinks/{{\tt\char92}}/%
\string\Configure\string{crosslinks\string}
   \string{{\sl left-delimiter}\string}
   \string{{\sl right-delimiter}\string}
   \string{{\sl next}\string}
   \string{{\sl prev}\string}
   \string{{\sl prev-tail}\string}
   \string{{\sl front}\string}
   \string{{\sl tail}\string}
   \string{{\sl   up}\string}}%
\ContItem

This command configures the appearance of the cross-links
between hypertext pages obtained for sectioning commands.


   \Code\xxxx{XXXX}<<<
\Configure{crosslinks}
  {}{}{$\scriptstyle\Rightarrow$}
  {$\scriptstyle\Leftarrow$}
  {}{}{}{$\scriptstyle\Uparrow$}
>>>  \ShowCode-\xxxx\ShowIndent




\Item{\tt
\--Configure/crosslinks+/{{\tt\char92}}/%
\string\Configure\string{crosslinks+\string}
   \string{{\sl before-top-links}\string}
   \string{{\sl after-top-links}\string}
   \string{{\sl before-bottom-links}\string}
   \string{{\sl after-bottob-links}\string}}%
\ContItem

%%%%%%%%%%%%%%%%%%%%%%%%%%%%%%%%%%%%%%%%%%%%%%%
\EndList


\DocChapter{Tables}

\--Tables///%
\--Tables/number of compilations//%
Tables with \''\multicolum' entries need a few \LaTeX{}
compilations to stabilize.




\List{button}
\Item
  {\tt
\--Configure/tabular/{{\tt\char92}}/%
\--Configure/array/{{\tt\char92}}/%
\--Configure/eqnarray/{{\tt\char92}}/%
\string\Configure\string{{\it table}\string} 
      \string{{\it before-tbl}\string} 
      \string{{\it after-tbl}\string}
      \string{{\it before-row}\string}
      \string{{\it after-row}\string}
      \string{{\it before-entry}\string}
      \string{{\it after-entry}\string}}%
\ContItem

\--HRow//{{\tt\char92}}/%
\--HCol//{{\tt\char92}}/%
The parameter {\it table} stands for \`'tabular', \`'array',
or  \`'eqnarray'.
The macros  \''\HRow' and \''\HCol' provide 
the row and  column indexes, respectively.  The macro \''\HMultispan'
records the number of columns.



\EXAMPLE
\Same
<table><tr><td class="try">(1,1)</td> <td class="try">(1,2)</td> <td class="try">(1,3)</td>
</tr><tr><td class="try">(2,1)</td> <td class="try">(2,2)</td> <td class="try">(2,3)</td> </tr></table>    
\EndSame

\Css{.try{ border:solid 4px; }}

\ContEXAMPLE
\Verbatim
\Configure{tabular}
   {\HCode{<table>}}   
   {\HCode{</table>}}
   {\HCode{<tr>}}   
   {\HCode{</tr>}}
   {\HCode{<td class="try">
               \ifnum \HMultispan>1 
                   colspan="\HMultispan"\fi}(\HRow,\HCol)}
   {\HCode{</td>}}
\Css{.try{ border:solid 4px; }}

\begin{tabular}{lll} 
  &&\\ &&
\end{tabular}
\EndVerbatim
\EndEXAMPLE



\EndList






\DocChapter{Lists and Environments}

\--Environments///%
\--Lists///%
The appearances of lists and {\tt\string\begin}-{\tt\string\end}
 environments are configured with the following commands.


\List{button}
\Item
 {\tt\INDEX\string\ConfigureList 
     \string{{\sl list-name}\string}
     \string{{\sl before-list}\string}
     \string{{\sl after-list}\string}
     \string{{\sl before-label}\string}
     \string{{\sl after-label}\string}}%
\ContItem

\--ConfigureList/list/{{\tt\char92}}/%
\--ConfigureList/trivlist/{{\tt\char92}}/%
Environments that directly or indirectly are built on top of  the
\`'\begin{list}...\end{list}' and
\`'\begin{trivlist}...\end{trivlist}' environments, inherit the
appearances of these base environments. The
\''\ConfigureList' command may be used to change the default
configuration.


   \Code\xxxx{XXXX}<<<
\ConfigureList {description} 
                            {}
                {\HCode{<hr />}} 
        {\HCode{<hr /><strong>}}
           {\HCode{</strong>}}   
>>>  \ShowCode-\xxxx\ShowIndent

\--ConfigureList/description/{{\tt\char92}}/%
\--ConfigureList/itemize/{{\tt\char92}}/%
\--ConfigureList/enumerate/{{\tt\char92}}/%
The \`'description', \`'itemize', and \`'enumerate' environments
are the more obvious extensions of the \`'list' and \`'trivlist'
environments.
 
\--ConfigureList/verse/{{\tt\char92}}/%
\--ConfigureList/quotation/{{\tt\char92}}/%
\--ConfigureList/quote/{{\tt\char92}}/%
\--ConfigureList/flushleft/{{\tt\char92}}/%
\--ConfigureList/flushright/{{\tt\char92}}/%
 The  \`'verse', \`'quotation', \`'quote', \`'center',
 \`'flushleft', and \`'flushright' are examples of non obvious 
 extensions.  The latter extensions are made up in \LaTeX{} from a
 single-item lists, to get the appearance of displayed paragraphs.
 


\Item 
{\tt \INDEX\string \ConfigureEnv
     \string{{\sl environment-name}\string}
     \string{{\sl before-environment}\string}
     \string{{\sl after-environment}\string}
     \string{{\sl before-list}\string}
     \string{{\sl after-list}\string}}
\ContItem

If either {\it before-environment} or  {\it
after-environment} is not empty, then these parameters
specify insertions that should be placed before and after the
specified environment.  


   \Code\xxxx{XXXX}<<<
\ConfigureEnv {tabular}
    {\HCode{<hr />}} {\HCode{<hr />}}
    {} {}
>>>  \ShowCode-\xxxx\ShowIndent



If the parameter {\sl before-list} or the parameter {\sl
after-list} is not empty, a call is made to
 {{\tt\char92}\bf ConfigureList \tt
     \string{{\sl list-name}\string}
     \string{{\sl before-list}\string}
     \string{{\sl before-label}\string}
     \string{\string}
     \string{\string}
} for configuring the base-list of  the environment. 


   \Code\xxxx{XXXX}<<<
\ConfigureEnv {flushright}
    {\HCode{<div class="flushright">}}
    {\HCode{</div>}}
    {\HCode{<h2>}} {\HCode{</h2>}}
>>>   \ShowCode-\xxxx\ShowIndent


\EndList




\DocChapter{Pictures}


\--Pictures///The next command imports external pictures, and the two
commands that follow request pictorial representations for local
content.  The attributes, and the replacement parameters with their
enclosing rectangular brackets, are optional.

\List{button}

\Item{\tt \INDEX\string\Picture[{\sl
replacement-for-textual-browser}]\string{{\sl file-name
attributes}\string}}% \ContItem

This command references the specified pictorial file. The
component `{\tt [{\sl replacement-for-textual-browsers}]}'
is optional.

\EXAMPLE
\Picture[**OSU logo**]{%
   http://www.cse.ohio-state.edu/images/OSU.gif}
\ContEXAMPLE
\''\Picture[**OSU logo**]{http://www.cse.ohio-state.edu/images/OSU.gif}'
\EndEXAMPLE


\Item{\tt\string\Picture+[{\sl replacement-for-text-browsers}]\string{{\sl
 file-name attributes}\string}{\sl content}\string\EndPicture} %
\ContItem

This command produces a picture for the provided content, stores the
outcome within a file of the specified name, and creates a reference
to the picture within the document. 


\Code\xxxx{XXXX}<<<
\Picture+{ align="right"}%
   Text within a picture.
\EndPicture
>>>   \OutputCode[log]\xxxx

\EXAMPLE
\input xxxx.log
\ContEXAMPLE
      \ShowCode-\xxxx\ShowIndent
\EndEXAMPLE

The component `{\tt [{\sl replacement-for-textual-browser}]}'
and the file name can be omitted.  If no name is provided for the
file, the system assigns a name of its own.


\Item{\tt\string\Picture*[{\sl replacement-for-text-browsers}]\string{{\sl
 file-name attributes}\string}{\sl content}\string\EndPicture} %
\ContItem

This is a variant of the previous command, that produces a picture 
of the content within a vertical box.


\EndList


\DocChapter{Mathematical Formulas}


%%\relax\relax\--(//{{\tt\char92}}/%
\relax\relax\--!//{\string\tt\relax\char \space 92 (}\string\csname\space :gobble\string\endcsname/%
\--Math environments///%
%\relax\relax\--[//{{\tt \char92}}/%
\relax\relax\--(//{\string\tt\relax\char \space 92 [}\string\csname\space :gobble\string\endcsname/%
%%\relax\relax\--\string$\relax///%
\relax\relax\--!//{\string\tt\char\space 36 }\string\csname\space :gobble\string\endcsname/%
%%\relax\relax\--\string$\string\string\string$///%
\relax\relax\--!//{\string\tt\char\space 36\char\space 36}\string\csname\space :gobble\string\endcsname/%
In the default setting, the math
environments \`'\(...\)', and the display math environments
\`'\[...\]'
and \`'$$...$$', request pictorial representations for their content.  On
the other hand, the math environments \`'$...$' ask for no special treatment.  Simple
features like mathematical symbols, subscripts, and superscripts, are
translated into html, and more complex entities like roots and
fractions are translated into pictures (\HPage{example}

\EXAMPLE
         $a^x + \frac{b}{c+d}$ and
         \(a^x + \frac{b}{c+d}\)
\ContEXAMPLE
         \''$a^x + \frac{b}{c+d}$ and \(a^x + \frac{b}{c+d}\)'
\EndEXAMPLE  

\ExitHPage{up}   
\EndHPage{}). 

\List{button}

\Item {\tt
\--Configure/[]/{{\tt\char92}}/%
\--Configure/()/{{\tt\char92}}/%
\--Configure/{{\tt\char36\char36}}/{{\tt\char92}}/%
\--Configure/{{\tt\char36\space}}/{{\tt\char92}}/%
 \string\Configure
   \string{[]\string}
   \string{{\it before}\string$\string${\it at-start}\string}
   \string{{\it at-end}\string$\string${\it after}\string}%
},
 {\tt \string\Configure
   \string{()\string}%
   \string{{\it before}\string${\it at-start}\string}%
   \string{{\it at-end}\string${\it after}\string}%%
}\BR {\tt \string\Configure
   \string{\string$\string$\string}%
   \string{{\it before}\string}%
   \string{{\it after}\string}%
   \string{{\it at-start}\string}%
}\BR {\tt \string\Configure
   \string{\string$\string}%
   \string{{\it before}\string}%
   \string{{\it after}\string}%
   \string{{\it at-start}\string}%
}
\ContItem



\Code\xxxx{XXXX}<<<
\Configure{[]}{An equation: $$}{$$}
\[a^b\]
>>> 
 \OutputCode[log]\xxxx

\EXAMPLE
   \input  xxxx.log
\ContEXAMPLE
   \ShowCode-\xxxx\ShowIndent
\EndEXAMPLE

\--PicDisplay//{{\tt\char92}}/%
\--PicMath//{{\tt\char92}}/%
The default configuration is obtained from\BR
\`'\Configure{[]}{\PicDisplay$$}{$$\EndPicDisplay}',\BR
\`'\Configure{$$}{\PicDisplay}{\EndPicDisplay}{}', \BR
\`'\Configure{()}{\PicMath$}{$\EndPicMath}', and\BR
\`'\Configure{$}{}{}{}'.



\Item {\tt
\--Configure/SUB/{{\tt\char92}}/%
\--Configure/SUP/{{\tt\char92}}/%
\--Configure/SUBSUP/{{\tt\char92}}/%
 \string\Configure
   \string{SUB\string}%
   \string{{\it before}\string}%
   \string{{\it after}\string}%
}\BR {\tt \string\Configure
   \string{SUP\string}%
   \string{{\it before}\string}%
   \string{{\it after}\string}%
}\BR {\tt \string\Configure
   \string{SUBSUP\string}%
   \string{{\it before}\string}%
   \string{{\it between}\string}%
   \string{{\it after}\string}%
}
\ContItem

These commands configure subscripts appearing in isolation,
superscripts given in isolation, and subscripts provided together with
superscripts.  If the last configuration command gets empty
parameters, the corresponding cases use the settings that apply to
isolated subscripts and superscripts.

The default setting results from\BR
\`'\Configure{SUB}{\HCode{<sub>}}{\HCode{</sub>}}', \BR
\`'\Configure{SUP}{\HCode{<sup>}}{\HCode{</sup>}}', and\BR
\`'\Configure{SUBSUP}{}{}'.



\Item{\tt
%%\relax\relax\--\string_//\string\tt/%
\relax\relax\--!//{\string\tt\char \space 95}\string\csname\space :gobble\string\endcsname/%
%%\relax\relax\--\string^///%
\relax\relax\--!//{\string\tt\char \space 94}\string\csname\space :gobble\string\endcsname/%
\--Options, package/no\string\tt\space\string\string\string_//%
\--Options, package/no\string\tt\space\string\string\string^//%
 no\string_}, {\tt no\string^}\ContItem

\TeX4ht modifies the implementation of \`'_' and \`'^', to create
hypertext subscripts and superscripts in non pictorial formulas---a
modification that occasionally might clash with other interpretations
in the source documents.  The current package parameters ask \TeX4ht
not to modify the implementation of these commands, respectively.

\EndList





\DocChapter{Paragraphs}

\--Paragraphs///The insertions of code at paragraph breaks are controlled by the
following commands.

\List{button}
\Item {\tt
\--Configure/HtmlPar/{{\tt\char92}}/%
\--Par//{{\tt \char92}}/%
 \string\Configure
   \string{HtmlPar\string}
   \string{{\sl noindent-P}\string}
   \string{{\sl indent-P}\string}
   \string{{\sl from-noindent-P}\string}
   \string{{\sl from-indent-P}\string}}
\BR
{\tt \string\EndP}
\ContItem

The first two parameters of this command determine the kind of code to
be inserted at the start of, respectively, nonindented and indented
paragraphs. 

\EXAMPLE
\ShowPar
\everypar{\HtmlPar}
\Configure{HtmlPar}
     {} {* }  {} {}
\par 1\par 2\par
     3\par 4\par 5
\ContEXAMPLE
   \Code\xxxx{XXXX}<<<
\Configure{HtmlPar}
     {} {* }  {} {}
\par 1\par 2\par
     3\par 4\par 5
>>>   \ShowCode-\xxxx\ShowIndent
\EndEXAMPLE

The last two parameters specify the code to be stored in
\''\EndP', when the first two parameters are introduced, respectively,
into the output.

\EXAMPLE
\ShowPar
\everypar{\HtmlPar}
\Configure{HtmlPar}
     {[ } {\EndP [ }
     { ]} { ]}

\par 1\par 2\par
     3\par 4\par 5

\Configure{HtmlPar}
     {}{\EndP} {}{}

\par\leavevmode
\ContEXAMPLE
   \Code\xxxx{XXXX}<<<
\Configure{HtmlPar}
     {[ } {\EndP [ } { ]} { ]}
\par 1\par 2\par 3\par 4\par 5
\Configure{HtmlPar}
     {}{\EndP} {}{}
\par\leavevmode
>>>   \ShowCode-\xxxx\ShowIndent
\EndEXAMPLE

The default setting assumes a configuration of the form
\`'\Configure{HtmlPar}{\HCode{<p class="noindent">}}
                   {\HCode{<p class="indent">}} {}{}' (and it is
implemented through an \`'\everypar{\HtmlPar}' command).

\Item {\tt
\--IgnorePar//{{\tt \char92}}/
 \string\IgnorePar}\ContItem

This command asks that no code will be inserted at the 
beginning of the next paragraph.

\Item {\tt
\--ShowPar//{{\tt \char92}}/
 \string\ShowPar}\ContItem

This command asks that  code will be inserted at the 
beginning of the next paragraph.

\Item {\tt 
\--IgnoreIndent//{{\tt \char92}}/
\string\IgnoreIndent}\ContItem

This command asks to treat the next paragraph as nonindented.


\Item {\tt
        \--Indent//{{\tt \char92}}/
 \string\ShowIndent}\ContItem

This command asks to treat the following paragraphs as indented.

\EndList

        








\DocChapter{Cascade Style Sheets (CSS)}


\--Cascade Style Sheets (CSS)///%
Cascade style sheets attach
presentations to the content of hypertext pages, in a manner similar
to the way that \`'.sty' files define the presentations to the content
of source \LaTeX{} files.  \TeX4ht produces a CSS file for each
document that is translated to HTML transitional 4.0 code. The
following are related commands.

\List{button}
\Item
{\tt
\string\Css\string{{\sl content}\string}%
}%
\ContItem

This command sends its content to the CSS file of the document.

\Item
{\tt
\INDEX
\string\Css\space{\sl content}\string\EndCss
}%
\ContItem

This command introduces the specified content, at the location of the
command as an inline CSS code fragment. The content should not start
with the left brace character \`'{'.



\Item
{\tt
\INDEX\string\CssFile[{\sl list-of-css-files}]{\sl content}\string\EndCssFile
}%
\ContItem

The default CSS file \TeX4ht produces is initially a file consisting
just of a single line of the form \`'/* css.sty */'.  That line is later
replaced with the code submitted by the \`'\Css{...}' commands.

The current command allows to specify an alternative to the initial
CSS file. The alternative  file consists of the code loaded
from listed files, and of the content explicitly specified in its body.



\Verbatim
  \ConfigureList{mylist} 
    {\HCode{<div class="mylist">}} {\HCode{</div>}} {* }{}

\begin{document}

  \HCode{<!--created by me-->}
  \CssFile
     /* css.sty */
     .mylist { color : red; }
  \EndCssFile
\EndVerbatim




The names in the list of files should be separated by commas, and the
rectangular brackets are optional when the list is empty. 

The file should include a line having the content of \`'/* css.sty */'.
If more than one such line is included, the content of the
\`'\Css{...}' commands replace the first occurrence of this line.
Arbitrary many space characters may appear around the substrings
\`'/*' and \`'*/'. 


\EndList









\DocChapter{Fonts}

\--Fonts/htf//%
\TeX4ht has an elaborated machinery
for handling fonts, through special virtual hypertext fonts stored in
\`'.htf' files. Instead of providing a design for each symbol, as is
the case in standard fonts, the virtual fonts provide a content for
each symbol.  The following commands offer some control, from within
the source \LaTeX{} documents, over the content provided to the
symbols.

\List{button}

\Item{\tt
\INDEX\string\NoFonts}\ContItem
This command asks that information provided in the font files will be
used for the symbols, but not the information requested for the fonts
in the source \LaTeX{} file through the \`'\Configure{htf}'
command.


\Item{\tt \INDEX\string\EndNoFonts}\ContItem

This command asks to end the effect of the most recently encountered
\''\NoFonts' command that is still active.

\Code\xxxx{XXXX}<<<
{\it italic \NoFonts\NoFonts
not italic \EndNoFonts
not italic \EndNoFonts
italic}.
>>>      \OutputCode[log]\xxxx

\EXAMPLE
   \input  xxxx.log
\ContEXAMPLE
   \ShowCode-\xxxx\ShowIndent
\EndEXAMPLE

\Item{\tt
\--Configure/htf\string\empty/{{\tt\char92}}/%
\string\Configure
\string{htf\string}
\string{{\sl class}\string}
\string{{\sl delimiter}\string}
\string{{\sl template-1}\string}
\string{{\sl template-2}\string}
\string{{\sl template-3}\string}
\string{{\sl template-4}\string}
\string{{\sl template-5}\string}
\string{{\sl template-6}\string}
\string{{\sl template-7}\string}%
}\ContItem


Each character is provided two entries in its virtual font file: a
string and an integer number.  
The integer number is considered to be the class of the character.  An
even number requests that the character will be represented by the
string. An odd number requests that the character will be represented
by a picture, with the string acting as an alternative representation.

The current \''\Configure' command provides a  template for
introducing, into the hypertext document, the representations of the
symbols of the specified class.  The template is consisted of the
seven specified components,  where the delimiter must be a character
that does not appear in  these components.

In even classes, the template is used for outputing a tuple,
consisting of the following information, for the given symbol: the
font name, the font size, the font magnification when it differs from
100\%, and the corresponding string field from the virtual font.

The first component is printed unconditionally at the beginning.
The font name is printed only if the second component of the template
is not empty and,  when it is not
empty, the second component should be a template for printing a string
in a C program. Similarly, the font size is printed only if the third
component of the template is not empty, and in such a case the
component should be a template for printing a string in a C program.
On the other hand, the font magnification is printed only if the
fourth component of the template is not empty, and in such a case the
component should be a template for printing an integer in a C program.
The rest of the components of the template are added literally into
the output, where either the fifth or the sixth component must be
empty. The string field from the virtual font is introduced just
before the last component.


\Code\xxxx{XXXX}<<<
{\it
\Configure{htf}
   {0}{+}{<span class="}
   {}{}{}{}{underline">}
   {</span>}
text%
\Configure{htf}
   {0}{+}{<span class="}
   {\%s}{-\%s}{--\%d}{}
   {">}{</span>}
text} 
>>> \OutputCode[log]\xxxx

\EXAMPLE
   \input xxxx.log
\BR
\BR
\Verbatim
<span class="underline">text</span> 
<span class="cmti-10">text</span>
\EndVerbatim
\ContEXAMPLE
   \ShowCode-\xxxx\ShowIndent
\EndEXAMPLE



The \''\Configure' defines for a symbol of an even class, a prefix and
a postfix to be inserted around the string assigned to the symbol in
its virtual font file.  The \''\Configure' for
characters of class 0 has the extra property
that it provides extra prefixes and postfixes also for all the
pictorial representations of symbols.


A \''\Configure' command for an odd class defines a template to output
a tuple, consisting of the following information: the font name, the
alternative string from the virtual font, the font name, the font
size, the font magnification when it differs from 100\%, and the
character code of the symbol.  The output is determined in a manner
similar to  that implied for symbols from characters of odd
classes.


\Code\xxxx{XXXX}<<<
\Configure{htf}
   {1}{+}{<sup><img src="}
   {" alt="}{}{}{}{}
   {" /></sup>}

$\alpha$%
\Configure{htf}
   {1}{+}{<img src="}
   {" alt="}{" class="\%s}
   {\%s}{-\%d}{--\%x}{" />}%
$\alpha$
>>> \OutputCode[log]\xxxx


\EXAMPLE
   \input xxxx.log
\ContEXAMPLE
   \ShowCode-\xxxx\ShowIndent
\EndEXAMPLE


\Item{\tt
\--Configure/htf\relax-sty/{{\tt\char92}}/%
\string\Configure
\string{htf-sty\string}
\string{{\sl class/font}\string}
\string{{\sl CSS-instructions}\string}%
}\ContItem
This command specifies CSS content for font classes and 
virtual hypertext fonts.

\EndList


The \''htf' fonts might request pictorial representations for 
symbols.  In such cases, the sizes of the pictures depend on
the sizes of the \TeX{} fonts in use.  Size changes through 
the \''\magnification' command should be made before loading the
\''tex4ht.sty' package.


The design of a  virtual hypertext font might take some labor,
but it does not 
\NextFile{\jobname-htf.html}%
\HPage{require}

\ExitHPage{up}

%%%%%%%%%%%%%%%%%%%%%%%%%%%%%%%%%%%%%%%%%%%%%%%%%%%%%%%%%%%%%%
\DocSection{Designing Virtual Hypertext  Fonts} 
%%%%%%%%%%%%%%%%%%%%%%%%%%%%%%%%%%%%%%%%%%%%%%%%%%%%%%%%%%%%%%


\List{1}
%
\item 
\--Fonts/htf//%
 If you are creating a new htf font, scan the information
   \TeX4ht issues in the \''log' file of the compilation
  regarding the missing htf font. For instance,

   {\tt--- warning --- Couldn't find font `ectt1000.htf' (char codes:
   0--255)}
%
\item
     Produce a document showing the character maps. Employ
 a script similar to the following one, 
using a standard compilation, say, for an output in PDF or PostScript.

%    {\tt \string\font\string\x=ectt1000  \string\ShowFont\string\x}

   \Example
\Verbatim
         \documentclass{article}
           \input showfonts.4ht
         \begin{document}
           \showfonts 
              {eccc1000}
              {ecss1000}
              {ecsx1200}
              {ecti1000}
              {ectt1000}
              {}
         \end{document}
\EndVerbatim

\EndExample
%          
\item
 Create htf fonts of the form 

\Verbatim
       first line:    prefix_of_font_name  first_index  last_index
                      ...............
                      ...............
                      ...............
                      ...............
       last line:     prefix_of_font_name  first_index  last_index
\EndVerbatim


    where
\List{a}
\item  The first and last lines must agree on their content, with
        \''first_index' and \''last_index' being equal to the character codes 
        mentioned in the message

         {\tt--- warning --- Couldn't find font `....' (char
                    codes: first\string_index--last\string_index)}

\item  The number of intermediate lines should equal

          \''(last_index) - (first_index) + 1'.
    
        Each of these intermediate lines provides a representation for
        a corresponding character code.    

        \Example
\Verbatim
            ectt 0 255
            '&#x0060;'    ''        0
            '&#x00B4;'    ''        1
            '&#x02C6;'    ''        2
            '~'           ''        3
            ..................
            'i'          '1'       25 dotless i
            ..................
            ''           ''        255
            ectt 0 255
\EndVerbatim  

\EndExample

\item Each intermediate line consists of three fields

            \''string   class  comment'

         The first two fields must be enclosed by a delimiter,
         determined by the first character in the line. The
         comment may be empty.

A \`'class' specified by an odd integer value asks for a pictorial
character.  An even integer number asks for a non-pictorial
character, specified in the \`'string' field. An empty class field 
is treated as a zero value.

The manner the characters of the different classes are packaged, is
determined by commands of the form \`'\Configure{htf}{class-number}...'.   
For instance,

\Verbatim
   \Configure{htf}{0}{+}{<span\Hnewline
      class="}{\%s}{-\%s}{x-x-\%d}{}{">}{</span>}
   \Configure{htf}{1}{+}{<img\Hnewline
      src="}{" alt="}{" class="}{\%s}{-\%d}{x-x-\%x}{" />}
   \Configure{htf}{4}{+}{<small\Hnewline
      class="}{}{}{}{}{small-caps">}{</small>}
   \Configure{htf}{6}{+}{<u\Hnewline
      class="}{}{}{}{}{underline">}{</u>}
\EndVerbatim



  When no
         special requirements are in place, it is advisable to
         use just the classes of 0 and 1.

\item The \`'string' field may include any sequence of characters,
         except for its delimiters.  The backslash
         character \`'\' acts there as an escaped character. It may 
         act as a delimiter for a character code, or be followed
         by another backslash (that is,  \`'\\'  represents the 
         character \`'\' ).
\item    In the string part, use \`'&lt;' for  the character \`'<',
         \`'&gt;' for \`'>', and \`'&amp;' for \`'&';



\EndList
%
\item
\--Cascade Style Sheets (CSS)///%
\--Fonts/htf//%%
 If you want specific information for a font, to be included in
  the .css file, add to the end of the file an entry consisting of the
  font name and the information in discourse.  The two fields must be
  separated by space.  The second field may span over more than one
  line; the extra lines must start with space. The  lines of the
  entries must be prefixed with \`'htfcss: '.

If more than one entry applies for a
given font, the first one is the only one that counts.
\EndList


{\noindent \bf Note.} It is highly recommended to set up fonts just of
Unicode entries, and let \TeX4ht automatically map the symbols to the
appropriate character encodings (using {\tt unicode.4hf} mapping files).










Instead of explicitly specifying the encodings for the characters, an
htf font can be declared an alias to another htf font by specifying in
the first line the aliased font name prepended by a period.

\Example{}

{\bf cmss.htf}

\Verbatim
.cmti
htfcss: cmss   font-family: sans-serif;
htfcss: cmssbx font-weight: bold;
htfcss: cmssi  font-style: italic; font-family: sans-serif;
\EndVerbatim
\EndExample

\ExitHPage{up}
\EndHPage{}
 too much sophistication.


A font of \TeX{} may have more than one htf font to map
to. The search  for a desired version can be regulated
within \Link{alt-htf}{}scripts\EndLink.

%%%%%%%%%%%%%%%%%%%%%%%%
\DocChapter{Scripts}
%%%%%%%%%%%%%%%%%%%%%

Scripts produce the content in verbatim format with  no decorations.

%-------------------------------------------------------------------

\List{button}

\Item{\tt
\--ScriptEnv//{{\tt\char92}}/%
\string\ScriptEnv
\string{{\sl environment}\string}
\string{{\sl prefix}\string}
\string{{\sl postfix}\string}
}\ContItem

This command defines a \LaTeX{} environment

\Template
\char92begin\string{{\sl environment}\string}\BR
{\sl body}\BR
\char92end\string{{\sl environment}\string}%
\EndTemplate

\IgnoreIndent which outputs its body in plain format, between the specified 
prefix and postfix.

\Item
{\tt \--ScriptCommand//{{\tt\char92}}/%
\string\ScriptCommand
\string{{\char92\sl command}\string}
\string{{\sl prefix}\string}
\string{{\sl postfix}\string}
}\ContItem

This command defines an environment

\Template
{\char92\sl command}\BR
{\sl body}\BR
{\char92End\sl command}
\EndTemplate

\IgnoreIndent which outputs its body in plain format, between the specified 
prefix and postfix.

\Item{\tt
\--JavaScript//{{\tt\char92}}/%
\string\JavaScript...\string\EndJavaScript}\ContItem

\EndList







%%%%%%%%%%%%%%%%%%%%%%%%%%%%%%%%%%%
\DocChapter{Configurable Hooks}
%%%%%%%%%%%%%%%%%%%%%%%%%%%%%%%%%%%




Much of the look and feel of \TeX4ht is achieved through
hooks that are introduced and configured with the following commands.


\List{button}

\Item {\tt
\--NewConfigure//{{\tt\char92}}/%
 \string\NewConfigure\string{{\it name}\string}[{\it
      i}]\string{{\it body}\string}} \ContItem 

Hooks are just macro names seeded within the bodies of other macros.
  This command introduces a configuration command for a group of
  hooks, whose cardinality is given by a digit {\it i} and whose
  name is provided by the first parameter. The body determines 
  the relationship between the hooks and the configurations
  provided for them.

\Item {\tt
\--Configure//{{\tt\char92}}/%
 \string\Configure\string{{\it name}\string}\string{{\it parameter-1}\string}...\string{{\it parameter-i}\string}}
\ContItem


\Code\xxxx{XXXX}<<<
\NewConfigure{try}[2]{%
   \def\hookI{#1}\def\hookII{#2}}
\def\try#1{\hookI#1\hookII}
\Configure{try}{* }{}  \try{ho}
\Configure{try}{}{ *}  \try{ha}
>>>   \OutputCode[log]\xxxx

\EXAMPLE
\input xxxx.log
\ContEXAMPLE
      \ShowCode-\xxxx\ShowIndent
\EndEXAMPLE





\EndList

 Block \Verb+{\begin}+\dots\Verb+\end+ environments may also be
 configured through the \Verb+\ConfigureEnv+ command, and lists may
 also employ the \Verb+\ConfigureList+ command.


For help configuring hooks already seeded in the system, compile the
source files in use with the \`'info' option  active and review the
information in log files.
Much of the information in the log files may also be obtained by
running \`'xhlatex mktex4ht.4ht' and reviewing the entries in the outcome
page \`'mktex4ht.html => index => mktex4ht'.






%%%%%%%%%%%%%%%%%%%%%%%%%%%%%%%%%%%%%%%%%
\DocChapter{General Configuration Files}
%%%%%%%%%%%%%%%%%%%%%%%%%%%%%%%%%%%%%%%%%

\--Files/Configuration//% 
\Tag{confFiles}{General Configuration Files}%
\Link{}{confFiles}{}\EndLink%
\--tex4ht.sty///%
A compilation starts by opening \''tex4ht.sty' and loading a
fraction of its code.  The main purpose of this phase is to request
the loading of the system at a later time (for instance, upon reaching
\''\begin{document}').  The motivation for the late loading is to
allow \TeX4ht to collect as much information as possible about the
environment requested by the source file, and help the system reshape
that environment with minimal interference from elsewhere.

\--Files/4ht//% 
The system uses two kinds of (4ht) configuration
files.  The files of the first kind mainly seed hooks into the macros
loaded by the source file (for instance, \''latex.4ht',
\''fontmath.4ht', and \''article.4ht').  The files of the second kind
mainly attach meaning to the hooks (for instance, \''html4.4ht',
\''unicode.4ht', and \''mathml.4ht').

\--tex4ht.4ht///% 
Different source files may request the loading of different style
files and in different orders.  The hook seeding files are loaded in
response to the loading of the style files, and in a compatible order.
Since the different style files may redefine the syntax and semantics
of macros, \TeX4t follows a similar route of defining and redefining
the hooks and their meanings.  

\--tex4ht.usr///% 
The meaning attaching files are normally requested
through option names introduced in the \''tex4ht.4ht' system file.
For instance, the \''mzlatex' command refers to the \''mozilla' option
name of \''tex4ht.4ht', and the \''oolatex' command refers to the
\''ooffice' option name.  The user may add option names, and redefine
old ones, within a new file named \''tex4ht.usr'.

 A new \''tex4ht.usr' file should group references to \''*.4ht'
 configuration files under arbitrarily chosen option names.  For that
 purpose, \''\Configure' commands similar to those provided in
 \''tex4ht.4ht' should be
 employed.

Variants of the htlatex-like scripts may be
produced in the following manner.

\List{a}
\item
Adjust the \''latex' (\''tex', \''texi') command of a 
given script to use a desired option name, 
and rename the new script.
\item
Make sure the \''tex4ht' and \''t4ht' commands  receive
appropriate switches in the new script.  (These commands show the 
available options when invoked without parameters.)
\EndList

The definition of new meaning assigning configuration files can be
considerable simplified by relying on literate programming 
and the file
\''mktex4t.4ht'.
For additional information, compile this file into a hypertext document,
visit the \`'index' page, and from there reach into
the  \`'mktex4ht' page.


\Example{}
\List{a}
\item Add a configuration file \''myooconfig.4ht' with the following  
content.

\Verbatim
    \exit:ifnot{jurabib}

    %%%%%%%%%%%%%%%%%%%%%%%%%%%%%%%%%%%%%%%%%%%%%%%%%%%%%%%%%%%%%%%%
                     \ConfigureHinput{jurabib}
    %%%%%%%%%%%%%%%%%%%%%%%%%%%%%%%%%%%%%%%%%%%%%%%%%%%%%%%%%%%%%%%%
    \def\jbNoLink#1#2{}
    \Configure{jblink}{\jbNoLink}{}
    \Configure{jbanchor}{\jbNoLink}{}

    %%%%%%%%%%%%%%%%%%%%%%%%%%%%%%%%%%%%%%%%%%%%%%%%%%%%%%%%%%%%%%%%

    \endinput\empty\empty\empty\empty\empty\empty
    %%%%%%%%%%%%%%%%%%%%%%%%%%%%%%%%%%%%%%%%%%%%%%%%%%%%%%%%%%%%%%%%
    \endinput
\EndVerbatim


\item Add to \''tex4ht.usr' the following script.

\Verbatim
      \Configure{myooffice}{%
         \:CheckOption{info}\if:Option
                     \Hinclude[*]{infoht4.4ht}\fi
         \:CheckOption{info}\if:Option
                     \Hinclude[*]{infomml.4ht}\fi
         \Hinclude[*]{ooffice.4ht}%
         \Hinclude[*]{unicode.4ht}%
         \Hinclude[*]{mathml.4ht}%
         \Hinclude[*]{ooffice-mml.4ht}%
         \Hinclude[*]{myooconfig.4ht}%
      }
\EndVerbatim

It is the ooffice script from \''tex4ht.4ht', with the added record
         \`'\Hinclude[*]{myooconfig.4ht}%'.

\item Invoke the compilations with a variant of the following form of
the \''oolatex' command.

\centerline{{\tt htlatex filename "xhtml,myooffice" "ooffice/!~-cmozhtf" "-coo"}}



\EndList

\EndExample

\EndHPage{here} 
\relax
the current document.
%
\item
\Link[http://www.cse.ohio-state.edu/\string
    ~gurari/publications.html]{}{}Conference presentations\EndLink
%
%\item In the .log files of the compilations.
\WAIT
\item 
\HPage{Examples}

\rightline{\ExitHPage{up}}
\EndHPage{} for using the \`'htlatex' and \`'httex' commands.
\ENDWAIT
\EndList



% 
% The documents \Link[http://www.cse.ohio-state.edu/\string
% ~gurari/docs/mml-00/mml-00.html]{}{}From \LaTeX{} to MathML and Back
% with \TeX4ht and Passive\TeX\EndLink,
% \Link[http://www.cse.ohio-state.edu/\string
% ~gurari/tug99/]{}{}\LaTeX{} to XML/MathML\EndLink{}, 
% \Link[http://www.cse.ohio-state.edu/\string
% ~gurari/tug97/tug97-h.html]{}{}A demonstration of TeX4ht\EndLink{},
% and
% \Link[http://www.cse.ohio-state.edu/\string
% ~gurari/docs/tug-03/tug-03.html]{}{}From LaTeX to MathML and Beyond\EndLink{}
%  may
% provide additional insight into the system (and some outdated
% details).




%%%%%%%%%%%%%%%%%%%%%%
\DocPart{Installation}
%%%%%%%%%%%%%%%%%%%%%%

To be installed, the system needs a port made up of native utilities
of \TeX4ht and of non-native utilities.  The easiest way to establish
an up to date port is to download an installed distribution
%  \HPage{}
% \List{*}
% \item Linux
% %%%%%
% \List{*}
% \item\Link[http://packages.debian.org/unstable/tex/tex4ht]{}{}Debian Linux\EndLink{} 
% % \item
% % For RedHut Linux at \Link[http://hunch.net]{}{}Hunch\EndLink:
% % \Link[http://hunch.net/tex4ht/tex4ht-1.0-1.src.rpm]{}{}source code\EndLink,
% % \Link[http://hunch.net/tex4ht/tex4ht-1.0-1.i386.rpm]{}{}i386 linux binary\EndLink
% %
% % From: John Langford <jcl@cs.cmu.edu>
% % To: Eitan Gurari <gurari@cse.ohio-state.edu>
% % Subject: Re: tex4ht
% % Date: Tue, 23 Sep 2003 17:59:14 -0400
% \item
% For Fedora
% at
% \Link[http://www.sbc.su.se/\string ~esjolund/tex4ht/]{}{}sbc.su.se\EndLink{}
% % From: Erik Sj�lund <erik.sjolund@sbc.su.se>
% % To: Eitan Gurari <gurari@cis.ohio-state.edu>
% % Cc: jl@tti-c.org, arne@sbc.su.se
% % Subject: tex4ht rpm on our home page
% % Date: 13 Jul 2004 12:08:08 +0200
% % maintains % http://xml2hostconf.sourceforge.net  
% %  and at
% %\Link[http://download.fedora.redhat.com/pub/fedora/linux/extras/development/SRPMS/]{}{}Fedora Extras\EndLink{} (invoke with `\Verb+yum install tetex-tex4ht+'; see \Link[http://www.fedoratracker.org/]{}{}http://www.fedoratracker.org/\EndLink)
% % From: Patrice Dumas <pertusus@free.fr> 
% % To: Eitan Gurari <gurari@cse.ohio-state.edu> 
% % Subject: tex4ht in fedora extras 
% % Date: Wed, 16 Nov 2005 00:47:04 +0100 
% %  
% % Hello, 
% %  
% % tex4ht (called tetex-tex4ht) has been accepted in fedora extras. The srpm 
% % is based on the one you provide on your site with changes by Michael A.  
% % Peters and me. 
% %  
% % The submission is there: 
% %  
% % https://bugzilla.redhat.com/bugzilla/show_bug.cgi?id=172521 
% %  
% % If you want to put the srpm on your website, I can give the link to you, and 
% % tell you when there are new srpm releases. 
% %
% %%%%%%%%%%%%%%%%%%%%%%%%%%%%%%%%% 
% %There is the main SRPM ftp repository 
% % http://download.fedora.redhat.com/pub/fedora/linux/extras/development/SRPMS/ 
% %            
% % And there is an index there... 
% % http://www.fedoratracker.org/ 
% %                                 
% % But as it is in fedora extras, a user only has to add the fedora extras  
% % repository to the yum config (done by default on fedora core 4) and do 
% %                                                   
% % yum install tetex-tex4ht
% \item
% \Link[http://packages.gentoo.org/search/?sstring=tex4ht]{}{}Gentoo\EndLink{}
% %  > http://packages.gentoo.org/search/?sstring=tex4ht 
% %  >  
% %  > The the ``ebuild'' script lists the file  
% %  > http://www.cse.ohio-state.edu/~gurari/TeX4ht/${P}.zip as the source file,  
% %  > where P == tex4ht. 
% % \item\Link[http://www.novell.com/products/linuxpackages/professional/tex4ht.html]{}{}SuSE\EndLink{} 
% %   (Linux) %(\Link[http://www.rpm.org/]{}{}RPM\EndLink)
% %%%
% % comp.text.tex #252972 (0 + 14 more)                          (1)--[1]
% % From: Karl Eichwalder <ke@suse.de>
% % [1] Re: tex4ht -> DocBook XML
% % Date: Fri Jun 14 07:24:43 EDT 2002
% % Lines: 17
% % 
% % Karl Eichwalder <ke@suse.de> writes:
% % 
% % > Using tex4ht I can convert LaTeX files into DocBook XML.  Great tool!
% % 
% % Forgot to mention: Here you may find a test unofficial upgrade package
% % for SuSE Linux 8.0 (i386):
% % 
% % ftp://ftp.suse.com/pub/people/ke/8.0-i386/
% % 
% % 45e049338013222baade6c3eaa9e8272  tex4ht-20020613-0.i386.rpm
% % 13a63ea28bc563eaef5df7c8f2ce779d  tex4ht-20020613-0.src.rpm
% % 
% % --
% % Linux frechet 2.4.18-4GB #1 Fri Apr 5 15:14:39 UTC 2002 i686 unknown
% %   1:22pm  up 66 days, 22:38, 13 users,  load average: 0.25, 0.16, 0.05
% %                                              work    :      ke@suse.de
% % Karl Eichwalder                              home    : keichwa@gmx.net
% %%%
% \EndList
% %%%%%%
% \item Mac
% %%%%%
% \List{*}
% \item\Link[http://fink.sourceforge.net/pdb/package.php/tex4ht]{}{}Fink\EndLink{} 
% \item \Link[http://www.rna.nl/]{}{}R\&A\EndLink ,
% \Link[http://ii2.sourceforge.net/]{}{}i-Installer\EndLink
% \EndList
% %%%%%%
% \item MS Windows
% %%%%%
% \List{*}
% \item  \Link[http://www.miktex.org/]{}{}Mik\TeX\EndLink{} 
% %\item \Link[ftp://ftp.simtel.net/pub/simtelnet/gnu/djgpp/v2apps/tex/]{}{}Simtelnet\EndLink{}
% \item \Link[http://www.metz.supelec.fr/\string
%         ~popineau/xemtex-1.html]{}{}XEm\TeX\EndLink{} 
% \EndList
% %%%%%%%%%%%%
% \item\Link[http://www.tug.org/texlive/]{}{}\TeX{} Live\EndLink{}
% % \item\Link[http://4tex.ntg.nl/]{}{}4all\TeX\EndLink{}
% 
% %\item \Link[https://dev.livingreviews.org/repos/tex4ht/trunk/]{}{}livingreviews\EndLink
% \EndList{}
% \EndHPage{}
 of the system, and %%%%%%%%%%%%%%%%%%%%%%%%%%%%%%%%%%%
\NextFile{\jobname-upgrade.html}\HPage{upgrade}\ExitHPage{}
\bgroup                       %%%%%%%%%%%%%%%%%%%%%%%%%%%%%%%%%%%%%%%

\parindent=0pt
\def\TEST{\item Issue the `{\tt htlatex test}' command.  If the compilation fails, fix the problem
encountered before proceeding.}

\InstallSection{TeX4ht Upgrading}



The following instructions describe how TeX4ht can be upgraded
employing Unix conventions. Apply only
the steps that pertain for the outdated parts of the TeX4ht environment. 



{\bf Note.} The target directory paths in the following instructions might differ
from those in your platform. If that is the case, adjust the paths in the
instructions to match the ones in your machine.



\List{1}





% 1. Make a new temporary directory, download the following files there 
%  
%        http://www.cse.ohio-state.edu/~gurari/TeX4ht/fix/newt4ht.zip 
%  
%     and unzip them. 
%  


\item  Create a file {\tt test.tex} in a work directory:


\Verbatim
\documentclass{article}
\begin{document}
  Hello \(alpha\)
\end{document}
\EndVerbatim




\item
 Create a new temporary directory, download  
\Link[tex4ht.zip]{}{}tex4ht.zip\EndLink{}
to that directory, and uncompress the file. 



\item {\bf[Configuration files]} Move the files from 
  
\centerline{\tt
        texmf/tex/generic/tex4ht/* }

  
     to 
  
\centerline{\tt
        /usr/share/texmf/tex/generic/tex4ht/.}



\List{*}
\item
 A super user access mode might be required  for the updating.

\item   The relative paths here and below are with respect to the 
   temporary directory where tex4ht.tar.gz got uncompressed.


\item
 The messages `htlatex test' issues during a
compilation  can be consulted to find out whether a path alternative to
\Verb!/usr/share/texmf/tex/generic/tex4ht/!
should be used.
\EndList


  
\item Refresh the latex registry.  Probably one of  the following 
     commands would do the job. 


\centerline{\tt
          texhash, 
          mktexlsr, initexmf -u}

\TEST



\item
 {\bf[Postprocessors tex4ht.c, t4ht.c]}
 Find where the executables {\tt tex4ht} and {\tt t4ht} reside (e.g., with 
   the commands `{\tt which tex4ht}' and `{\tt which t4ht}')
 
Temporarily save the files, say, by renaming them to {\tt tex4ht.old} and {\tt t4ht.old}. 

On Linux platforms,  set the following executable files 
 
\Verbatim
       bin/linux/tex4ht 
       bin/linux/t4ht 
\EndVerbatim

\noindent   as replacement to the ones renamed.
 
On non-Linux platforms, new executables should be compiled for the replacements.


\TEST
 
\item 
 {\bf[Fonts]}
  Remove the subdirectory
 
\centerline{\tt
     /usr/share/texmf/tex4ht/ht-fonts} 

 
  and introduce there instead the subtree  `{\tt ht-fonts}' obtained from 
  the zip file.


 The messages `htlatex test' issues during a
compilation  can be consulted to find out whether a path alternative to
\Verb!/usr/share/texmf/tex4ht/ht-fonts!
should be used.


 
 
 
\item Refresh the latex registry.

\centerline{\tt
          texhash, 
          mktexlsr, initexmf -u}


\TEST

\item
 {\bf[Environment File]}
 Determine the location of \Verb!tex4ht.env! 
from the messages issued during the compilation invoked  by  `{\tt htlatex test}'

 
\item Temporarily save the \Verb!tex4ht.env! file (under a different name?) 
    and set the following file as a replacement 
 
\centerline{\tt       texmf/tex4ht/base/unix/tex4ht.env} 


\item Replace the substrings

\centerline{\tt   \%\%\string~/texmf-dist}

with the substrings

\centerline{\tt /usr/share/texmf}

within the records of tex4ht.env that invoke Java.


\item
 {\bf[Invocation Scripts]}
 Find where {\tt htlatex}
 resides (e.g., `{\tt which htlatex}').  Save the files
\Verb!htlatex!, \Verb!httex!, \Verb!httexi!, and \Verb!htcontext!
   elsewhere and get new scripts from
 
\centerline{\tt
        bin/unix/}
 
   for a replacement.   (Variants of the above scripts are
available at {\tt
        bin/ht/unix/}.)
 
\TEST


\item
 Find where {\tt mk4ht} resides (e.g., `{\tt which mk4ht}').  Save the file 
   elsewhere and set the file 
 
\centerline{\tt
        bin/unix/mk4ht} 

 
   for a replacement.
 

\item Issue the command 
 `{\tt
        mk4ht htlatex test}'.
   Check that the outcome is the same way as from the `{\tt htlatex test}' command. 







\item
 {\bf[Backend Filters]}
 If they exist, delete the subdirectories 
 
\Verbatim
        /usr/share/texmf/tex4ht/bin 
        /usr/share/texmf/tex4ht/xtpipes 
\EndVerbatim


\item Add the subtrees
 
\Verbatim
        texmf/tex4ht/bin 
        texmf/tex4ht/xtpipes 
\EndVerbatim


    to 
    
\centerline{\tt
    /usr/share/texmf/tex4ht/} 


\item  The platform should have Java installed, with version no earlier than 1.5 


\TEST

% 
%  
% 8. Issue the command 
%  
%        htlatex filename  "xhtml,emspk" " -cemspkhtf -s4es" "-cemspk" 
%  
%    and email me the messages it gives.  Don't worry about 
%    error messages 
% 
% 
% 
% 
% 1.  About three lines before the end of 
%  
%        /etc/tex4ht/tex4ht.env 
%  
%    there is a record 
%  
%        java -classpath  ~/tex4ht.dir/texmf/tex4ht/bin/tex4ht.jar xtpipes -i 
%             ~/tex4ht.dir/texmf/tex4ht/xtpipes/ -o filename.html filenam.tmp 
%  
%    Replace there the two prefixes 
%  
%          ~/tex4ht.dir 
%  
%    with 
%  
%          /usr/share 
%  
% 2. Issue the command 
%  
%        htlatex filename  "xhtml,emspk" " -cemspkhtf -s4es" "-cemspk" 
%  
%    and email me the messages it gives. 
%  
%  
% The 
%  
%      htlatex filename  "xhtml,emspk" " -cemspkhtf -s4es" "-cemspk" 
%  
% is too painful to use.  The 
%  
%      mk4ht eslatex filename 
%  
% can be invoked for the translations. Another alternative is to invoke 
% the command 
%  
%      eslatex filename 
%  
% However, for this last command the script 
%  
%      bin/ht/unix/eslatex 
%  
% needs to be installed in the same directory as htlatex, with modifications 
% similar to those made in htlatex. 
%  
% A few comments. 
%  
% 1. The html file(s) are to be fed into emacspeak.  If you encounter 
%    latex commands that are not translated into emacspeak 
%    at all or to your liking,  I'll be glad to fix the configurations. 
%    However, I'll need  miniature latex files demonstrating the problems. 
%  
% 2. I have a variant of the system that instead of producing ACSS 
%    translations for emacspeak provides JSML output for java-based voice 
%    browsers 
%     
% (http://java.sun.com/products/java-media/speech/forDevelopers/JSML/JSML.html). 
%    I hope to have a simple browser for the job in the coming 
%    weeks and will announce then the JSML option. 

\EndList

Contributed instructions:

\List{*}
\item \Link[http://www.exstrom.com/journal/comp/tex4ht.html]{}{}%        
 OpenSuSE 10.2\EndLink
\EndList

\egroup                       %%%%%%%%%%%%%%%%%%%%%%%%%%%%%%%%%%%%%%
\EndHPage{} 
%%%%%%%%%%%%%%%%%%%%%%%%%%%%%%%%%%%
 it with the
files provided here.

 Establishing ports from scratch for 
%
\NextFile{\jobname-unix.html}\HPage[]{Unix}
%
\ExitHPage{}

\InstallSection{A Setup for Unix Environments}




\List{a}
\item Establish a  directory, say,  `{\tt \string~/tex4ht.dir}'.


\item
\--Files/Download//%
 Download the file \Link[tex4ht.zip]{}{}tex4ht.zip\EndLink{}
 into the directory \''tex4ht.dir' and \UNZIP{}  it.





\Code\Uht{}<<<
$1 $2
$1 $2
$1 $2
tex4ht $2
t4ht $2  $3     # -d~/WWW/temp/ -m644 
>>>

\OutputCode[foo]\Uht

%%%%%%%%%%%%%%%%%%%%%%%%%%%%%%%%%%%%%%%%%
\setup{Compile the Postprocessors}
%%%%%%%%%%%%%%%%%%%%%%%%%%%%%%%%%%%%%%%%%




\item
\--tex4ht.c/Compiling//\--t4ht.c/Compiling//Compile 
\''~/tex4ht.dir/src/tex4ht.c' into an executable \''tex4ht' file with a
command similar to the following one.
%, where \`'path' stands for the absolute
% path to  directory \''tex4ht.dir'.


\'+gcc -o tex4ht tex4ht.c 
-DENVFILE='"~/tex4ht.dir/texmf/tex4ht/base/unix/tex4ht.env"'
 -DHAVE_DIRENT_H+


The switch
 \`+-DENVFILE='"~/tex4ht.dir/texmf/tex4ht/base/unix/tex4ht.env"'+ may be omitted,
if the program can reach the environment file
in an \Link{envloc}{}alternative\EndLink{} manner.

% #define ENVFILE "/n/soda/export/0/gurari/tex4ht.dir/tex4ht.env" 
% #define HTFDIR  "/n/soda/export/0/gurari/tex4ht.dir"



\item Compile  \''~/tex4ht.dir/src/t4ht.c'  with a command similar to following one.
%, where \`'path' stands for the absolute
% path to  directory \''tex4ht.dir'.


\'+gcc -o t4ht t4ht.c 
-DENVFILE='"~/tex4ht.dir/texmf/tex4ht/base/unix/tex4ht.env"'+

Again, the switch
 \`@-DENVFILE='"~/tex4ht.dir/texmf/tex4ht/base/unix/tex4ht.env"'@ 
may be omitted,
if the program can reach the environment file
in an \Link{envloc}{}alternative\EndLink{} manner.



\item
 Move the
executable files \''tex4ht' and \''t4ht' to
directory \''~/tex4ht.dir/bin/unix/'.
%
\setup{Update the Pointers in the Environment File}
\item
\--Fonts/tfm//%
\--tex4ht.env/Unix//Replace in
\Link[tex4ht-env-unix.txt]{}{}%
\''~/tex4ht.dir/texmf/tex4ht/base/unix/tex4ht.env'\EndLink{}
the line(s)  starting with the character \`'t', with alternative lines
which state
what directories should be searched for the tfm files
of \TeX{} and \LaTeX. The directory names must  be preceded with the character
\`'t' at column 1 and, if their subdirectories are also to be searched, 
the names should be appended with the character \`'!'
(\Link{alt-tfm}{}insight\EndLink{}).

\item
\--Fonts/htf//%
 If needed, adjust the paths in the \`'i' records of
\''tex4ht.env'.  These records are used for searching htf fonts, and they
are similar to the \`'t'  (\Link{alt-htf}{}insight\EndLink{}).




\item
\--tex4ht.fls///%
  The entry \`'l~/tex4ht.dir/filename'
  in \''tex4ht.env' points to the address where the
\Link{TEX4HTWR}{}bookkeeping file\EndLink{} should reside. 
  Modify the path to fit
  your platform.  The character \`'l' should precede the address, and
  be placed at the first column.
 Make sure the access mode
of the directory permits writing into files.


\setup{Update the Bitmap Generating Scripts in the Environment File}

\item
\--Bitmaps and graphics/png//%
\--dvips/png//%
The file \''tex4ht.env' contains the following default script, of
calls to system utilities for translating dvi pictures into
bitmaps.\Link{}{dv2png}\EndLink{}

\Verbatim
Gdvips -Ppdf -mode ibmvga -D 110 -f %%1 -pp %%2  > zz%%4.ps
Gconvert zz%%4.ps -trim +repage -density 110x110 -transparent '#FFFFFF' %%3
Grm zz%%4.ps 
\EndVerbatim

The entry \''%%1' is a parameter  refering to a dvi file, the \''%%2' is a
parameter indicating a page number,  the \''%%3' is a parameter
standing for an output file name,
and \''%%4' is a parameter providing the jobname.

You may replace this script with an alternative sequence of system
calls.  In such a case, place one command per line, and mark
each of these lines with the character \`'G' at the first column.
The literate version tex4ht-env.tex of tex4ht.env offers a few
suggestions.

The \Link[http://www.radicaleye.com/dvips.html]{}{}dvips\EndLink{}
utility  translates dvi files into postscript.  The
\''convert' utility, provided within the distribution of
\Link[http://www.imagemagick.org/]{}{}ImageMagick\EndLink{},
translates postscript files into png. 

The script 
employs the Metafont mode \`'ibmvga' of resolution \`'110';
the available modes are listed in file \''modes.mf' of Metafont.


Use the option `{\tt -crop 0x0 +page}' or `{\tt -crop 0x0 +repage}'
instead of `{\tt -trim}' for old convert utilities that do not
recognize the latter argument.


\item Instead of employing the G scripts, glyphs can rely on
specialized 
\HPage{F scripts}
\List{*}
\item For instance
\Verbatim
Fdvips -Ppdf -mode ibmvga -D 110 -f %%1 -pp %%2  > zz%%4.ps
Fconvert zz%%4.ps -trim +repage -density 110x110 -transparent '#FFFFFF' %%3
Frm zz%%4.ps
\EndVerbatim


\item The specialized scripts may, for instance, maintain global caches of
   png bitmaps for cutting down on recompilation time.  \EndList
  \EndHPage{} of similar nature for creating bitmaps.

\item
\--Bitmaps and graphics/fonts//%
\--Fonts/Bitmaps//%
\--Pictures/Bitmaps//%
\--tex4ht.c/LGTYP//%
The bitmap formats can be controlled by a `g' record of tex4ht.env,  
a `-g' switch
 of {\tt tex4ht.c},
and a -LGTYP switch in the compilation of tex4ht.c.
 The default setting assumes the `png' format.

\setup{Update the Other Scripts in the Environment File}



\item  \Link{}{mvscript}\EndLink{}If needed, replace
 the scripts \`'Mmv %%1 %%2%%3' and \`'Ccp %%1 %%2%%3'
 in \''tex4ht.env' with alternative scripts 
for moving and copying files.  The parameter \''%%1' stands for the
source file(s), the parameter \''%%2' provides the target directory
name, and the parameter \''%%3' refers to the target file name(s).



% \item  If needed, replace the script
% \`'Ecp empty.gif %%1%%2' in \''tex4ht.env' with an
%  alternative script 
% for replacing empty files.  The parameter \''%%1' stands 
% for the 
% target directory name, and the parameter \''%%2' refers to
% the empty file name.







\item  \Link{}{accscript}\EndLink{}If needed,
replace the script \`'Achmod %%1 %%2%%3' 
in \''tex4ht.env' with an  alternative script for changing access mode of
files.  The parameter \''%%1' stands for access mode, the parameter
\''%%2' refers to a directory name, and the parameter \''%%3' refers
to file(s).

\item \--Validation///% 
\relax\relax\--!//{\string\tt\char \space 37\char \space 37\char \space 126}\string\csname\space :gobble\string\endcsname/%
Postprocessing of files can be requested with
\`'.' scripts.  The files are selected by their extension names, as
 listed following the period symbols.   The parameter
\`'%%1' provides the file names, and the parameter \`'%%0' provides
the jobnames. 
  Under kpathsea, the substring `%%~' may be employed 
 to indirectly obtain the value of \Verb=${SELFAUTOPARENT}=.



%%%%%%%%%%%%%%%%%%%%%%%%%%%%%%%%%%%%%%%%%%%%%%%%%%%%%%%%%%%%%%%%
The environment file
\Link[tex4ht-env-unix.txt]{}{}tex4ht.env\EndLink{} 
offers the following inactive dot  script  for 
{\bf validating} output of compilations.


\Verbatim
<validatehtml>
 .html xmllint --noout --valid --html %%1.html
</validatehtml>
<validate>
 .xml xmllint --noout --valid %%1.xml
 .html xmllint --noout --valid %%1.html
 .css mycssparser %%1.css
</validate> 
\EndVerbatim


The dot script may be activated in the following manner.

\List{1}

\item
Bind a CSS validator to the .css record. 
(\Link[http://jigsaw.w3.org/css-validator/DOWNLOAD.html]{}{}{\tt
http://jigsaw.w3.org/css-validator/DOWNLOAD.html}\EndLink)


\item  Remove the leading space characters from the above record. 




\EndList


%%%%%%%%%%%%%%%%%%%%%%%%%%%%%%%%%%%%%%%%%%%%%%%%%%



\item
 Postprocessing of  files can also be requested with \`'X' 
scripts.  The file names are accessed through the parameter \`'%%1',
and their extensions through the parameter \`'%%2'.  



\setup{Set the Script Files}
\item
Ensure proper paths within the htlatex-like scripts in
\`'~/tex4ht.dir/bin/unix/'.
       If you use a command different than \''latex' for compiling \LaTeX{}
       source files, 
fix also
the references to \''latex' in the scripts. 
Check also the appropriateness of the commands for compiling \TeX{} and
\TeX i files.

\--htlatex/number of compilations//%
\--Tables/number of compilations//%
The default scripts provide for three calls to LaTeX.
  The file \Verb+tex4ht-auto-script.tex+ in
\Link[tex4ht-lit.zip]{}{}tex4ht-lit.zip\EndLink{} offers examples of
bash scripts that automatically determine the number of compilations
needed from LaTeX (contributed by
Kai-Mikael J\"a\"a-Aro).
%
                              \setup{Make the System Globally Known}
%
\item
Inform the operating system where  the scripts reside, say, by
 adding the directory   `{\tt \string ~/tex4ht.dir/bin/unix/}' into  the
\''path' variable within the \''.login' file. For instance,

  \''set path=($path ~/tex4ht.dir/bin/unix/)'

%
\item Let \LaTeX{} and \TeX{} know where the new style files reside, say, by
adding the directory   `{\tt \string~/tex4ht.dir/texmf/tex/generic/tex4ht/}' to  the
environment variable \''TEXINPUTS'. For instance, 

\''setenv TEXINPUTS .:~/tex4ht.dir/texmf/tex//:/usr/local/share/texmf/tex//'

\item
If your \TeX{} system uses a registry database to locate files, make sure to refresh
it (e.g., run {\tt texhash} for kpathsea, 
or {\tt mktexlsr} or `{\tt initexmf -u}' for some Linux systems).


%\Verbatim
%There is an environment variables, TEXINPUTS, which controls from where
%TeX should take its input files; if not set, TeX has a default value
%(at CIS, the value is ".:/usr/local/lib/tex/macros//").
%
%So if we set
%       setenv TEXINPUTS .:/usr/local/lib/tex/macros//:~gurari/cis788//
%then it would continue looking where it used to (in the current dir and
%in the system dir) plus use whatever possible from your directory (the
%added convenience is that if we're going to use more than just IprTeX.sty
%we wouldn't need to link every file).
%===
%Albert Meltzer
%%-------------------
% > latex (teTeX) cann't find the style files and so on.
% > So I don't know where to set the style files and
% > what kind of commands I have to give so that
% > latex can handle these files.
%
%

\item Some output modes assume  
\Link[http://www.sun.com/]{}{}Java\EndLink{}  
is also available in the computer in use. 
%%%%%%%%%%%%%%%%%%%%%%%%%%%%
\setup{Test the Installation}
%%%%%%%%%%%%%%%%%%%%%%%%%%%%%%%%%
\item 
Move \''testa.tex' and  \''testb.tex'
from
\''~/tex4ht.dir/temp/'
 to your work directory
\item Compile  \`'testa.tex'  with the command  \`'ht latex testa'
\item Compile  \`'testb.tex'  with the command  \`'htlatex testb'


\Code\demo{}<<<
% \def\CALL{{tex}{latex}}
% \let\PC=Y
% \let\DOS=Y

%%%%%%%%%%%%%%%%%%%%%%%%%%%%%%%%%%%%%%%%%%%%%%%%%%%%%%%%%%%%%
% demo.tex                                                  %
%                                                           %
% Please DON'T try to understand the code of this file---it %
% can be harmful to your eyes and brain!                    %
%                                                           %
%                                gurari@cse.ohio-state.edu  %
%                    http://www.cse.ohio-state.edu/~gurari  %
%%%%%%%%%%%%%%%%%%%%%%%%%%%%%%%%%%%%%%%%%%%%%%%%%%%%%%%%%%%%%


\ifx \SCRIPT\UnDef
   \ifx \PC\UnDef
       \def\SCRIPT{
          \Needs{ht \TEX\space
                 \FN\space
                 }
       }
   \else          % dos batch file
       \def\SCRIPT{
             \Needs{call ht \TEX\space  \FN}
             \Needs{if exist \FN.txt del \FN.txt}
             \Needs{ren \FN.tex \FN.txt}
       }
\fi\fi

%       \def\SCRIPT{
%             \Needs{call ht \TEX\space  \FN}
%             \Needs{if exist \FN.txt del /q \FN.txt}
%             \Needs{ren /q \FN.tex \FN.txt}
%       }




\ifx \PC\UnDef             
\else 

\openin15=clean.bat
\ifeof15
\closein15
\immediate\openout15=clean.bat
\immediate\write15{DEL hti.bat}
\immediate\write15{DEL htii.bat}
\immediate\write15{DEL htiii.bat}
\immediate\write15{DEL htiv.bat}
\immediate\write15{DEL htv.bat}
\immediate\write15{DEL htvi.bat}
\immediate\write15{DEL htvii.bat}
\immediate\write15{DEL htdemo.bat}
\immediate\write15{DEL i.*}
\immediate\write15{DEL ia.*}
\immediate\write15{DEL ii.*}
\immediate\write15{DEL iia.*}
\immediate\write15{DEL iii.*}
\immediate\write15{DEL iiia.*}
\immediate\write15{DEL iv.*}
\immediate\write15{DEL iv2.*}
\immediate\write15{DEL iva2.*}
\immediate\write15{DEL iva.*}
\immediate\write15{DEL demo*.aux}
\immediate\write15{DEL demo*.bat}
\immediate\write15{DEL demo*.dvi}
\immediate\write15{DEL demo*.htm*}
\immediate\write15{DEL demo*.css}
\immediate\write15{DEL demo*.lg}
\immediate\write15{DEL demo*.idv}
\immediate\write15{DEL demo*.log}
\immediate\write15{DEL demo*.otc}
\immediate\write15{DEL demo*.toc}
\immediate\write15{DEL demo*.txt}
\immediate\write15{DEL demo*.xre*}
\immediate\write15{DEL tex4ht.ps}
\immediate\write15{DEL v.*}
\immediate\write15{DEL va.*}
\immediate\write15{DEL vi.*}
\immediate\write15{DEL viIa.*}
\immediate\write15{DEL via.*}
\immediate\write15{DEL vii.*}
\immediate\write15{DEL viia.*}
\immediate\write15{DEL clean.bat}
\immediate\closeout15
\else \closein15 \fi

\fi

\ifx \DOS\UnDef
       \def\HTML{html}  \def\ARG{}
 \else \def\HTML{htm}   \def\ARG{[htm]}
 \fi

\ifx \documentclass\undef \else
   \documentclass{article}
\fi
\input tex4ht.sty
\def\temp{htm}\ifx \HTML\temp
   \Preamble{htm,no_,no^}         
\else
   \Preamble{html,no_,no^}          
\fi
   \ifx \documentclass\undef \else
      \begin{document}
   \fi

   \Configure{Needs}{l. \the\inputlineno\space--- needs --- "#1" ---}


\newlinechar=`\^^J

\ifx \WWW\UnDef \let\WWW=\empty \fi
\ifx \DOS\UnDef\else
%      \Needs{CLS} % clear screen
\fi


\ifx \CALL\UnDef
   \def\TEX{\ifx \documentclass\undef \else la\fi tex}
\else
   \def\TEX#1#2{\def\TEX{\ifx \documentclass\undef #1\else #2\fi}}
   \expandafter\TEX\CALL
\fi

\edef\type{\ifx \documentclass\undef \else  a\fi}

\def\\#1\\{\expandafter\noexpand\csname#1\endcsname}
\catcode`\#=11 \def\Sharp{#} \catcode`\#=6
\def\source#1#2{
   \def\doc##1##2##3{#2}
   \HAdvance\FileNo by 1
   \NextFileName
   \immediate\openout15=\FN.tex
   \immediate\write16{--- Writing file \FN.tex}%
   \SCRIPT
   \immediate\write15{\ifx \documentclass\undef 
      \noexpand \input tex4ht.sty                            ^^J
      \noexpand \Preamble{\HTML#1}                            ^^J
      \noexpand \EndPreamble                              ^^J^^J\else
      \noexpand \documentclass{article}                ^^J
      \noexpand \usepackage\ARG{tex4ht}                            ^^J
      \noexpand \begin{document}                           ^^J^^J\fi
      \doc{PSALMS (131:1)}
          {My heart is not haughty, nor                      ^^J
             mine eyes lofty:}
          {neither do I                                      ^^J
             exercise  myself in great                       ^^J
             matters, or in things too high                  ^^J
            for me.}                                      ^^J^^J
      {\\it\\ Example \FileNo\space (out of \examples)}   ^^J^^J
      \noexpand \end\ifx \documentclass\undef \else
          {document}\fi    }
   \immediate\closeout15
   \Link[\jobname\FileNo\type f.\HTML\space target="X"]{}{}\FileNo\EndLink
   \FileStream+{\jobname\FileNo\type f.\HTML}
      \HorFrames{*,*}
         \Frame[\romannumeral\FileNo\type.txt]{}
         \Frame[\romannumeral\FileNo\type.\HTML]{}
      \EndPreamble
         \Link[\romannumeral\FileNo\type.txt]{}{}outcome\EndLink{} /
         \Link[\romannumeral\FileNo\type.\HTML]{}{}source\EndLink{} /
         \Link{1}{}pointers\EndLink
   \EndFileStream{\jobname\FileNo\type f.\HTML}
}

\def\NextFileName{\edef\FN{\romannumeral\FileNo\type}}

\HAssign\examples=0\LikeRef{count}
\HAssign\FileNo=1  \NextFileName

  \VerFrames{7*,*}
  \Frame[\jobname\FileNo\type f.\HTML\space NAME="X"]{}
  \Frame[ NAME="Y"]{1}

\EndPreamble

\Link[\FN.txt]{}{}source\EndLink{}  /
\Link[\FN.\HTML]{}{}outcome\EndLink{} /

\HAssign\FileNo=0

\edef\temp{\noexpand\HPage<\jobname p.\HTML>}
\temp{Pointers}\Link{}{1}~\EndLink


Goto to Example:
\source{%,fonts
}{#1^^J^^J#2^^J^^J#3}
  \def\HR{^^J\\HCode\\{<HR>}^^J}
\source{%,fonts
}{\HR #1 \HR #2 \HR #3 \HR}
\source{%,fonts
}{{\\bf\\#1}^^J^^J{\\it\\#2}^^J^^J{\\tt\\#3}}
\source{%,fonts
}{\\HPage\\{#1}^^J^^J#2^^J^^J#3^^J^^J
    \\ExitHPage\\{}\\EndHPage\\{}}
%
\ifx \documentclass\undef
\source{%,fonts
}{\\Contribute\\{halign}{BORDER<>0 1 < 0 2 - 0 3 > }^^J
\\halign\\{&\Sharp\\cr\\#1^^J&^^J#2^^J&^^J#3\\cr\\}}
\else
\source{%,fonts
}{\string\begin{tabular}{|l|c|r|}^^J
#1^^J&^^J#2^^J&^^J#3^^J
\string\end{tabular}}
\fi
%

  \def\d{\\scriptstyle\\ \\diamondsuit\\
    \\diamondsuit\\ \\diamondsuit\\ }
\ifx \documentclass\undef
\source{%,fonts
}{^^J
\\settabs\\ 6 \\columns\\^^J
\\+\\PSALMS & (131:1)\\cr\\^^J
\\+\\My heart & is not haughty,\\cr\\ ^^J
\\+\\nor mine &  eyes lofty:\\cr\\^^J
\\settabs\\ \\+\\ neither & do I exercise & \\cr\\^^J
\\+\\neither & do I exercise \\cr\\^^J
\\+\\        & myself in & great matters, \\cr\\^^J
\\+\\or      & in things & too high for me. \\cr\\^^J
}
\else
\source{%,fonts
}{^^J
\\begin\\{tabbing}^^J
XXXXXXXXX \\=\\\\kill\\^^J
PSALMS \\>\\ (131:1) \string\\^^J
My heart \\>\\ is not haughty, \string\\^^J
nor mine \\>\\  eyes lofty:\string\\^^J
neither \\=\\ do I exercise \\=\\ \\kill\\^^J
neither \\>\\ do I exercise \string\\^^J
        \\>\\ myself in \\>\\ great matters, \string\\^^J
or      \\>\\ in things \\>\\ too high for me. \string\\^^J
\\end\\{tabbing}^^J
}
\fi
%
\source{%,fonts
}{\ifx \documentclass\undef\space
   \string\def\string\(\string{\string\Picture+\string{\string}\string
   $\string}^^J
   \string\def\string\)\string{\string$\string\EndPicture\string}^^J
\fi
\\Picture\\+{ ALIGN="LEFT"}\\vtop\\{\\hsize\\=2in^^J
     \\hrule\\\\smallskip\\ #1\\smallskip\\\\hrule\\}\string
     \EndPicture^^J #2^^J\\(\\ \d \\)\\ ^^J#3 $\d$}
\EndHPage{}

 \Tag{count}{\FileNo}

\end{document}
\endinput
>>>

\OutputCode[tex]\demo






\Code\testa{}<<<
\documentclass {article} 
  \usepackage {tex4ht} 
\begin {document} 

A single $\heartsuit$ and a full 
suit: \( \clubsuit \diamondsuit
\heartsuit \spadesuit \).

\end {document}
\endinput
>>>

\OutputCode[tex]\testa



\Code\testb{}<<<
\documentclass {article} 
\begin {document} 

A single $\heartsuit$ and a full 
suit: \( \clubsuit \diamondsuit
\heartsuit \spadesuit \).

\end {document}
\endinput
>>>

\OutputCode[tex]\testb















% \Code\HT{}<<<
% echo "**************** Temporary ht ***********************"
% $1 $2
% $1 $2
% $1 $2
% tex4ht $2
% t4ht $2  -m644 -d?DIR
% echo "**************** Temporary ht ***********************"
% >>>
% 
% \OutputCode[log]\HT
% \csname SysNeeds\endcsname{"mv ht SVht"}
% \csname SysNeeds\endcsname{"mv HT.log ht"}
% \csname SysNeeds\endcsname{"chmod 700 ht"}
%
%
% \csname SysNeeds\endcsname{"mv SVht ht"}
% \csname SysNeeds\endcsname{"chmod 700 ht"}





\EndList







\EndHPage{} and
\NextFile{\jobname-mswin.html}\HPage[]{MS Windows}
%
\ExitHPage{}

\InstallSection{A Setup for MS Windows} %  / DOS


\List{a}
\item 
\--Files/Download//%
Establish a directory, say, \`'c:\tex4ht'.


\item
\--Files/Download//%
 Download the file \Link[tex4ht.zip]{}{}tex4ht.zip\EndLink{}
 into the directory \''tex4ht.dir' and \UNZIP{}  it.




\Code\Wht{}<<<
%1 %2 
%1 %2 
%1 %2 
tex4ht %2 -ic:\tex4ht\texmf\tex4ht\ht-fonts\%3 -ec:\tex4ht\texmf\tex4ht\base\win32\tex4ht.env
t4ht %2  -ec:\tex4ht\texmf\tex4ht\base\win32\tex4ht.env
>>>

\OutputCode[tab]\Wht

%%%%%%%%%%%%%%%%%%%%%%%%


\setup{Update the Pointers in the Environment File}



%
\item
\--tex4ht.env/MS Windows//%
\--Fonts/tfm//%
 Replace in 
\Link[tex4ht-env-win32.txt]{}{}%
\''c:\tex4ht\texmf\tex4ht\base\win32\tex4ht.env'\EndLink{}
the lines starting with the character
  \`'t', with alternative lines which state what directories should be
  searched for the tfm files of \TeX{} and \LaTeX. The directory names
  must be preceded with the character \`'t' at the first column. If
  their subdirectories are also to be searched, the names should be
  appended with the character \`'!'.

Note that long file names on MS Windows, and file names with spaces, 
might have short space-free aliases.  For instance, a directory name
\`'c:\progra~1\texmf' instead of \`'c:\program files\texmf'.
The alternative names  can be checked by issuing the \`'dir' command
in a DOS session.


\item
\--Fonts/htf//%
 If needed, adjust the paths in the \`'i' records of
\''tex4ht.env'.  These records are used for searching htf fonts, and they
are similar to the \`'t' records.




\setup{Update the Bitmap Generating Scripts}




\item
\--Bitmaps and graphics/png//%
The file \''tex4ht.env' contains the following default script, of calls
to system utilities for translating dvi pictures into png.

\Verbatim
Gif exist zz%%4.ps DEL zz%%4.ps >nul
Gif exist %%3 DEL %%3 >nul
Gdvips -E -Ppdf -mode ibmvga -D 110 -f %%1 -pp %%2 > zz%%4.ps
Gconvert zz%%4.ps -trim +repage -density 110x110 -transparent "#FFFFFF" %%3
Gif exist zz%%4.ps DEL zz%%4.ps >nul 
\EndVerbatim

You may replace this script with an alternative sequence of system
calls.  In such a case, place one command per line, and mark
each of these lines with the character \`'G' at the first column.
The literate version tex4ht-env.tex of tex4ht.env offers a few
suggestions.

The entry \''%%1' is a parameter referring to a dvi file, the \''%%2' is a
parameter indicating a page number,  the \''%%3' is a parameter
standing for an output file name,
and \''%%4' is a parameter providing the jobname.


% Some variants of the \''convert' utility
% require the  \''-transparent' switch, instead of \''-transparency'.

The \Link[http://www.radicaleye.com/dvips.html]{}{}dvips\EndLink{}
utility  translates dvi files into postscript.  The
\''convert' utility, provided within the distribution of
\Link[http://www.imagemagick.org/]{}{}ImageMagick\EndLink{},
translates  postscript files into png. 

The script 
employs the Metafont mode \`'ibmvga' of resolution \`'110';
the available modes are listed in file \''modes.mf' of Metafont.


Use the option `{\tt -crop 0x0 +page}' or `{\tt -crop 0x0 +repage}'
instead of `{\tt -trim}' for old convert utilities that do not
recognize the latter argument.

\item Instead of employing the G scripts, glyphs can rely on
specialized 
\HPage{F scripts}
\List{*}
\item For instance
\Verbatim
Fif exist zz%%4.ps DEL zz%%4.ps >nul
Fif exist %%3 DEL %%3 >nul
Fdvips -E -Ppdf -mode ibmvga -D 110 -f %%1 -pp %%2 > zz%%4.ps
Fconvert zz%%4.ps -trim +repage -density 110x110 -transparent "#FFFFFF" %%3
Fif exist zz%%4.ps DEL zz%%4.ps >nul 
\EndVerbatim

\item The specialized scripts may, for instance, maintain global caches of
   png bitmaps for cutting down on recompilation time.  \EndList
  \EndHPage{} of similar nature for creating pngs.



\item
\--Bitmaps and graphics/fonts//%
\--Fonts/Bitmaps//%
\--Pictures/Bitmaps//%
\--tex4ht.c/LGTYP//%
The bitmap formats can be controlled by a `g' record of tex4ht.env,  
a `-g' switch
 of {\tt tex4ht.c},
and a -LGTYP switch in the compilation of tex4ht.c.
 The default setting assumes the `png' format.


\setup{Update the Other Scripts in the Environment File}

\item  If needed, replace the scripts \`'Mmove %%1 %%2%%3',
and \`'Ccopy %%1 %%2%%3'
in \''tex4ht.env' with \HPage{alternative scripts}

\Verbatim
Mif exist %%2%%3 del %%2%%3
Mif exists %%1 move %%1 %%2%%3
Cif exist %%2%%3 del %%2%%3
Cif exists %%1 copy %%1 %%2%%3
\EndVerbatim
\EndHPage{} 
for moving and copying files.  The parameter \''%%1' stands 
for the source file(s), the parameter \''%%2' provides the 
target directory name, and the parameter \''%%3' refers to
the target file name(s).

% \item  If needed, replace the script
% \`'Ecopy empty.gif %%1%%2' in \''tex4ht.env' with an
%  alternative script 
% for replacing empty files.  The parameter \''%%1' stands 
% for the 
% target directory name, and the parameter \''%%2' refers to
% the empty file name.



\item  If applicable, replace the scripts \`'Achmod %%1 %%2%%3' 
in \''tex4ht.env' with an alternative script for changing the access mode of
files.  The parameter \''%%1' stands for access mode, the parameter
\''%%2' refers to a directory name, and the parameter \''%%3' refers
to the file(s).



\item \--Validation///% 
Postprocessing of files can be requested with
\`'.' scripts.  The files are selected by their extension names, as
 listed following the period symbols.   The parameter
\`'%%1' provides the file names, and the parameter \`'%%0' provides
the jobnames.  

%%%%%%%%%%%%%%%%%%%%%%%%%%%%%%%%%%%%%%%%%%%%%%%%%%%%%%%%%%%%%%%%
The environment file
\Link[tex4ht-env-win32.txt]{}{}tex4ht.env\EndLink{} 
offers the following draft of a dot  script  for 
{\bf validating} output of compilations.


\Verbatim
<validate>
 .xml xmllint --noout --valid %%1.xml 
 .html xmllint --noout --valid %%1.html
 .css mycssparser %%1.css
</validate>
\EndVerbatim


The dot script may be activated in the following manner.

\List{1}
\item  Fix the paths in the first two records.


\item
Bind a CSS validator to the .css record. 
(\Link[http://jigsaw.w3.org/css-validator/DOWNLOAD.html]{}{}{\tt
http://jigsaw.w3.org/css-validator/DOWNLOAD.html}\EndLink)


\item  Remove the leading space characters from the above record. 




\EndList


%%%%%%%%%%%%%%%%%%%%%%%%%%%%%%%%%%%%%%%%%%%%%%%%%%





\item
 Postprocessing of  files can also be requested with \`'X' 
scripts.  The file names are accessed through the parameter \`'%%1',
and their extensions through the parameter \`'%%2'.  


\setup{Set the Script Files}

\item
       If you use a command different than \`'latex' for compiling \LaTeX{}
       source files, replace
the references to \''latex' in \''c:\tex4ht\bin\win32\*.bat' with the appropriate
command name.
The  \''tex' and \''texi'
commands require similar attention.



                              \setup{Make the System Globally Known}
\item Add \`'c:\tex4ht\bin\win32' to your path variable in \`'c:\AUTOEXEC.BAT'.

\item 
Place the files \`'tex4ht.sty' and \`'*.4ht' 
of \''c:\texmf\tex\generic\tex4ht\'
within the
\LaTeX/\TeX{} tree (either by moving the files to a directory which already
has sty files, or by modifying the environment variable TEXINPUTS  to point also to \`'c:\texmf\tex\generic\tex4ht\').

\item
If your \TeX{} system uses a database to locate files, make sure to refresh
it (e.g., select \''Start -> Programs -> MiKTeX -> Refresh', or run \`'initexmf -u' from
a DOS session, to update MiK\TeX{}).



\setup{Didn't Use '{\tt c:\string\tex4ht'}?}
\item
If \TeX4ht  is installed in a directory other than \''c:\tex4ht',
make sure to adjust the related paths in 
\''tex4ht.env' the \''.bat' files.


\item Some output modes assume  
\Link[http://www.sun.com/]{}{}Java\EndLink{}  
is also available in the computer in use. 
 

%%%%%%%%%%%%%%%%%%%%%%%%%%%%%%%%
\setup{Test the Installation}
%%%%%%%%%%%%%%%%%%%%%%%%%%%%%%%%%
\item 
Move \''testa.tex' and  \''testb.tex'
from
\''c:\tex4ht.dir\temp\'
 to your work directory
\item Compile  \`'testa.tex'  with the command  \`'ht latex testa'
\item Compile  \`'testb.tex'  with the command  \`'htlatex testb'



\EndList



[\Link[http://www.csulb.edu/\string
~murdock/dosindex.html]{}{}MS Commands\EndLink]



\EndHPage{} require additional effort, mainly because of the need to  set up
non-native utilities. 
%
Alternative ports for these and other platforms can be tailored
in a 
%
\NextFile{\jobname-port.html}\HPage{similar manner}

\ExitHPage{}

\InstallSection{Establishing a Port}

\--Files/Download//%
A port can be established in the following manner. 

\List{1}
 \item Set a  directory, say,`{\tt \string~/tex4ht.dir}'.
\item
 Download the file \Link[tex4ht.zip]{}{}tex4ht.zip\EndLink{}
 into the directory \''tex4ht.dir' and \UNZIP{}  it.
\item Add to
\`'tex4ht.dir/texmf/tex4ht/base/' 
and
\`'tex4ht.dir/bin/' 
a subdirectory named, say, \''mydir'.
\item  Copy  
\`'tex4ht.dir/texmf/tex4ht/base/unix/tex4ht.env' 
or
\`'tex4ht.dir/texmf/tex4ht/base/win32/tex4ht.env' 
into
\`'tex4ht.dir/texmf/tex4ht/base/mydir/tex4ht.env'.
%%%%%%%%%%%%%%%%%%%%%%%%%%%%%%%%%%%%%%%%%%%%%%%%%%
\HPage{}
\ExitHPage{exit}
\InstallSection{Placement of the Environment File}

\--tex4ht.env/ENVFILE//%
\--tex4ht.c/Compiling//%
\--Kpathsea///%
\--tex4ht.env/TEX4HTENV//%
\--Environment Variables///%
\Link{}{envloc}\EndLink{}
The \''tex4ht.c'
and \''t4ht.c' programs retrieve the information about their
platform from an environment file. The programs search  the file at the
following locations, in the given order.
\List{disc}
%
%
\item The address specified within the
\Link[\RefFile{overview}]{}{}command lines\EndLink{} of \''tex4ht'
and \''t4ht',
and identified there by the prefix \`'-e'.
%
\item The address \''tex4ht.env' in the work directory.
%
\item At the location specified by an optional  environment variable named
\''TEX4HTENV'.  
%
\item The address \''tex4ht.env' in the \HPage{root}
 The address of the root directory
is assumed to be stored in an environment variable named \''HOME'.  

On MS Windows, the directory
\''c:/', and the residence  directory of  \''tex4ht.exe' and
\''t4ht.exe', are also candidates.
\EndHPage{}  directory.                 
\item
The address provided in  the variable \SysVar{ENVFILE} of tex4ht
and t4ht during compilation.
\item At the directories of kpathsea, to be searched by that utility,
 if the \''tex4ht.c' and \''t4ht.c' programs are compiled
with the  \`'-DKPATHSEA' switch on.

For instance,

\'@gcc  -o tex4ht tex4ht.c
        -DENVFILE='"~/tex4ht.dir/texmf/tex4ht/base/unix/tex4ht.env"' 
        -DKPATHSEA
        -DHAVE_DIRENT_H
        -lkpathsea@

\noindent or

\'@gcc -O2 
       -DKPATHSEA 
       -I/usr/include
       -L/usr/lib
       -o tex4ht tex4ht.c
       -DHAVE_DIRENT_H
       -lkpathsea@
%

When the address of \''tex4ht.env' is not explicitly given, the
\''texmf.cnf' file might need records similar to the following ones.

\Verbatim
  TEX4HTFONTSET=alias,iso8859
  TEX4HTINPUTS=.;$TEXMF/tex4ht/base//;$TEXMF/tex4ht/ht-fonts/{$TEX4HTFONTSET}//
  T4HTINPUTS=.;$TEXMF/tex4ht/base//
\EndVerbatim


\item 
The instruction

\centerline{\tt   apt-get install gcc libkpathsea4 libkpathsea-dev}

\noindent when executed as root
installs the following
packages 
on
 Ubuntu/Debian. 

\List{*}
\item 
gcc: the standard C compiler 
\item
libkpathsea4: library for TeX path searching, part of the texlive
 distribution 
              provides files like /usr/lib/libkpathsea.so.4.0.0 
\item
libkpathsea-dev: development files for libkpathsea 
                 provides numerous header files (*.h) in 
/usr/include/kpathsea 
\EndList 

If a file is already present,  a warning
is issued 
 and the instruction can  be
retried after omitting the request for the installed package.



\EndList


A given address may start with the character \`'~'.  This character is
interpreted to be the directory address of the root (as provided in an
environment variable named \''HOME').



The file \''tex4ht.env' may be renamed to \`'.tex4ht',
if your system allows names consisting only of the extension part.





\EndHPage{}%
%%%%%%%%%%%%%%%%%%%%%%%%%%%%%%%%%%%%%%%%%%%%%%%%%%%
\HPage{Modify}

\ExitHPage{}

\InstallSection{The Composition of tex4ht.env}

\--tex4ht.env///%
\TeX4ht consults this file for system-dependent information.  The file
consists of a sequence of directives, where each directive occupies a line
and is identified with a distinguished character code in the first column.
The following are the possible character codes and their meaning.
\List{}


\item{b} This character identifies
for \''tex4ht'
 a comment to be placed
in the \`'.lg' file, before the 
\Link{bsc}{bsc-1}scripts\EndLink{}
 for creating
pictures for symbols.

\item{g}
\--Bitmaps and graphics/png//%
 This character identifies the extension  tex4ht
should associate to names of the files of pictures the postprocessor
 requests (for instance, bitmap files of glyphs.).  The
default corresponds to a setting of the form \`'g.png'.

An extension name can also be encoded into tex4ht
during compilation time through the variable
\SysVar{LGTYP}.

Alternatively, an extension can be provided
in the command line of tex4ht, in which case the character code
should be present and immediately preceded by a \`'-' character.



\item{i} 
This character identifies to \''tex4ht'
a directory 
\Link{ch-i}{ch.i}where\EndLink{}
the hypertext  font (.htf) files of \''tex4ht' are stored. 

\item{l}
\--tex4ht.fls///%
 This character identifies a 
%
\HPage{bookkeeping file}
\ExitHPage{exit}
\InstallSection{Placement of the Bookkeeping File}


\--tex4ht.c///%
\--Kpathsea///%
\--tex4ht.fls/TEX4HTWR//%
\--Environment Variables///%
\Link{}{TEX4HTWR}\EndLink{}To cut down on the time invested to locate files,
the \''tex4ht.c' program maintains a bookkeeping 
 file where it records the
addresses of the files it finds.  Whenever the program  needs to locate
a file, it first searches the address in the bookkeeping file.  If it doesn't
find the address there, the program embarks on a search
throughout the physical directories of its platform. 

The program places the bookkeeping file at the first applicable
location of the following list.

\List{1}
\item
 The pointer specified by an optional \`'-l' switch  provided to the
invocation of the \''tex4ht' program.
\item 
 A file named \''tex4ht.fls' in
 the directory specified by an optional  environment variable named
\''TEX4HTWR'.  
\item
 The location specified by an optional \`'l' record  provided 
in the environment file.
\item  A file named \''tex4ht.fls'  in the work directory.
\EndList



A given address may start with the character \`'~' or the 
character pair \`'~~'.  The single character \`'~' is interpreted to be the
directory address of the root,  provided in an environment variable
named \''HOME'.
The  character pair \`'~~' is interpreted to be a
directory address  provided in an environment variable
named \''TEX4HTWR'.  The address stored in the latter environment
variable may also start with a single character \`'~'.



 The bookkeeping file must have writable access mode, and the directory
which contains the file should have a compatible access mode.  On
multi-user  platforms,  it is recommended not to share the bookkeeping
file, but to ask for such files in the users' directories.
 

The bookkeeping file is a 
dynamically constructed variant of the ls-R file of kpathsea.  TeX4ht
ignores its own bookkeeping mechanism, if the program \''tex4ht.c' 
is compiled with a raised \''-DKPATHSEA'  switch.
\EndHPage{}
%
  where \''tex4ht' can
  record  information about  paths to files it uses.


\item{s} This character identifies 
to \''tex4ht'
a \Link{sc}{sc-1}command\EndLink{}
for requesting in the \''lg' file the translation of dvi
 pictures to bitmaps.  When needed, a sequence of such
commands can be placed in consecutive lines to form a block of
commands for handling the translation.




\item{t}
\--Fonts/tfm//%
 This character identifies
to \''tex4ht'
 the directory \HPage{where}


\ExitHPage{exit}

\InstallSection{Directories of the \TeX\space Fonts   Metric (tfm)}

\--Fonts/tfm//%%
\--tex4ht.c/Compiling//%
\--Kpathsea///%
\--Environment Variables///%
\Link{}{alt-tfm}\EndLink\TeX4ht searches
  the hypertext fonts in the working directory, and
 in the following optional directories, in the given order.  


\List{disc}
\item
The directory named
in the \Link[\RefFile{overview}]{}{}command line\EndLink{}
of tex4ht, identified with the  prefix \`'-t'.


\item The   directory identified with the character \`'t'
in the tex4ht.env file.

\item
The directory whose name is fed into tex4ht.c
during compilation time,  through   the program variable
\SysVar{TFMDIR}.

\item In the directory holding the executable \''tex4ht' (only 
for MS Windows).

\EndList


A given address may start with the character \`'~' or the 
character pair \`'~~'.  The single character \`'~' is interpreted to be the
directory address of the root,  provided in an environment variable
named \''HOME'.
The  character pair \`'~~' is interpreted to be a
directory address  provided in an environment variable
named \''TEX4HTTFM'.  The address stored in the latter environment
variable may also start with a single character \`'~'.

The optional environment variable \''TEX4HTTFM' may hold one
or more addresses. The addresses must be separated by a 
character which does not appear in the addresses, and that
character must also delimit the content at the start and end points.


 The system variable \SysVar{MAXFONTS} of tex4ht.c places a
 limit on the number of fonts allowed in the documents. The default
 setting doesn't provide such a limit.

 If the program \''tex4ht.c' is compiled
with the  \`'-DKPATHSEA' switch on,
\TeX4ht ignores its own search and delegate it to the kpathsea utility.

% \leavevmode \--Fonts/tfm/aliases/\Link{}{tfmalias}\EndLink 
% Some
% \TeX{} systems provide virtual fonts without tfm files. Instead, they
% provide file of aliases with records of the form `{\it virtual font
% name}={\it actual font name}'. In such a case, the location of a file of
% aliases can be provided within an a-record in \''tex4ht.env'.
% 
% The tex4ht program ignores records having form different than
% `{\it virtual font name}={\it actual font name}'.


A given address may start with the character \`'~' or the 
character pair \`'~~'.  The single character \`'~' is interpreted to be the
directory address of the root,  provided in an environment variable
named \''HOME'.
The  character pair \`'~~' is interpreted to be a
directory address  provided in an environment variable
named \''TEX4HTA'.  The address stored in the latter environment
variable may also start with a single character \`'~'.




\EndHPage{}
the  font metric (.tfm) files  of \TeX{} are stored. 


\item{A}
This character identifies to \''t4ht' a
\Link{accscript}{}script\EndLink{} for changing
access mode of files.

\item{C}
This character identifies a \Link{mvscript}{}script\EndLink{} for
satisfying requests made in the
\''lg' file to copy files between directories.


\item{E}
This character identifies a \Link{mvscript}{}script\EndLink{} for
satisfying requests made in the
\''lg' file to get substitutions for empty pictures.


\item{F}
This character identifies 
to \''t4ht'
a \Link{dv2png}{}script\EndLink{} for translating  into other formats
characters
from dvi files.

A `F.ext' record marks a conditional F-subscript. It states that the
following F-records  will apply only to pictures whose extension
names are `ext'.

A `F.' record marks a default F-subscript. It applies to pictures
whose extension names do not get dedicated F-subscripts.

The `F.ext' and `F.' records are not needed, in case all the pictures
are to be processed by a single set of F-records.

\noindent{\bf Example}:

\Verbatim
F.gif
F-gif-script
F.png
F-png-script
F.
F-script
\EndVerbatim


\item{G}
This character identifies 
to \''t4ht'
a \Link{dv2png}{}script\EndLink{} for translating general
 dvi figures into other formats.


A `G.ext' record marks a conditional G-subscript. It states that the
following G-records  will apply only to pictures whose extension
names are `ext'.

A `G.' record marks a default G-subscript. It applies to pictures
whose extension names do not get dedicated G-subscripts.

The `G.ext' and `G.' records are not needed, in case all the pictures
are to be processed by a single set of G-records.

\noindent{\bf Example}:

\Verbatim
G.gif
G-gif-script
G.png
G-png-script
G.
G-script
\EndVerbatim


\item{M}
This character identifies a \Link{mvscript}{}script\EndLink{} for
satisfying requests made in the
\''lg' file to move files between directories.



\item{S} This character specifies what security measures 
  \''t4ht' should take when invoking other utilities.  In the
  absence of this directive, all the calls to system services are
  ignored. On the other hand, a directive of the form \`'S*' allows
  all system calls.  A selective access to system calls can be
  obtained with \''S' directives, which specify the prefixes of command
  names that should be allowed to go through.
  
  The \''S' directives can be fed as switches to the command lines   \''t4ht'.

\item{P} A variant of he \''S' switch for \''tex4ht'. Rarely useful.

  
\item{X} This character identifies to \''t4ht' a script for
  postprocessing the files which \''tex4ht' outputs.  The file names
  are represented
by \`'%%1', and their extensions by \`'%%2'. 
Such scripts, for instance, may invoke validators to check
the correctness of the files against given DTD's and request
XSL transformations.  

The command line of \''t4ht' may include a flag \`'-X' whose content 
is represented by \`'%%3' in the script.


Example:  \''Xmake -f mymake name=%%1  ext=%%2 %%3'

\item{{\bf.} (dot)} A variant of the X script applied to files whose extension
names are provided after the dot. The file names are represented by
\`'%%1', and the jobnames by \`'%%0'.

Example:  \''.xml echo "name=%%1.xml"'


\item{{\tt <}}
Tagged script segments \''<tag>...</tag>' are scanned only if     
     their names are specified within \''-ctag' switches of \''tex4ht.c'   
     and \''t4ht.c'. When such switches are not supplied, a the switch \''-cdefault'
     is implicitly assumed.  

\EndList
Lines starting with other characters are treated as comments,
and empty lines may be treated as file terminators
 by \''tex4ht' and \''t4ht'.



{\noindent \bf Examples:}
\Link[tex4ht-env-unix.txt]{}{}Unix-oriented\EndLink{},
\Link[tex4ht-env-win32.txt]{}{}MS-oriented\EndLink{}




\EndHPage{} the file to meet the setting
of your environment.
%
%
%
\item 
\HPage{Compile}

\ExitHPage{}
\Link{}{doc-c}\EndLink

\InstallSection{Compiling tex4ht.c}


\List{disc}
\item
\--tex4ht.c/Compiling//You might want to activate some of the options at the start of the
file by uncommenting the corresponding lines.

\Verbatim
/* **********************************************
    Compiler options                            *
    (uncommented | command line)                *
------------------------------------------------*
       Classic C (CC)             default
#define ANSI                      ansi-c, c++
#define DOS_C
#define HAVE_STRING_H             <string.h>
#define HAVE_DIRENT_H             <dirent.h>
#define HAVE_SYS_NDIR_H           <sys/ndir.h>
#define HAVE_SYS_DIR_H            <sys/dir.h>
#define HAVE_NDIR_H               <dir.h>
#define WIN32
#define KPATHSEA
#define BCC32                    bordland c++

*************************************************
\EndVerbatim

\item 
\--tex4ht.c/Compiling//%
\--tex4ht.c/HTFDIR//%
\--tex4ht.c/LGPIC//%
\--tex4ht.c/LGSEP//%
\--tex4ht.c/LGTYP//%
\--tex4ht.c/MAXFONTS//%
\--tex4ht.c/TFMDIR//%
Values may be assigned in the command line or
the beginning of the source file, to the environment variables
\def\SysVar#1{\Link{#1}{x-#1}#1\EndLink}
\SysVar{ENVFILE},
\SysVar{HTFDIR},
\SysVar{LGPIC},
\SysVar{LGSEP},
\SysVar{LGTYP},
%          ???   \SysVar{MAXFDIRS},
\SysVar{MAXFONTS},
and
\SysVar{TFMDIR}.
\gdef\SysVar#1{\Link{-#1}{#1}#1\EndLink}

For instance,

     \'@gcc -o tex4ht tex4ht.c
     -DENVFILE='"path/tex4ht.dir/texmf/tex4ht/base/unix/tex4ht.env"'
     -DHAVE_DIRENT_H@

\EndList

The switch \`+-DENVFILE+ is optional, if the program can reach the
environment file in an \Link{envloc}{}alternative\EndLink{} manner.


\InstallSection{Compiling t4ht.c}


\--t4ht.c/Compiling//%
The environment variable \''ENVFILE' is available also in \`'t4ht.c'.

For instance,

\'+gcc -o t4ht t4ht.c -DENVFILE='"gold/tex4ht.dir/texmf/tex4ht/base/unix/tex4ht.env"'+



\EndHPage{}     
the source files
\`'tex4ht.dir/src/tex4ht.c'
and
\`'tex4ht.dir/src/t4ht.c'
 into executable programs \''tex4ht' and \''t4ht',
and move the latter programs into \`'tex4ht.dir/bin/'.
%
%
%
\item Set  a 
\HPage{driver} 
\ExitHPage{}

\InstallSection{Driving the Translation of Pictures and Other Tasks}


\--Files/idv//%
\--Files/lg//%
\--Bitmaps and graphics/Requests//%
\TeX4ht outputs a script file (\`'.lg') describing how the dvi file
(extension \`'.idv') of pictures should be processed, the CSS
instructions to be included for the file, and user-initiated requests
from the operating system. In the default setting, the script file
holds abstract commands similar to the following ones.
\List{}
\item{{\tt--- needs --- source.idv[i] ==> target.png ---}}

An abstract command of this form requests that the i'th page in
the dvi file will be translated into a target file
whose name is provided.

\item{{\tt
--- characters ---
}}

This abstract command is a  identifies where
the requests for pictorial characters start.

\EndList

A manual brute-force execution of the abstract
commands can be a tedious job for large number of
pictures.  The \''t4ht' is in essence an interpreter
for these abstract commands.

 Another
possible approach for automating the process
is to request lg scripts in the form of
shell scripts
or batch files.  


\HCode{<big>}{\bf Alternatives to `{{\tt--- needs --- 
source.idv[i] ==> target.png ---}}'}\HCode{</big>}

The default setting is made with a request of the form
\`'--- needs --- %%1.idv[%%2] ==> %%3.png ---', where
the parameters \''%%1', \''%%2', and \''%%3' respectively represent
the name of the source file without its extension, a page number, and
a name of the target file.  Alternative patterns to these abstract
commands can be requested in the following locations, with the order
reflecting on the priority given to the requests.


\List{disc}
\item In the  \Link[\RefFile{overview}]{}{}command line\EndLink{}
of \''tex4ht', where the pattern should be prefixed with \`'-s'.

\item In the environment file \''tex4ht.env'
within a block of consecutive lines, where the lines should be
identified with the character \Link{sc-1}{sc}\`'s'\EndLink{}.

\gdef\SysVar#1{\Link{x-#1}{#1}#1\EndLink}   %%%%%%%%#

\item  In \''tex4ht' with the pattern provided  through the variable
\SysVar{LGPIC} of \''tex4ht.c'.

\def\SysVar#1{\Link{#1}{x-#1}#1\EndLink}

\EndList

The character \`'%' can be introduced into a pattern through
the entry \`'%%%'.  On the other hand, the parameters \`'%%1',
\`'%%2', and \''%%3' can specify, 
between the first two percentage characters (i.e., \`'%...%1',
\`'%...%2', and \`'%...%3'), any format  for the outcome that is
compatible with the print formats of C.

\HCode{<big>}{\bf Alternatives to  `{{\tt--- characters ---}}'}\HCode{</big>}

Substitutions for this abstract command
can  be requested in the following locations.





\List{disc}
\item
In the  
%\Link[\RefFile{overview}]{}{}command line\EndLink{}
 \HPage[]{overview}

\ExitHPage{}

\InstallSection{The Translation Process}


\--tex4ht.sty///\--t4ht.c///\--tex4ht.c///\--\LaTeX///%
\--Scripts///%
The system can be activated with a sequence of
commands of the following form, typically embedded within a script.

\Verbatim
       latex      x            (or `tex x')
       latex      x
       latex      x
       tex4ht     x
       t4ht       x
\EndVerbatim


The three compilations with La(\TeX) are needed to ensure proper links.
The approach is illustrated in the following picture.


\Draw
   \TreeSpec(n,\Node & r,\SRectNode)()()
   \TreeAlign(H,0,0)(0,0,0)
   \MinNodeSize(1,30)
\Tree()(
   1,n,x.tex//
   1,r,TeX //
   1,n,x.dvi //
   3,r,tex4ht //
   0,n,html~~files  & 0,n,x.idv &
                      0,n,x.lg //
)
\MoveToLoc(4..1) 
\CSeg[0.5]\Move(4..1,4..2) 


  \CSeg\Move(0..0,1..0) 
 \SRectNode(5..0)(--t4ht--)
 
 \CSeg\Move(0..0,1..0) 
 \Node(6..0)(--png~\&~css~~files--)


\Edge(5..0,6..0)

\MoveToLoc(5..0) 
\CSeg[0.5]\Move(1..0,0..0)  \MarkLoc(x)
\HHEdge(4..1,5..0,x)
\HHEdge(4..2,5..0,x)

 \EndDraw





\List{}
\item{x.tex}
  
  This is a source \TeX/\LaTeX/Other\TeX{} file that imports the style
  files \''tex4ht.sty' and \''*.4ht'.  The style files define
  the features for the output.

\item{tex4ht}

\--dvi///%
\--Files/dvi//%
The output of \TeX{} is a standard dvi file interleaved with special
instructions for the postprocessor \''tex4ht' to use.  The special
instructions come from implicit and explicit requests made in the
source file through commands of \TeX4ht.

The utility \''tex4ht' translates the dvi code into standard text,
while obeying the requests it gets from the special instructions. The
special instructions may request the creation of files, insertion
of html code, filtering of pictures, and so forth.

In the extreme case that the source code contains no commands of
\TeX4ht{}, tex4ht gets pure dvi code and it outputs (almost) plain
text with no hypertext elements in it.

The special ({\tt\string\special}) instructions seeded in the dvi code
are not understood by dvi processors other than those of \TeX4ht.

\item{x.idv}

\--dvi///%
\--Files/idv//%
This is a dvi file extracted from \''x.dvi', and it contains the
pictures needed in the html files.

\item{x.lg}

\--Files/lg//%
This is a log file listing the pictures of x.idv, the png files that
should be created, CSS information, and user directives introduced
through the \`'\Needs{...}' command.

\item{t4ht}

This is an interpreter for executing the requests made in the \''x.lg'
script.

\EndList

\InstallSection{A Reflection at the System Messages}

\bgroup
\def\.#1.{$\Leftarrow$ {\bf #1}}

\Verbatim-@
This is TeX, Version 3.14159 (Web2C 7.3.1)             @.invoke `latex x'.
LaTeX2e <1998/12/01> patch level 1
Babel <v3.6x> and hyphenation patterns for american, french, german, ngerman, n
ohyphenation, loaded.
(x.tex (/usr/share/texmf/tex/latex/base/article.cls
Document Class: article 1999/01/07 v1.4a Standard LaTeX document class
(/usr/share/texmf/tex/latex/base/size10.clo))
(n/tex4ht.dir/tex4ht.sty)
(n/tex4ht.dir/tex4ht.sty
--- needs --- tex4ht x ---
(tex4ht.tmp) (x.xref) (n/tex4ht.dir/html4.4ht)
(n/tex4ht.dir/picmath4.4ht)
(n/tex4ht.dir/latex.4ht (n/tex4ht.dir/html4.4ht)
(n/tex4ht.dir/picmath4.4ht))
(n/tex4ht.dir/fontmath.4ht
(n/tex4ht.dir/html4.4ht)
(n/tex4ht.dir/picmath4.4ht))
(n/tex4ht.dir/article.4ht (n/tex4ht.dir/html4.4ht
) (n/tex4ht.dir/picmath4.4ht))) (x.aux)
--- file x.css ---
[1] (x.aux) )
Output written on x.dvi (1 page, 4460 bytes).
Transcript written on x.log.
This is TeX, Version 3.14159 (Web2C 7.3.1)             @.invoke `latex x'.
LaTeX2e <1998/12/01> patch level 1
.......................
This is TeX, Version 3.14159 (Web2C 7.3.1)             @.invoke `latex x'.
LaTeX2e <1998/12/01> patch level 1
.......................
tex4ht.c (1999-11-10-03-50)                            @.invoke `tex4ht x'.
(tex4ht.env)
(n/tex4ht.dir/tex4ht.fls)
(/usr/share/texmf/fonts/tfm/public/cm/cmr10.tfm)
(n/tex4ht.dir/ht-fonts/iso8859/cm/cmr.htf)
[1 file x.html
 file x.css
 file tex4ht.tmp
]
Execute script `x.lg'
t4ht.c (1999-12-30-21-14)                              @.invoke `t4ht x'.
Entering tex4ht.env
Entering x.lg
.......................
\EndVerbatim \egroup




\EndHPage{}
of \''tex4ht', prefixed with \`'-b'.

\item In a line within the environment file
\`'tex4ht.file', where the line should be identified with the character
`\Link{bsc-1}{bsc}b\EndLink{}'.

\gdef\SysVar#1{\Link{x-#1}{#1}#1\EndLink}

\item  In \''tex4ht' with the substitution provided
during compilation time through the variable \SysVar{LGSEP} of
\''tex4ht.c'.

\def\SysVar#1{\Link{#1}{x-#1}#1\EndLink}

\EndList

\EndHPage{}          
for the t4ht utility.
\EndList


%%%%%%%%%%%%%%%%%%%%%%%%

\HPage{}

\ExitHPage{exit}

\InstallSection{Directories for the Virtual Hypertext   Fonts (htf)}

\--Fonts/htf//%%
\--tex4ht.c/Compiling//%
\--Kpathsea///%
\--Environment Variables///%
\Link{}{alt-htf}\EndLink%
Each font of TeX may have numerous virtual hypertext font to map to.
The \Link{ch.i}{}i directives\EndLink{} of \''tex4ht.env', as well as
the i switches of the \''tex4ht' command, may be used to identify the
subdirectories where the target virtual fonts are to be found.

\TeX4ht searches  the hypertext fonts in the working directory, and
 in the following optional directories, in the given order.  


\List{disc}
\item
The directory named
in the \Link[\RefFile{overview}]{}{}command line\EndLink{}
of tex4ht, identified with the  prefix \`'-i'.


\item The   directory identified with the character
\Link{ch.i}{ch-i} \`'i'\EndLink{}
in the tex4ht.env file.

\item
The directory whose name is encoded into tex4ht
during compilation time,
within a string \''"..."' fed into the variable
\SysVar{HTFDIR} of tex4ht.c.

\item In the directory holding the executable \''tex4ht' (only 
for MS Windows).

\EndList


A given address may start with the character \`'~' or the 
character pair \`'~~'.  The single character \`'~' is interpreted to be the
directory address of the root,  provided in an environment variable
named \''HOME'.
The  character pair \`'~~' is interpreted to be the
directory address  provided in an environment variable
named \''TEX4HTHTF'.  The address stored in the latter environment
variable may also start with a single character \`'~'.


The optional environment variable \''TEX4HTHTF' may hold one
or more addresses. The addresses must be separated by a 
character which does not appear in the addresses, and that
character must also delimit the content at the start and end points.

 If the program \''tex4ht.c' is compiled
with the  \`'-DKPATHSEA' switch on,
\TeX4ht ignores its own search and delegate it to the kpathsea utility.
\EndHPage{} 
%
%%%%%%%%%%%%%%%%%%%%%%%%%%%%%%%%%%%%%%%%%%%


 \EndHPage{}.
The distribution assumes compilations through command lines but
 \HPage{graphical user inferfaces}

\--Graphical User Interfaces///%

\List{*}
\item
\Link[http://www.mayer.dial.pipex.com/tex.htm]{}{}\TeX{} Converter\EndLink{} 
(Steve Mayer). Supports a set of converters on MS Windows,
including  \TeX4ht.



\item
 \Link[http://www.simpletex4ht.free.fr/
 target="\string_blank"]{}{} SimpleTeX4ht\EndLink{} 
(Yves Gesnel). A Mac OS X GUI.


\item
\TeX4ht
\HPage{integrated into the
WinShell}
I was successful with integrating TeXh4t win WinShell 2.2.1
from Texlive 7 distribution. By inserting ``user tools'' you can
have an editor from which there is a quick access to not only
LaTeX, (standard makeindex - not provided but explained in help support),
PDFLaTeX, Dvi, GsView, Acrobat Reader, but also to TeXh4t (DOS switch
MUST be OFF as logspace has not enough room for long log files produced
by TeXh4t), to two stage makeindex procedure and to Mozilla browser.




\EndHPage{}
 graphical user interface 
of Ingo H. de Boer
 for working with TeX.
(Piotr Grabowski) %<pgrab@IA.AGH.edu.PL>

\item
\Link[http://www.ctan.org/tex-archive/systems/win32/bakoma/programs/texword.html]{}{}BaKoMa
\TeX{} Word\EndLink .  A WYSIWYG \LaTeX{} editor with HTML expot
through \TeX4ht .
\EndList

\EndHPage{} may also be employed.



 Philip A. Viton discusses in details issues of installing \TeX4ht under
 \Link[http://facweb.knowlton.ohio-state.edu/pviton/support/tex4ht.html]{}{}Mik\TeX\EndLink{} and
 \Link[http://facweb.knowlton.ohio-state.edu/pviton/support/swpht.html]{}{}Scientific
 Word/WorkPlace\EndLink, but many of the topics apply
 also to other platforms.  Steven Zeil offers
 \Link[http://www.cs.odu.edu/\string~zeil/tex4ht/tex4ht\string_discussion.html
 target="\string_blank"]{}{}improvements\EndLink{} for the above
 settings.




 The \Link[tex4ht-lit.zip]{}{}literate sources\EndLink{} of \TeX4ht
 are also available, but they are not needed for installing the
 system.  The literate views are very far from being in a desirable
 state for a review by a public eye--they reflect their true nature as
 being privately used for developing and maintaining the available
 code.  The views follow a basement mentality: throw in without much
 scrutiny any item of possible value at some point of time, and clean
 a corner when the need arises for working on a specific issue of the
 code. The leading lines in the files indicate how the files can be compiled.



\noindent \Link{trbl-sht}{}trouble shooting\EndLink
{\tt |}
%%%%%%%%%%%%%%%%%%%%%%%%%%%%%
 \HCode{<span class="bugfixes">}\Link[bugfixes.html]{}{}bug fixes\EndLink
 \HCode{</span>}

%%%%%%%%%%%%%%%%%%%%%%%%%%
\DocPart{Bug Reports}
%%%%%%%%%%%%%%%%%%%%%%%%%%

The development of the \TeX4ht system is to a large degree 
driven by users' bug reports
and requests.  
In most cases, when providing feedback, it is essential to include the
following information.

\List{*} 
\item A 
\NextFile{\jobname-bug.html}\HPage{minimal}\ExitHPage{}
%%%%%%%%%%%%%%%%%%%%%%%%%%%%%%%%%%%%%%%%%%%%%%%%%%%%%%%%%%%%%%
\List{*} 
\item Remove the calls to packages that are not essential to reproduce the problem 
  (e.g., \Verb=\RequirePackage=, \Verb=\usepakage=) 
 
\item Remove  the local definitions that are not essential.
For instance, use

\Verbatim
\begin{figure} 
   \includegraphics{filename} 
   \caption{What ever.}
\end{figure}
\EndVerbatim

instead of

\Verbatim
\def\fig#1#2{\begin{figure} \includegraphics{#1} \caption{#2} \end{figure}}
\fig{filename}{What ever.}
\EndVerbatim
 
\item Use core classes when possible (e.g., \Verb=article= instead of \Verb=scrbook=).
 
\item Cut to a minimum the content where possible 
      (e.g., `Abc' instead of `Blah blah ... for ever.') 
\EndList

 
I work in a \LaTeX{} environment different from where the bug reports 
typically originate.  To understand the bugs I need to recreate the 
environments producing them.  A sample file that isolates the  
problem reduces the debugging effort to a minimum. 
 

%%%%%%%%%%%%%%%%%%%%%%%%%%%%%%%%%%%%%%%%%%%%%%%%%%%%%%%%%%%%%%
\EndHPage{} miniature complete source file illustrating the
issues involved, but do not include bitmap files loaded by the 
sources (e.g., PNG files). Since similar constructs are introduced within different
style files, a minimal source is required for correctly pinpointing the 
definitions of the constructs involved in the translation.



\item A mentioning of the command used to invoke the translation. This
information is required since the command determines which configurations
participate in the translation.

\item
Occasionally it might also be
useful to have
 an URL to a public domain where the files and messages produced
during the compilation can be viewed.  Unless explicitly requested to to do so,
please don't email such information.

\EndList


Translations of source files are centered around logical structures.
Formatting instructions receive only limited attention.



\DocPart{Resources}


%  From: Sebastian Rahtz <s.rahtz@elsevier.co.uk>
%  Date: Fri, 4 Sep 1998 10:16:12 +0100 (`)
%  To: plaice@cse.unsw.edu.au
%  cc: gurari@cse.ohio-state.edu
%  Subject: Re: tex to mathml
%  
%  John Plaice writes:
%  
%  
%   > i've changed the tex engine so that entities get recognized at the
%   > font level or at the mathcode/delcode level.  there are also new macros for
%   > \sgmlstarttag, \sgmlendtag, \sgmlemptytag and \sgmlentity.
%   > plus with versioned macros, you can have highly optimized translations
%   > that do everything correctly.
%  ok so far. you do in the TeX engine what Eitan has to do in the dvi
%  post-processor. but
%  
%   - you still have to write an extensive macro package to do the actual
%      mapping, I assume?
%   - how do you get the XML file out the end? is Omega writing a text
%      file, or are you going to extract from the dvi file?
%  
%   > it should be finished in about 2 weeks.
%  shall be fascinated to see it
%  
%  sebastian


\HTable^/\Css{\#TBL-\TableNo\space td{font-size:75\%;}}                  
Languages: \&
\HPage{LaTeX/TeX}
\ExitHPage{}\SubSection{LaTeX/TeX}
\List{disc}
\item
 \Link[http://www.mech.gla.ac.uk/\string
         ~donald/talks/LaTeX/slides.html
   target="\string_blank"]{}{}%
  Introduction to \LaTeX \EndLink{} (A. J. Hildebrand)
\item 
 \Link[http://www.cs.rug.nl/\string
     ~rein/csrugonly/latex/latexdoc/latexdoc.html
   target="\string_blank"]{}{}%
  What's LaTeX all about?\EndLink{} (Rein Smedinga)
\item
 \Link[http://web.mit.edu/olh/Latex/ess-latex.html
   target="\string_blank"]{}{}%
  Essential \LaTeX \EndLink{} (Sharon Belville, Matthew Swift)
\item 
 \Link[http://www.cs.cornell.edu/Info/Misc/LaTeX-Tutorial/LaTeX-Home.html
   target="\string_blank"]{}{}%
  Beginning LaTeX\EndLink{} (Denise Moore)
\item
 \Link[http://www.cs.stir.ac.uk/guides/latex/guide.html
   target="\string_blank"]{}{}%
  Document Preparation with \LaTeX\EndLink{} (David Budgen, Sam Nelson)
\item
 \Link[http://www.sci.usq.edu.au/staff/robertsa/LaTeX/latexintro.html
   target="\string_blank"]{}{}\LaTeX: from quick and dirty to style and finesse\EndLink{} (Tony Roberts)
\item
 \Link[http://www.maths.tcd.ie/\string
      ~dwilkins/LaTeXPrimer/Index.html
   target="\string_blank"]{}{}Getting Started with \LaTeX\EndLink{} (David R. Wilkins)
\item \Link[http://www-h.eng.cam.ac.uk/help/tpl/textprocessing/latex\string_advanced/latex\string_advanced.html
   target="\string_blank"]{}{}Advanced \LaTeX \EndLink {} (Tim Love)
\item
\Link[http://www.giss.nasa.gov/latex/ltx-2.html
   target="\string_blank"]{}{}\LaTeX{} commands\EndLink{} (Sheldon Green)
\item
\Link[http://tex.loria.fr/ctan-doc/macros/latex/doc/html/fntguide/fntguide.html
   target="\string_blank"]{}{}\LaTeX2e
 font selection\EndLink{} 
\item
 \Link[http://makingtexwork.sourceforge.net/mtw/
   target="\string_blank"]{}{}Making \TeX{} Work\EndLink{} (Norman Walsh)
\item
 \Link[http://profs.sci.univr.it/\string ~gregorio/orrori.pdf
   target="\string_blank"]{}{}Horrors in LaTeX: how to mistreat LaTeX and make a copy
>            editor unhappy\EndLink{} (Enrico Gregorio). Commented in Italian.
\item
 \Link[http://www.non.com/books/TeX\string_cc.html
   target="\string_blank"]{}{}Books\EndLink{} 
\item
 \Link[news:comp.text.tex
   target="\string_blank"]{}{}newsgroup comp.text.tex\EndLink
\item
 \Link[http://www.esm.psu.edu/mac-tex/
   target="\string_blank"]{}{}MacOSX TeX/LaTeX Web Site\EndLink
\item
 \Link[http://www.ctan.org/tex-archive/help/Catalogue/
   target="\string_blank"]{}{}\TeX{} Catalogue\EndLink{} (Graham Williams)
\item
 \Link[http://www.dante.de/cgi-bin/ctan-index
   target="\string_blank"]{}{}Archive Search\EndLink{} 
\item
 \Link[http://en.wikibooks.org/wiki/LaTeX
   target="\string_blank"]{}{}LaTeX Wiki\EndLink{} 
\EndList
\EndHPage{},
\HPage{HTML}\ExitHPage{}\SubSection{HTML}
\List{disc}
\item
\Link[http://www.w3.org/MarkUp/Guide/
                target="\string_blank"]{}{}Getting started
    with HTML\EndLink{} (Dave Raggett)
\item
   \Link[http://www.htmlhelp.com/reference/wilbur/overview.html
          target="\string_blank"]{}{}%
HTML        3.2: Wilbur\EndLink{} (WDG: Web Design Group)
\item
   \Link[http://www.w3.org/TR/html401/
          target="\string_blank"]{}{}%
        HTML 4.01 Specification\EndLink{} (W3C),
   \Link[http://www.w3.org/MarkUp/Test/HTML401/current/tests/index.html
          target="\string_blank"]{}{}%
        examples\EndLink{} 
\item
   \Link[http://www.w3.org/MarkUp/
          target="\string_blank"]{}{}%
        XHTML\EndLink{} (W3C)
\EndList
\EndHPage{},
\HPage{XML/XSLT}\ExitHPage{}\SubSection{XML/XSLT}
\List{disc}
\item
   \Link[http://www.w3.org/TR/REC-xml
          target="\string_blank"]{}{}%
        XML\EndLink{} (W3C),
   \Link[http://www.garshol.priv.no/download/text/xml-intro/index-en.html
          target="\string_blank"]{}{}%
         An Introduction to XML\EndLink{}  (Lars Marius Garshol)
\item
XSLT:
\Link[http://www.w3.org/TR/xslt
        target="\string_blank"]{}{}W3C\EndLink,
 \Link[http://www.devguru.com/Technologies/xslt/quickref/xslt\string
         _index.html
        target="\string_blank"]{}{}DevGuru index\EndLink,
 \Link[http://metalab.unc.edu/xml/books/bible/updates/14.html
        target="\string_blank"]{}{}tutorial\EndLink{} (XML Bible),
 \Link[http://nwalsh.com/docs/tutorials/xsl/xsl/slides.html
        target="\string_blank"]{}{}tutorial\EndLink{} (Groso and Walsh)
\item
 \Link[http://www.w3.org/TR/xpath
        target="\string_blank"]{}{}XPATH\EndLink{} (W3C)
\item
Pointers (Robin Cover):
   \Link[http://www.oasis-open.org/cover/sgml-tex.html  
          target="\string_blank"]{}{}%
        SGML/XML and (La)\TeX\EndLink{},
   \Link[http://www.oasis-open.org/cover/xml.html
          target="\string_blank"]{}{}%
        SGML/XML\EndLink{},
   \Link[http://www.oasis-open.org/cover/xsl.html
        target="\string_blank"]{}{}%
                 XSL\EndLink{} 
\item
   \Link[http://www.xml.com/pub/pt/3
          target="\string_blank"]{}{}%
        editors\EndLink{} (XML.com)


\EndList
\EndHPage{},
%
\HPage{MathML}\ExitHPage{}\SubSection{MathML}
\List{disc}
%-----------------------------------------------------------------------------------
\item
\Link[http://www.w3.org/TR/MathML2/
        target="\string_blank"]{}{}%
Specifications\EndLink{}.
%-----------------------------------------------------------------------------------
\item
\Link[http://www.w3.org/Math/
        target="\string_blank"]{}{}%
W3C's Math Home Page\EndLink{}.
%-----------------------------------------------------------------------------------
\item
\Link[http://www.w3.org/Amaya/
        target="\string_blank"]{}{}%
Amaya\EndLink{}.
 A browser capable of viewing MathML.
%-----------------------------------------------------------------------------------
\item
\Link[http://www.dessci.com/webmath/mathplayer/
        target="\string_blank"]{}{}%
MathPlayer\EndLink{}.
A MathML display engine for the MicroSoft's Internet Explorer web browser.
%-----------------------------------------------------------------------------------
\item
\Link[http://www.mozilla.org
        target="\string_blank"]{}{}%
Mozilla\EndLink{}.
A MathML enabled browser
%-----------------------------------------------------------------------------------
\item
\Link[http://www.integretechpub.com/
        target="\string_blank"]{}{}%
TechExplorer\EndLink{}.
 A plug-in for viewing 
 a large subset of
\TeX, \LaTeX, and AMS-\LaTeX, as well as MathML.
% %---------------------------------------------------------------------------
\item
\Link[http://www.dessci.com
      target="\string_blank"]{}{}WebEQ\EndLink{}. A
Java-based renderer for MathML.
%---------------------------------------------------------------------------
\item
\Link[ftp://ftp.tex.ac.uk/tex-archive/macros/xmltex/base/manual.html
      target="\string_blank"]{}{}XML\TeX\EndLink{}. 
LaTeX-based formatter for MathML.
%---------------------------------------------------------------------------
\item
\Link[http://www.pragma-ade.com/
      target="\string_blank"]{}{}Con\TeX t\EndLink{}. 
TeX-based formatter for MathML
%---------------------------------------------------------------------------
\item
\Link[http://www.alanwood.net/unicode/
        target="\string_blank"]{}{}%
Unicode fonts\EndLink{}  for browsers (Alan Wood).
%
% Piotr Grabowski <pgrab@IA.AGH.edu.PL>
% I would like to share withe you a solution I found for zingbats unicode fonts.
% The solution followed Alan Wood's unicode resource.
% 
% This is really a big resource. I studied what is written therein. Soon it
% turned out
% that I do not have enough font characters. From zingbats test I found that
% I had only
% a few characters. The remedium was to install ARIAL UNICODE MS for Win.
% The web page address is attached to the list of Arial fonts on Alan Wood's
% page.
% I downloaded and installed it - the problem disappeared all zingbats
% unicode characters
% are now visible. Alan Wood's web page is also recommending such tools as,
% e.g. Character Agent
% to verify what type of fonts you really have in your machine. Such tools as
% ``character map''
% normally available in Win systems are toys!
% 
% New Mozilla is quite nice - red fonts are now all in black as the rest.

%-----------------------------------------------------------------------------------


\EndList
\EndHPage{},
%
\HPage{OpenDocument}\ExitHPage{}\SubSection{OpenDocument}\--OpenDocument///%
\List{disc}
\item
     \Link[http://books.evc-cit.info/odbook/book.html
        target="\string_blank"]{}{}%
      OpenDocument Essentials\EndLink{}  (OASIS)
\item
     \Link[http://www.oasis-open.org/committees/download.php/12572/OpenDocument-v1.0-os.pd
        target="\string_blank"]{}{}%
      Open Document Format for Office Applications\EndLink{}  (OASIS)
\EndList
\EndHPage{},
%
\HPage{DocBook}\ExitHPage{}\SubSection{DocBook}\--DocBook///%
\List{disc}
\item
     \Link[http://www.oasis-open.org/docbook/xml/
        target="\string_blank"]{}{}%
      The official DocBook Homepage\EndLink{}  (Oasis)
\item
     \Link[http://www.docbook.org/tdg5/en/html/docbook.html
        target="\string_blank"]{}{}%
      DocBook: The Definitive Guide\EndLink{}  (Norman Walsh and Leonard Muellner)

\item
     \Link[http://db2latex.sourceforge.net/
        target="\string_blank"]{}{}%
      DB2LaTeX XSL Stylesheets\EndLink{}  

\EndList
\EndHPage{},
%
\HPage{TEI}\ExitHPage{}\SubSection{TEI}
\--TEI///%
\List{disc}
\item
\Link[http://www.tei-c.org/ 
     target="\string_blank" ]{}{}Home page\EndLink{} 
\item
\Link[http://www.oasis-open.org/cover/tei.html
     target="\string_blank" ]{}{}XML TEI\EndLink{} 
(OASIS)
\EndList 
\EndHPage{},
%
\HPage{Style Sheets}\ExitHPage{}\SubSection{Style Sheets}
\List{disc}
\item
  \Link[http://www.w3.org/Style/CSS/
        target="\string_blank"]{}{}%
     CSS (W3C)\EndLink{}  
       (\Link[http://www.w3.org/TR/REC-CSS1
            target="\string_blank"]{}{}%
         CSS1\EndLink{},
        \Link[http://www.w3.org/TR/REC-CSS2
             target="\string_blank"]{}{}%
           CSS2\EndLink{},
        \Link[http://www.w3.org/Style/CSS/Test/
             target="\string_blank"]{}{}%
           examples\EndLink{})
\item
        \Link[http://library.thinkquest.org/15074/cssmain.html
        target="\string_blank"]{}{}%
                 tutorial\EndLink{} (Oracle ThinkQuest)
\item
     \Link[http://www.w3.org/MarkUp/Guide/Style
        target="\string_blank"]{}{}%
CSS      tutorial\EndLink{}  (Dave Raggett)
\item
  \Link[http://www.w3.org/TR/WD-xsl/
        target="\string_blank"]{}{}%
     XSL\EndLink{} (W3C)
\item
 \Link[http://www.w3.org/TR/xsl/
        target="\string_blank"]{}{}XSL-FO\EndLink{} (W3C)
\item
 \Link[http://nwalsh.com/docs/tutorials/xsl/xsl/slides.html
        target="\string_blank"]{}{}XSL Concepts and Practical Use\EndLink{} (Paul Grosso and Norman Walsh)
\EndList 
\EndHPage{},
%
\HPage{Validators}\ExitHPage{}\SubSection{Validators}

\--Validation///%
\List{disc}
\item XML: xmllint
\item
CSS:
\Link[http://jigsaw.w3.org/css-validator/
     target="\string_blank" ]{}{}WDG 
    jigsaw.w3.org\EndLink{} 



\EndList
\EndHPage{}
%
\CR
Converters\BR into\BR HTML/XML: \&

% http://union.ncsa.uiuc.edu/HyperNews/get/www/html/converters.html

\ResourceList


%--------------------------------------------------------------------------
\Item
\Link[http://www.astro.gla.ac.uk/users/norman/distrib/bibhtml.html
     target="\string_blank"]{}{}BibHTML\EndLink{}\author
     (Norman Gray)\ContItem
A Bib\TeX{} to HTML converter.
%--------------------------------------------------------------------------
% \Item
% \Link[http://sgalland.multimania.com/english/tools/bib2html/ 
%      target="\string_blank"]{}{}Bib2HTML\EndLink{}\author
%      (Stephane Galland)\ContItem
% Perl script for creating HTML output from a Bib\TeX{} database.
%  St�phane Galland <stephane.galland@emse.fr> 
% at home sgalland@multimania.com 
%--------------------------------------------------------------------------
% \Item
% \Link[http://rikblok.cjb.net/scripts/bbl2html.awk
%      target="\string_blank"]{}{}Bib2HTML\EndLink{}\author
%      (Rik Blok)\ContItem
% AWK script for creating HTML output from a Bib\TeX{} database.
% %--------------------------------------------------------------------------
% \Item
% \Link[http://www.uni-koblenz.de/ag-ki/ftp/bib2html/
%      target="\string_blank"]{}{}Bib2HTML\EndLink{}\author
%      (David Hull)\ContItem
% Perl script for creating HTML output from a Bib\TeX{} database.
%--------------------------------------------------------------------------
\Item
\Link[http://www.ibiblio.org/pub/packages/TeX/biblio/bibtex/utils/bibtools/bib2html
     target="\string_blank"]{}{}Bib2HTML\EndLink{}\author
     (David Kotz)\ContItem
Perl script for creating HTML output from a Bib\TeX{} database.
%--------------------------------------------------------------------------
%--------------------------------------------------------------------------
\Item
\Link[http://www.shelldorado.com/scripts/cmds/bib2html.txt
     target="\string_blank"]{}{}Bib2HTML\EndLink{}\author
     (Heiner Steven)\ContItem
AWK script for creating HTML output from a Bib\TeX{} database.
%--------------------------------------------------------------------------
\Item
\Link[http://www.spinellis.gr/sw/textproc/bib2xhtml/
     target="\string_blank"]{}{}Bib2XHTML\EndLink{}\author
     (Diomidis Spinelli)\ContItem
Bibtex styles and a Perl script driver for creating XHTML output from 
BibTeX citations.
%--------------------------------------------------------------------------
\Item
\Link[http://tug.ctan.org/tex-archive/biblio/bibtex/utils/bib2ml
     target="\string_blank"]{}{}Bib2ML\EndLink{}\author
     (St\'ephane Galland)\ContItem
 A Perl script for generating HTML, XML and SQL files 
from Bib\TeX{} databases.
%-----------------------------------------------------------------------------------
\Item
\Link[http://www.lri.fr/\string ~filliatr/bibtex2html/index.en.html
     target="\string_blank"]{}{}Bib\TeX2HTML\EndLink{}\author
     (Jean-Christophe Filliatre)\ContItem
A Bib\TeX{} to HTML converter written in Objective Caml.
%--------------------------------------------------------------------------
\Item
\Link[http://www.blahtex.org/
     target="\string_blank"]{}{}Bib\TeX2HTML\EndLink{}\author
     (David Harvey)\ContItem
A tool that translates TeX math into MathML for  MediaWiki
%-----------------------------------------------------------------------------------
%%---------------------------------------------------------------------------
\Item
\Link[http://www.mostang.com/\%7Edavidm/dlh.html
     target="\string_blank"]{}{}Dlh\EndLink{}\author (David Mosberger)\ContItem
 A C translator for converting from a subset of \LaTeX{} to   HTML.
%-----------------------------------------------------------
% \Item
% \Link[http://www.dcs.fmph.uniba.sk/\string~emt/EmSystem.html
%      target="\string_blank"]{}{}Euromath\EndLink{} \ContItem
%  A converter from
% \LaTeX{} to Euromath SGML
%-----------------------------------------------------------------------------------
\Item
\Link[http://math.albany.edu:8000/math/pers/hammond/igl.html
     target="\string_blank"]{}{}GELLMU\EndLink{}\author
     (William F. Hammond)\ContItem
 A general-purpose SGML authoring language based on 
\LaTeX{} syntax.
%
% GELLMU is not LaTeX nor is it an HTML generator (but see below); rather it
% is a general-purpose SGML authoring language that is based on traditional
% LaTeX syntax (to the extent possible). 
% 
% The design of GELLMU envisions a three stage system. 
% 
% Stage 1 uses an E-lisp program to convert GELLMU input into an SGML
% document. (E-lisp is the dialect of Lisp that underlies GNU Emacs.) 
% 
% Stage 2 uses the SGML parser ``nsgmls'' of James Clark to produced a fully
% parsed text stream from the SGML arising out of stage 1. 
% 
% Stage 3 uses author-provided collections of Perl codelets for David
% Megginson's ``sgmlspl'' SGML-processing program, actually more of a
% % framework, that is part of his ``sgmlspm'' Perl library.
%-----------------------------------------------------------------------------------
% \Item
% \Link[http://www.psyx.org/mowgli/
%      target="\string_blank"]{}{}Hermes\EndLink{}\author
%      (Romeo Anghelache)\ContItem
% A converted for a subset of \LaTeX{} concentrating on getting content-MathML 
% out of math.
% From: Romeo Anghelache <romeo@psyx.org>
%     Date: Fri, 26 Sep 2003 14:03:23 +0200
%     Message-ID: <3F742B0B.4040703@psyx.org>
%     To: www-math@w3.org
%     
% 
% Hi,
% 
% a new tool for getting Content-MathML out of LaTeX sources is in current
% development, its name is Hermes.
% Current version is 0.7 and is relatively useful for testing and feedback.
% It is covered by GPL, the details and the sources are available here
% (http://www.psyx.org/mowgli/).
% 
% The official Hermes pages/presentations will grow here:
% http://relativity.livingreviews.org/Info/AboutLR/mowgli/index.html
% 
% The updated source distributions versions of hermes will be also
% available on the MoWGLI project's website (http://www.mowgli.cs.unibo.it).
%-----------------------------------------------------------------------------------
\Item
\Link[http://www.aei.mpg.de/hermes/
     target="\string_blank"]{}{}Hermes\EndLink{}\author
     (Romeo Anghelache)\ContItem
A LaTeX to XML convertor with a presentation MathML target for math.  
When possible, content MathML code is introduced as a wrapper that supplies semantics to the expressions.
Sources are compiled
with the native \TeX{} compiler to obtain output seeded with DVI specials. 
A postprocessor extracts the output from the DVI code.  
 % http://www.ictp.trieste.it/~its/2004/casestudies/1/Hermes_files/frame.htm
%-----------------------------------------------------------------------------------
\Item
\Link[http://para.inria.fr/\string~maranget/hevea
     target="\string_blank"]{}{}HEVEA\EndLink{}\author
     (Luc Maranget)\ContItem
An Objective Caml translator from a subset of \LaTeX{} to HTML.
% From Luc.Maranget@inria.fr Mon Aug  3 00:29:53 EDT 1998
% Article: 145813 of comp.text.tex
% Path: news.cse.ohio-state.edu!nntp.sei.cmu.edu!bb3.andrew.cmu.edu!honeys.cs.cmu.edu!goldenapple.srv.cs.cmu.edu!not-for-mail
% From: maranget@conti.inria.fr (Luc Maranget)
% Newsgroups: comp.text.tex
% Subject: HEVEA 1.0, a LaTeX to HTML translator in Caml
% Date: 1 Aug 1998 17:49:11 GMT
% Organization: INRIA Rocquencourt
% Lines: 41
% Distribution: world
% Message-ID: <6pvken$3un$1@goldenapple.srv.cs.cmu.edu>
% Reply-To: Luc.Maranget@inria.fr
% NNTP-Posting-Host: raw.fox.cs.cmu.edu
% Originator: rowan@raw.fox.cs.cmu.edu
% Xref: news.cse.ohio-state.edu comp.text.tex:145813
% 
% It is my pleasure to anounce the first release of HEVEA.
% 
%       This is HEVEA, version 1.0, a fast LaTeX to HTML translator.
% 
% 
%     HEVEA is a LaTeX to HTML translator.  The input language is a fairly
%     complete subset of LaTeX2e (old LaTeX style is also accepted) and the
%     output language is HTML that is (hopefully) correct with respect to
%     version 3.2.
% 
%     Exotic symbols are translated into symbols
%     pertaining to the symbol font of the HTML browser, using the
%     non-standard FACE attribute of the FONT tag.
%     This allows the translation to HTML of quite a lot of the symbols
%     used in LaTeX.
% 
%     HEVEA understands LaTeX macro definitions. Simple user style
%     files are understood with little or no modifications.
%     Furthermore, HEVEA customization is done by writing LaTeX code.
%     
%     HEVEA is written in Objective Caml, as many lexers. It is quite fast
%     and flexible. Using HEVEA it is possible to translate large documents
%     such as manuals, books, etc. very quickly. All documents are
%     translated as one single HTML file. Then, the output file can be cut
%     into smaller files, using the companion program HACHA.
% 
% CONTACTS
%     mail:      Luc.Maranget@inria.fr
%     distribution:  ftp://ftp.inria.fr/INRIA/Project/para/hevea
% 
% DOCUMENTATION
%     On-line documentation is available at
%     http://para.inria.fr/~maranget/hevea
%     The first sections explain how to use hevea.
% 
% IMPORTANT REQUIREMENT
%     To compile HEVEA, you need Objective Caml 1.07, see
%     http://caml.inria.fr/ocaml/
% 
% --Luc Maranget
%
%-----------------------------------------------------------------------------------
\Item
\Link[http://www.gams.com/contrib/htex/htex.htm
     target="\string_blank"]{}{}H\TeX\EndLink{}\author
     (Thomas F. Rutherford)\ContItem
 A preprocessor which expands  
html by a new tag: \`'<eq>..</eq>'. An \''<eq>' 
signifies the start of a \LaTeX-readable equation, 
and a \''</eq>' indicates the end of an equation. Anything appearing
between these tags is treated as \LaTeX{} commands within a \`'displaymath'
 section of a \LaTeX{} document.
%
%---------------------------------------------------------------------------
\Item
\Link[http://www-db.stanford.edu/\string~sergey/htmltex/
     target="\string_blank"]{}{}Html\TeX\EndLink{} \author (Sergey Brin)\ContItem
Perl script for converting a subset of \LaTeX{} to HTML
%-----------------------------------------------------------------------------------
\Item
\Link[http://www.southernct.edu/\string ~fields/htmx/
      target="\string_blank"]{}{}HTMX\EndLink{}\author (Joe Fields)\ContItem
Perl script for translating   \LaTeX{} equations embedded within HTML into gif.
%
%-----------------------------------------------------------------------------------
\Item
\Link[http://hyperlatex.sourceforge.net/
      target="\string_blank"]{}{}Hyper\LaTeX\EndLink{}\author (Otfried Cheong)\ContItem
A processor for a subset of \LaTeX{} enriched with hypertext-oriented
commands.
% http://www.cs.uu.nl/\string~otfried/Hyperlatex
%
% \Link[http://www.postech.ac.kr/\string~otfried/html/hyperlatex.html
%
%    In anticipation of my moving to Hong Kong, I have already moved the
%    Hyperlatex homepage to ``http://www.cs.ust.hk/~otfried/hyperlatex.html''.
%    The primary ftp distribution site is now ``ftp.cs.ust.hk:/pub/ipe''.
%
%    Please update all links you may have to the Hyperlatex pages.
%    (Roland, can you update the introductory text on the mailing list?)
%
%    Otfried
%
%------------------------------------------------------------------------------          says that -----
\Item
\Link[http://xxx.lanl.gov/hypertex/
      target="\string_blank"]{}{}Hyper\TeX\EndLink{}\author
(Tanmoy Bhattacharya, David Carlisle, Mark Doyle, Paul Ginsparg, Alan
Jeffrey, Hiroshi Kubo, Kasper Peeters, Sebastian Rahtz and Arthur
Smith)\ContItem{}  A convention for inclusion of hyperlinks in TeX and LaTeX
documents, supported by packages that automatically introduce the
links into the code.
%
%     4.Invest in the hyperTeX conventions (standardised \special commands);
%     there are supporting macro
%          packages for plain TeX and LaTeX). 
%     
%     The HyperTeX project aims to extend the functionality of all the LaTeX
%     cross-referencing commands (including
%     the table of contents) to produce \special commands which are parsed
%     by DVI processors conforming to
%     the HyperTeX guidelines; it provides general hypertext links,
%     including those to external documents. 
%     
%     The HyperTeX spec viewers/translators
%     must recognize the following set of
%     \special commands: 
%     
%     href:
%          html:<a href = ``href_string''> 
%     name:
%          html:<a name = ``name_string''> 
%     end:
%          html:</a> 
%     image:
%          html:<img src = ``href_string''> 
%     base_name:
%          html:<base href = ``href_string''> 
%     
%     The href, name and end commands are used to do the basic hypertext
%     operations of establishing links between
%     sections of documents. 
%     
%     Further details are available on http://xxx.lanl.gov/hypertex/; there
%     are two commonly-used implementations of
%     the specification, a modified xdvi and a modified dvips. Output from
%     the latter may be used in a modified
%     GhostScript or Acrobat Distiller. 
%-----------------------------------------------------------------------------------
\Item
\Link[http://pear.math.pitt.edu/mathzilla/itex2mmlItex.html
      target="\string_blank"]{}{}Itex2mml\EndLink{}\author
 (Paul Gartside)\ContItem
A C program for translating a dialect of \LaTeX{} into MathML.
%-----------------------------------------------------------------------------------
\Item
\Link[http://dlmf.nist.gov/LaTeXML/
      target="\string_blank"]{}{}\LaTeX ML\EndLink{}\author
 (Bruce Miller)\ContItem
A program for translating \LaTeX{} into XML (including XHTML and MathML).
%-----------------------------------------------------------------------------------
\Item
\Link[http://www.tug.org/mailman/listinfo/latex2html
      target="\string_blank"]{}{}\LaTeX 2HTML\EndLink{}\author
 (Nikos Drakos)\ContItem
 A perl script for translating \LaTeX{} into HTML.
%
%  http://www.latex2html.org    
%  http://tug.org/mailman/listinfo/latex2html    
%
%     LaTeX2HTML needs Perl, the PBM utilities, dvips, GhostScript, and
%     other sundries; it assumes it is running on a Unix system. Michel
%     Goossens and Janne Saarela published a detailed discussion of
%     LaTeX2HTML, and how to tailor it, in TUGboat 16(2). 
%-----------------------------------------------------------------------------------
% \Item
% \Link[http://sunsite.sut.ac.jp/pub/archives/ctan/support/latex2hyp/
%     target="\string_blank"]{}{}\LaTeX 2hyp\EndLink{}\author
%     (Roger Nelson)\ContItem
% A C program for converting  \TeX{} and \LaTeX{} documents
% into plain text, capable of introducing HTML links.
%%-----------------------------------------------------------------------------------
\Item
\Link[http://ctan.tug.org/tex-archive/support/latex2man/latex2man.html
    target="\string_blank"]{}{}\LaTeX 2man\EndLink{}\author
    (J\"urgen Vollmer)\ContItem
A Perl script for translating  UNIX manual pages written with \LaTeX{}
 into  UNIX man(1)-command,  HTML, or TexInfo code.
%%-----------------------------------------------------------------------------------
% \Item
% \Link[http://www.soft4science.com/fr\string
%         _index.html?/products/LaTeX2MathML/s4s\string_LaTeX2MathML.html
%     target="\string_blank"]{}{}\LaTeX 2MathML\EndLink{}\author
%     (Soft4Science)\ContItem
% A Microsoft .NET Framework program for translating LaTeX math expressions into MathML.
%
%%-----------------------------------------------------------------------------------
 \Item
 \Link[http://www.localghost.at/latex2mathml/
     target="\string_blank"]{}{}\LaTeX 2MathML\EndLink{}\author
     (Jens Breit)\ContItem
A Python program to translate \LaTeX{} math into MathML.
%%-----------------------------------------------------------------------------------
 \Item
 \Link[http://www.cucat.org/projects/latex2mathml/index.php
     target="\string_blank"]{}{}\LaTeX 2MathML\EndLink{}\author
     (Greg Kearney)\ContItem
 A  Macintosh program to translate \LaTeX{} math into MathML.
%
% >From: Greg Kearney <gkearney@gmail.com> 
% >To: Technical Developments Discussion List  
% ><technical-developments@mail.daisy.org>, 
% >         Discussion of Digital Talking Books <dtb-talk@nfbnet.org>, 
% >         macvoiceover <macvoiceover@freelists.org> 
% >Subject: LaTeX2MathML 
% >Date: Sat, 31 Jan 2009 18:21:47 -0700 
% >X-Mailer: Apple Mail (2.930.3) 
% > 
% > 
% >With the advent of MathML support in DAISY I have put together a 
% >simple MacOS X program that will convert LaTeX math code into MathML 
% >code. This is a GUI for blahtex. It is a universal application (intal 
% >and PPC). You can get it and the Xcode source code from the following 
% >URL: 
% > 
% >http://www.cucat.org/projects/latex2mathml/index.php 
% > 
% >Greg Kearney 
% >535 S. Jackson St. 
% >Casper, Wyoming 82601 
% >307-224-4022 
% >gkearney@gmail.com 

%%-----------------------------------------------------------------------------------
\Item
\Link[http://perso.wanadoo.fr/eric.chopin/latex/latex_subset.htm
    target="\string_blank"]{}{}\LaTeX 4Web\EndLink{}\author
    (Eric Chopin)\ContItem
A javascript for translating a subset of \LaTeX{} to HTML.
%
%-----------------------------------------------------------------------------------
\Item
\Link[http://yum.math.hmc.edu/ctan/support/ltoh/readme.html
    target="\string_blank"]{}{}Ltoh\EndLink{}\author
 (Russell W. Quong)\ContItem
A customizable Perl converter from a subset of \LaTeX{} to HTML.
%
%-----------------------------------------------------------------------------------
\Item
\Link[ftp://ftp.dante.de/tex-archive/support/ltx2x/ltx2x.html
    target="\string_blank"]{}{}Ltx2x\EndLink{}\author (Pete R.r Wilson)\ContItem
A table-driven program that will replace \LaTeX{} commands by user
     defined text. 
%
%-----------------------------------------------------------------------------------
% \Item
% \Link[http://www.go.dlr.de/fresh/unix/src/www/.warix/math2html-1.3.tar.gz.html
%      target="\string_blank"]{}{}Math2HTML\EndLink{}\author
%    (Janne Saarela)\ContItem
% A bison program for converting \LaTeX{} mathematics and tables into HTML 3.0.
%
%-----------------------------------------------------------
% \Item
% \Link[http://www.stilo.com/news/pr-1999-06-12b.html
%      target="\string_blank"]{}{}MathWriter\EndLink{}\author
%      (Stilo)\ContItem
% A MathML (``MathWriter'') and XML (``WebWriter'') editors
% capable of inputing \TeX{} expressions.
%-----------------------------------------------------------------------------------
\Item
\Link[http://www.mathtran.org/index.html
     target="\string_blank"]{}{}MathTran\EndLink{}\author Jonathan Fine()\ContItem
A modified \TeX{} compiler aiming at  translating mathematical content from TeX to MathML
 and vice-versa, and to graphics formats.
%
%-----------------------------------------------------------------------------------
\Item
\Link[http://www.micropress-inc.com/webb/wbstart.htm
     target="\string_blank"]{}{} MicroPress
                   \TeX pider\EndLink{}\author (MicroPress)\ContItem
A modified \TeX{} compiler  for PCs  for translating \LaTeX{} to HTML.
%
%-----------------------------------------------------------
% \Item
% \Link[http://www.mathmlstudio.com/
%      target="\string_blank"]{}{} MathMLStudio\EndLink{}\author (?)\ContItem
% A translator for a subset of \TeX, \LaTeX, and AMS\TeX{} math into MathML.
%
%-----------------------------------------------------------
% \Item
% \Link[http://www.maths.tcd.ie/pub/openmath/
%      target="\string_blank"]{}{}Ml2om\EndLink{}\author
%      (Richard M. Timoney)\ContItem
%  \LaTeX{} math to OpenMath converter. 
%
%-----------------------------------------------------------
 \Item
 \Link[http://www.orcca.on.ca/MathML/projects.html
      target="\string_blank"]{}{}ORCCA\EndLink{}\author
      (?)\ContItem
A Java program for translating TeX mathematics  into MathML
%-----------------------------------------------------------
% \Item
% \Link[http://omega.enstb.org
%      target="\string_blank"]{}{}Omega\EndLink{}\author
%      (Yannis Haralambous and John Plaice)\ContItem
% A \TeX{} compiler capable of producing MathML code.
% %http://sourceforge.net/projects/omega-system/
%-----------------------------------------------------------
\Item
\Link[http://plastex.sourceforge.net/
     target="\string_blank"]{}{}plasTeX\EndLink{}\author
     (Kevin Smith)\ContItem
A \LaTeX{} document processing framework written entirely in Python.
A successor to 
\Link[http://pylatex.sourceforge.net/
     target="\string_blank"]{}{}pyLaTeX\EndLink{}.
%  Kevin.Smith@themorgue.org (Kevin Smith)
%  equations are parsed into pseudo-mathml
%-----------------------------------------------------------
\Item
\Link[http://richdoc.sourceforge.net/doc/en/html/index.html
     target="\string_blank"]{}{}RichDoc\EndLink{}\author
     (Michal \v Sev\v cenko)\ContItem
A document preparation system capable to import \LaTeX{} and export HTML
%-----------------------------------------------------------
\Item
\Link[http://dione.zcu.cz/\string ~toman40/selathco/
     target="\string_blank"]{}{}Selathco\EndLink{}\author
     (Petr Toman)\ContItem
A simple extensible \LaTeX{} to HTML converter written in Java.
%-----------------------------------------------------------
\Item
\Link[http://www.ph.ed.ac.uk/snuggletex/
     target="\string_blank"]{}{}SnuggleTeX\EndLink{}\author
     (David McKain)\ContItem
A converter from a subset of  \LaTeX{} to XHTML+MAthML  written in Java.
%-----------------------------------------------------------------------------------
\Item
\Link[http://web.informatik.uni-bonn.de/\string
      ~zach/vim/index.html
      target="\string_blank"]{}{}\TeX2HTML\EndLink{}\author
(Gabriel Zachmann)\ContItem
 Vim macros for converting from
  \LaTeX{} to HTML
%%-----------------------------------------------------------------------------------
\Item
\Link[http://www.ccs.neu.edu/home/dorai/tex2page/tex2page-doc.html
      target="\string_blank"]{}{}\TeX2Page\EndLink{}\author
(Dorai Sitaram)\ContItem
A Scheme script for translating a subset of plain \TeX{}  to
HTML. 
% was
% http://www.cs.rice.edu/\string~dorai/tex2html/tex2html.html   
%-----------------------------------------------------------------------------------
\Item
\Link[http://www.ctan.org/tex-archive/support/tex2rtf/
     target="\string_blank"]{}{}\TeX 2RTF\EndLink{}\author
 (Julian Smart)\ContItem
tmEA A C++ utility for converting from a subset of \LaTeX{} to   HTML and other formats.
%%%-----------------------------------------------------------------------------------%-----------------------------------------------------------------------------------
\Item
\Link[http://www.cmt.phys.kyushu-u.ac.jp/\string
          ~M.Sakurai/java/sdoc/index-e.html
     target="\string_blank"]{}{}tex2sdoc\EndLink{}\author
 (Masashi Sakurai)\ContItem
A Java program for translating a subset of \LaTeX{} into
\Link[http://www.asahi-net.or.jp/\string
    ~dp8t-asm/java/tools/SmartDoc/]{}{}SmartCode\EndLink
%%%-----------------------------------------------------------------------------------
\Item
\Link[http://www.mathematik.uni-kl.de/\string~obachman/Texi2html/
      target="\string_blank"]{}{}Texi2HTML\EndLink{}\author
(Lionel Cons)\ContItem
A Perl script for  converting Texinfo manuals to HTML. 
%-----------------------------------------------------------------------
% \Item
% \Link[http://sunland.gsfc.nasa.gov/info/texi2www/
%       target="\string_blank"]{}{}Texi2www\EndLink{}\author
% (Tim Singletary)\ContItem
%  Converts Texinfo manuals to HTML. 
% Another Perl script for  converting Texinfo manuals to HTML. 
% 
%---------------------------------------------------------------------------
\Item
\Link[http://www.ktalk.com/texport-web.html    
      target="\string_blank"]{}{}\TeX Port WEB\EndLink{}\author (K-talk Communications)\ContItem
A program running on MS-DOS or
          MS-Windows. 
%---------------------------------------------------------------------------
\Item
\Link[http://www.xm1math.net/ttwp/index.html
      target="\string_blank"]{}{}TexToWebPublishing\EndLink{}\author (P.Brachet / J.Amblard)\ContItem
A perl script for producing HTML from \LaTeX{} 
%---------------------------------------------------------------------------
% \Item
% \Link[http://www-unix.mcs.anl.gov/\string~gropp/manuals/tohtml/tohtml.html
%       target="\string_blank"]{}{}Tohtml\EndLink{}\author (William Gropp)\ContItem
% A C program for translating a subset of \LaTeX{} to HTML.
% %-----------------------------------------------------------------------------------
%---------------------------------------------------------------------------
\Item
\Link[http://www-sop.inria.fr/miaou/Jose.Grimm/tralics/
       target="\string_blank"]{}{}Tralics\EndLink{}\author (Jos� Grimm)\ContItem
 A C++ program for translating a subset of \LaTeX{} to a local XML.
-----------------------------------------------------------------------------------
\Item
\Link[http://hutchinson.belmont.ma.us/tth/ 
      target="\string_blank"]{}{}Tth\EndLink{}\author
(Ian Hutchinson)\ContItem
%hutch@pfc.mit.edu
A C program for translating a subset of plain \TeX{} and \LaTeX{}
into HTML. Uses special fonts and tables to represent formulas.
%
% omp.text.tex #118834 (0 + 10 more)
% From: Ian Hutchinson <hutch@pfc.mit.edu>
% Newsgroups: comp.text.tex,comp.infosystems.browsers.x
% Subject: tth Plain TeX to HTML translator does LaTeX too.
% Date: Fri May 23 20:53:46 EDT 1997
% Organization: MIT
% Lines: 16
% Mime-Version: 1.0
% Content-Type: text/plain
% Content-Transfer-Encoding: 7bit
% X-Mailer: Mozilla 3.0 (X11; I; Linux 2.0.28 i486)
% CC: hutch@pfc.mit.edu
% 
% tth is a TeX to HTML translator that renders mathematics and most of the
% relevant aspects of TeX into HTML3.2.
% 
% It does not use images for hard-to-translate items like mathematics.
% Instead it uses fonts and HTML tables to give compact output that is
% semantically correct. It consists of a single executable available as a
% single C file and therefore portable to any system. Some precompiled
% executables are also available. Installation is trivial.
% 
% tth is extremely fast, and hence suitable for a CGI script.
% 
% In the latest version, 0.60, tth simultaneously supports a major subset
% of LaTeX.
%
%     omp.text.tex #114492 (0 + 4 more)
%        [1]
%     From: Ian Hutchinson <hutch@pfc.mit.edu>
%     [1] Plain TeX to HTML Translator: tth
%     Date: Sat Mar 15 21:19:39 EST 1997
%     Organization: MIT
%     Lines: 18
%     Mime-Version: 1.0
%     Content-Type: text/plain
%     Content-Transfer-Encoding: 7bit
%     X-Mailer: Mozilla 3.0 (X11; I; Linux 2.0.28 i486)
%     
%     Version 0.2 of tth, may be obtained from.
%     
%     http://venus.pfc.mit.edu/tth/
%     
%     It translates TeX source that uses the plain macro package (not LaTeX)
%     into a near equivalent in HTML.
%     
%      It can't do a perfect job, of course, because HTML is not capable of
%     many of TeX's more subtle tricks. Nevertheless, on TeX files that are
%%     predominantly text and use standard constructs and macros, the results
%%     are rather satisfactory.
%%     
%%     The new version adds support for \def, \edef, \xdef, and counters.
%%     
%%     LaTeX has its own translator, latex2html, which is recommended for
%%     LaTeX
%%     code.
%%     
%%     Ian Hutchinson.
%%
%%-----------------------------------------------------------
%From rubber@bounce.com Fri Feb 20 10:25:23 EST 1998
%Article: 134658 of comp.text.tex
%Path: news.cse.ohio-state.edu!news.maxwell.syr.edu!nntp.news.xara.net!xara.net!server6.netnews.ja.net!server4.netnews.ja.net!server2.netnews.ja.net!news.ox.ac.uk!not-for-mail
%From: rubber@bounce.com (Keith Refson - real email address in signature)
%Newsgroups: comp.text.tex
%Subject: Re: Alternatives to latex2HTML
%Date: 20 Feb 1998 13:55:14 +0000
%Organization: Oxford University
%Lines: 65
%Sender: keith@everest
%Message-ID: <wxbtw2bcu5.fsf@earth.ox.ac.uk>
%References: <pfroehli-1602981824130001@uss-excalibur.ics.uci.edu>
%       <3463E1.4E3E337F@pfc.mit.edu> <lbzg1lhnz3j.fsf@tag.uio.no>
%       <34873.349FD14@pfc.mit.edu> <bonard.887900444@hqedqr48>
%       <wxpvkjtxdi.fsf@earth.ox.ac.uk>
%NNTP-Posting-Host: everest.earth.ox.ac.uk
%Mime-Version: 1.0 (generated by -edit 7.106)
%Content-Type: text/plain; charset=US-ASCII
%X-Newsreader: Gnus v5.3/Emacs 19.34
%Xref: news.cse.ohio-state.edu comp.text.tex:134658
%
%
%In some earlier ramblings I wrote:
%
%> I did take a look at ttH but I couldn't even look at the homepage
%> without reconfiguring my browser. 
%
%Although that was factually correct it could give a misleading
%impression that I was criticizing ttH, which I hadn't even tried.
%Well I feel a bit guilty about that, so now I *have* tried it out.
%
%1) It certainly does work as advertised.  In fact you can display
%   simple maths by adding a 1 line X resource for Netscape.  
%   The documentation claims you can improve the output further,
%   but this is at the cost of some highly undesirable mucking
%   about with font aliases in the X server which I imagine
%   would cause many other problems not just for Netscape but
%   for other X programs.
%
%   Unfortunately this does really rule it out for web *publishing*
%   for my purposes.  But it could be very useful in, say, a 
%   teaching laboratory or local environment where you have
%   administrative control over which browsers are installed and
%   their setup.
%  
%2) The maths handling *is* limited.  It doesn't do well on
%   complex or sophisticated maths, a limitation of it's
%   all-html approach.  But simple display formulae, equations,
%   matrices, sums and integals do come out surprisingly readable. 
%
%3) You get 1 huge html file at the end.  There is a version which
%   handles this better but it costs money.
%
%4) Handling of included graphics is rudimentary compared to
%   latex2html.  Simple \includegraphics of ps or 
%   should work, but my figures are more complicated than that!
%
%5) There are some bugs in the parser:  eg it can't handle
%   switching back to text mode inside an equation properly
%   and crashes.  However this is no worse than I found with
%   the earlier versions of latex2html.
%
%6) It does run about 1000 times faster than latex2html!  No, that
%   isn't a mistype!
%
%7) It doesn't seem to be available in source form so presumably
%   only the author can fix bugs.  (You get a  compilable C
%   file but the real sources are in lex and yacc and you don't get
%   those.)
%
%8) It's only free for non-commercial use but can be purchased for
%   commercial applications.
%
%
%I expect there are many circumstances where ttH will be very useful,
%notably where the maths is simple or unimportant.  In its present form
%it can't replace latex2html for mathematical documents.
%
%Keith Refson
%---------------------------------------------------------------------------
\Item
\Link[http://www.plover.com/\string ~mjd/vulcanize.html
      target="\string_blank"]{}{}Vulcanize\EndLink{}\author (Mark-Jason Dominus)\ContItem
A Perl program for translating a subset of \LaTeX{} to HTML.
%
%     What is Texinfo?
%     
%     Texinfo is a documentation system that uses one source file to produce
%     both on-line information and printed
%     output. So instead of writing two different documents, one for the
%     on-line help and the other for a typeset
%     manual, you need write only one document source file. When the work is
%     revised, you need only revise one
%     document. You can read the on-line information, known as an ``Info
%     file'', with an Info documentation-reading
%     program. By convention, Texinfo source file names end with a .texi or
%     .texinfo extension. You can write
%     and format Texinfo files into Info files within GNU emacs, and read
%     them using the emacs Info reader. If you do
%     not have emacs, you can format Texinfo files into Info files using
%     makeinfo and read them using info. 
%     
%     The Texinfo distribution, including a set of TeX macros for formatting
%     Texinfo files is available as
%     macros/texinfo/texinfo-3.9.tar.gz (also available as a .zip file
%     macros/texinfo/texinfo-3.9.zip). 
%
%     Making Acrobat documents from LaTeX
%     
%     In the simplest case, use your DVI to PostScript driver, and run the
%     result through Adobe's Acrobat Distiller;
%     even simpler, if you use a Mac or Windows TeX system, is to install
%     Acrobat Exchange, and use PDFwriter like
%     a printer from your application. The latter is a dead end, though fine
%     for simple documents, since you can't
%     insert extra hyperlinks in the PDF output. For that, you  most
%     other translators) and gives access to all the
%     functionality of pdfmark. 
%     
%     Sadly, there are no free implementations of Distiller, nor any signs
%     of them. GhostScript (versions 3.51
%     onwards) can display and print PDF files, however, if you are on a
%     platform with no Acrobat Reader. You may
%     see a DVI to PDF translator soon, but do not hold your breath. 
%
%---------------------------------------------------------------------------
% \Item
% \Link[http://mathosphere.net/editeurml/WeM.html
%       target="\string_blank"]{}{}VeM\EndLink{}\author (St\'ephan
%      S\'emirat)\ContItem  
% A PHP program for translating a subset of \LaTeX{} to MathML.
%---------------------------------------------------------------------------
\Item
\Link[http://www.dessci.com/en/products/webeq
      target="\string_blank"]{}{}WebEQ\EndLink{}\author (Mark-Jason Dominus)\ContItem
MathML fro a \TeX{} variant named WebTeX
%-----------------------------------------------------------------------------------
%-----------------------------------------------------------------------------------
\Item
\Link[http://www.gold-saucer.org/mathml/greasemonkey/
%http://planetx.cc.vt.edu/AsteroidMeta/Display\string_LaTeX
      target="\string_blank"]{}{}LaTeX on Web pages\EndLink{}\author
 (Steve Cheng)\ContItem
A JavaScript program for translating  \LaTeX{} into MathML for rendering on the web.
%-----------------------------------------------------------------------------------
\Item
\Link[http://www.math.union.edu/\string~dpvc/jsMath/
      target="\string_blank"]{}{}jsMath\EndLink{}\author
 (Davide P. Cervone)\ContItem
A JavaScript program for rendering  \LaTeX{} on the web.
%%--------------------------------------------------------------------------
\EndResourceList 
%--------------------------------------------------------------------------
\CR
Backward\BR Converters: \&
\ResourceList
\Item
   \Link[http://www.chemie.fu-berlin.de/chemnet/use/gf.html
          target="\string_blank"]{}{}%
       Gf\EndLink{}\author   (Gary Houston)\ContItem
 SGML to \LaTeX
% \Item
%    \Link[http://www.hut.fi/\string~jkorpela/h2l/
%           target="\string_blank"]{}{}%
%        H2l\EndLink{}\author (Jukka Korpela)\ContItem  A C program converting HTML to \LaTeX
\Item
   \Link[http://www.wildfire.dircon.co.uk/htex.html
          target="\string_blank"]{}{}%
       Htex\EndLink{}\author (Toby Thurston)\ContItem  HTML to \LaTeX
\Item
   \Link[http://htmltolatex.sourceforge.net/
          target="\string_blank"]{}{}%
       HTMLtoLaTeX\EndLink{}\author (Michal Kebrt)\ContItem     
A Java program
\Item
   \Link[http://www.lib.rpi.edu/dept/acs/rpinfo/filters/GChtml2latex/
          target="\string_blank"]{}{}%
       HTML2\LaTeX\EndLink{}\author   (Davide Cervone \string& Jeffrey Schaefer, Geometry Center)\ContItem
A Perl script for converting HTML into \LaTeX
\Item
   \Link[ftp://ftp.dante.de/tex-archive/support/html2latex/
          target="\string_blank"]{}{}%
       HTML2\LaTeX\EndLink{}\author   (Nathan Torkington)\ContItem
A C-program relying on the HTML parser of the
NCSA Mosaic HTML browser.
%-----------------------------------------------------------------------------------
\Item
   \Link[http://html2latex.sourceforge.net/
          target="\string_blank"]{}{}%
       HTML2\LaTeX\EndLink{}\author   (Peter Thatcher)\ContItem
A Perl script for converting HTML into \LaTeX
% also at http://www.go.dlr.de/fresh/unix/src/www/html2latex-1.1.tar.gz
%-----------------------------------------------------------------------------------
\Item
\Link[http://home.planet.nl/\string ~faase009/html2tex.html
      target="\string_blank"]{}{}HTML2\TeX\EndLink{}\author
(Frans J. Faase)\ContItem
C macros for converting from
HTML 2 to  \LaTeX{} 
% %-----------------------------------------------------------------------------------
% \Item
% \Link[http://www.igd.fhg.de/\string~zach/vim/
%       target="\string_blank"]{}{}HTML2\TeX\EndLink{}\author
% (Gabriel Zachmann)\ContItem
%  Vim macros for converting from
% HTML to  \LaTeX{} 
%-----------------------------------------------------------------------------------
\Item
   \Link[http://sourceforge.net/projects/jadetex
%http://www.tug.org/applications/jadetex/
          target="\string_blank"]{}{}%
       Jade\TeX\EndLink{}\author (Sebastian Rahtz)\ContItem     
SGML to \TeX{} using  DSSSL Formatting Objects  and
the  Jade program  of James Clark
% From: Sebastian Rahtz <sebastian.rahtz@computing-         |         |-(2)--(2)
% +     services.oxford.ac.uk>                              |         \-(2)
% [1] Re: ANN: SGMLbase and HTMLbase                        \-(1)--(1)+-(1)
% Date: Fri Jan 14 04:27:29 EST 2000                                  \-(1)--[1]
% > I am most interested in an alternative to Jade+DSSSL+JadeTeX for hard
% > copy output from SGML.
% 
% write a trivial application using a Perl module like SGMLSpm to
% generate plain LaTeX. that works, no doubt about it. JadeTeX is a joke
% (I speak as the author :-})
\Item
   \Link[http://www.tei-c.org.uk/Software/passivetex/
          target="\string_blank"]{}{}%
       Passive \TeX\EndLink{}\author (Sebastian Rahtz)\ContItem
XML to  \LaTeX{} using XSL
   Formatting Objects
\Item
   \Link[http://www.ucc.ie/info/TeX/sgml2tex.html
          target="\string_blank"]{}{}%
       SGML2\TeX\EndLink{}\author   (Peter Flynn)\ContItem
A PCL (Personal Computer Language, an interpreted language for DOS)
program
% \Item
%    \Link[http://www.active-tex.demon.co.uk
%           target="\string_blank"]{}{}%
%         SGMLbase\EndLink{}\author (Jonathan Fine)\ContItem
% A TeX-based parser for SGML with a style file HTMLbase for typsetting 
% HTML documents.
%---------------------------------------------------------------------
\Item
   \Link[http://www.alphaworks.ibm.com/aw.nsf/techreqs/texml
          target="\string_blank"]{}{}%
       \TeX ML\EndLink{}\author (Doug Lovell)\ContItem
A JAVA translator from \TeX ML-conforming XML into \TeX.
% TeXML? No, dont believe all you read. TeXML is an XML DTD for
% expressing LaTeX. You write a transformation specification which
% expresses your XML in TeXML, and then they have a little processor to
% make `traditional' LaTeX. You still get to do all the hard work!
%---------------------------------------------------------------------
\Item
   \Link[http://getfo.sourceforge.net/texml/
          target="\string_blank"]{}{}%
       \TeX ML\EndLink{}\author (Oleg Paraschenko)\ContItem
A Perl translator from \TeX ML-conforming XML into \TeX.
%---------------------------------------------------------------------
\Item
   \Link[http://www.ctan.org/tex-archive/macros/latex/contrib/carlisle/
          target="\string_blank"]{}{}%
       TypeHTML\EndLink{}\author (David Carlisle)\ContItem     
HTML to \LaTeX
% \Item
%    \Link[http://www.tex.ac.uk/tex-archive/macros/passivetex/
%           target="\string_blank"]{}{}%
%        TypeXML\EndLink{}\author (Sebastian Rahtz)\ContItem     
% TypeHTML modified for XML 
\Item
   \Link[http://kebrt.webz.cz/programs/word-to-latex/index.html
          target="\string_blank"]{}{}%
       Word-to-LaTeX\EndLink{}\author (Michal Kebrt)\ContItem     
A C\# program  running  on Windows with Microsoft Word installed.
\Item
   \Link[http://www.hj-gym.dk/\string~hj/writer2latex/
          target="\string_blank"]{}{}%
       Writer2\LaTeX\EndLink{}\author (Henrik Just)\ContItem     
A Java application for translation OpenOffice documents into \LaTeX
%
% \Item
%    \Link[http://www.jclark.com/xml/xt.html
%           target="\string_blank"]{}{}%
%        XT\EndLink{}\author (James Clark)\ContItem Implementation in Java for 
%    \Link[http://www.w3.org/TR/xslt
%           target="\string_blank"]{}{}%
%         XSL Transformations (XSLT) 1.0\EndLink{}:
% A language for expressing transformations on XML documents.
\Item
   \Link[ftp://ftp.dante.de/pub/tex/support/word2latex/
          target="\string_blank"]{}{}%
       W2LTX\EndLink{}\author (Ingo H. de Boer)\ContItem     
A mostly Unix based API for translating
MS Word documents to LaTeX.
\Item
   \Link[ftp://ftp.tex.ac.uk:/tex-archive/macros/xmltex/base/manual.html
          target="\string_blank"]{}{}%
       XML\TeX\EndLink{}\author (David Carlisle)\ContItem     
A system for typesetting XML files with \TeX.
%
\Item
   \Link[http://www.cse.ohio-state.edu/\string~gurari/tug99/indexSl57.html
          target="\string_blank"]{}{}%
       Brute Force\EndLink{} \ContItem
\EndResourceList 
%--------------------------------------------------------------------------
\CR
Converters\BR for other\BR formats: \&
\ResourceList
%%----------------------------------------------------------------------------
\Item
\Link[http://dvi2bitmap.sourceforge.net/
      target="\string_blank"]{}{}Dvi2bitmap\EndLink{}\author (Norman Gray)\ContItem
 A DVI to GIF and XBM translator.
%%----------------------------------------------------------------------------
\Item
\Link[http://www.ags.uni-sb.de/\string ~adrianf/dvi2svg/help.html
      target="\string_blank"]{}{}Dvi2Svg\EndLink{}\author (Adrian Frischauf)\ContItem
 A DVI to SVG translator.
%%----------------------------------------------------------------------------
\Item
\Link[http://gaspra.kettering.edu/dvipdfm/
      target="\string_blank"]{}{}Dvipdfm\EndLink{}\author (Mark A. Wicks)\ContItem
 A DVI to PDF translator.
%%----------------------------------------------------------------------------
\Item
\Link[http://dvipng.sourceforge.net/dvipng.html
      target="\string_blank"]{}{}dvipng{}\author (Jan-Ake Larsson)\ContItem
A DVI to PNG and GIF convertor. It supports PostScript inclusion. 
%%----------------------------------------------------------------------------
\Item
\Link[http://dvisvg.sourceforge.net/
      target="\string_blank"]{}{}DviSvg\EndLink{}\author (Rudolf Sabo)\ContItem
 A DVI to SVG translator.

% dvisvgm is a command line utility for converting DVI files to SVG
% (Scalable Vector Graphics). The latest release 0.4.2 fixes some bugs and
% introduces options for SVG transformations (rotation, translation, scaling
% etc.)
% 
% dvisvgm has been sucessfully tested on various Linux (teTeX) and Windows
% (MikTeX) systems. Precompiled versions for Windows/MikTeX can be
% downloaded from the project homepage (see README for further information).

%%----------------------------------------------------------------------------
\Item
\Link[http://dvisvgm.sourceforge.net/
      target="\string_blank"]{}{}DviSvgm\EndLink{}\author (Martin Gieseking)\ContItem
 A DVI to SVG translator.
%%----------------------------------------------------------------------------
% \Item
% \Link[http://www.tug.org/TUGboat/Articles/tb22-4/tb72goo.pdf
%       target="\string_blank"]{}{}Dvi2Svg\EndLink{}\author (Michel Goossens and Vesa Sivunen)\ContItem
%  A DVI to SVG translator.
%%----------------------------------------------------------------------------
\Item
\Link[http://www.grindeq.com/
      target="\string_blank"]{}{}GrindEQ\EndLink{}\author ()\ContItem
 Converter from LaTeX equations to  MS Word and backward.
%%%----------------------------------------------------------------------------
\Item
\Link[http://www.forkosh.com/
      target="\string_blank"]{}{}MimeTeX\EndLink{}\author (John Forkosh)\ContItem
A server side parses for LaTeX math expressions, emitting either mime
   xbitmaps or gif images of them
%
%%----------------------------------------------------------------------------
% \Item
% \Link[
%      target="\string_blank"]{}{}ltx2rtf\EndLink{}\author (Georg Lehner)\ContItem
%  La\TeX{} to MicroSoft Word 
% %%----------------------------------------------------------------------------
% \Item
% \Link[
%      target="\string_blank"]{}{}ltx2rtf\EndLink{}\author (Georg Lehner)\ContItem
%  La\TeX{} to MicroSoft Word 
%%----------------------------------------------------------------------------
\Item
\Link[http://www.tug.org/utilities/texconv/ltx2rtf3.html
     target="\string_blank"]{}{}ltx2rtf\EndLink{}\author (Daniel Taupin)\ContItem
 La\TeX{} to MicroSoft Word (rtf: Rich Text Format)
%%----------------------------------------------------------------------------
% \Item
% \Link[http://pantheon.yale.edu/\string~pmm34/leq.html     
%      target="\string_blank"]{}{}LEQ\EndLink{}\author (Paul Magwene)\ContItem
%  La\TeX{} to MicroSoft Word
%%----------------------------------------------------------------------------
\Item
\Link[http://tug.org/applications/pdftex/
     target="\string_blank"]{}{}Pdf\TeX\EndLink{}\author (Han The Thanh)\ContItem
An extension of the \TeX{} compiler 
offering an option for Adobe's Portable Document Format (pdf) output.
%
%ftp://ftp.cstug.cz/pub/tex/local/cstug/thanh/
%
%          Re: tex2pdf binary for Linux ?
%          Date: Fri Mar 07 07:16:40 EST 1997
%          Organization: Elsevier Science Ltd
%          Lines: 44
%          In-reply-to: mnikunen@mat-48.pc.helsinki.fi's message of 6 Mar 1997 17:18:19
%          +            GMT
%          X-Newsreader: Gnus v5.1
%          
%             Changing libpng headers and the library to newer versions helped me to get
%             it compiled and it seems to work OK for a couple of plain documents that I
%             have tried. The file example.tex, however, makes r less any Unix system). A \
%          compiled
%          binary for Windows NT/95 should be available soon.
%          
%          PLEASE do not jump into this stuff without accepting it as beta
%          status. I don't think Han The Thanh will thank me if announcements like
%          this get him deluged with mail saying `where is the Mac version' or
%          .
%          
%          In order to help promulgate information about this development, and
%          exchange ideas, the TeX Users Group  is hosting a mailing list. If you
%          want to join, send a mail message to
% 
% 
% 
% There are several routes.  One is to use DVIPS or similar PS driver
% and then pump the result through Adobe Acrobat Distiller. A less
% desirable method involves printing via the PDFWriter.  There are
% also programs being written to go direct from DVI to PDF, and
% direct from TeX to PDF.  These ar mentioned often on this news group.
% 
% Check out http://www.YandY.com/pdf-from.pdf and also links to
% other web sites discussion this issues on the `Other Information
% Sources' on that site.  Check the comp.text.pdf news group.
%
% 
%%----------------------------------------------------------------------------
% \Item
% \Link[http://shika.aist-nara.ac.jp/products/plain2/plain2.html
%      target="\string_blank"]{}{}Plain2\EndLink{}\author (Akihiro Uchida)\ContItem
%  Plain text to \TeX, ROFF and HTML. 
%%----------------------------------------------------------------------------
\item{\LaTeX{} to ... }
Convert to HTML, then according to the target
\List{}
\item{Text}
Save the view of the outcome on a
text-based browser
(e.g., \''lynx -dump foo.html > foo.txt', or
\Link[http://www.w3m.org/]{}{}w3m\EndLink{}; the latter one might 
treat tables better)
\item {Word}
\''Word', and some other word processors, can load HTML files. The \`'uxhlatex'
is probably the best setting of \TeX4ht for such translations.
\EndList 
%-----------------------------------------------------------------------------------
% \Item
% \Link[http://www.htmlscript.com/docs/userdocs.htm
%       target="\string_blank"]{}{}HtmlScript\EndLink{}\ContItem
%  An extended html development language to be translated by a
% \Link[http://hoohoo.ncsa.uiuc.edu/cgi/
%     target="\string_blank"]{}{}CGI\EndLink{}
%  into html.
% %
%
% \Link[http://www.geom.umn.edu/software/WebEQ/]{}{}webEQ\EndLink{}:
% A  java applet for including mathematics in documents.
%
%           TTTTo use WebEQ, you must use a browser that can run Java
%           . This immediately excludes a large fraction of your
% balise%           readers. Java is also known to pose security risks.  applet
%           takes a long time to load. Upon visiting the WebEQ home page, it took
%           40 seconds for me to see one equation. YYYYou can't print any of the
%           output from WebEQ. WWWWebEQ isn't extensible or
%           reconfigurable. WWWWebEQ expects you to write your equations in a
%           highly presentational language that is based half on TEX and half on
%           the expired HTML 3.0 draft. It is not especially complicated, but it
%           is tedious because there are no definitions for common mathematical
%           constructs: you basically do the layout yourself. Moreover, the
%           limited nature of the medium makes it difficult for your equation to
%           be presented in a text-based browser or using a speech synthesis
%           system. TTTTo put an equation into your document, you have to insert
%           an APPLET element and then place PARAM tags within it to provide the
%           expression, which increases typing and makes editing
%           inconvenien%          
%          t. This combines with the presentational syntax to make for
%          a lot of code even for the simplest expressions.

\EndResourceList 
% %%%%%%%%%%%%%%%%%%%%%%%%%%%%%%%%
% \CR[C2<]
% %%%%%%%%%%%%%%%%%%%%%%%%%%%%%%%
% \HPage{Conversion to Bitmaps}
% \ExitHPage{}\SubSection{Conversion to Bitmaps}
% \List{*}
% \item
% \Link[http://www.fourmilab.ch/webtools/textogif/textogif.html]{}{}\TeX togif\EndLink{} (John Walker)
% \item
% \Link[http://www.astro.gla.ac.uk/users/norman/star/dvi2bitmap/]{}{}Dvi2bitmap\EndLink{}:
% direct conversion of Dvi to Gif (Norman Gray)
% \item
% \Link[http://www.slurm.com/gf2gifs/]{}{}gf2gifs\EndLink{}:
% A Java program for converting MetaFont files into GIF images (Richard Blaylock )
% %  From: Richard Blaylock <blaylock@slurm.com>
% %  To: ``Eitan M. Gurari'' <gurari@cse.ohio-state.edu>
% %  Subject: gf2gifs link
% %  Date: Thu, 24 May 2001 10:05:36 -0700 (PDT)
% %  
% %  
% %  Hi,
% %  
% %  Thanks for putting the link to my GF2GIFs program on
% %    http://www.cse.ohio-state.edu/~gurari/TeX4ht/mn64.html
% %  
% %  When you get a chance, you could take away the question mark
% %  after my name since, yes, I really am the author.
% %  
% %  Thanks again,
% %  Richard Blaylock

%\EndList
%\EndHPage{}
%%%%%%%%%%%%%%%%%%%%%%%%%%%%%%%
% {\tt |}
% %%%%%%%%%%%%%%%%%%%%%%%%%%%%%%%
% \HPage{math on the web}
% \ExitHPage{}\SubSection{math on the web}
% \List{*}
% \item
% \Link[http://www.w3.org/Math/]{}{}W3C\EndLink 's math home page
% \item
% \Link[http://www.oasis-open.org/cover/topics.html]{sgml-math}{}OASIS\EndLink
% 's resources on SGML and math.
% % \item
% % \Link[http://www.oreilly.com/people/staff/crism/math/]{}{}Chris Maden\EndLink.
% % A note about the
% % semantic of  mathematical markup
% \item
% \Link[http://mathforum.org/typesetting/index.html]
% Typesetting for the Internet\EndLink. A review of methods.
% % \item
% % \Link[http://www.nag.co.uk/projects/OpenMath/]{}{}OpenMath\EndLink.
% % \item 
% % \Link[http:www.mathtype.com]{}{}MathType\EndLink{}. An interactive editor 
% % for mathematical formulas in \LaTeX{} and MathML.
% % \item 
% % \Link[http://www.webeq.com/WebEQ/2.3/docs/webtex/webtex.html]{}{}Web\TeX\EndLink.
% % A \LaTeX-like equation markup language understood by the MathML-based
% % viewer WebEQ.
% % \item
% % \Link[http://www.nikhef.nl/\string~t16/public/ndvi/ndvi\string_doc.html]{}{}nDVI\EndLink{}
% % (Kasper Peeters).
% %  A DVI viewer plugin for Unix Netscape.  
% % \item
% %   \Link[http://www.mnemonic.org//mnemonic/documentation/doc/www/index.html]{}{}Mnemonic\EndLink.
% %   A browser under development supporting mathematics typesetting in
% %   the form of MathML and \TeX.
% % \item
% % \Link[http://www-4.ibm.com/software/network/techexplorer/]{}{}Techexplorer\EndLink.
% %   A plug-in for browsers, enabling the display of TeX, LaTeX, and
% %   MathML 
% % \item
% % \Link[http://www.mozilla.org/projects/mathml]{}{}MathML in
% % Mozilla\EndLink
% % \item
% % \Link[http://hutchinson.belmont.ma.us/tth/]{}{}TTH\EndLink.
% % Translates a subset of \TeX{} and \LaTeX{} math 
% % to a table-based representation and to MathML.
% % \item
% % \Link[http://genepi.louis-jean.com/omega/lyonmathml.pdf]{}{}Omega\EndLink.
% % A \TeX{} compiler with extra primitive to support SGML (and MathML) output.
% %%%%%%%%%%%%%%%%%
%  \EndList \EndHPage{}
% %%%%%%%%%%%%%%%%%%%%%%%%%%%%%%%%
% {\tt |}
% %%%%%%%%%%%%%%%%%%%%%%%%%%%%%%%
% \HPage{web publishing with LaTeX}
% \ExitHPage{}\SubSection{Web Publishing}
% \List{*}
% \item
% \Link[ftp://ctan.tug.org/tex-archive/info/webguide/webguide.html]{}{}A
%  Brief Guide to LaTeX Tools for Web Publishing\EndLink{} (Peter R. Wilson)
% \EndList   \EndHPage{}
% %%%%%%%%%%%%%%%%%%%%%%%%%%%%%%%
% {\tt |}
%%%%%%%%%%%%%%%%%%%%%%%%%%%%%%%
% \HPage{conversion services}
% \ExitHPage{}\SubSection{Conversion Services}
% \List{*}
% \item
% \Link[http://wheel.compose.cs.cmu.edu:8001/cgi-bin/browse/objweb]{}{}TOM
%      (Typed Object Model)\EndLink
% \EndList \EndHPage{}
\EndHTable


\DocPart{License}

% http://www.ctan.org/tex-archive/macros/latex/base/lppl.txt

\TeX4ht is provided under the
\Link[ftp://tug.ctan.org/tex-archive/macros/latex/base/lppl.txt]{}{}LaTeX
Project Public License\EndLink{} (LPPL).  



\DocPart{Acknowledgment}

I am very grateful for the suggestions, contributions, and bug reports
offered by many people. In particular, thanks go to Carmen Fierro,
Piotr Grabowski, Gertjan Klein, Sebastian Rahtz, and Philip Viton for
extensive feedback and help at early stages of this project.

This work was partially sponsored by NSF grant IIS-0312487.

\Sign{Eitan M. Gurari\HCode{<br />}\Link
    [mailto:gurari@cse.ohio-state.edu]{}{}gurari@cse.ohio-state.edu\EndLink
   {}}{\ifcase \month \or
   January \or February \or March \or April \or May \or June \or July
   \or August \or September \or October \or November \or December \fi
   \the\day, \the\year}




% \IgnorePar\EndP \Tg<hr />

% \TableOfContents[DocPart]


%%%%%%%%%%%%%%%%%%%%%%%%%%%%%%%%
\HPage{}

%%%%%%%%%%%%%%%%%%%%%%%%%%%%%%%%%%%%%%%%%%%%
% index
%%%%%%%%%%%%%%%%%%%%%%%%%%%%%%%%%%%%%%%%%%%%
\NextFile{\jobname-index.html}\HPage{}
\rightline{\Link[\jobname.html]{}{Index}up\EndLink}
\DocPart{Index}



\HAssign\ICount=0

\let\svIndexEntry\IndexEntry
\def\IndexEntry#1#2#3#4{\bgroup 
   \let\sv=\Link
   \def\Link[##1]##2##3##4\EndLink{\sv[##1]{##2}{##3}\ICount\EndLink}%
   \gHAdvance\ICount by 1
   \IndexSec#1//\let\IndexSec=\relax
   \svIndexEntry{#3#1}{#2}{}{#4}%
   \egroup}

\catcode`\:=11
  \def\IndexSec#1#2//{%
      \tmp:cnt=`#1\relax
      \ifnum \tmp:cnt>`Z \advance\tmp:cnt by -32 \fi
      \ifnum \tmp:cnt<`A \else \ifnum \tmp:cnt>`Z \else
          \ifnum \Idx:ch<\tmp:cnt
             \Configure{leftline}
               {\HCode{<br /><br /><span \Hnewline
                   class="IndexSec">}}{\HCode{</span><br />}}
             \leftline{\bf \char\tmp:cnt }%
             \xdef\Idx:ch{\the\tmp:cnt}%
             \global\let\prev:A\:UnDef  
          \fi
      \fi \fi % #1%
   }
\catcode`\:=12


\Index


\csname SysNeeds\endcsname{"rm \jobname.xdi"}
\csname SysNeeds\endcsname{"sort -f \jobname.idx -o \jobname.xdi"}



\EndHPage{}

%%%%%%%%%%%%%%%%%%%%%%%%%%%%%%%%%

                        % can't clean the full directory because
                        % the new html stuff is already there
  %%%%%%%%%%%%%%% FIRST      %%%%%%%%%%%%%%%%%%%%%

\csname SysNeeds\endcsname{"rm -r -f \DIR *.zip"}
\csname SysNeeds\endcsname{"mkdir \DIR"}
\csname SysNeeds\endcsname{"mkdir \DIR"}
\csname SysNeeds\endcsname{"chmod 711 \DIR"}
\csname SysNeeds\endcsname{"chmod 711 \DIR"}
                                    %%%%%%%%%%%%%%%%%%%%%

\csname SysNeeds\endcsname{"mkdir \DIR tex4ht"}   
\csname SysNeeds\endcsname{"chmod 755 \DIR tex4ht"}
\csname SysNeeds\endcsname{"mkdir \DIR tex4ht/temp"}   
\csname SysNeeds\endcsname{"chmod 755 \DIR tex4ht/temp"}
\csname SysNeeds\endcsname{"mkdir \DIR tex4ht/src"}   
\csname SysNeeds\endcsname{"chmod 755 \DIR tex4ht/src"}
\csname SysNeeds\endcsname{"mkdir \DIR tex4ht/bin"}   
\csname SysNeeds\endcsname{"mkdir \DIR tex4ht/bin/unix"}   
\csname SysNeeds\endcsname{"mkdir \DIR tex4ht/bin/win32"}   
\csname SysNeeds\endcsname{"chmod 755 \DIR tex4ht/bin"}
\csname SysNeeds\endcsname{"chmod 755 \DIR tex4ht/bin/unix"}
\csname SysNeeds\endcsname{"chmod 755 \DIR tex4ht/bin/win32"}
\csname SysNeeds\endcsname{"mkdir \DIR tex4ht/texmf"}   
\csname SysNeeds\endcsname{"mkdir \DIR tex4ht/texmf/tex"}   
\csname SysNeeds\endcsname{"mkdir \DIR tex4ht/texmf/tex/generic"}   
\csname SysNeeds\endcsname{"mkdir \DIR tex4ht/texmf/tex/generic/tex4ht"}   
\csname SysNeeds\endcsname{"mkdir \DIR tex4ht/texmf/tex4ht"}   
\csname SysNeeds\endcsname{"mkdir \DIR tex4ht/texmf/tex4ht/ht-fonts"}   
\csname SysNeeds\endcsname{"mkdir \DIR tex4ht/texmf/tex4ht/base"}   
\csname SysNeeds\endcsname{"mkdir \DIR tex4ht/texmf/tex4ht/bin"}   
\csname SysNeeds\endcsname{"mkdir \DIR tex4ht/texmf/tex4ht/xtpipes"}   
\csname SysNeeds\endcsname{"chmod 755 \DIR tex4ht/texmf"}   
\csname SysNeeds\endcsname{"chmod 755 \DIR tex4ht/texmf/tex"}   
\csname SysNeeds\endcsname{"chmod 755 \DIR tex4ht/texmf/tex/generic"}   
\csname SysNeeds\endcsname{"chmod 755 \DIR tex4ht/texmf/tex/generic/tex4ht"}   
\csname SysNeeds\endcsname{"chmod 755 \DIR tex4ht/texmf/tex4ht"}   
\csname SysNeeds\endcsname{"chmod 755 \DIR tex4ht/texmf/tex4ht/ht-fonts"}
\csname SysNeeds\endcsname{"chmod 755 \DIR tex4ht/texmf/tex4ht/base"}
\csname SysNeeds\endcsname{"chmod 755 \DIR tex4ht/texmf/tex4ht/xtpipes"}
\csname SysNeeds\endcsname{"chmod 755 \DIR tex4ht/texmf/tex4ht/bin"}

\csname SysNeeds\endcsname{"mkdir \DIR tex4ht/texmf/tex4ht/base/unix"}   
\csname SysNeeds\endcsname{"mkdir \DIR tex4ht/texmf/tex4ht/base/win32"}   
\csname SysNeeds\endcsname{"chmod 755 \DIR tex4ht/texmf/tex4ht/base/unix"}
\csname SysNeeds\endcsname{"chmod 755 \DIR tex4ht/texmf/tex4ht/base/win32"}




\csname SysNeeds\endcsname{"cp -R 
     \GOLD tex4ht.dir/texmf/tex4ht/ht-fonts
         \DIR  tex4ht/texmf/tex4ht/."}
\csname SysNeeds\endcsname{"chmod 755 
    \DIR tex4ht/texmf/tex4ht/ht-fonts/*"}
\csname SysNeeds\endcsname{"chmod 644
    \DIR tex4ht/texmf/tex4ht/ht-fonts/*.htf"}
\csname SysNeeds\endcsname{"chmod 755 
    \DIR tex4ht/texmf/tex4ht/ht-fonts/*/*"}
\csname SysNeeds\endcsname{"chmod 644
    \DIR tex4ht/texmf/tex4ht/ht-fonts/*/*.htf"}
\csname SysNeeds\endcsname{"chmod 755 
    \DIR tex4ht/texmf/tex4ht/ht-fonts/*/*/*"}
\csname SysNeeds\endcsname{"chmod 644
    \DIR tex4ht/texmf/tex4ht/ht-fonts/*/*/*.htf"}
\csname SysNeeds\endcsname{"chmod 755 
    \DIR tex4ht/texmf/tex4ht/ht-fonts/*/*/*/*"}
\csname SysNeeds\endcsname{"chmod 644
    \DIR tex4ht/texmf/tex4ht/ht-fonts/*/*/*/*.htf"}
\csname SysNeeds\endcsname{"chmod 755 
    \DIR tex4ht/texmf/tex4ht/ht-fonts/*/*/*/*/*"}
\csname SysNeeds\endcsname{"chmod 644
    \DIR tex4ht/texmf/tex4ht/ht-fonts/*/*/*/*/*.htf"}
\csname SysNeeds\endcsname{"chmod 755 
    \DIR tex4ht/texmf/tex4ht/ht-fonts/*/*/*/*/*/*"}
\csname SysNeeds\endcsname{"chmod 644
    \DIR tex4ht/texmf/tex4ht/ht-fonts/*/*/*/*/*/*.htf"}


\csname SysNeeds\endcsname{"cp     \GOLD main.dir/html.dir/htcmd.c 
    \DIR tex4ht/src/."}
\csname SysNeeds\endcsname{"cp    htcmd.exe 
    \DIR tex4ht/bin/win32/."}

\csname SysNeeds\endcsname{"cp    tex4ht.exe 
    \DIR tex4ht/bin/win32/tex4ht.exe"}
\csname SysNeeds\endcsname{"cp    t4ht.exe
    \DIR tex4ht/bin/win32/t4ht.exe"}
\csname SysNeeds\endcsname{"chmod 644 \DIR tex4ht/bin/win32/*.exe"}

\csname SysNeeds\endcsname{"cp    \GOLD main.dir/html.dir/mk4ht.perl
                                      \DIR tex4ht/bin/win32/mk4ht"}
\csname SysNeeds\endcsname{"chmod 644 \DIR tex4ht/bin/win32/mk4ht"}

\csname SysNeeds\endcsname{"cp    \GOLD main.dir/html.dir/mk4ht.perl
                                      \DIR tex4ht/bin/unix/mk4ht"}
\csname SysNeeds\endcsname{"chmod 644 \DIR tex4ht/bin/unix/mk4ht"}





\csname SysNeeds\endcsname{"chmod 711  ht-unix"}
\csname SysNeeds\endcsname{"chmod 711  ht-win32"}

\csname SysNeeds\endcsname{"rm foo"}
\csname SysNeeds\endcsname{"rm foo1"}
\csname SysNeeds\endcsname{"
    latex mkht-scripts.4ht ;    
    mv *.bat ht-win32/. ;   
    ls -l *.unix > foo ;
    sed -e's/-r.*--.*:.. //g'
        -e's/.unix//g' 
        -e's/.*/mv \string&.unix ht-unix\string\/\string&/g'
        < foo > foo1 ;
    chmod 700 foo1;
    foo1
     "}

\csname SysNeeds\endcsname{"cp    ht-win32/ht*
    \DIR tex4ht/bin/win32/."}
\csname SysNeeds\endcsname{"chmod 644 \DIR tex4ht/bin/win32/*"}
\csname SysNeeds\endcsname{"cp    ht-unix/ht*
    \DIR tex4ht/bin/unix/."}
\csname SysNeeds\endcsname{"chmod 644 \DIR tex4ht/bin/unix/*"}


\csname SysNeeds\endcsname{"mkdir  \DIR tex4ht/bin/ht"}
\csname SysNeeds\endcsname{"mkdir  \DIR tex4ht/bin/ht/win32"}
\csname SysNeeds\endcsname{"mkdir  \DIR tex4ht/bin/ht/unix"}
\csname SysNeeds\endcsname{"chmod 755  \DIR tex4ht/bin/ht"}
\csname SysNeeds\endcsname{"chmod 755  \DIR tex4ht/bin/ht/win32"}
\csname SysNeeds\endcsname{"chmod 755  \DIR tex4ht/bin/ht/unix"}



\csname SysNeeds\endcsname{"cp    ht-win32/*
    \DIR tex4ht/bin/ht/win32/."}
\csname SysNeeds\endcsname{"chmod 644 \DIR tex4ht/bin/ht/win32/*"}
\csname SysNeeds\endcsname{"cp    ht-unix/*
    \DIR tex4ht/bin/ht/unix/."}
\csname SysNeeds\endcsname{"chmod 644 \DIR tex4ht/bin/ht/unix/*"}





\csname SysNeeds\endcsname{"mv Wht.tab temp.log"}  
\csname SysNeeds\endcsname{"addMM.perl  temp.log > Wht.tab"}  


\csname SysNeeds\endcsname{"mv Wht.tab  ht.tab"}  
\csname SysNeeds\endcsname{"/n/gold/5/gurari/main.dir/bin.dir/myunix2dos 
                           ht.tab"}
\csname SysNeeds\endcsname{"cp   \GOLD tex4ht.dir/bin/ht/win32/ht.bat
                                 \DIR tex4ht/bin/win32/ht.bat"}
\csname SysNeeds\endcsname{"chmod 755 \DIR tex4ht/bin/win32/*"}



\csname SysNeeds\endcsname{"cp \GOLD tex4ht.dir/bin/ht/unix/ht
                               \DIR tex4ht/bin/unix/ht"}
\csname SysNeeds\endcsname{"chmod 755   \DIR tex4ht/bin/unix/*"}






% \csname SysNeeds\endcsname{"cp \GOLD main.dir/html.dir/xv4ht.java
%                         \DIR tex4ht/src/."}
% 
% \csname SysNeeds\endcsname{"cp  \GOLD 
%                         tex4ht.dir/texmf/tex4ht/bin/xv4ht.jar 
%                         \DIR tex4ht/bin/."}
% 
% \csname SysNeeds\endcsname{"cp \GOLD main.dir/html.dir/xv4ht.cat-unix
%                         \DIR tex4ht/texmf/tex4ht/base/unix/xv4ht.cat"}
% 
% \csname SysNeeds\endcsname{"cp \GOLD main.dir/html.dir/xv4ht.cat-win
%                         \DIR tex4ht/texmf/tex4ht/base/win32/xv4ht.cat"}


 

 

 





\csname SysNeeds\endcsname{"cp    testa.tex  \DIR tex4ht/temp/."}
\csname SysNeeds\endcsname{"cp    testb.tex  \DIR tex4ht/temp/."}
\csname SysNeeds\endcsname{"chmod 644        \DIR tex4ht/temp/*.tex"}

\csname SysNeeds\endcsname{"cp \GOLD main.dir/html.dir/tex4ht.c
                        \DIR tex4ht/src/."}
\csname SysNeeds\endcsname{"cp \GOLD main.dir/html.dir/t4ht.c
                        \DIR tex4ht/src/."}
\csname SysNeeds\endcsname{"chmod 644        \DIR tex4ht/src/*.c"}


\csname SysNeeds\endcsname{"cp \GOLD main.dir/html.dir/*.4ht
                        \DIR  tex4ht/texmf/tex/generic/tex4ht/."}
\csname SysNeeds\endcsname{"cp \GOLD main.dir/html.dir/tex4ht.sty
                        \DIR  tex4ht/texmf/tex/generic/tex4ht/."}
\csname SysNeeds\endcsname{"chmod 644
                        \DIR  tex4ht/texmf/tex/generic/tex4ht/*"}


\csname SysNeeds\endcsname{"cp \GOLD main.dir/html.dir/tex4ht.env-unix
                        \DIR tex4ht/texmf/tex4ht/base/unix/tex4ht.env"}
\csname SysNeeds\endcsname{"chmod 644      
                        \DIR tex4ht/texmf/tex4ht/base/unix/*"}
\csname SysNeeds\endcsname{"cp \GOLD main.dir/html.dir/tex4ht.env-win32
                        \DIR tex4ht/texmf/tex4ht/base/win32/tex4ht.env"}
\csname SysNeeds\endcsname{"chmod 644      
                        \DIR tex4ht/texmf/tex4ht/base/unix/*"}

\csname SysNeeds\endcsname{"cp -r \GOLD tex4ht.dir/texmf/tex4ht/bin
                        \DIR tex4ht/texmf/tex4ht/."}
\csname SysNeeds\endcsname{"cp -r \GOLD tex4ht.dir/texmf/tex4ht/xtpipes
                        \DIR tex4ht/texmf/tex4ht/."}





%
\csname SysNeeds\endcsname{"cp jhsample.tex 
                               \DIR  . ; chmod 644 \DIR jhsample.tex"}
%
\csname SysNeeds\endcsname{"cp \GOLD main.dir/html.dir/tex4ht.env-unix
                               \DIR  tex4ht-env-unix.txt"}
\csname SysNeeds\endcsname{"cp \GOLD main.dir/html.dir/tex4ht.env-win32
                               \DIR  tex4ht-env-win32.txt"}
\csname SysNeeds\endcsname{"chmod 644 \DIR  *.txt"}



\bgroup
%%%%%%%%%%%%%%%%%%%%
\Needs{" mkdir \DIR lit "}

 \Needs{" mkdir \DIR lit/src"}
 \Needs{" mkdir \DIR lit/src/java"}
 \Needs{" mkdir \DIR lit/src/java/xtpipes"}
 \Needs{" mkdir \DIR lit/src/java/xtpipes/util"}


%     cp \GOLD main.dir/html.dir/brail.dir/xtpipes.dir/xtpipes.dir/work.dir/xtpipes.java 
%     src/java/.  ;
%     cp \GOLD main.dir/html.dir/brail.dir/xtpipes.dir/xtpipes.dir/work.dir/xtpipes/Xtpipes.java 
%     src/java/xtpipes/.  ;
%     cp \GOLD main.dir/html.dir/brail.dir/xtpipes.dir/xtpipes.dir/work.dir/xtpipes/XtpipesUni.java 
%     src/java/xtpipes/. ; 
%     cp \GOLD main.dir/html.dir/brail.dir/xtpipes.dir/xtpipes.dir/work.dir/xtpipes/lib/ScriptsManager.java 
%     src/java/xtpipes/lib/. ;
%     cp \GOLD main.dir/html.dir/brail.dir/xtpipes.dir/xtpipes.dir/work.dir/xtpipes/lib/ScriptsManagerLH.java 
%     src/java/xtpipes/lib/. ;

%%%%%%%%%%%%%%%%%%%%


\def\LitNeeds#1{%    
   \def\file{\GOLD main.dir/html.dir/#1.tex}% 
   \openin15=\file\relax 
   \ifeof15 
      \def\file{\GOLD main.dir/html.dir/brail.dir/xtpipes.dir/xtpipes.dir/#1.tex}% 
      \openin15=\file\relax 
   \fi 
   \ifeof15\else 
      \closein15 
      \Needs{" cd \DIR lit ; cp \file\space . "}% 
   \fi 
} 



  \LitNeeds{tex4ht-4ht}                
%  \LitNeeds{tex4ht-bibtex2}           
  \LitNeeds{tex4ht-c}                  
  \LitNeeds{tex4ht-cond4ht}            
  \LitNeeds{tex4ht-docbook}            
  \LitNeeds{tex4ht-docbook-xtpipes}            
  \LitNeeds{tex4ht-jsml-xtpipes}            
  \LitNeeds{tex4ht-jsml}            
  \LitNeeds{tex4ht-xhtml-xtpipes}            
  \LitNeeds{tex4ht-env}                
  \LitNeeds{tex4ht-fonts-cjk}          
  \LitNeeds{tex4ht-fonts-cjk-utf8}          
  \LitNeeds{tex4ht-fonts-noncjk}       
  \LitNeeds{tex4ht-fonts-4hf}          
  \LitNeeds{tex4ht-fonts-modern}          
  \LitNeeds{tex4ht-htcmd}           
  \LitNeeds{tex4ht-html-speech-xtpipes}               
  \LitNeeds{tex4ht-html-speech}               
  \LitNeeds{tex4ht-html0}              
  \LitNeeds{tex4ht-html32}             
  \LitNeeds{tex4ht-html4}              
  \LitNeeds{tex4ht-info-html4}         
  \LitNeeds{tex4ht-info-mml}           
  \LitNeeds{tex4ht-info-svg}           
  \LitNeeds{tex4ht-info}               
  \LitNeeds{tex4ht-info-javahelp}               
  \LitNeeds{tex4ht-info-ooffice}               
  \LitNeeds{tex4ht-bibtex2}               
  \LitNeeds{tex4ht-javahelp}           
  \LitNeeds{tex4ht-mathltx}            
  \LitNeeds{tex4ht-mathml}              
  \LitNeeds{tex4ht-jsmath}             
  \LitNeeds{tex4ht-mathplayer}         
  \LitNeeds{tex4ht-mkht}               
  \LitNeeds{tex4ht-moz}                
  \LitNeeds{tex4ht-ooffice}            
  \LitNeeds{tex4ht-options}            
  \LitNeeds{tex4ht-sty}                
       \LitNeeds{wripro}                    
  \LitNeeds{tex4ht-svg}                
  \LitNeeds{tex4ht-t4ht}               
  \LitNeeds{tex4ht-tei}                
%  \LitNeeds{tex4ht-texconverter} 
  \LitNeeds{tex4ht-unicode}            
  \LitNeeds{tex4ht-word}               
  \LitNeeds{tex4ht-dir}                
  \LitNeeds{tex4ht-auto-script}                
  \LitNeeds{tex4ht-oo-xtpipes} 
  \LitNeeds{xtpipes} 
  \LitNeeds{tex4ht-htcmd} 

%%%%%%%%%%%%%%%%%%%%%%%%%%%%%%%%%%%%%%%%%%%%%%



\Needs{"
cp /home/4/gurari/main.dir/html.dir/brail.dir/xtpipes.dir/xtpipes.dir/work.dir/xtpipes.java    \DIR lit/src/java/.  ;
cp /home/4/gurari/main.dir/html.dir/brail.dir/xtpipes.dir/xtpipes.dir/work.dir/xtpipes/Xtpipes.java    \DIR lit/src/java/xtpipes/.  ;
cp /home/4/gurari/main.dir/html.dir/brail.dir/xtpipes.dir/xtpipes.dir/work.dir/xtpipes/XtpipesUni.java    \DIR lit/src/java/xtpipes/.  ;
cp /home/4/gurari/main.dir/html.dir/brail.dir/xtpipes.dir/xtpipes.dir/work.dir/xtpipes/util/ScriptsManager.java    \DIR      lit/src/java/xtpipes/util/.  ;
cp /home/4/gurari/main.dir/html.dir/brail.dir/xtpipes.dir/xtpipes.dir/work.dir/xtpipes/util/ScriptsManagerLH.java \DIR lit/src/java/xtpipes/util/.  ;
cp /home/4/gurari/main.dir/html.dir/brail.dir/xtpipes.dir/xtpipes.dir/work.dir/xtpipes/InputObject.java    \DIR lit/src/java/xtpipes/.  ;
cp /home/4/gurari/main.dir/html.dir/brail.dir/xtpipes.dir/xtpipes.dir/work.dir/xtpipes/FileInfo.java    \DIR lit/src/java/xtpipes/.  ;
cp /home/4/gurari/main.dir/html.dir/GroupMn.java          \DIR lit/src/java/.  ;
cp /home/4/gurari/main.dir/html.dir/HtJsml.java           \DIR lit/src/java/.  ;
cp /home/4/gurari/main.dir/html.dir/HtSpk.java            \DIR lit/src/java/.  ;
cp /home/4/gurari/main.dir/html.dir/JsmlFilter.java       \DIR lit/src/java/.  ;
cp /home/4/gurari/main.dir/html.dir/JsmlMathBreak.java    \DIR lit/src/java/.  ;
cp /home/4/gurari/main.dir/html.dir/EmSpk.java            \DIR lit/src/java/.  ;
cp /home/4/gurari/main.dir/html.dir/OoFilter.java         \DIR lit/src/java/.  ;
cp /home/4/gurari/main.dir/html.dir/OoUtilities.java      \DIR lit/src/java/.  ;
cp /home/4/gurari/main.dir/html.dir/OomFilter.java        \DIR lit/src/java/. 
"}

%%%%%%%%%%%%%%%%%%%%
\Needs{"
          cd \DIR lit
;         zip tex4ht-lit *
;         mv tex4ht-lit.zip \DIR .
"}
\Needs{"
          chmod 644  \DIR tex4ht-lit.zip
"}
\Needs{" /usr/bin/rm -r \DIR lit
"}
%%%%%%%%%%%%%%%%%%%%%%%%%%
\egroup






\csname SysNeeds\endcsname{"cd \DIR; ln -s  mn.html index.html"}
\csname SysNeeds\endcsname{"chmod 644 \DIR index.html"}



\csname SysNeeds\endcsname{"cd \DIR tex4ht
                          ; zip -r tex4ht.zip *
                          ; mv \DIR tex4ht/tex4ht.zip \DIR .
                          ; cd \DIR
                          ; rm  -r tex4ht
                          ; chmod 644 *.zip 
                          ; zip -r tex4ht-all.zip *  
                          ; chmod 644 *.zip 
                          ; ln -s tex4ht.zip 
                                  tex4ht-\the\year 
                                         \ifnum \month<10 0\fi
                                         \the\month
                                         \ifnum \day<10 0\fi
                                         \the\day.zip
       "}

\csname SysNeeds\endcsname{"echo '*************** DONE ***************'"}

\bye


% LocalWords:  MathTran MathML
